\input webmac
% This program is copyright (C) 1985, 1988, 2010 by Oren Patashnik;
% all rights are reserved.
%
% This program, BibTeX, is available under the same terms as
% Donald Knuth's TeX program.
%
% (Request to implementors: The WEB system provides for alterations via
% an auxiliary file; the master file should stay intact.)
%
% See Appendix H of the WEB manual for hints on how to install this program.

% Version 0.98f was released in March 1985.
% Version 0.98g was released in April; it removed some system dependencies
%       (introducing term_in and term_out in place of just tty, and removing
%       some nonlocal goto's) and it gave context for certain parsing errors.
% Version 0.98h was released in April; it patched a bug in the output
%       line-breaking routine that can arise with some nonstandard style files.
% Version 0.98i was released in May; its main change split up the main program
%       and some procedures to help certain compilers cope with size
%       limitations, among other things changing error and warning macros so
%       they'd produce (much) less inline code; it also redefined the class of
%       legal style-file identifiers---although this affects only the bizarre
%       ones, it makes BibTeX's error messages more coherent; and it had many
%       minor changes, including about a 15% speed-up on TOPS-20.
% Version 0.99a was released in January 1988.  Its main changes: allowed the
%       inclusion of entire .bib files (rather than just those entries
%       \cited or \nocited); made the sorting algorithm stable; eliminated
%       any case conversion for file names; allowed concatenation in database
%       fields and string definitions; handled hyphenated names properly;
%       handled accented characters properly; implemented new empty$,
%       preamble$, text.length$, text.prefix$, and warning$ built-in functions;
%       allowed a new cross-referencing feature; and made many minor fixes,
%       including about a 40% speed-up on TOPS-20.
% Version 0.99b was released in February 1988.  It changed text.length$ and
%       text.prefix$ to not count braces as text characters, and it changed
%       text.prefix$ to add any necessary matching right braces.
% Version 0.99c was released in February 1988.  It removed two begin-end pairs
%       that, for convention only, surrounded entire modules, but that elicited
%       label-related complaints from some compilers.
% Version 0.99d was released in March 2010.  It made output lines breakable
%       only at white_space (so that, for example, URLs would not be broken).
%       Other known bugs (all minor) will be fixed in a subsequent release.
% Updated bibtex.web was released on 8 December 2010.  Still version
%       0.99d; this release clarified the license.

% Please report any bugs to biblio@tug.org

% Although considerable effort has been expended to make the BibTeX program
% correct and reliable, no warranty is implied; the author disclaims any
% obligation or liability for damages, including but not limited to
% special, indirect, or consequential damages arising out of or in
% connection with the use or performance of this software.

% This program was written by Oren Patashnik, in consultation with Leslie
% Lamport, to be used with Lamport's LaTeX document preparation system.
% Some modules were taken from Knuth's TeX and TeXware with his permission.

% Here is TeX material that gets inserted after \input webmac
\def\hang{\hangindent 3em\indent\ignorespaces}
\font\ninerm=cmr9
\let\mc=\ninerm % medium caps for names like PASCAL
\def\PASCAL{{\mc PASCAL}}
\def\ph{{\mc PASCAL-H}}
\def\<#1>{$\langle#1\rangle$}
\def\section{\mathhexbox278}

\def\(#1){} % this is used to make section names sort themselves better
\def\9#1{} % this is used for sort keys in the index via @:sort key}{entry@>

% Note: WEAVE will typeset an upper-case `E' in a PASCAL identifier a
% bit strangely so that the `TeX' in the name of this program is typeset
% correctly; if this becomes a problem remove these three lines to get
% normal upper-case `E's in PASCAL identifiers
\def\drop{\kern-.1667em\lower.5ex\hbox{E}\kern-.125em} % middle of TeX
\catcode`E=13 \uppercase{\def E{e}}
\def\\#1{\hbox{\let E=\drop\it#1\/\kern.05em}} % italic type for identifiers

\font\sc=cmcsc10

\def\BibTeX{{\rm B\kern-.05em{\sc i\kern-.025em b}\kern-.08em
    T\kern-.1667em\lower.7ex\hbox{E}\kern-.125emX}}

\def\LaTeX{{\rm L\kern-.36em\raise.3ex\hbox{\sc a}\kern-.15em
    T\kern-.1667em\lower.7ex\hbox{E}\kern-.125emX}}

\def\title{\BibTeX\ }
\def\today{\ifcase\month\or
  January\or February\or March\or April\or May\or June\or
  July\or August\or September\or October\or November\or December\fi
  \space\number\day, \number\year}
\def\topofcontents{\null\vfill
 \def\titlepage{F}
 \centerline{\:\titlefont The {\:\ttitlefont \BibTeX} preprocessor}
 \vskip 15pt \centerline{(Version 0.99d---\today)} \vfill}
\pageno=\contentspagenumber \advance\pageno by 1




\N1.  Introduction.
\BibTeX\ is a preprocessor (with elements of postprocessing as
explained below) for the \LaTeX\ document-preparation system.  It
handles most of the formatting decisions required to produce a
reference list, outputting a \.{.bbl} file that a user can edit to add
any finishing touches \BibTeX\ isn't designed to handle (in practice,
such editing almost never is needed); with this file \LaTeX\ actually
produces the reference list.

Here's how \BibTeX\ works.  It takes as input (a)~an \.{.aux} file
produced by \LaTeX\ on an earlier run; (b)~a \.{.bst} file (the style
file), which specifies the general reference-list style and specifies
how to format individual entries, and which is written by a style
designer (called a wizard throughout this program) in a
special-purpose language described in the \BibTeX\ documentation---see
the file {\.{btxdoc.tex}}; and (c)~\.{.bib} file(s) constituting a
database of all reference-list entries the user might ever hope to
use.  \BibTeX\ chooses from the \.{.bib} file(s) only those entries
specified by the \.{.aux} file (that is, those given by \LaTeX's
\.{\\cite} or \.{\\nocite} commands), and creates as output a \.{.bbl}
file containing these entries together with the formatting commands
specified by the \.{.bst} file (\BibTeX\ also creates a \.{.blg} log
file, which includes any error or warning messages, but this file
isn't used by any program).  \LaTeX\ will use the \.{.bbl} file,
perhaps edited by the user, to produce the reference list.

Many modules of \BibTeX\ were taken from Knuth's \TeX\ and \TeX ware,
with his permission.  All known system-dependent modules are marked in
the index entry ``system dependencies''; Dave Fuchs helped exorcise
unwanted ones.  In addition, a few modules that can be changed to make
\BibTeX\ smaller are marked in the index entry ``space savings''.

Megathanks to Howard Trickey, for whose suggestions future users and
style writers would be eternally grateful, if only they knew.

The \\{banner} string defined here should be changed whenever \BibTeX\
gets modified.

\Y\P\D \37$\\{banner}\S\.{\'This\ is\ BibTeX,\ Version\ 0.99d\'}$\C{printed
when the program starts}\par
\fi

\M2.
Terminal output goes to the file \\{term\_out}, while terminal input
comes from \\{term\_in}.  On our system, these (system-dependent) files
are already opened at the beginning of the program, and have the same
real name.

\Y\P\D \37$\\{term\_out}\S\\{tty}$\par
\P\D \37$\\{term\_in}\S\\{tty}$\par
\fi

\M3.
This program uses the term \\{print} instead of \\{write} when writing on
both the \\{log\_file} and (system-dependent) \\{term\_out} file, and it
uses \\{trace\_pr} when in  \&{trace}  mode, for which it writes on just the
\\{log\_file}.  If you want to change where either set of macros writes
to, you should also change the other macros in this program for that
set; each such macro begins with \\{print\_} or \\{trace\_pr\_}.

\Y\P\D \37$\\{print}(\#)\S$\1\6
\&{begin} \37$\\{write}(\\{log\_file},\39\#)$;\5
$\\{write}(\\{term\_out},\39\#)$;\6
\&{end}\2\par
\P\D \37$\\{print\_ln}(\#)\S$\1\6
\&{begin} \37$\\{write\_ln}(\\{log\_file},\39\#)$;\5
$\\{write\_ln}(\\{term\_out},\39\#)$;\6
\&{end}\2\par
\P\D \37$\\{print\_newline}\S\\{print\_a\_newline}$\C{making this a procedure
saves a little space}\Y\par
\P\D \37$\\{trace\_pr}(\#)\S$\1\6
\&{begin} \37$\\{write}(\\{log\_file},\39\#)$;\6
\&{end}\2\par
\P\D \37$\\{trace\_pr\_ln}(\#)\S$\1\6
\&{begin} \37$\\{write\_ln}(\\{log\_file},\39\#)$;\6
\&{end}\2\par
\P\D \37$\\{trace\_pr\_newline}\S$\1\6
\&{begin} \37$\\{write\_ln}(\\{log\_file})$;\6
\&{end}\2\par
\Y\P$\4\X3:Procedures and functions for all file I/O, error messages, and such%
\X\S$\6
\4\&{procedure}\1\  \37\\{print\_a\_newline};\2\6
\&{begin} \37$\\{write\_ln}(\\{log\_file})$;\5
$\\{write\_ln}(\\{term\_out})$;\6
\&{end};\par
\As18, 44, 45, 46, 47, 51, 53, 59, 82, 95, 96, 98, 99, 108, 111, 112, 113, 114,
115, 121, 128, 137, 138, 144, 148, 149, 150, 153, 157, 158, 159, 165, 166, 167,
168, 169, 188, 220, 221, 222, 226, 229, 230, 231, 232, 233, 234, 235, 240, 271,
280, 281, 284, 293, 294, 295, 310, 311, 313, 321, 356, 368, 373\ETs456.
\U12.\fi

\M4.
Some of the code below is intended to be used only when diagnosing the
strange behavior that sometimes occurs when \BibTeX\ is being
installed or when system wizards are fooling around with \BibTeX\
without quite knowing what they are doing. Such code will not normally
be compiled; it is delimited by the codewords
`$ \&{debug} \ldots  \&{gubed} $', with apologies to people who wish to
preserve the purity of English. Similarly, there is some conditional
code delimited by `$ \&{stat} \ldots  \&{tats} $' that is intended only for use
when statistics are to be kept about \BibTeX's memory/cpu usage,
and there is conditional code delimited by `$ \&{trace} \ldots  \&{ecart} $'
that is intended to be a trace facility for use mainly when debugging
\.{.bst} files.

\Y\P\D \37$\\{debug}\S\B$\C{ remove the `$\B$' when debugging }\par
\P\D \37$\\{gubed}\S\hbox{}\T$\C{ remove the `$\T$' when debugging }\par
\P\F \37$\\{debug}\S\\{begin}$\par
\P\F \37$\\{gubed}\S\\{end}$\Y\par
\P\D \37$\\{stat}\S\B$\C{ remove the `$\B$' when keeping statistics }\par
\P\D \37$\\{tats}\S\hbox{}\T$\C{ remove the `$\T$' when keeping statistics }\par
\P\F \37$\\{stat}\S\\{begin}$\par
\P\F \37$\\{tats}\S\\{end}$\Y\par
\P\D \37$\\{trace}\S\B$\C{ remove the `$\B$' when in  \&{trace}  mode }\par
\P\D \37$\\{ecart}\S\hbox{}\T$\C{ remove the `$\T$' when in  \&{trace}  mode }%
\par
\P\F \37$\\{trace}\S\\{begin}$\par
\P\F \37$\\{ecart}\S\\{end}$\par
\fi

\M5.
We assume that   \&{case}  statements may include a
default case that applies if no matching label is found,
since most \PASCAL\ compilers have plugged this hole in the language by
incorporating some sort of default mechanism. For example, the \ph\
compiler allows `\\{others}:' as a default label, and other \PASCAL s allow
syntaxes like `\ignorespaces \&{else} \unskip' or `\\{otherwise}' or
`\\{otherwise}:', etc. The definitions of  \&{othercases}  and   \&{endcases}
should be changed to agree with local conventions.   Note that no semicolon
appears before   \&{endcases}  in this program, so the definition of   %
\&{endcases}
should include a semicolon if the compiler wants one.  (Of course, if no
default mechanism is available, the   \&{case}  statements of \BibTeX\ will
have
to be laboriously extended by listing all remaining cases. People who are
stuck with such \PASCAL s have in fact done this, successfully but not
happily!)

\Y\P\D \37$\\{othercases}\S\\{others}$: \37\C{default for cases not listed
explicitly}\par
\P\D \37$\\{endcases}\S$\ \&{end} \C{follows the default case in an extended   %
\&{case}  statement}\par
\P\F \37$\\{othercases}\S\\{else}$\par
\P\F \37$\\{endcases}\S\\{end}$\par
\fi

\M6.
Labels are given symbolic names by the following definitions, so that
occasional \&{goto}  statements will be meaningful.  We insert the label
`\\{exit}:' just before the `\ignorespaces  \&{end} \unskip' of a procedure
in which we have used the `\&{return}' statement defined below (and this
is the only place `\\{exit}:' appears).  This label is sometimes used
for exiting loops that are set up with the  \~ \&{loop} construction defined
below.  Another generic label is `\\{loop\_exit}:'; it appears
immediately after a loop.

Incidentally, this program never declares a label that isn't actually used,
because some fussy \PASCAL\ compilers will complain about redundant labels.

\Y\P\D \37$\\{exit}=10$\C{go here to leave a procedure}\par
\P\D \37$\\{loop\_exit}=15$\C{go here to leave a loop within a procedure}\par
\P\D \37$\\{loop1\_exit}=16$\C{the first generic label for a procedure with
two}\par
\P\D \37$\\{loop2\_exit}=17$\C{the second}\par
\fi

\M7.
And  \&{while}  we're discussing loops: This program makes into  \&{while}
loops many that would otherwise be  \&{for}  loops because of Standard
\PASCAL\ limitations (it's a bit complicated---standard \PASCAL\
doesn't allow a global variable as the index of a  \&{for}  loop inside a
procedure; furthermore, many compilers have fairly severe limitations
on the size of a block, including the main block of the program; so
most of the code in this program occurs inside procedures, and since
for other reasons this program must use primarily global variables, it
doesn't use many  \&{for}  loops).


\fi

\M8.
This program uses this convention: If there are several quantities in
a boolean expression, they are ordered by expected frequency (except
perhaps when an error message results) so that execution will be
fastest; this is more an attempt to understand the program than to
make it faster.


\fi

\M9.
Here are some macros for common programming idioms.

\Y\P\D \37$\\{incr}(\#)\S\#\K\#+1$\C{increase a variable by unity}\par
\P\D \37$\\{decr}(\#)\S\#\K\#-1$\C{decrease a variable by unity}\par
\P\D \37$\\{loop}\S$\ \&{while} $\\{true}$ \1\&{do}\ \C{repeat over and over
until a \&{goto}  happens}\par
\P\F \37$\\{loop}\S\\{xclause}$\C{\.{WEB}'s  \~ \&{xclause} acts like `%
\ignorespaces \&{while} $\\{true}$ \&{do}\unskip'}\par
\P\D \37$\\{do\_nothing}\S$\C{empty statement}\par
\P\D \37$\\{return}\S$\1\5
\&{goto} \37\\{exit}\C{terminate a procedure call}\2\par
\P\F \37$\\{return}\S\\{nil}$\par
\P\D \37$\\{empty}=0$\C{symbolic name for a null constant}\par
\P\D \37$\\{any\_value}=0$\C{this appeases \PASCAL's boolean-evaluation scheme}%
\par
\fi

\N10.  The main program.
This program first reads the \.{.aux} file that \LaTeX\ produces,
(\romannumeral1) determining which \.{.bib} file(s) and \.{.bst} file
to read and (\romannumeral2) constructing a list of cite keys in order
of occurrence.  The \.{.aux} file may have other \.{.aux} files nested
within.  Second, it reads and executes the \.{.bst} file,
(\romannumeral1) determining how and in which order to process the
database entries in the \.{.bib} file(s) corresponding to those cite
keys in the list (or in some cases, to all the entries in the \.{.bib}
file(s)), (\romannumeral2) determining what text to be output for each
entry and determining any additional text to be output, and
(\romannumeral3) actually outputting this text to the \.{.bbl} file.
In addition, the program sends error messages and other remarks to the
\\{log\_file} and terminal.

\Y\P\D \37$\\{close\_up\_shop}=9998$\C{jump here after fatal errors}\par
\P\D \37$\\{exit\_program}=9999$\C{jump here if we couldn't even get started}%
\par
\Y\P\hbox{\4}\X11:Compiler directives\X\6
\4\&{program}\1\  \37\\{BibTEX};\C{all files are opened dynamically}\6
\4\&{label} \37$\\{close\_up\_shop},\39\\{exit\_program}$\1\5
\X109:Labels in the outer block\X;\2\6
\4\&{const} \X14:Constants in the outer block\X \6
\4\&{type} \X22:Types in the outer block\X \6
\4\&{var} \37\X16:Globals in the outer block\X\6
\X12:Procedures and functions for about everything\X\6
\X13:The procedure \\{initialize}\X\7
\&{begin} \37\\{initialize};\5
$\\{print\_ln}(\\{banner})$;\6
\X110:Read the \.{.aux} file\X;\6
\X151:Read and execute the \.{.bst} file\X;\6
\4\\{close\_up\_shop}: \37\X455:Clean up and leave\X;\6
\4\\{exit\_program}: \37\&{end}.\par
\fi

\M11.
If the first character of a \PASCAL\ comment is a dollar sign,
\ph\ treats the comment as a list of ``compiler directives'' that will
affect the translation of this program into machine language.  The
directives shown below specify full checking and inclusion of the \PASCAL\
debugger when \BibTeX\ is being debugged,
but they cause range checking and other
redundant code to be eliminated when the production system is being generated.
Arithmetic overflow will be detected in all cases.

\Y\P$\4\X11:Compiler directives\X\S$\6
$\B\J\$\|C-,\39\|A+,\39\|D-\T$\C{no range check, catch arithmetic overflow, no
debug overhead}\6
\&{debug} \37$\B\J\$\|C+,\39\|D+\T$\ \&{gubed}\C{but turn everything on when
debugging}\par
\U10.\fi

\M12.
All procedures in this program (except for \\{initialize}) are grouped
into one of the seven classes below, and these classes are dispersed
throughout the program.  However: Much of this program is written top
down, yet \PASCAL\ wants its procedures bottom up.  Since mooning is
neither a technically nor a socially acceptable solution to the
bottom-up problem, this section instead performs the topological
gymnastics that \.{WEB} allows, ordering these classes to satisfy
\PASCAL\ compilers.  There are a few procedures still out of place
after this ordering, though, and the other modules that complete the
task have ``gymnastics'' as an index entry.

\Y\P$\4\X12:Procedures and functions for about everything\X\S$\6
\X3:Procedures and functions for all file I/O, error messages, and such\X\6
\X38:Procedures and functions for file-system interacting\X\6
\X54:Procedures and functions for handling numbers, characters, and strings\X\6
\X83:Procedures and functions for input scanning\X\6
\X367:Procedures and functions for name-string processing\X\6
\X307:Procedures and functions for style-file function execution\X\6
\X100:Procedures and functions for the reading and processing of input files\X%
\par
\U10.\fi

\M13.
This procedure gets things started properly.

\Y\P$\4\X13:The procedure \\{initialize}\X\S$\6
\4\&{procedure}\1\  \37\\{initialize}; \6
\4\&{var}  \X23:Local variables for initialization\X \6
\&{begin} \37\X17:Check the ``constant'' values for consistency\X;\6
\&{if} $(\\{bad}>0)$ \1\&{then}\6
\&{begin} \37$\\{write\_ln}(\\{term\_out},\39\\{bad}:0,\39\.{\'\ is\ a\ bad\
bad\'})$;\5
\&{goto} \37\\{exit\_program};\6
\&{end};\2\6
\X20:Set initial values of key variables\X;\6
\\{pre\_def\_certain\_strings};\6
\\{get\_the\_top\_level\_aux\_file\_name};\6
\&{end};\par
\U10.\fi

\M14.
These parameters can be changed at compile time to extend or reduce
\BibTeX's capacity.  They are set to accommodate about 750 cites when
used with the standard styles, although \\{pool\_size} is usually the
first limitation to be a problem, often when there are 500 cites.

\Y\P$\4\X14:Constants in the outer block\X\S$\6
$\\{buf\_size}=1000$;\C{maximum number of characters in an input line (or
string)}\6
$\\{min\_print\_line}=3$;\C{minimum \.{.bbl} line length: must be $\G3$}\6
$\\{max\_print\_line}=79$;\C{the maximum: must be $>\\{min\_print\_line}$ and
$<\\{buf\_size}$}\6
$\\{aux\_stack\_size}=20$;\C{maximum number of simultaneous open \.{.aux}
files}\6
$\\{max\_bib\_files}=20$;\C{maximum number of \.{.bib} files allowed}\6
$\\{pool\_size}=65000$;\C{maximum number of characters in strings}\6
$\\{max\_strings}=4000$;\C{maximum number of strings, including pre-defined;
                                                     must be $\L\\{hash%
\_size}$}\6
$\\{max\_cites}=750$;\C{maximum number of distinct cite keys; must be
                                              $\L\\{max\_strings}$}\6
$\\{min\_crossrefs}=2$;\C{minimum number of cross-refs required for automatic
                                                      \\{cite\_list} inclusion}%
\6
$\\{wiz\_fn\_space}=3000$;\C{maximum amount of \\{wiz\_defined}-function space}%
\6
$\\{single\_fn\_space}=100$;\C{maximum amount for a single \\{wiz%
\_defined}-function}\6
$\\{max\_ent\_ints}=3000$;\C{maximum number of \\{int\_entry\_var}s
                            (entries $\times$ \\{int\_entry\_var}s)}\6
$\\{max\_ent\_strs}=3000$;\C{maximum number of \\{str\_entry\_var}s
                            (entries $\times$ \\{str\_entry\_var}s)}\6
$\\{ent\_str\_size}=100$;\C{maximum size of a \\{str\_entry\_var}; must be $\L%
\\{buf\_size}$}\6
$\\{glob\_str\_size}=1000$;\C{maximum size of a \\{str\_global\_var};
                                              must be $\L\\{buf\_size}$}\6
$\\{max\_fields}=17250$;\C{maximum number of fields (entries $\times$ fields,
                                      about $23\ast\\{max\_cites}$ for
consistency)}\6
$\\{lit\_stk\_size}=100$;\C{maximum number of literal functions on the stack}%
\par
\A333.
\U10.\fi

\M15.
These parameters can also be changed at compile time, but they're
needed to define some \.{WEB} numeric macros so they must be so
defined themselves.

\Y\P\D \37$\\{hash\_size}=5000$\C{must be $\G\\{max\_strings}$ and $\G\\{hash%
\_prime}$}\par
\P\D \37$\\{hash\_prime}=4253$\C{a prime number about 85\% of \\{hash\_size}
and $\G128$                                                 and $<%
\hbox{$2^{14}-2^6$}$}\par
\P\D \37$\\{file\_name\_size}=40$\C{file names shouldn't be longer than this}%
\par
\P\D \37$\\{max\_glob\_strs}=10$\C{maximum number of \\{str\_global\_var}
names}\par
\P\D \37$\\{max\_glb\_str\_minus\_1}=\\{max\_glob\_strs}-1$\C{to avoid wasting
a \\{str\_global\_var}}\par
\fi

\M16.
In case somebody has inadvertently made bad settings of the ``constants,''
\BibTeX\ checks them using a global variable called \\{bad}.

This is the first of many sections of \BibTeX\ where global variables are
defined.

\Y\P$\4\X16:Globals in the outer block\X\S$\6
\4\\{bad}: \37\\{integer};\C{is some ``constant'' wrong?}\par
\As19, 24, 30, 34, 37, 41, 43, 48, 65, 74, 76, 78, 80, 89, 91, 97, 104, 117,
124, 129, 147, 161, 163, 195, 219, 247, 290, 331, 337, 344\ETs365.
\U10.\fi

\M17.
Each digit-value of \\{bad} has a specific meaning.

\Y\P$\4\X17:Check the ``constant'' values for consistency\X\S$\6
$\\{bad}\K0$;\6
\&{if} $(\\{min\_print\_line}<3)$ \1\&{then}\5
$\\{bad}\K1$;\2\6
\&{if} $(\\{max\_print\_line}\L\\{min\_print\_line})$ \1\&{then}\5
$\\{bad}\K10\ast\\{bad}+2$;\2\6
\&{if} $(\\{max\_print\_line}\G\\{buf\_size})$ \1\&{then}\5
$\\{bad}\K10\ast\\{bad}+3$;\2\6
\&{if} $(\\{hash\_prime}<128)$ \1\&{then}\5
$\\{bad}\K10\ast\\{bad}+4$;\2\6
\&{if} $(\\{hash\_prime}>\\{hash\_size})$ \1\&{then}\5
$\\{bad}\K10\ast\\{bad}+5$;\2\6
\&{if} $(\\{hash\_prime}\G(16384-64))$ \1\&{then}\5
$\\{bad}\K10\ast\\{bad}+6$;\2\6
\&{if} $(\\{max\_strings}>\\{hash\_size})$ \1\&{then}\5
$\\{bad}\K10\ast\\{bad}+7$;\2\6
\&{if} $(\\{max\_cites}>\\{max\_strings})$ \1\&{then}\5
$\\{bad}\K10\ast\\{bad}+8$;\2\6
\&{if} $(\\{ent\_str\_size}>\\{buf\_size})$ \1\&{then}\5
$\\{bad}\K10\ast\\{bad}+9$;\2\6
\&{if} $(\\{glob\_str\_size}>\\{buf\_size})$ \1\&{then}\5
$\\{bad}\K100\ast\\{bad}+11$;\C{well, almost each}\2\par
\A302.
\U13.\fi

\M18.
A global variable called \\{history} will contain one of four values at
the end of every run: \\{spotless} means that no unusual messages were
printed; \\{warning\_message} means that a message of possible interest
was printed but no serious errors were detected; \\{error\_message} means
that at least one error was found; \\{fatal\_message} means that the
program terminated abnormally. The value of \\{history} does not
influence the behavior of the program; it is simply computed for the
convenience of systems that might want to use such information.

\Y\P\D \37$\\{spotless}=0$\C{\\{history} value for normal jobs}\par
\P\D \37$\\{warning\_message}=1$\C{\\{history} value when non-serious info was
printed}\par
\P\D \37$\\{error\_message}=2$\C{\\{history} value when an error was noted}\par
\P\D \37$\\{fatal\_message}=3$\C{\\{history} value when we had to stop
prematurely}\par
\Y\P$\4\X3:Procedures and functions for all file I/O, error messages, and such%
\X\mathrel{+}\S$\6
\4\&{procedure}\1\  \37\\{mark\_warning};\2\6
\&{begin} \37\&{if} $(\\{history}=\\{warning\_message})$ \1\&{then}\5
$\\{incr}(\\{err\_count})$\6
\4\&{else} \&{if} $(\\{history}=\\{spotless})$ \1\&{then}\6
\&{begin} \37$\\{history}\K\\{warning\_message}$;\5
$\\{err\_count}\K1$;\6
\&{end};\2\2\6
\&{end};\7
\4\&{procedure}\1\  \37\\{mark\_error};\2\6
\&{begin} \37\&{if} $(\\{history}<\\{error\_message})$ \1\&{then}\6
\&{begin} \37$\\{history}\K\\{error\_message}$;\5
$\\{err\_count}\K1$;\6
\&{end}\6
\4\&{else} \C{$\\{history}=\\{error\_message}$}\2\6
$\\{incr}(\\{err\_count})$;\6
\&{end};\7
\4\&{procedure}\1\  \37\\{mark\_fatal};\2\6
\&{begin} \37$\\{history}\K\\{fatal\_message}$;\6
\&{end};\par
\fi

\M19.
For the two states \\{warning\_message} and \\{error\_message} we keep track
of the number of messages given; but since \\{warning\_message}s aren't
so serious, we ignore them once we've seen an \\{error\_message}.  Hence
we need just the single variable \\{err\_count} to keep track.


\Y\P$\4\X16:Globals in the outer block\X\mathrel{+}\S$\6
\4\\{history}: \37$\\{spotless}\to\\{fatal\_message}$;\C{how bad was this run?}%
\6
\4\\{err\_count}: \37\\{integer};\par
\fi

\M20.
The \\{err\_count} gets set or reset when \\{history} first changes to
\\{warning\_message} or \\{error\_message}, so we don't need to initialize
it.

\Y\P$\4\X20:Set initial values of key variables\X\S$\6
$\\{history}\K\\{spotless}$;\par
\As25, 27, 28, 32, 33, 35, 67, 72, 119, 125, 131, 162, 164, 196\ETs292.
\U13.\fi

\N21.  The character set.
(The following material is copied (almost) verbatim from \TeX.
Thus, the same system-dependent changes should be made to both programs.)

In order to make \TeX\ readily portable between a wide variety of
computers, all of its input text is converted to an internal seven-bit
code that is essentially standard ASCII, the ``American Standard Code for
Information Interchange.''  This conversion is done immediately when each
character is read in. Conversely, characters are converted from ASCII to
the user's external representation just before they are output to a
text file.

Such an internal code is relevant to users of \TeX\ primarily because it
governs the positions of characters in the fonts. For example, the
character `\.A' has ASCII code $65=\O{101}$, and when \TeX\ typesets
this letter it specifies character number 65 in the current font.
If that font actually has `\.A' in a different position, \TeX\ doesn't
know what the real position is; the program that does the actual printing from
\TeX's device-independent files is responsible for converting from ASCII to
a particular font encoding.

\TeX's internal code is relevant also with respect to constants
that begin with a reverse apostrophe.


\fi

\M22.
Characters of text that have been converted to \TeX's internal form
are said to be of type \\{ASCII\_code}, which is a subrange of the integers.

\Y\P$\4\X22:Types in the outer block\X\S$\6
$\\{ASCII\_code}=0\to127$;\C{seven-bit numbers}\par
\As31, 36, 42, 49, 64, 73, 105, 118, 130, 160, 291\ETs332.
\U10.\fi

\M23.
The original \PASCAL\ compiler was designed in the late 60s, when six-bit
character sets were common, so it did not make provision for lower-case
letters. Nowadays, of course, we need to deal with both capital and small
letters in a convenient way, especially in a program for typesetting;
so the present specification of \TeX\ has been written under the assumption
that the \PASCAL\ compiler and run-time system permit the use of text files
with more than 64 distinguishable characters. More precisely, we assume that
the character set contains at least the letters and symbols associated
with ASCII codes \O{40} through \O{176}; all of these characters are now
available on most computer terminals.

Since we are dealing with more characters than were present in the first
\PASCAL\ compilers, we have to decide what to call the associated data
type. Some \PASCAL s use the original name \\{char} for the
characters in text files, even though there now are more than 64 such
characters, while other \PASCAL s consider \\{char} to be a 64-element
subrange of a larger data type that has some other name.

In order to accommodate this difference, we shall use the name \\{text\_char}
to stand for the data type of the characters that are converted to and
from \\{ASCII\_code} when they are input and output. We shall also assume
that \\{text\_char} consists of the elements $\\{chr}(\\{first\_text\_char})$
through
$\\{chr}(\\{last\_text\_char})$, inclusive. The following definitions should be
adjusted if necessary.

\Y\P\D \37$\\{text\_char}\S\\{char}$\C{the data type of characters in text
files}\par
\P\D \37$\\{first\_text\_char}=0$\C{ordinal number of the smallest element of %
\\{text\_char}}\par
\P\D \37$\\{last\_text\_char}=127$\C{ordinal number of the largest element of %
\\{text\_char}}\par
\Y\P$\4\X23:Local variables for initialization\X\S$\6
\4\|i: \37$0\to\\{last\_text\_char}$;\C{this is the first one declared}\par
\A66.
\U13.\fi

\M24.
The \TeX\ processor converts between ASCII code and
the user's external character set by means of arrays \\{xord} and \\{xchr}
that are analogous to \PASCAL's \\{ord} and \\{chr} functions.

\Y\P$\4\X16:Globals in the outer block\X\mathrel{+}\S$\6
\4\\{xord}: \37\&{array} $[\\{text\_char}]$ \1\&{of}\5
\\{ASCII\_code};\C{specifies conversion of input characters}\2\6
\4\\{xchr}: \37\&{array} $[\\{ASCII\_code}]$ \1\&{of}\5
\\{text\_char};\C{specifies conversion of output characters}\2\par
\fi

\M25.
Since we are assuming that our \PASCAL\ system is able to read and write the
visible characters of standard ASCII (although not necessarily using the
ASCII codes to represent them), the following assignment statements initialize
most of the \\{xchr} array properly, without needing any system-dependent
changes. On the other hand, it is possible to implement \TeX\ with
less complete character sets, and in such cases it will be necessary to
change something here.

\Y\P$\4\X20:Set initial values of key variables\X\mathrel{+}\S$\6
$\\{xchr}[\O{40}]\K\.{\'\ \'}$;\5
$\\{xchr}[\O{41}]\K\.{\'!\'}$;\5
$\\{xchr}[\O{42}]\K\.{\'"\'}$;\5
$\\{xchr}[\O{43}]\K\.{\'\#\'}$;\5
$\\{xchr}[\O{44}]\K\.{\'\$\'}$;\5
$\\{xchr}[\O{45}]\K\.{\'\%\'}$;\5
$\\{xchr}[\O{46}]\K\.{\'\&\'}$;\5
$\\{xchr}[\O{47}]\K\.{\'\'}\.{\'\'}$;\6
$\\{xchr}[\O{50}]\K\.{\'(\'}$;\5
$\\{xchr}[\O{51}]\K\.{\')\'}$;\5
$\\{xchr}[\O{52}]\K\.{\'*\'}$;\5
$\\{xchr}[\O{53}]\K\.{\'+\'}$;\5
$\\{xchr}[\O{54}]\K\.{\',\'}$;\5
$\\{xchr}[\O{55}]\K\.{\'-\'}$;\5
$\\{xchr}[\O{56}]\K\.{\'.\'}$;\5
$\\{xchr}[\O{57}]\K\.{\'/\'}$;\6
$\\{xchr}[\O{60}]\K\.{\'0\'}$;\5
$\\{xchr}[\O{61}]\K\.{\'1\'}$;\5
$\\{xchr}[\O{62}]\K\.{\'2\'}$;\5
$\\{xchr}[\O{63}]\K\.{\'3\'}$;\5
$\\{xchr}[\O{64}]\K\.{\'4\'}$;\5
$\\{xchr}[\O{65}]\K\.{\'5\'}$;\5
$\\{xchr}[\O{66}]\K\.{\'6\'}$;\5
$\\{xchr}[\O{67}]\K\.{\'7\'}$;\6
$\\{xchr}[\O{70}]\K\.{\'8\'}$;\5
$\\{xchr}[\O{71}]\K\.{\'9\'}$;\5
$\\{xchr}[\O{72}]\K\.{\':\'}$;\5
$\\{xchr}[\O{73}]\K\.{\';\'}$;\5
$\\{xchr}[\O{74}]\K\.{\'<\'}$;\5
$\\{xchr}[\O{75}]\K\.{\'=\'}$;\5
$\\{xchr}[\O{76}]\K\.{\'>\'}$;\5
$\\{xchr}[\O{77}]\K\.{\'?\'}$;\6
$\\{xchr}[\O{100}]\K\.{\'@\'}$;\5
$\\{xchr}[\O{101}]\K\.{\'A\'}$;\5
$\\{xchr}[\O{102}]\K\.{\'B\'}$;\5
$\\{xchr}[\O{103}]\K\.{\'C\'}$;\5
$\\{xchr}[\O{104}]\K\.{\'D\'}$;\5
$\\{xchr}[\O{105}]\K\.{\'E\'}$;\5
$\\{xchr}[\O{106}]\K\.{\'F\'}$;\5
$\\{xchr}[\O{107}]\K\.{\'G\'}$;\6
$\\{xchr}[\O{110}]\K\.{\'H\'}$;\5
$\\{xchr}[\O{111}]\K\.{\'I\'}$;\5
$\\{xchr}[\O{112}]\K\.{\'J\'}$;\5
$\\{xchr}[\O{113}]\K\.{\'K\'}$;\5
$\\{xchr}[\O{114}]\K\.{\'L\'}$;\5
$\\{xchr}[\O{115}]\K\.{\'M\'}$;\5
$\\{xchr}[\O{116}]\K\.{\'N\'}$;\5
$\\{xchr}[\O{117}]\K\.{\'O\'}$;\6
$\\{xchr}[\O{120}]\K\.{\'P\'}$;\5
$\\{xchr}[\O{121}]\K\.{\'Q\'}$;\5
$\\{xchr}[\O{122}]\K\.{\'R\'}$;\5
$\\{xchr}[\O{123}]\K\.{\'S\'}$;\5
$\\{xchr}[\O{124}]\K\.{\'T\'}$;\5
$\\{xchr}[\O{125}]\K\.{\'U\'}$;\5
$\\{xchr}[\O{126}]\K\.{\'V\'}$;\5
$\\{xchr}[\O{127}]\K\.{\'W\'}$;\6
$\\{xchr}[\O{130}]\K\.{\'X\'}$;\5
$\\{xchr}[\O{131}]\K\.{\'Y\'}$;\5
$\\{xchr}[\O{132}]\K\.{\'Z\'}$;\5
$\\{xchr}[\O{133}]\K\.{\'[\'}$;\5
$\\{xchr}[\O{134}]\K\.{\'\\\'}$;\5
$\\{xchr}[\O{135}]\K\.{\']\'}$;\5
$\\{xchr}[\O{136}]\K\.{\'\^\'}$;\5
$\\{xchr}[\O{137}]\K\.{\'\_\'}$;\6
$\\{xchr}[\O{140}]\K\.{\'\`\'}$;\5
$\\{xchr}[\O{141}]\K\.{\'a\'}$;\5
$\\{xchr}[\O{142}]\K\.{\'b\'}$;\5
$\\{xchr}[\O{143}]\K\.{\'c\'}$;\5
$\\{xchr}[\O{144}]\K\.{\'d\'}$;\5
$\\{xchr}[\O{145}]\K\.{\'e\'}$;\5
$\\{xchr}[\O{146}]\K\.{\'f\'}$;\5
$\\{xchr}[\O{147}]\K\.{\'g\'}$;\6
$\\{xchr}[\O{150}]\K\.{\'h\'}$;\5
$\\{xchr}[\O{151}]\K\.{\'i\'}$;\5
$\\{xchr}[\O{152}]\K\.{\'j\'}$;\5
$\\{xchr}[\O{153}]\K\.{\'k\'}$;\5
$\\{xchr}[\O{154}]\K\.{\'l\'}$;\5
$\\{xchr}[\O{155}]\K\.{\'m\'}$;\5
$\\{xchr}[\O{156}]\K\.{\'n\'}$;\5
$\\{xchr}[\O{157}]\K\.{\'o\'}$;\6
$\\{xchr}[\O{160}]\K\.{\'p\'}$;\5
$\\{xchr}[\O{161}]\K\.{\'q\'}$;\5
$\\{xchr}[\O{162}]\K\.{\'r\'}$;\5
$\\{xchr}[\O{163}]\K\.{\'s\'}$;\5
$\\{xchr}[\O{164}]\K\.{\'t\'}$;\5
$\\{xchr}[\O{165}]\K\.{\'u\'}$;\5
$\\{xchr}[\O{166}]\K\.{\'v\'}$;\5
$\\{xchr}[\O{167}]\K\.{\'w\'}$;\6
$\\{xchr}[\O{170}]\K\.{\'x\'}$;\5
$\\{xchr}[\O{171}]\K\.{\'y\'}$;\5
$\\{xchr}[\O{172}]\K\.{\'z\'}$;\5
$\\{xchr}[\O{173}]\K\.{\'\{\'}$;\5
$\\{xchr}[\O{174}]\K\.{\'|\'}$;\5
$\\{xchr}[\O{175}]\K\.{\'\}\'}$;\5
$\\{xchr}[\O{176}]\K\.{\'\~\'}$;\6
$\\{xchr}[0]\K\.{\'\ \'}$;\5
$\\{xchr}[\O{177}]\K\.{\'\ \'}$;\C{ASCII codes 0 and \O{177} do not appear in
text}\par
\fi

\M26.
Some of the ASCII codes without visible characters have been given symbolic
names in this program because they are used with a special meaning.  The
\\{tab} character may be system dependent.

\Y\P\D \37$\\{null\_code}=\O{0}$\C{ASCII code that might disappear}\par
\P\D \37$\\{tab}=\O{11}$\C{ASCII code treated as \\{white\_space}}\par
\P\D \37$\\{space}=\O{40}$\C{ASCII code treated as \\{white\_space}}\par
\P\D \37$\\{invalid\_code}=\O{177}$\C{ASCII code that should not appear}\par
\fi

\M27.
The ASCII code is ``standard'' only to a certain extent, since many
computer installations have found it advantageous to have ready access
to more than 94 printing characters. Appendix~C of {\sl The \TeX book\/}
gives a complete specification of the intended correspondence between
characters and \TeX's internal representation.

If \TeX\ is being used
on a garden-variety \PASCAL\ for which only standard ASCII
codes will appear in the input and output files, it doesn't really matter
what codes are specified in $\\{xchr}[1\to\O{37}]$, but the safest policy is to
blank everything out by using the code shown below.

However, other settings of \\{xchr} will make \TeX\ more friendly on
computers that have an extended character set, so that users can type things
like `\.^^Z' instead of `\.{\\ne}'. At MIT, for example, it would be more
appropriate to substitute the code
$$\hbox{ \&{for} $\|i\K1\mathrel{\&{to}}\O{37}$ \&{do} $\\{xchr}[\|i]\K\\{chr}(%
\|i)$;}$$
\TeX's character set is essentially the same as MIT's, even with respect to
characters less than~\O{40}. People with extended character sets can
assign codes arbitrarily, giving an \\{xchr} equivalent to whatever
characters the users of \TeX\ are allowed to have in their input files.
It is best to make the codes correspond to the intended interpretations as
shown in Appendix~C whenever possible; but this is not necessary. For
example, in countries with an alphabet of more than 26 letters, it is
usually best to map the additional letters into codes less than~\O{40}.

\Y\P$\4\X20:Set initial values of key variables\X\mathrel{+}\S$\6
\&{for} $\|i\K1\mathrel{\&{to}}\O{37}$ \1\&{do}\5
$\\{xchr}[\|i]\K\.{\'\ \'}$;\2\6
$\\{xchr}[\\{tab}]\K\\{chr}(\\{tab})$;\par
\fi

\M28.
This system-independent code makes the \\{xord} array contain a suitable
inverse to the information in \\{xchr}. Note that if $\\{xchr}[\|i]=\\{xchr}[%
\|j]$
where $\|i<\|j<\O{177}$, the value of $\\{xord}[\\{xchr}[\|i]]$ will turn out
to be
\|j or more; hence, standard ASCII code numbers will be used instead
of codes below \O{40} in case there is a coincidence.

\Y\P$\4\X20:Set initial values of key variables\X\mathrel{+}\S$\6
\&{for} $\|i\K\\{first\_text\_char}\mathrel{\&{to}}\\{last\_text\_char}$ \1%
\&{do}\5
$\\{xord}[\\{chr}(\|i)]\K\\{invalid\_code}$;\2\6
\&{for} $\|i\K1\mathrel{\&{to}}\O{176}$ \1\&{do}\5
$\\{xord}[\\{xchr}[\|i]]\K\|i$;\2\par
\fi

\M29.
Also, various characters are given symbolic names; all the ones this
program uses are collected here.  We use the sharp sign as the
\\{concat\_char}, rather than something more natural (like an ampersand),
for uniformity of database syntax (ampersand is a valid character in
identifiers).

\Y\P\D \37$\\{double\_quote}=\.{""}\.{""}$\C{delimits strings}\par
\P\D \37$\\{number\_sign}=\.{"\#"}$\C{marks an \\{int\_literal}}\par
\P\D \37$\\{comment}=\.{"\%"}$\C{ignore the rest of a \.{.bst} or \TeX\ line}%
\par
\P\D \37$\\{single\_quote}=\.{"\'"}$\C{marks a quoted function}\par
\P\D \37$\\{left\_paren}=\.{"("}$\C{optional database entry left delimiter}\par
\P\D \37$\\{right\_paren}=\.{")"}$\C{corresponding right delimiter}\par
\P\D \37$\\{comma}=\.{","}$\C{separates various things}\par
\P\D \37$\\{minus\_sign}=\.{"-"}$\C{for a negative number}\par
\P\D \37$\\{equals\_sign}=\.{"="}$\C{separates a field name from a field value}%
\par
\P\D \37$\\{at\_sign}=\.{"@"}$\C{the beginning of a database entry}\par
\P\D \37$\\{left\_brace}=\.{"\{"}$\C{left delimiter of many things}\par
\P\D \37$\\{right\_brace}=\.{"\}"}$\C{corresponding right delimiter}\par
\P\D \37$\\{period}=\.{"."}$\C{these are three}\par
\P\D \37$\\{question\_mark}=\.{"?"}$\C{string-ending characters}\par
\P\D \37$\\{exclamation\_mark}=\.{"!"}$\C{of interest in \.{add.period\$}}\par
\P\D \37$\\{tie}=\.{"\~"}$\C{the default space char, in \.{format.name\$}}\par
\P\D \37$\\{hyphen}=\.{"-"}$\C{like \\{white\_space}, in \.{format.name\$}}\par
\P\D \37$\\{star}=\.{"*"}$\C{for including entire database}\par
\P\D \37$\\{concat\_char}=\.{"\#"}$\C{for concatenating field tokens}\par
\P\D \37$\\{colon}=\.{":"}$\C{for lower-casing (usually title) strings}\par
\P\D \37$\\{backslash}=\.{"\\"}$\C{used to recognize accented characters}\par
\fi

\M30.
These arrays give a lexical classification for the \\{ASCII\_code}s;
\\{lex\_class} is used for general scanning and \\{id\_class} is used for
scanning identifiers.

\Y\P$\4\X16:Globals in the outer block\X\mathrel{+}\S$\6
\4\\{lex\_class}: \37\&{array} $[\\{ASCII\_code}]$ \1\&{of}\5
\\{lex\_type};\2\6
\4\\{id\_class}: \37\&{array} $[\\{ASCII\_code}]$ \1\&{of}\5
\\{id\_type};\2\par
\fi

\M31.
Every character has two types of the lexical classifications.  The
first type is general, and the second type tells whether the character
is legal in identifiers.

\Y\P\D \37$\\{illegal}=0$\C{the unrecognized \\{ASCII\_code}s}\par
\P\D \37$\\{white\_space}=1$\C{things like \\{space}s that you can't see}\par
\P\D \37$\\{alpha}=2$\C{the upper- and lower-case letters}\par
\P\D \37$\\{numeric}=3$\C{the ten digits}\par
\P\D \37$\\{sep\_char}=4$\C{things sometimes treated like \\{white\_space}}\par
\P\D \37$\\{other\_lex}=5$\C{when none of the above applies}\par
\P\D \37$\\{last\_lex}=5$\C{the same number as on the line above}\Y\par
\P\D \37$\\{illegal\_id\_char}=0$\C{a few forbidden ones}\par
\P\D \37$\\{legal\_id\_char}=1$\C{most printing characters}\par
\Y\P$\4\X22:Types in the outer block\X\mathrel{+}\S$\6
$\\{lex\_type}=0\to\\{last\_lex}$;\6
$\\{id\_type}=0\to1$;\par
\fi

\M32.
Now we initialize the system-dependent \\{lex\_class} array.  The \\{tab}
character may be system dependent.  Note that the order of these
assignments is important here.

\Y\P$\4\X20:Set initial values of key variables\X\mathrel{+}\S$\6
\&{for} $\|i\K0\mathrel{\&{to}}\O{177}$ \1\&{do}\5
$\\{lex\_class}[\|i]\K\\{other\_lex}$;\2\6
\&{for} $\|i\K0\mathrel{\&{to}}\O{37}$ \1\&{do}\5
$\\{lex\_class}[\|i]\K\\{illegal}$;\2\6
$\\{lex\_class}[\\{invalid\_code}]\K\\{illegal}$;\5
$\\{lex\_class}[\\{tab}]\K\\{white\_space}$;\5
$\\{lex\_class}[\\{space}]\K\\{white\_space}$;\5
$\\{lex\_class}[\\{tie}]\K\\{sep\_char}$;\5
$\\{lex\_class}[\\{hyphen}]\K\\{sep\_char}$;\6
\&{for} $\|i\K\O{60}\mathrel{\&{to}}\O{71}$ \1\&{do}\5
$\\{lex\_class}[\|i]\K\\{numeric}$;\2\6
\&{for} $\|i\K\O{101}\mathrel{\&{to}}\O{132}$ \1\&{do}\5
$\\{lex\_class}[\|i]\K\\{alpha}$;\2\6
\&{for} $\|i\K\O{141}\mathrel{\&{to}}\O{172}$ \1\&{do}\5
$\\{lex\_class}[\|i]\K\\{alpha}$;\2\par
\fi

\M33.
And now the \\{id\_class} array.

\Y\P$\4\X20:Set initial values of key variables\X\mathrel{+}\S$\6
\&{for} $\|i\K0\mathrel{\&{to}}\O{177}$ \1\&{do}\5
$\\{id\_class}[\|i]\K\\{legal\_id\_char}$;\2\6
\&{for} $\|i\K0\mathrel{\&{to}}\O{37}$ \1\&{do}\5
$\\{id\_class}[\|i]\K\\{illegal\_id\_char}$;\2\6
$\\{id\_class}[\\{space}]\K\\{illegal\_id\_char}$;\5
$\\{id\_class}[\\{tab}]\K\\{illegal\_id\_char}$;\5
$\\{id\_class}[\\{double\_quote}]\K\\{illegal\_id\_char}$;\5
$\\{id\_class}[\\{number\_sign}]\K\\{illegal\_id\_char}$;\5
$\\{id\_class}[\\{comment}]\K\\{illegal\_id\_char}$;\5
$\\{id\_class}[\\{single\_quote}]\K\\{illegal\_id\_char}$;\5
$\\{id\_class}[\\{left\_paren}]\K\\{illegal\_id\_char}$;\5
$\\{id\_class}[\\{right\_paren}]\K\\{illegal\_id\_char}$;\5
$\\{id\_class}[\\{comma}]\K\\{illegal\_id\_char}$;\5
$\\{id\_class}[\\{equals\_sign}]\K\\{illegal\_id\_char}$;\5
$\\{id\_class}[\\{left\_brace}]\K\\{illegal\_id\_char}$;\5
$\\{id\_class}[\\{right\_brace}]\K\\{illegal\_id\_char}$;\par
\fi

\M34.
The array \\{char\_width} gives relative printing widths of each
\\{ASCII\_code}, and \\{string\_width} will be used later to sum up
\\{char\_width}s in a string.

\Y\P$\4\X16:Globals in the outer block\X\mathrel{+}\S$\6
\4\\{char\_width}: \37\&{array} $[\\{ASCII\_code}]$ \1\&{of}\5
\\{integer};\2\6
\4\\{string\_width}: \37\\{integer};\par
\fi

\M35.
Now we initialize the system-dependent \\{char\_width} array, for which
\\{space} is the only \\{white\_space} character given a nonzero printing
width.  The widths here are taken from Stanford's June~'87
$cmr10$~font and represent hundredths of a point (rounded), but since
they're used only for relative comparisons, the units have no meaning.

\Y\P\D \37$\\{ss\_width}=500$\C{character \O{31}'s width in the $cmr10$ font}%
\par
\P\D \37$\\{ae\_width}=722$\C{character \O{32}'s width in the $cmr10$ font}\par
\P\D \37$\\{oe\_width}=778$\C{character \O{33}'s width in the $cmr10$ font}\par
\P\D \37$\\{upper\_ae\_width}=903$\C{character \O{35}'s width in the $cmr10$
font}\par
\P\D \37$\\{upper\_oe\_width}=1014$\C{character \O{36}'s width in the $cmr10$
font}\par
\Y\P$\4\X20:Set initial values of key variables\X\mathrel{+}\S$\6
\&{for} $\|i\K0\mathrel{\&{to}}\O{177}$ \1\&{do}\5
$\\{char\_width}[\|i]\K0$;\2\6
$\\{char\_width}[\O{40}]\K278$;\5
$\\{char\_width}[\O{41}]\K278$;\5
$\\{char\_width}[\O{42}]\K500$;\5
$\\{char\_width}[\O{43}]\K833$;\5
$\\{char\_width}[\O{44}]\K500$;\5
$\\{char\_width}[\O{45}]\K833$;\5
$\\{char\_width}[\O{46}]\K778$;\5
$\\{char\_width}[\O{47}]\K278$;\5
$\\{char\_width}[\O{50}]\K389$;\5
$\\{char\_width}[\O{51}]\K389$;\5
$\\{char\_width}[\O{52}]\K500$;\5
$\\{char\_width}[\O{53}]\K778$;\5
$\\{char\_width}[\O{54}]\K278$;\5
$\\{char\_width}[\O{55}]\K333$;\5
$\\{char\_width}[\O{56}]\K278$;\5
$\\{char\_width}[\O{57}]\K500$;\5
$\\{char\_width}[\O{60}]\K500$;\5
$\\{char\_width}[\O{61}]\K500$;\5
$\\{char\_width}[\O{62}]\K500$;\5
$\\{char\_width}[\O{63}]\K500$;\5
$\\{char\_width}[\O{64}]\K500$;\5
$\\{char\_width}[\O{65}]\K500$;\5
$\\{char\_width}[\O{66}]\K500$;\5
$\\{char\_width}[\O{67}]\K500$;\5
$\\{char\_width}[\O{70}]\K500$;\5
$\\{char\_width}[\O{71}]\K500$;\5
$\\{char\_width}[\O{72}]\K278$;\5
$\\{char\_width}[\O{73}]\K278$;\5
$\\{char\_width}[\O{74}]\K278$;\5
$\\{char\_width}[\O{75}]\K778$;\5
$\\{char\_width}[\O{76}]\K472$;\5
$\\{char\_width}[\O{77}]\K472$;\5
$\\{char\_width}[\O{100}]\K778$;\5
$\\{char\_width}[\O{101}]\K750$;\5
$\\{char\_width}[\O{102}]\K708$;\5
$\\{char\_width}[\O{103}]\K722$;\5
$\\{char\_width}[\O{104}]\K764$;\5
$\\{char\_width}[\O{105}]\K681$;\5
$\\{char\_width}[\O{106}]\K653$;\5
$\\{char\_width}[\O{107}]\K785$;\5
$\\{char\_width}[\O{110}]\K750$;\5
$\\{char\_width}[\O{111}]\K361$;\5
$\\{char\_width}[\O{112}]\K514$;\5
$\\{char\_width}[\O{113}]\K778$;\5
$\\{char\_width}[\O{114}]\K625$;\5
$\\{char\_width}[\O{115}]\K917$;\5
$\\{char\_width}[\O{116}]\K750$;\5
$\\{char\_width}[\O{117}]\K778$;\5
$\\{char\_width}[\O{120}]\K681$;\5
$\\{char\_width}[\O{121}]\K778$;\5
$\\{char\_width}[\O{122}]\K736$;\5
$\\{char\_width}[\O{123}]\K556$;\5
$\\{char\_width}[\O{124}]\K722$;\5
$\\{char\_width}[\O{125}]\K750$;\5
$\\{char\_width}[\O{126}]\K750$;\5
$\\{char\_width}[\O{127}]\K1028$;\5
$\\{char\_width}[\O{130}]\K750$;\5
$\\{char\_width}[\O{131}]\K750$;\5
$\\{char\_width}[\O{132}]\K611$;\5
$\\{char\_width}[\O{133}]\K278$;\5
$\\{char\_width}[\O{134}]\K500$;\5
$\\{char\_width}[\O{135}]\K278$;\5
$\\{char\_width}[\O{136}]\K500$;\5
$\\{char\_width}[\O{137}]\K278$;\5
$\\{char\_width}[\O{140}]\K278$;\5
$\\{char\_width}[\O{141}]\K500$;\5
$\\{char\_width}[\O{142}]\K556$;\5
$\\{char\_width}[\O{143}]\K444$;\5
$\\{char\_width}[\O{144}]\K556$;\5
$\\{char\_width}[\O{145}]\K444$;\5
$\\{char\_width}[\O{146}]\K306$;\5
$\\{char\_width}[\O{147}]\K500$;\5
$\\{char\_width}[\O{150}]\K556$;\5
$\\{char\_width}[\O{151}]\K278$;\5
$\\{char\_width}[\O{152}]\K306$;\5
$\\{char\_width}[\O{153}]\K528$;\5
$\\{char\_width}[\O{154}]\K278$;\5
$\\{char\_width}[\O{155}]\K833$;\5
$\\{char\_width}[\O{156}]\K556$;\5
$\\{char\_width}[\O{157}]\K500$;\5
$\\{char\_width}[\O{160}]\K556$;\5
$\\{char\_width}[\O{161}]\K528$;\5
$\\{char\_width}[\O{162}]\K392$;\5
$\\{char\_width}[\O{163}]\K394$;\5
$\\{char\_width}[\O{164}]\K389$;\5
$\\{char\_width}[\O{165}]\K556$;\5
$\\{char\_width}[\O{166}]\K528$;\5
$\\{char\_width}[\O{167}]\K722$;\5
$\\{char\_width}[\O{170}]\K528$;\5
$\\{char\_width}[\O{171}]\K528$;\5
$\\{char\_width}[\O{172}]\K444$;\5
$\\{char\_width}[\O{173}]\K500$;\5
$\\{char\_width}[\O{174}]\K1000$;\5
$\\{char\_width}[\O{175}]\K500$;\5
$\\{char\_width}[\O{176}]\K500$;\par
\fi

\N36.  Input and output.
The basic operations we need to do are
(1)~inputting and outputting of text characters to or from a file;
(2)~instructing the operating system to initiate (``open'')
or to terminate (``close'') input or output to or from a specified file; and
(3)~testing whether the end of an input file has been reached.

\Y\P$\4\X22:Types in the outer block\X\mathrel{+}\S$\6
$\\{alpha\_file}=$\1\5
\&{packed} \37\&{file} \1\&{of}\5
\\{text\_char};\C{files that contain textual data}\2\2\par
\fi

\M37.
Most of what we need to do with respect to input and output can be handled
by the I/O facilities that are standard in \PASCAL, i.e., the routines
called \\{get}, \\{put}, \\{eof}, and so on. But
standard \PASCAL\ does not allow file variables to be associated with file
names that are determined at run time, so it cannot be used to implement
\BibTeX; some sort of extension to \PASCAL's ordinary \\{reset} and \\{rewrite}
is crucial for our purposes. We shall assume that \\{name\_of\_file} is a
variable
of an appropriate type such that the \PASCAL\ run-time system being used to
implement \BibTeX\ can open a file whose external name is specified by
\\{name\_of\_file}. \BibTeX\ does no case conversion for file names.

\Y\P$\4\X16:Globals in the outer block\X\mathrel{+}\S$\6
\4\\{name\_of\_file}: \37\&{packed} \37\&{array} $[1\to\\{file\_name\_size}]$ %
\1\&{of}\5
\\{char};\C{on some systems this is a \&{record} variable}\2\6
\4\\{name\_length}: \37$0\to\\{file\_name\_size}$;\C{this many characters are
relevant in \\{name\_of\_file} (the rest are blank)}\6
\4\\{name\_ptr}: \37$0\to\\{file\_name\_size}+1$;\C{index variable into \\{name%
\_of\_file}}\par
\fi

\M38.
The \ph\ compiler with which the present version of \TeX\ was prepared has
extended the rules of \PASCAL\ in a very convenient way. To open file~\|f,
we can write
$$\vbox{\halign{#\hfil\qquad&#\hfil\cr
$\\{reset}(\|f,\hbox{\\{name}},\.{\'/O\'})$&for input;\cr
$\\{rewrite}(\|f,\hbox{\\{name}},\.{\'/O\'})$&for output.\cr}}$$
The `\\{name}' parameter, which is of type `\ignorespaces\&{packed} \&{array}
$[\hbox{\<\\{any}>}]$ \&{of} \\{text\_char}', stands for the name of
the external file that is being opened for input or output.
Blank spaces that might appear in \\{name} are ignored.

The `\.{/O}' parameter tells the operating system not to issue its own
error messages if something goes wrong. If a file of the specified name
cannot be found, or if such a file cannot be opened for some other reason
(e.g., someone may already be trying to write the same file), we will have
$\\{erstat}(\|f)\I0$ after an unsuccessful \\{reset} or \\{rewrite}.  This
allows
\TeX\ to undertake appropriate corrective action.

\TeX's file-opening procedures return \\{false} if no file identified by
\\{name\_of\_file} could be opened.

\Y\P\D \37$\\{reset\_OK}(\#)\S\\{erstat}(\#)=0$\par
\P\D \37$\\{rewrite\_OK}(\#)\S\\{erstat}(\#)=0$\par
\Y\P$\4\X38:Procedures and functions for file-system interacting\X\S$\6
\4\&{function} \1\  \\{erstat} ( $\mathop{\&{var}}\|f:$ \&{file} ) : %
\\{integer};\5
\\{extern};\C{in the runtime library}\7
\hbox{\2} \6
\4\&{function}\1\  \37$\\{a\_open\_in}(\mathop{\&{var}}\|f:\\{alpha\_file})$: %
\37\\{boolean};\C{open a text file for input}\2\6
\&{begin} \37$\\{reset}(\|f,\39\\{name\_of\_file},\39\.{\'/O\'})$;\5
$\\{a\_open\_in}\K\\{reset\_OK}(\|f)$;\6
\&{end};\7
\4\&{function}\1\  \37$\\{a\_open\_out}(\mathop{\&{var}}\|f:\\{alpha\_file})$: %
\37\\{boolean};\C{open a text file for output}\2\6
\&{begin} \37$\\{rewrite}(\|f,\39\\{name\_of\_file},\39\.{\'/O\'})$;\5
$\\{a\_open\_out}\K\\{rewrite\_OK}(\|f)$;\6
\&{end};\par
\As39, 58, 60\ETs61.
\U12.\fi

\M39.
Files can be closed with the \ph\ routine `$\\{close}(\|f)$', which should
be used when all input or output with respect to \|f has been
completed.  This makes \|f available to be opened again, if desired;
and if \|f was used for output, the \\{close} operation makes the
corresponding external file appear on the user's area, ready to be
read.

\Y\P$\4\X38:Procedures and functions for file-system interacting\X\mathrel{+}%
\S$\6
\4\&{procedure}\1\  \37$\\{a\_close}(\mathop{\&{var}}\|f:\\{alpha\_file})$;%
\C{close a text file}\2\6
\&{begin} \37$\\{close}(\|f)$;\6
\&{end};\par
\fi

\M40.
Text output is easy to do with the ordinary \PASCAL\ \\{put} procedure,
so we don't have to make any other special arrangements.
The treatment of text input is more difficult, however, because
of the necessary translation to \\{ASCII\_code} values, and because
\TeX's conventions should be efficient and they should
blend nicely with the user's operating environment.


\fi

\M41.
Input from text files is read one line at a time, using a routine
called \\{input\_ln}. This function is defined in terms of global
variables called \\{buffer} and \\{last}.  The \\{buffer} array contains
\\{ASCII\_code} values, and \\{last} is an index into this array marking
the end of a line of text.  (Occasionally, \\{buffer} is used for
something else, in which case it is copied to a temporary array.)

\Y\P$\4\X16:Globals in the outer block\X\mathrel{+}\S$\6
\4\\{buffer}: \37\\{buf\_type};\C{usually, lines of characters being read}\6
\4\\{last}: \37\\{buf\_pointer};\C{end of the line just input to \\{buffer}}\par
\fi

\M42.
The type \\{buf\_type} is used for \\{buffer}, for saved copies of it, or
for scratch work.  It's not \&{packed}  because otherwise the program
would run much slower on some systems (more than 25 percent slower,
for example, on a TOPS-20 operating system).  But on systems that are
byte-addressable and that have a good compiler, packing \\{buf\_type}
would save lots of space without much loss of speed.  Other modules
that have packable arrays are also marked with a ``space savings''
index entry.

\Y\P$\4\X22:Types in the outer block\X\mathrel{+}\S$\6
$\\{buf\_pointer}=0\to\\{buf\_size}$;\C{an index into a \\{buf\_type}}\6
$\\{buf\_type}=$\1\5
\&{array} $[\\{buf\_pointer}]$ \1\&{of}\5
\\{ASCII\_code};\C{for various buffers}\2\2\par
\fi

\M43.
And while we're at it, we declare another buffer for general use.
Because buffers are not packed and can get large, we use \\{sv\_buffer}
several purposes; this is a bit kludgy, but it helps make the stack
space not overflow on some machines.  It's used when reading the
entire database file (in the \.{read} command) and when doing
name-handling (through the alias \\{name\_buf}) in the \\{built\_in}
functions \.{format.names\$} and \.{num.names\$}.

\Y\P$\4\X16:Globals in the outer block\X\mathrel{+}\S$\6
\4\\{sv\_buffer}: \37\\{buf\_type};\6
\4\\{sv\_ptr1}: \37\\{buf\_pointer};\6
\4\\{sv\_ptr2}: \37\\{buf\_pointer};\6
\4$\\{tmp\_ptr},\39\\{tmp\_end\_ptr}$: \37\\{integer};\C{copy pointers only,
usually for buffers}\par
\fi

\M44.
When something in the program wants to be bigger or something out
there wants to be smaller, it's time to call it a run.  Here's the
first of several macros that have associated procedures so that they
produce less inline code.

\Y\P\D \37$\\{overflow}(\#)\S$\1\6
\&{begin} \37\C{fatal error---close up shop}\6
\\{print\_overflow};\5
$\\{print\_ln}(\#:0)$;\5
\&{goto} \37\\{close\_up\_shop};\6
\&{end}\2\par
\Y\P$\4\X3:Procedures and functions for all file I/O, error messages, and such%
\X\mathrel{+}\S$\6
\4\&{procedure}\1\  \37\\{print\_overflow};\2\6
\&{begin} \37$\\{print}(\.{\'Sorry---you\'}\.{\'ve\ exceeded\ BibTeX\'}\.{\'s\ %
\'})$;\5
\\{mark\_fatal};\6
\&{end};\par
\fi

\M45.
When something happens that the program thinks is impossible,
call the maintainer.

\Y\P\D \37$\\{confusion}(\#)\S$\1\6
\&{begin} \37\C{fatal error---close up shop}\6
$\\{print}(\#)$;\5
\\{print\_confusion};\5
\&{goto} \37\\{close\_up\_shop};\6
\&{end}\2\par
\Y\P$\4\X3:Procedures and functions for all file I/O, error messages, and such%
\X\mathrel{+}\S$\6
\4\&{procedure}\1\  \37\\{print\_confusion};\2\6
\&{begin} \37$\\{print\_ln}(\.{\'---this\ can\'}\.{\'t\ happen\'})$;\5
$\\{print\_ln}(\.{\'*Please\ notify\ the\ BibTeX\ maintainer*\'})$;\5
\\{mark\_fatal};\6
\&{end};\par
\fi

\M46.
When a buffer overflows, it's time to complain (and then quit).

\Y\P$\4\X3:Procedures and functions for all file I/O, error messages, and such%
\X\mathrel{+}\S$\6
\4\&{procedure}\1\  \37\\{buffer\_overflow};\2\6
\&{begin} \37$\\{overflow}(\.{\'buffer\ size\ \'},\39\\{buf\_size})$;\6
\&{end};\par
\fi

\M47.
The \\{input\_ln} function brings the next line of input from the
specified file into available positions of the buffer array and
returns the value \\{true}, unless the file has already been entirely
read, in which case it returns \\{false} and sets $\\{last}\K0$.  In
general, the \\{ASCII\_code} numbers that represent the next line of the
file are input into $\\{buffer}[0]$, $\\{buffer}[1]$, \dots, $\\{buffer}[%
\\{last}-1]$;
and the global variable \\{last} is set equal to the length of the line.
Trailing \\{white\_space} characters are removed from the line
(\\{white\_space} characters are explained in the character-set section%
---most likely they're blanks); thus, either $\\{last}=0$ (in which case
the line was entirely blank) or $\\{lex\_class}[\\{buffer}[\\{last}-1]]\I%
\\{white\_space}$.
An overflow error is given if the normal actions of \\{input\_ln} would
make $\\{last}>\\{buf\_size}$.

Standard \PASCAL\ says that a file should have \\{eoln} immediately
before \\{eof}, but \BibTeX\ needs only a weaker restriction: If \\{eof}
occurs in the middle of a line, the system function \\{eoln} should return
a \\{true} result (even though $\|f\^$ will be undefined).

\Y\P$\4\X3:Procedures and functions for all file I/O, error messages, and such%
\X\mathrel{+}\S$\6
\4\&{function}\1\  \37$\\{input\_ln}(\mathop{\&{var}}\|f:\\{alpha\_file})$: \37%
\\{boolean};\C{inputs the next line or returns \\{false}}\6
\4\&{label} \37\\{loop\_exit};\2\6
\&{begin} \37$\\{last}\K0$;\6
\&{if} $(\\{eof}(\|f))$ \1\&{then}\5
$\\{input\_ln}\K\\{false}$\6
\4\&{else} \&{begin} \37\&{while} $(\R\\{eoln}(\|f))$ \1\&{do}\6
\&{begin} \37\&{if} $(\\{last}\G\\{buf\_size})$ \1\&{then}\5
\\{buffer\_overflow};\2\6
$\\{buffer}[\\{last}]\K\\{xord}[\|f\^]$;\5
$\\{get}(\|f)$;\5
$\\{incr}(\\{last})$;\6
\&{end};\2\6
$\\{get}(\|f)$;\6
\&{while} $(\\{last}>0)$ \1\&{do}\C{remove trailing \\{white\_space}}\6
\&{if} $(\\{lex\_class}[\\{buffer}[\\{last}-1]]=\\{white\_space})$ \1\&{then}\5
$\\{decr}(\\{last})$\6
\4\&{else} \&{goto} \37\\{loop\_exit};\2\2\6
\4\\{loop\_exit}: \37$\\{input\_ln}\K\\{true}$;\6
\&{end};\2\6
\&{end};\par
\fi

\N48.  String handling.
\BibTeX\ uses variable-length strings of seven-bit characters.
Since \PASCAL\ does not have a well-developed string mechanism,
\BibTeX\ does all its string processing by home-grown
(predominantly \TeX's) methods.
Unlike \TeX, however, \BibTeX\ does not use a \\{pool\_file} for
string storage; it creates its few pre-defined strings at run-time.

The necessary operations are handled with a simple data structure.
The array \\{str\_pool} contains all the (seven-bit) ASCII codes in all
the strings \BibTeX\ must ever search for (generally identifiers
names), and the array \\{str\_start} contains indices of the starting
points of each such string. Strings are referred to by integer
numbers, so that string number \|s comprises the characters
$\\{str\_pool}[\|j]$ for $\\{str\_start}[\|s]\L\|j<\\{str\_start}[\|s+1]$.
Additional integer
variables \\{pool\_ptr} and \\{str\_ptr} indicate the number of entries used
so far in \\{str\_pool} and \\{str\_start}; locations $\\{str\_pool}[\\{pool%
\_ptr}]$
and $\\{str\_start}[\\{str\_ptr}]$ are ready for the next string to be
allocated.  Location $\\{str\_start}[0]$ is unused so that hashing will
work correctly.

Elements of the \\{str\_pool} array must be ASCII codes that can actually be
printed; i.e., they must have an \\{xchr} equivalent in the local
character set.

\Y\P$\4\X16:Globals in the outer block\X\mathrel{+}\S$\6
\4\\{str\_pool}: \37\&{packed} \37\&{array} $[\\{pool\_pointer}]$ \1\&{of}\5
\\{ASCII\_code};\C{the characters}\2\6
\4\\{str\_start}: \37\&{packed} \37\&{array} $[\\{str\_number}]$ \1\&{of}\5
\\{pool\_pointer};\C{the starting pointers}\2\6
\4\\{pool\_ptr}: \37\\{pool\_pointer};\C{first unused position in \\{str%
\_pool}}\6
\4\\{str\_ptr}: \37\\{str\_number};\C{start of the current string being
created}\6
\4\\{str\_num}: \37\\{str\_number};\C{general index variable into \\{str%
\_start}}\6
\4$\\{p\_ptr1},\39\\{p\_ptr2}$: \37\\{pool\_pointer};\C{several procedures use
these locally}\par
\fi

\M49.
Where \\{pool\_pointer} and \\{str\_number} are pointers into \\{str\_pool} and
\\{str\_start}.

\Y\P$\4\X22:Types in the outer block\X\mathrel{+}\S$\6
$\\{pool\_pointer}=0\to\\{pool\_size}$;\C{for variables that point into \\{str%
\_pool}}\6
$\\{str\_number}=0\to\\{max\_strings}$;\C{for variables that point into \\{str%
\_start}}\par
\fi

\M50.
These macros send a string in \\{str\_pool} to an output file.

\Y\P\D \37$\\{max\_pop}=3$\C{---see the \\{built\_in} functions section}\Y\par
\P\D \37$\\{print\_pool\_str}(\#)\S\\{print\_a\_pool\_str}(\#)$\C{making this a
procedure saves a little space}\Y\par
\P\D \37$\\{trace\_pr\_pool\_str}(\#)\S$\1\6
\&{begin} \37$\\{out\_pool\_str}(\\{log\_file},\39\#)$;\6
\&{end}\2\par
\fi

\M51.
And here are the associated procedures.  Note: The \\{term\_out} file is
system dependent.

\Y\P$\4\X3:Procedures and functions for all file I/O, error messages, and such%
\X\mathrel{+}\S$\6
\4\&{procedure}\1\  \37$\\{out\_pool\_str}(\mathop{\&{var}}\|f:\\{alpha\_file};%
\,\35\|s:\\{str\_number})$;\6
\4\&{var} \37\|i: \37\\{pool\_pointer};\2\6
\&{begin} \37\C{allowing $\\{str\_ptr}\L\|s<\\{str\_ptr}+\\{max\_pop}$ is a %
\.{.bst}-stack kludge}\6
\&{if} $((\|s<0)\V(\|s\G\\{str\_ptr}+\\{max\_pop})\V(\|s\G\\{max\_strings}))$ %
\1\&{then}\5
$\\{confusion}(\.{\'Illegal\ string\ number:\'},\39\|s:0)$;\2\6
\&{for} $\|i\K\\{str\_start}[\|s]\mathrel{\&{to}}\\{str\_start}[\|s+1]-1$ \1%
\&{do}\5
$\\{write}(\|f,\39\\{xchr}[\\{str\_pool}[\|i]])$;\2\6
\&{end};\7
\4\&{procedure}\1\  \37$\\{print\_a\_pool\_str}(\|s:\\{str\_number})$;\2\6
\&{begin} \37$\\{out\_pool\_str}(\\{term\_out},\39\|s)$;\5
$\\{out\_pool\_str}(\\{log\_file},\39\|s)$;\6
\&{end};\par
\fi

\M52.
Several of the elementary string operations are performed using \.{WEB}
macros instead of using \PASCAL\ procedures, because many of the
operations are done quite frequently and we want to avoid the
overhead of procedure calls. For example, here is
a simple macro that computes the length of a string.

\Y\P\D \37$\\{length}(\#)\S(\\{str\_start}[\#+1]-\\{str\_start}[\#])$\C{the
number of characters in string number \#}\par
\fi

\M53.
Strings are created by appending character codes to \\{str\_pool}.
The macro called \\{append\_char}, defined here, does not check to see if the
value of \\{pool\_ptr} has gotten too high; this test is supposed to be
made before \\{append\_char} is used.

To test if there is room to append \|l more characters to \\{str\_pool},
we shall write $\\{str\_room}(\|l)$, which aborts \BibTeX\ and gives an
error message if there isn't enough room.

\Y\P\D \37$\\{append\_char}(\#)\S$\C{put \\{ASCII\_code} \# at the end of %
\\{str\_pool}}\6
\&{begin} \37$\\{str\_pool}[\\{pool\_ptr}]\K\#$;\5
$\\{incr}(\\{pool\_ptr})$;\6
\&{end}\Y\par
\P\D \37$\\{str\_room}(\#)\S$\C{make sure that the pool hasn't overflowed}\6
\&{begin} \37\&{if} $(\\{pool\_ptr}+\#>\\{pool\_size})$ \1\&{then}\5
\\{pool\_overflow};\2\6
\&{end}\par
\Y\P$\4\X3:Procedures and functions for all file I/O, error messages, and such%
\X\mathrel{+}\S$\6
\4\&{procedure}\1\  \37\\{pool\_overflow};\2\6
\&{begin} \37$\\{overflow}(\.{\'pool\ size\ \'},\39\\{pool\_size})$;\6
\&{end};\par
\fi

\M54.
Once a sequence of characters has been appended to \\{str\_pool}, it
officially becomes a string when the function \\{make\_string} is called.
It returns the string number of the string it just made.

\Y\P$\4\X54:Procedures and functions for handling numbers, characters, and
strings\X\S$\6
\4\&{function}\1\  \37\\{make\_string}: \37\\{str\_number};\C{current string
enters the pool}\2\6
\&{begin} \37\&{if} $(\\{str\_ptr}=\\{max\_strings})$ \1\&{then}\5
$\\{overflow}(\.{\'number\ of\ strings\ \'},\39\\{max\_strings})$;\2\6
$\\{incr}(\\{str\_ptr})$;\5
$\\{str\_start}[\\{str\_ptr}]\K\\{pool\_ptr}$;\5
$\\{make\_string}\K\\{str\_ptr}-1$;\6
\&{end};\par
\As56, 57, 62, 63, 68, 77, 198, 265, 278, 300, 301, 303, 335\ETs336.
\U12.\fi

\M55.
These macros destroy and recreate the string at the end of the pool.

\Y\P\D \37$\\{flush\_string}\S$\1\6
\&{begin} \37$\\{decr}(\\{str\_ptr})$;\5
$\\{pool\_ptr}\K\\{str\_start}[\\{str\_ptr}]$;\6
\&{end}\2\par
\P\D \37$\\{unflush\_string}\S$\1\6
\&{begin} \37$\\{incr}(\\{str\_ptr})$;\5
$\\{pool\_ptr}\K\\{str\_start}[\\{str\_ptr}]$;\6
\&{end}\2\par
\fi

\M56.
This subroutine compares string \|s with another string that appears
in the buffer \\{buf} between positions \\{bf\_ptr} and $\\{bf\_ptr}+%
\\{len}-1$; the
result is \\{true} if and only if the strings are equal.

\Y\P$\4\X54:Procedures and functions for handling numbers, characters, and
strings\X\mathrel{+}\S$\6
\4\&{function}\1\  \37$\\{str\_eq\_buf}(\|s:\\{str\_number};\,\35\mathop{%
\&{var}}\\{buf}:\\{buf\_type};\,\35\\{bf\_ptr},\39\\{len}:\\{buf\_pointer})$: %
\37\\{boolean};\C{test equality of strings}\6
\4\&{label} \37\\{exit};\6
\4\&{var} \37\|i: \37\\{buf\_pointer};\C{running}\6
\|j: \37\\{pool\_pointer};\C{indices}\2\6
\&{begin} \37\&{if} $(\\{length}(\|s)\I\\{len})$ \1\&{then}\C{strings of
unequal length}\6
\&{begin} \37$\\{str\_eq\_buf}\K\\{false}$;\5
\&{return};\6
\&{end};\2\6
$\|i\K\\{bf\_ptr}$;\5
$\|j\K\\{str\_start}[\|s]$;\6
\&{while} $(\|j<\\{str\_start}[\|s+1])$ \1\&{do}\6
\&{begin} \37\&{if} $(\\{str\_pool}[\|j]\I\\{buf}[\|i])$ \1\&{then}\6
\&{begin} \37$\\{str\_eq\_buf}\K\\{false}$;\5
\&{return};\6
\&{end};\2\6
$\\{incr}(\|i)$;\5
$\\{incr}(\|j)$;\6
\&{end};\2\6
$\\{str\_eq\_buf}\K\\{true}$;\6
\4\\{exit}: \37\&{end};\par
\fi

\M57.
This subroutine compares two \\{str\_pool} strings and returns true
\\{true} if and only if the strings are equal.

\Y\P$\4\X54:Procedures and functions for handling numbers, characters, and
strings\X\mathrel{+}\S$\6
\4\&{function}\1\  \37$\\{str\_eq\_str}(\\{s1},\39\\{s2}:\\{str\_number})$: \37%
\\{boolean};\6
\4\&{label} \37\\{exit};\2\6
\&{begin} \37\&{if} $(\\{length}(\\{s1})\I\\{length}(\\{s2}))$ \1\&{then}\6
\&{begin} \37$\\{str\_eq\_str}\K\\{false}$;\5
\&{return};\6
\&{end};\2\6
$\\{p\_ptr1}\K\\{str\_start}[\\{s1}]$;\5
$\\{p\_ptr2}\K\\{str\_start}[\\{s2}]$;\6
\&{while} $(\\{p\_ptr1}<\\{str\_start}[\\{s1}+1])$ \1\&{do}\6
\&{begin} \37\&{if} $(\\{str\_pool}[\\{p\_ptr1}]\I\\{str\_pool}[\\{p\_ptr2}])$ %
\1\&{then}\6
\&{begin} \37$\\{str\_eq\_str}\K\\{false}$;\5
\&{return};\6
\&{end};\2\6
$\\{incr}(\\{p\_ptr1})$;\5
$\\{incr}(\\{p\_ptr2})$;\6
\&{end};\2\6
$\\{str\_eq\_str}\K\\{true}$;\6
\4\\{exit}: \37\&{end};\par
\fi

\M58.
This procedure copies file name \\{file\_name} into the beginning of
\\{name\_of\_file}, if it will fit.  It also sets the global variable
\\{name\_length} to the appropriate value.

\Y\P$\4\X38:Procedures and functions for file-system interacting\X\mathrel{+}%
\S$\6
\4\&{procedure}\1\  \37$\\{start\_name}(\\{file\_name}:\\{str\_number})$;\6
\4\&{var} \37\\{p\_ptr}: \37\\{pool\_pointer};\C{running index}\2\6
\&{begin} \37\&{if} $(\\{length}(\\{file\_name})>\\{file\_name\_size})$ \1%
\&{then}\6
\&{begin} \37$\\{print}(\.{\'File=\'})$;\5
$\\{print\_pool\_str}(\\{file\_name})$;\5
$\\{print\_ln}(\.{\',\'})$;\5
\\{file\_nm\_size\_overflow};\6
\&{end};\2\6
$\\{name\_ptr}\K1$;\5
$\\{p\_ptr}\K\\{str\_start}[\\{file\_name}]$;\6
\&{while} $(\\{p\_ptr}<\\{str\_start}[\\{file\_name}+1])$ \1\&{do}\6
\&{begin} \37$\\{name\_of\_file}[\\{name\_ptr}]\K\\{chr}(\\{str\_pool}[\\{p%
\_ptr}])$;\5
$\\{incr}(\\{name\_ptr})$;\5
$\\{incr}(\\{p\_ptr})$;\6
\&{end};\2\6
$\\{name\_length}\K\\{length}(\\{file\_name})$;\6
\&{end};\par
\fi

\M59.
Yet another complaint-before-quiting.

\Y\P$\4\X3:Procedures and functions for all file I/O, error messages, and such%
\X\mathrel{+}\S$\6
\4\&{procedure}\1\  \37\\{file\_nm\_size\_overflow};\2\6
\&{begin} \37$\\{overflow}(\.{\'file\ name\ size\ \'},\39\\{file\_name%
\_size})$;\6
\&{end};\par
\fi

\M60.
This procedure copies file extension \\{ext} into the array
\\{name\_of\_file} starting at position $\\{name\_length}+1$.  It also sets the
global variable \\{name\_length} to the appropriate value.

\Y\P$\4\X38:Procedures and functions for file-system interacting\X\mathrel{+}%
\S$\6
\4\&{procedure}\1\  \37$\\{add\_extension}(\\{ext}:\\{str\_number})$;\6
\4\&{var} \37\\{p\_ptr}: \37\\{pool\_pointer};\C{running index}\2\6
\&{begin} \37\&{if} $(\\{name\_length}+\\{length}(\\{ext})>\\{file\_name%
\_size})$ \1\&{then}\6
\&{begin} \37$\\{print}(\.{\'File=\'},\39\\{name\_of\_file},\39\.{\',\
extension=\'})$;\5
$\\{print\_pool\_str}(\\{ext})$;\5
$\\{print\_ln}(\.{\',\'})$;\5
\\{file\_nm\_size\_overflow};\6
\&{end};\2\6
$\\{name\_ptr}\K\\{name\_length}+1$;\5
$\\{p\_ptr}\K\\{str\_start}[\\{ext}]$;\6
\&{while} $(\\{p\_ptr}<\\{str\_start}[\\{ext}+1])$ \1\&{do}\6
\&{begin} \37$\\{name\_of\_file}[\\{name\_ptr}]\K\\{chr}(\\{str\_pool}[\\{p%
\_ptr}])$;\5
$\\{incr}(\\{name\_ptr})$;\5
$\\{incr}(\\{p\_ptr})$;\6
\&{end};\2\6
$\\{name\_length}\K\\{name\_length}+\\{length}(\\{ext})$;\5
$\\{name\_ptr}\K\\{name\_length}+1$;\6
\&{while} $(\\{name\_ptr}\L\\{file\_name\_size})$ \1\&{do}\C{pad with blanks}\6
\&{begin} \37$\\{name\_of\_file}[\\{name\_ptr}]\K\.{\'\ \'}$;\5
$\\{incr}(\\{name\_ptr})$;\6
\&{end};\2\6
\&{end};\par
\fi

\M61.
This procedure copies the default logical area name \\{area} into the
array \\{name\_of\_file} starting at position 1, after shifting up the
rest of the filename.  It also sets the global variable \\{name\_length}
to the appropriate value.

\Y\P$\4\X38:Procedures and functions for file-system interacting\X\mathrel{+}%
\S$\6
\4\&{procedure}\1\  \37$\\{add\_area}(\\{area}:\\{str\_number})$;\6
\4\&{var} \37\\{p\_ptr}: \37\\{pool\_pointer};\C{running index}\2\6
\&{begin} \37\&{if} $(\\{name\_length}+\\{length}(\\{area})>\\{file\_name%
\_size})$ \1\&{then}\6
\&{begin} \37$\\{print}(\.{\'File=\'})$;\5
$\\{print\_pool\_str}(\\{area})$;\5
$\\{print}(\\{name\_of\_file},\39\.{\',\'})$;\5
\\{file\_nm\_size\_overflow};\6
\&{end};\2\6
$\\{name\_ptr}\K\\{name\_length}$;\6
\&{while} $(\\{name\_ptr}>0)$ \1\&{do}\C{shift up name}\6
\&{begin} \37$\\{name\_of\_file}[\\{name\_ptr}+\\{length}(\\{area})]\K\\{name%
\_of\_file}[\\{name\_ptr}]$;\5
$\\{decr}(\\{name\_ptr})$;\6
\&{end};\2\6
$\\{name\_ptr}\K1$;\5
$\\{p\_ptr}\K\\{str\_start}[\\{area}]$;\6
\&{while} $(\\{p\_ptr}<\\{str\_start}[\\{area}+1])$ \1\&{do}\6
\&{begin} \37$\\{name\_of\_file}[\\{name\_ptr}]\K\\{chr}(\\{str\_pool}[\\{p%
\_ptr}])$;\5
$\\{incr}(\\{name\_ptr})$;\5
$\\{incr}(\\{p\_ptr})$;\6
\&{end};\2\6
$\\{name\_length}\K\\{name\_length}+\\{length}(\\{area})$;\6
\&{end};\par
\fi

\M62.
This system-independent procedure converts upper-case characters to
lower case for the specified part of \\{buf}.  It is system independent
because it uses only the internal representation for characters.

\Y\P\D \37$\\{case\_difference}=\.{"a"}-\.{"A"}$\par
\Y\P$\4\X54:Procedures and functions for handling numbers, characters, and
strings\X\mathrel{+}\S$\6
\4\&{procedure}\1\  \37$\\{lower\_case}(\mathop{\&{var}}\\{buf}:\\{buf\_type};%
\,\35\\{bf\_ptr},\39\\{len}:\\{buf\_pointer})$;\6
\4\&{var} \37\|i: \37\\{buf\_pointer};\2\6
\&{begin} \37\&{if} $(\\{len}>0)$ \1\&{then}\6
\&{for} $\|i\K\\{bf\_ptr}\mathrel{\&{to}}\\{bf\_ptr}+\\{len}-1$ \1\&{do}\6
\&{if} $((\\{buf}[\|i]\G\.{"A"})\W(\\{buf}[\|i]\L\.{"Z"}))$ \1\&{then}\5
$\\{buf}[\|i]\K\\{buf}[\|i]+\\{case\_difference}$;\2\2\2\6
\&{end};\par
\fi

\M63.
This system-independent procedure is the same as the previous except
that it converts lower- to upper-case letters.

\Y\P$\4\X54:Procedures and functions for handling numbers, characters, and
strings\X\mathrel{+}\S$\6
\4\&{procedure}\1\  \37$\\{upper\_case}(\mathop{\&{var}}\\{buf}:\\{buf\_type};%
\,\35\\{bf\_ptr},\39\\{len}:\\{buf\_pointer})$;\6
\4\&{var} \37\|i: \37\\{buf\_pointer};\2\6
\&{begin} \37\&{if} $(\\{len}>0)$ \1\&{then}\6
\&{for} $\|i\K\\{bf\_ptr}\mathrel{\&{to}}\\{bf\_ptr}+\\{len}-1$ \1\&{do}\6
\&{if} $((\\{buf}[\|i]\G\.{"a"})\W(\\{buf}[\|i]\L\.{"z"}))$ \1\&{then}\5
$\\{buf}[\|i]\K\\{buf}[\|i]-\\{case\_difference}$;\2\2\2\6
\&{end};\par
\fi

\N64.  The hash table.
All static strings that \BibTeX\ might have to search for, generally
identifiers, are stored and retrieved by means of a fairly standard
hash-table algorithm (but slightly altered here) called the method of
``coalescing lists''
(cf.\ Algorithm 6.4C in {\sl The Art of Computer Programming}).
Once a string enters the table, it is never removed.  The actual
sequence of characters forming a string is stored in the \\{str\_pool}
array.

The hash table consists of the four arrays \\{hash\_next}, \\{hash\_text},
\\{hash\_ilk}, and \\{ilk\_info}.  The first array, $\\{hash\_next}[\|p]$,
points
to the next identifier belonging to the same coalesced list as the
identifier corresponding to~\|p.  The second, $\\{hash\_text}[\|p]$, points
to the \\{str\_start} entry for \|p's string. If position~\|p of the hash
table is empty, we have $\\{hash\_text}[\|p]=0$; if position \|p is either
empty or the end of a coalesced hash list, we have
$\\{hash\_next}[\|p]=\\{empty}$; an auxiliary pointer variable called \\{hash%
\_used}
is maintained in such a way that all locations $\|p\G\\{hash\_used}$ are
nonempty.  The third, $\\{hash\_ilk}[\|p]$, tells how this string is used (as
ordinary text, as a variable name, as an \.{.aux} file command, etc).
The fourth, $\\{ilk\_info}[\|p]$, contains information specific to the
corresponding \\{hash\_ilk}---for \\{integer\_ilk}s: the integer's value;
for \\{cite\_ilk}s: a pointer into \\{cite\_list}; for \\{lc\_cite\_ilk}s: a
pointer to a \\{cite\_ilk} string; for \\{command\_ilk}s: a constant to be
used in a   \&{case}  statement; for \\{bst\_fn\_ilk}s: function-specific
information; for \\{macro\_ilk}s: a pointer to its definition string; for
\\{control\_seq\_ilk}s: a constant for use in a   \&{case}  statement; for all
other \\{ilk}s it contains no information.  This \\{ilk}-specific
information is set in other parts of the program rather than here in
the hashing routine.

\Y\P\D \37$\\{hash\_base}=\\{empty}+1$\C{lowest numbered hash-table location}%
\par
\P\D \37$\\{hash\_max}=\\{hash\_base}+\\{hash\_size}-1$\C{highest numbered
hash-table location}\par
\P\D \37$\\{hash\_is\_full}\S(\\{hash\_used}=\\{hash\_base})$\C{test if all
positions are occupied}\Y\par
\P\D \37$\\{text\_ilk}=0$\C{a string of ordinary text}\par
\P\D \37$\\{integer\_ilk}=1$\C{an integer (possibly with a \\{minus\_sign})}\par
\P\D \37$\\{aux\_command\_ilk}=2$\C{an \.{.aux}-file command}\par
\P\D \37$\\{aux\_file\_ilk}=3$\C{an \.{.aux} file name}\par
\P\D \37$\\{bst\_command\_ilk}=4$\C{a \.{.bst}-file command}\par
\P\D \37$\\{bst\_file\_ilk}=5$\C{a \.{.bst} file name}\par
\P\D \37$\\{bib\_file\_ilk}=6$\C{a \.{.bib} file name}\par
\P\D \37$\\{file\_ext\_ilk}=7$\C{one of \.{.aux}, \.{.bst}, \.{.bib}, \.{.bbl},
                                                                or \.{.blg}}\par
\P\D \37$\\{file\_area\_ilk}=8$\C{one of \.{texinputs:} or \.{texbib:}}\par
\P\D \37$\\{cite\_ilk}=9$\C{a \.{\\citation} argument}\par
\P\D \37$\\{lc\_cite\_ilk}=10$\C{a \.{\\citation} argument converted to lower
case}\par
\P\D \37$\\{bst\_fn\_ilk}=11$\C{a \.{.bst} function name}\par
\P\D \37$\\{bib\_command\_ilk}=12$\C{a \.{.bib}-file command}\par
\P\D \37$\\{macro\_ilk}=13$\C{a \.{.bst} macro or a \.{.bib} string}\par
\P\D \37$\\{control\_seq\_ilk}=14$\C{a control sequence specifying a foreign
character}\par
\P\D \37$\\{last\_ilk}=14$\C{the same number as on the line above}\par
\Y\P$\4\X22:Types in the outer block\X\mathrel{+}\S$\6
$\\{hash\_loc}=\\{hash\_base}\to\\{hash\_max}$;\C{a location within the hash
table}\6
$\\{hash\_pointer}=\\{empty}\to\\{hash\_max}$;\C{either \\{empty} or a \\{hash%
\_loc}}\7
$\\{str\_ilk}=0\to\\{last\_ilk}$;\C{the legal string types}\par
\fi

\M65.
\Y\P$\4\X16:Globals in the outer block\X\mathrel{+}\S$\6
\4\\{hash\_next}: \37\&{packed} \37\&{array} $[\\{hash\_loc}]$ \1\&{of}\5
\\{hash\_pointer};\C{coalesced-list link}\2\6
\4\\{hash\_text}: \37\&{packed} \37\&{array} $[\\{hash\_loc}]$ \1\&{of}\5
\\{str\_number};\C{pointer to a string}\2\6
\4\\{hash\_ilk}: \37\&{packed} \37\&{array} $[\\{hash\_loc}]$ \1\&{of}\5
\\{str\_ilk};\C{the type of string}\2\6
\4\\{ilk\_info}: \37\&{packed} \37\&{array} $[\\{hash\_loc}]$ \1\&{of}\5
\\{integer};\C{\\{ilk}-specific info}\2\6
\4\\{hash\_used}: \37$\\{hash\_base}\to\\{hash\_max}+1$;\C{allocation pointer
for hash table}\6
\4\\{hash\_found}: \37\\{boolean};\C{set to \\{true} if it's already in the
hash table}\6
\4\\{dummy\_loc}: \37\\{hash\_loc};\C{receives \\{str\_lookup} value whenever
it's useless}\par
\fi

\M66.
\Y\P$\4\X23:Local variables for initialization\X\mathrel{+}\S$\6
\4\|k: \37\\{hash\_loc};\par
\fi

\M67.
Now it's time to initialize the hash table; note that $\\{str\_start}[0]$
must be unused if $\\{hash\_text}[\|k]\K0$ is to have the desired effect.

\Y\P$\4\X20:Set initial values of key variables\X\mathrel{+}\S$\6
\&{for} $\|k\K\\{hash\_base}\mathrel{\&{to}}\\{hash\_max}$ \1\&{do}\6
\&{begin} \37$\\{hash\_next}[\|k]\K\\{empty}$;\5
$\\{hash\_text}[\|k]\K0$;\C{thus, no need to initialize \\{hash\_ilk} or \\{ilk%
\_info}}\6
\&{end};\2\6
$\\{hash\_used}\K\\{hash\_max}+1$;\C{nothing in table initially}\par
\fi

\M68.
Here is the subroutine that searches the hash table for a
(string,~\\{str\_ilk}) pair, where the string is of length $\|l\G0$ and
appears in $\\{buffer}[\|j\to(\|j+\|l-1)]$.  If it finds the pair, it returns
the
corresponding hash-table location and sets the global variable
\\{hash\_found} to \\{true}.  Otherwise it sets \\{hash\_found} to \\{false},
and if the parameter \\{insert\_it} is \\{true}, it inserts the pair into
the hash table, inserts the string into \\{str\_pool} if not previously
encountered, and returns its location.  Note that two different pairs
can have the same string but different \\{str\_ilk}s, in which case the
second pair encountered, if \\{insert\_it} were \\{true}, would be inserted
into the hash table though its string wouldn't be inserted into
\\{str\_pool} because it would already be there.

\Y\P\D \37$\\{max\_hash\_value}=\\{hash\_prime}+\\{hash\_prime}-2+127$\C{\|h's
maximum value}\par
\P\D \37$\\{do\_insert}\S\\{true}$\C{insert string if not found in hash table}%
\par
\P\D \37$\\{dont\_insert}\S\\{false}$\C{don't insert string}\Y\par
\P\D \37$\\{str\_found}=40$\C{go here when you've found the string}\par
\P\D \37$\\{str\_not\_found}=45$\C{go here when you haven't}\par
\Y\P$\4\X54:Procedures and functions for handling numbers, characters, and
strings\X\mathrel{+}\S$\6
\4\&{function}\1\  \37$\\{str\_lookup}(\mathop{\&{var}}\\{buf}:\\{buf\_type};\,%
\35\|j,\39\|l:\\{buf\_pointer};\,\35\\{ilk}:\\{str\_ilk};\,\35\\{insert\_it}:%
\\{boolean})$: \37\\{hash\_loc};\C{search the hash table}\6
\4\&{label} \37$\\{str\_found},\39\\{str\_not\_found}$;\6
\4\&{var} \37\|h: \37$0\to\\{max\_hash\_value}$;\C{hash code}\6
\|p: \37\\{hash\_loc};\C{index into \\{hash\_} arrays}\6
\|k: \37\\{buf\_pointer};\C{index into \\{buf} array}\6
\\{old\_string}: \37\\{boolean};\C{set to \\{true} if it's an already
encountered string}\6
\\{str\_num}: \37\\{str\_number};\C{pointer to an already encountered string}\2%
\6
\&{begin} \37\X69:Compute the hash code \|h\X;\6
$\|p\K\|h+\\{hash\_base}$;\C{start searching here; note that $0\L\|h<\\{hash%
\_prime}$}\6
$\\{hash\_found}\K\\{false}$;\5
$\\{old\_string}\K\\{false}$;\6
\~ \1\&{loop}\6
\&{begin} \37\X70:Process the string if we've already encountered it\X;\6
\&{if} $(\\{hash\_next}[\|p]=\\{empty})$ \1\&{then}\C{location \|p may or may
not be empty}\6
\&{begin} \37\&{if} $(\R\\{insert\_it})$ \1\&{then}\5
\&{goto} \37\\{str\_not\_found};\2\6
\X71:Insert pair into hash table and make \|p point to it\X;\6
\&{goto} \37\\{str\_found};\6
\&{end};\2\6
$\|p\K\\{hash\_next}[\|p]$;\C{old and new locations \|p are not empty}\6
\&{end};\2\6
\4\\{str\_not\_found}: \37\\{do\_nothing};\C{don't insert pair; function value
meaningless}\6
\4\\{str\_found}: \37$\\{str\_lookup}\K\|p$;\6
\&{end};\par
\fi

\M69.
The value of \\{hash\_prime} should be roughly 85\% of \\{hash\_size}, and
it should be a prime number
(it should also be less than $2^{14} + 2^{6} = 16320$ because of
\.{WEB}'s simple-macro bound).  The theory of hashing tells us to expect
fewer than two table probes, on the average, when the search is
successful.

\Y\P$\4\X69:Compute the hash code \|h\X\S$\6
\&{begin} \37$\|h\K0$;\C{note that this works for zero-length strings}\6
$\|k\K\|j$;\6
\&{while} $(\|k<\|j+\|l)$ \1\&{do}\C{not a  \&{for}  loop in case $\|j=\|l=0$}\6
\&{begin} \37$\|h\K\|h+\|h+\\{buf}[\|k]$;\6
\&{while} $(\|h\G\\{hash\_prime})$ \1\&{do}\5
$\|h\K\|h-\\{hash\_prime}$;\2\6
$\\{incr}(\|k)$;\6
\&{end};\2\6
\&{end}\par
\U68.\fi

\M70.
Here we handle the case in which we've already encountered this
string; note that even if we have, we'll still have to insert the pair
into the hash table if \\{str\_ilk} doesn't match.

\Y\P$\4\X70:Process the string if we've already encountered it\X\S$\6
\&{begin} \37\&{if} $(\\{hash\_text}[\|p]>0)$ \1\&{then}\C{there's something
here}\6
\&{if} $(\\{str\_eq\_buf}(\\{hash\_text}[\|p],\39\\{buf},\39\|j,\39\|l))$ \1%
\&{then}\C{it's the right string}\6
\&{if} $(\\{hash\_ilk}[\|p]=\\{ilk})$ \1\&{then}\C{it's the right \\{str\_ilk}}%
\6
\&{begin} \37$\\{hash\_found}\K\\{true}$;\5
\&{goto} \37\\{str\_found};\6
\&{end}\6
\4\&{else} \&{begin} \37\C{it's the wrong \\{str\_ilk}}\6
$\\{old\_string}\K\\{true}$;\5
$\\{str\_num}\K\\{hash\_text}[\|p]$;\6
\&{end};\2\2\2\6
\&{end}\par
\U68.\fi

\M71.
This code inserts the pair in the appropriate unused location.

\Y\P$\4\X71:Insert pair into hash table and make \|p point to it\X\S$\6
\&{begin} \37\&{if} $(\\{hash\_text}[\|p]>0)$ \1\&{then}\C{location \|p isn't
empty}\6
\&{begin} \37\1\&{repeat} \37\&{if} $(\\{hash\_is\_full})$ \1\&{then}\5
$\\{overflow}(\.{\'hash\ size\ \'},\39\\{hash\_size})$;\2\6
$\\{decr}(\\{hash\_used})$;\6
\4\&{until}\5
$(\\{hash\_text}[\\{hash\_used}]=0)$;\C{search for an empty location}\2\6
$\\{hash\_next}[\|p]\K\\{hash\_used}$;\5
$\|p\K\\{hash\_used}$;\6
\&{end};\C{now location \|p is empty}\2\6
\&{if} $(\\{old\_string})$ \1\&{then}\C{it's an already encountered string}\6
$\\{hash\_text}[\|p]\K\\{str\_num}$\6
\4\&{else} \&{begin} \37\C{it's a new string}\6
$\\{str\_room}(\|l)$;\C{make sure it'll fit in \\{str\_pool}}\6
$\|k\K\|j$;\6
\&{while} $(\|k<\|j+\|l)$ \1\&{do}\C{not a  \&{for}  loop in case $\|j=\|l=0$}\6
\&{begin} \37$\\{append\_char}(\\{buf}[\|k])$;\5
$\\{incr}(\|k)$;\6
\&{end};\2\6
$\\{hash\_text}[\|p]\K\\{make\_string}$;\C{and make it official}\6
\&{end};\2\6
$\\{hash\_ilk}[\|p]\K\\{ilk}$;\6
\&{end}\par
\U68.\fi

\M72.
Now that we've defined the hash-table workings we can initialize the
string pool.  Unlike \TeX, \BibTeX\ does not use a \\{pool\_file} for
string storage; instead it inserts its pre-defined strings into
\\{str\_pool}---this makes one file fewer for the \BibTeX\ implementor
to deal with.  This section initializes \\{str\_pool}; the pre-defined
strings will be inserted into it shortly; and other strings are
inserted while processing the input files.

\Y\P$\4\X20:Set initial values of key variables\X\mathrel{+}\S$\6
$\\{pool\_ptr}\K0$;\5
$\\{str\_ptr}\K1$;\C{hash table must have $\\{str\_start}[0]$ unused}\6
$\\{str\_start}[\\{str\_ptr}]\K\\{pool\_ptr}$;\par
\fi

\M73.
The longest pre-defined string determines type definitions used to
insert the pre-defined strings into \\{str\_pool}.

\Y\P\D \37$\\{longest\_pds}=12$\C{the length of `\.{change.case\$}'}\par
\Y\P$\4\X22:Types in the outer block\X\mathrel{+}\S$\6
$\\{pds\_loc}=1\to\\{longest\_pds}$;\5
$\\{pds\_len}=0\to\\{longest\_pds}$;\5
$\\{pds\_type}=$\1\5
\&{packed} \37\&{array} $[\\{pds\_loc}]$ \1\&{of}\5
\\{char};\2\2\par
\fi

\M74.
The variables in this program beginning with \\{s\_} specify the
locations in \\{str\_pool} for certain often-used strings.  Those here
have to do with the file system; the next section will actually insert
them into \\{str\_pool}.

\Y\P$\4\X16:Globals in the outer block\X\mathrel{+}\S$\6
\4\\{s\_aux\_extension}: \37\\{str\_number};\C{\.{.aux}}\6
\4\\{s\_log\_extension}: \37\\{str\_number};\C{\.{.blg}}\6
\4\\{s\_bbl\_extension}: \37\\{str\_number};\C{\.{.bbl}}\6
\4\\{s\_bst\_extension}: \37\\{str\_number};\C{\.{.bst}}\6
\4\\{s\_bib\_extension}: \37\\{str\_number};\C{\.{.bib}}\6
\4\\{s\_bst\_area}: \37\\{str\_number};\C{\.{texinputs:}}\6
\4\\{s\_bib\_area}: \37\\{str\_number};\C{\.{texbib:}}\par
\fi

\M75.
It's time to insert some of the pre-defined strings into \\{str\_pool}
(and thus the hash table).  These system-dependent strings should
contain no upper-case letters, and they must all be exactly
\\{longest\_pds} characters long (even if fewer characters are actually
stored).  The \\{pre\_define} routine appears shortly.

Important notes: These pre-definitions must not have any glitches or
the program may bomb because the \\{log\_file} hasn't been opened yet,
and \\{text\_ilk}s should be pre-defined later, for
\.{.bst}-function-execution purposes.

\Y\P$\4\X75:Pre-define certain strings\X\S$\6
$\\{pre\_define}(\.{\'.aux\ \ \ \ \ \ \ \ \'},\394,\39\\{file\_ext\_ilk})$;\5
$\\{s\_aux\_extension}\K\\{hash\_text}[\\{pre\_def\_loc}]$;\5
$\\{pre\_define}(\.{\'.bbl\ \ \ \ \ \ \ \ \'},\394,\39\\{file\_ext\_ilk})$;\5
$\\{s\_bbl\_extension}\K\\{hash\_text}[\\{pre\_def\_loc}]$;\5
$\\{pre\_define}(\.{\'.blg\ \ \ \ \ \ \ \ \'},\394,\39\\{file\_ext\_ilk})$;\5
$\\{s\_log\_extension}\K\\{hash\_text}[\\{pre\_def\_loc}]$;\5
$\\{pre\_define}(\.{\'.bst\ \ \ \ \ \ \ \ \'},\394,\39\\{file\_ext\_ilk})$;\5
$\\{s\_bst\_extension}\K\\{hash\_text}[\\{pre\_def\_loc}]$;\5
$\\{pre\_define}(\.{\'.bib\ \ \ \ \ \ \ \ \'},\394,\39\\{file\_ext\_ilk})$;\5
$\\{s\_bib\_extension}\K\\{hash\_text}[\\{pre\_def\_loc}]$;\5
$\\{pre\_define}(\.{\'texinputs:\ \ \'},\3910,\39\\{file\_area\_ilk})$;\5
$\\{s\_bst\_area}\K\\{hash\_text}[\\{pre\_def\_loc}]$;\5
$\\{pre\_define}(\.{\'texbib:\ \ \ \ \ \'},\397,\39\\{file\_area\_ilk})$;\5
$\\{s\_bib\_area}\K\\{hash\_text}[\\{pre\_def\_loc}]$;\par
\As79, 334, 339\ETs340.
\U336.\fi

\M76.
This global variable gives the hash-table location of pre-defined
strings generated by calls to \\{str\_lookup}.

\Y\P$\4\X16:Globals in the outer block\X\mathrel{+}\S$\6
\4\\{pre\_def\_loc}: \37\\{hash\_loc};\par
\fi

\M77.
This procedure initializes a pre-defined string of length at most
\\{longest\_pds}.

\Y\P$\4\X54:Procedures and functions for handling numbers, characters, and
strings\X\mathrel{+}\S$\6
\4\&{procedure}\1\  \37$\\{pre\_define}(\\{pds}:\\{pds\_type};\,\35\\{len}:%
\\{pds\_len};\,\35\\{ilk}:\\{str\_ilk})$;\6
\4\&{var} \37\|i: \37\\{pds\_len};\2\6
\&{begin} \37\&{for} $\|i\K1\mathrel{\&{to}}\\{len}$ \1\&{do}\5
$\\{buffer}[\|i]\K\\{xord}[\\{pds}[\|i]]$;\2\6
$\\{pre\_def\_loc}\K\\{str\_lookup}(\\{buffer},\391,\39\\{len},\39\\{ilk},\39%
\\{do\_insert})$;\6
\&{end};\par
\fi

\M78.
These constants all begin with \\{n\_} and are used for the   \&{case}
statement that determines which command to execute.  The variable
\\{command\_num} is set to one of these and is used to do the branching,
but it must have the full \\{integer} range because at times it can
assume an arbitrary \\{ilk\_info} value (though it will be one of the
values here when we actually use it).

\Y\P\D \37$\\{n\_aux\_bibdata}=0$\C{\.{\\bibdata}}\par
\P\D \37$\\{n\_aux\_bibstyle}=1$\C{\.{\\bibstyle}}\par
\P\D \37$\\{n\_aux\_citation}=2$\C{\.{\\citation}}\par
\P\D \37$\\{n\_aux\_input}=3$\C{\.{\\@input}}\Y\par
\P\D \37$\\{n\_bst\_entry}=0$\C{\.{entry}}\par
\P\D \37$\\{n\_bst\_execute}=1$\C{\.{execute}}\par
\P\D \37$\\{n\_bst\_function}=2$\C{\.{function}}\par
\P\D \37$\\{n\_bst\_integers}=3$\C{\.{integers}}\par
\P\D \37$\\{n\_bst\_iterate}=4$\C{\.{iterate}}\par
\P\D \37$\\{n\_bst\_macro}=5$\C{\.{macro}}\par
\P\D \37$\\{n\_bst\_read}=6$\C{\.{read}}\par
\P\D \37$\\{n\_bst\_reverse}=7$\C{\.{reverse}}\par
\P\D \37$\\{n\_bst\_sort}=8$\C{\.{sort}}\par
\P\D \37$\\{n\_bst\_strings}=9$\C{\.{strings}}\Y\par
\P\D \37$\\{n\_bib\_comment}=0$\C{\.{comment}}\par
\P\D \37$\\{n\_bib\_preamble}=1$\C{\.{preamble}}\par
\P\D \37$\\{n\_bib\_string}=2$\C{\.{string}}\par
\Y\P$\4\X16:Globals in the outer block\X\mathrel{+}\S$\6
\4\\{command\_num}: \37\\{integer};\par
\fi

\M79.
Now we pre-define the command strings; they must all be exactly
\\{longest\_pds} characters long.

Important note: These pre-definitions must not have any glitches or
the program may bomb because the \\{log\_file} hasn't been opened yet.

\Y\P$\4\X75:Pre-define certain strings\X\mathrel{+}\S$\6
$\\{pre\_define}(\.{\'\\citation\ \ \ \'},\399,\39\\{aux\_command\_ilk})$;\5
$\\{ilk\_info}[\\{pre\_def\_loc}]\K\\{n\_aux\_citation}$;\5
$\\{pre\_define}(\.{\'\\bibdata\ \ \ \ \'},\398,\39\\{aux\_command\_ilk})$;\5
$\\{ilk\_info}[\\{pre\_def\_loc}]\K\\{n\_aux\_bibdata}$;\5
$\\{pre\_define}(\.{\'\\bibstyle\ \ \ \'},\399,\39\\{aux\_command\_ilk})$;\5
$\\{ilk\_info}[\\{pre\_def\_loc}]\K\\{n\_aux\_bibstyle}$;\5
$\\{pre\_define}(\.{\'\\@input\ \ \ \ \ \'},\397,\39\\{aux\_command\_ilk})$;\5
$\\{ilk\_info}[\\{pre\_def\_loc}]\K\\{n\_aux\_input}$;\7
$\\{pre\_define}(\.{\'entry\ \ \ \ \ \ \ \'},\395,\39\\{bst\_command\_ilk})$;\5
$\\{ilk\_info}[\\{pre\_def\_loc}]\K\\{n\_bst\_entry}$;\5
$\\{pre\_define}(\.{\'execute\ \ \ \ \ \'},\397,\39\\{bst\_command\_ilk})$;\5
$\\{ilk\_info}[\\{pre\_def\_loc}]\K\\{n\_bst\_execute}$;\5
$\\{pre\_define}(\.{\'function\ \ \ \ \'},\398,\39\\{bst\_command\_ilk})$;\5
$\\{ilk\_info}[\\{pre\_def\_loc}]\K\\{n\_bst\_function}$;\5
$\\{pre\_define}(\.{\'integers\ \ \ \ \'},\398,\39\\{bst\_command\_ilk})$;\5
$\\{ilk\_info}[\\{pre\_def\_loc}]\K\\{n\_bst\_integers}$;\5
$\\{pre\_define}(\.{\'iterate\ \ \ \ \ \'},\397,\39\\{bst\_command\_ilk})$;\5
$\\{ilk\_info}[\\{pre\_def\_loc}]\K\\{n\_bst\_iterate}$;\5
$\\{pre\_define}(\.{\'macro\ \ \ \ \ \ \ \'},\395,\39\\{bst\_command\_ilk})$;\5
$\\{ilk\_info}[\\{pre\_def\_loc}]\K\\{n\_bst\_macro}$;\5
$\\{pre\_define}(\.{\'read\ \ \ \ \ \ \ \ \'},\394,\39\\{bst\_command\_ilk})$;\5
$\\{ilk\_info}[\\{pre\_def\_loc}]\K\\{n\_bst\_read}$;\5
$\\{pre\_define}(\.{\'reverse\ \ \ \ \ \'},\397,\39\\{bst\_command\_ilk})$;\5
$\\{ilk\_info}[\\{pre\_def\_loc}]\K\\{n\_bst\_reverse}$;\5
$\\{pre\_define}(\.{\'sort\ \ \ \ \ \ \ \ \'},\394,\39\\{bst\_command\_ilk})$;\5
$\\{ilk\_info}[\\{pre\_def\_loc}]\K\\{n\_bst\_sort}$;\5
$\\{pre\_define}(\.{\'strings\ \ \ \ \ \'},\397,\39\\{bst\_command\_ilk})$;\5
$\\{ilk\_info}[\\{pre\_def\_loc}]\K\\{n\_bst\_strings}$;\7
$\\{pre\_define}(\.{\'comment\ \ \ \ \ \'},\397,\39\\{bib\_command\_ilk})$;\5
$\\{ilk\_info}[\\{pre\_def\_loc}]\K\\{n\_bib\_comment}$;\5
$\\{pre\_define}(\.{\'preamble\ \ \ \ \'},\398,\39\\{bib\_command\_ilk})$;\5
$\\{ilk\_info}[\\{pre\_def\_loc}]\K\\{n\_bib\_preamble}$;\5
$\\{pre\_define}(\.{\'string\ \ \ \ \ \ \'},\396,\39\\{bib\_command\_ilk})$;\5
$\\{ilk\_info}[\\{pre\_def\_loc}]\K\\{n\_bib\_string}$;\par
\fi

\N80.  Scanning an input line.
This section describes the various \\{buffer} scanning routines.  The
two global variables \\{buf\_ptr1} and \\{buf\_ptr2} are used in scanning an
input line.  Between scans, \\{buf\_ptr1} points to the first character
of the current token and \\{buf\_ptr2} points to that of the next.  The
global variable \\{last}, set by the function \\{input\_ln}, marks the end
of the current line; it equals 0 at the end of the current file.  All
the procedures and functions in this section will indicate an
end-of-line when it's the end of the file.

\Y\P\D \37$\\{token\_len}\S(\\{buf\_ptr2}-\\{buf\_ptr1})$\C{of the current
token}\par
\P\D \37$\\{scan\_char}\S\\{buffer}[\\{buf\_ptr2}]$\C{the current character}\par
\Y\P$\4\X16:Globals in the outer block\X\mathrel{+}\S$\6
\4\\{buf\_ptr1}: \37\\{buf\_pointer};\C{points to the first position of the
current token}\6
\4\\{buf\_ptr2}: \37\\{buf\_pointer};\C{used to find the end of the current
token}\par
\fi

\M81.
These macros send the current token, in $\\{buffer}[\\{buf\_ptr1}]$ to
$\\{buffer}[\\{buf\_ptr2}-1]$, to an output file.

\Y\P\D \37$\\{print\_token}\S\\{print\_a\_token}$\C{making this a procedure
saves a little space}\Y\par
\P\D \37$\\{trace\_pr\_token}\S$\1\6
\&{begin} \37$\\{out\_token}(\\{log\_file})$;\6
\&{end}\2\par
\fi

\M82.
And here are the associated procedures.  Note: The \\{term\_out} file is
system dependent.

\Y\P$\4\X3:Procedures and functions for all file I/O, error messages, and such%
\X\mathrel{+}\S$\6
\4\&{procedure}\1\  \37$\\{out\_token}(\mathop{\&{var}}\|f:\\{alpha\_file})$;\6
\4\&{var} \37\|i: \37\\{buf\_pointer};\2\6
\&{begin} \37$\|i\K\\{buf\_ptr1}$;\6
\&{while} $(\|i<\\{buf\_ptr2})$ \1\&{do}\6
\&{begin} \37$\\{write}(\|f,\39\\{xchr}[\\{buffer}[\|i]])$;\5
$\\{incr}(\|i)$;\6
\&{end};\2\6
\&{end};\7
\4\&{procedure}\1\  \37\\{print\_a\_token};\2\6
\&{begin} \37$\\{out\_token}(\\{term\_out})$;\5
$\\{out\_token}(\\{log\_file})$;\6
\&{end};\par
\fi

\M83.
This function scans the \\{buffer} for the next token, starting at the
global variable \\{buf\_ptr2} and ending just before either the single
specified stop-character or the end of the current line, whichever
comes first, respectively returning \\{true} or \\{false}; afterward,
\\{scan\_char} is the first character following this token.

\Y\P$\4\X83:Procedures and functions for input scanning\X\S$\6
\4\&{function}\1\  \37$\\{scan1}(\\{char1}:\\{ASCII\_code})$: \37\\{boolean};\2%
\6
\&{begin} \37$\\{buf\_ptr1}\K\\{buf\_ptr2}$;\C{scan until end-of-line or the
specified character}\6
\&{while} $((\\{scan\_char}\I\\{char1})\W(\\{buf\_ptr2}<\\{last}))$ \1\&{do}\5
$\\{incr}(\\{buf\_ptr2})$;\2\6
\&{if} $(\\{buf\_ptr2}<\\{last})$ \1\&{then}\5
$\\{scan1}\K\\{true}$\6
\4\&{else} $\\{scan1}\K\\{false}$;\2\6
\&{end};\par
\As84, 85, 86, 87, 88, 90, 92, 93, 94, 152, 183, 184, 185, 186, 187, 228, 248%
\ETs249.
\U12.\fi

\M84.
This function is the same but stops at \\{white\_space} characters as well.

\Y\P$\4\X83:Procedures and functions for input scanning\X\mathrel{+}\S$\6
\4\&{function}\1\  \37$\\{scan1\_white}(\\{char1}:\\{ASCII\_code})$: \37%
\\{boolean};\2\6
\&{begin} \37$\\{buf\_ptr1}\K\\{buf\_ptr2}$;\C{scan until end-of-line, the
specified character, or \\{white\_space}}\6
\&{while} $((\\{lex\_class}[\\{scan\_char}]\I\\{white\_space})\W(\\{scan\_char}%
\I\\{char1})\W(\\{buf\_ptr2}<\\{last}))$ \1\&{do}\5
$\\{incr}(\\{buf\_ptr2})$;\2\6
\&{if} $(\\{buf\_ptr2}<\\{last})$ \1\&{then}\5
$\\{scan1\_white}\K\\{true}$\6
\4\&{else} $\\{scan1\_white}\K\\{false}$;\2\6
\&{end};\par
\fi

\M85.
This function is similar to \\{scan1}, but stops at either of two
stop-characters as well as the end of the current line.

\Y\P$\4\X83:Procedures and functions for input scanning\X\mathrel{+}\S$\6
\4\&{function}\1\  \37$\\{scan2}(\\{char1},\39\\{char2}:\\{ASCII\_code})$: \37%
\\{boolean};\2\6
\&{begin} \37$\\{buf\_ptr1}\K\\{buf\_ptr2}$;\C{scan until end-of-line or the
specified characters}\6
\&{while} $((\\{scan\_char}\I\\{char1})\W(\\{scan\_char}\I\\{char2})\W(\\{buf%
\_ptr2}<\\{last}))$ \1\&{do}\5
$\\{incr}(\\{buf\_ptr2})$;\2\6
\&{if} $(\\{buf\_ptr2}<\\{last})$ \1\&{then}\5
$\\{scan2}\K\\{true}$\6
\4\&{else} $\\{scan2}\K\\{false}$;\2\6
\&{end};\par
\fi

\M86.
This function is the same but stops at \\{white\_space} characters as well.

\Y\P$\4\X83:Procedures and functions for input scanning\X\mathrel{+}\S$\6
\4\&{function}\1\  \37$\\{scan2\_white}(\\{char1},\39\\{char2}:\\{ASCII%
\_code})$: \37\\{boolean};\2\6
\&{begin} \37$\\{buf\_ptr1}\K\\{buf\_ptr2}$;\C{scan until end-of-line, the
specified characters, or \\{white\_space}}\6
\&{while} $((\\{scan\_char}\I\\{char1})\W(\\{scan\_char}\I\\{char2})\W(\\{lex%
\_class}[\\{scan\_char}]\I\\{white\_space})\W(\\{buf\_ptr2}<\\{last}))$ \1%
\&{do}\5
$\\{incr}(\\{buf\_ptr2})$;\2\6
\&{if} $(\\{buf\_ptr2}<\\{last})$ \1\&{then}\5
$\\{scan2\_white}\K\\{true}$\6
\4\&{else} $\\{scan2\_white}\K\\{false}$;\2\6
\&{end};\par
\fi

\M87.
This function is similar to \\{scan2}, but stops at either of three
stop-characters as well as the end of the current line.

\Y\P$\4\X83:Procedures and functions for input scanning\X\mathrel{+}\S$\6
\4\&{function}\1\  \37$\\{scan3}(\\{char1},\39\\{char2},\39\\{char3}:\\{ASCII%
\_code})$: \37\\{boolean};\2\6
\&{begin} \37$\\{buf\_ptr1}\K\\{buf\_ptr2}$;\C{scan until end-of-line or the
specified characters}\6
\&{while} $((\\{scan\_char}\I\\{char1})\W(\\{scan\_char}\I\\{char2})\W(\\{scan%
\_char}\I\\{char3})\W(\\{buf\_ptr2}<\\{last}))$ \1\&{do}\5
$\\{incr}(\\{buf\_ptr2})$;\2\6
\&{if} $(\\{buf\_ptr2}<\\{last})$ \1\&{then}\5
$\\{scan3}\K\\{true}$\6
\4\&{else} $\\{scan3}\K\\{false}$;\2\6
\&{end};\par
\fi

\M88.
This function scans for letters, stopping at the first nonletter; it
returns \\{true} if there is at least one letter.

\Y\P$\4\X83:Procedures and functions for input scanning\X\mathrel{+}\S$\6
\4\&{function}\1\  \37\\{scan\_alpha}: \37\\{boolean};\2\6
\&{begin} \37$\\{buf\_ptr1}\K\\{buf\_ptr2}$;\C{scan until end-of-line or a
nonletter}\6
\&{while} $((\\{lex\_class}[\\{scan\_char}]=\\{alpha})\W(\\{buf\_ptr2}<%
\\{last}))$ \1\&{do}\5
$\\{incr}(\\{buf\_ptr2})$;\2\6
\&{if} $(\\{token\_len}=0)$ \1\&{then}\5
$\\{scan\_alpha}\K\\{false}$\6
\4\&{else} $\\{scan\_alpha}\K\\{true}$;\2\6
\&{end};\par
\fi

\M89.
These are the possible values for \\{scan\_result}; they're set by the
\\{scan\_identifier} procedure and are described in the next section.

\Y\P\D \37$\\{id\_null}=0$\par
\P\D \37$\\{specified\_char\_adjacent}=1$\par
\P\D \37$\\{other\_char\_adjacent}=2$\par
\P\D \37$\\{white\_adjacent}=3$\par
\Y\P$\4\X16:Globals in the outer block\X\mathrel{+}\S$\6
\4\\{scan\_result}: \37$\\{id\_null}\to\\{white\_adjacent}$;\par
\fi

\M90.
This procedure scans for an identifier, stopping at the first
\\{illegal\_id\_char}, or stopping at the first character if it's
\\{numeric}.  It sets the global variable \\{scan\_result} to \\{id\_null} if
the identifier is null, else to \\{white\_adjacent} if it ended at a
\\{white\_space} character or an end-of-line, else to
\\{specified\_char\_adjacent} if it ended at one of \\{char1} or \\{char2} or
\\{char3}, else to \\{other\_char\_adjacent} if it ended at a nonspecified,
non\\{white\_space} \\{illegal\_id\_char}.  By convention, when some calling
code really wants just one or two ``specified'' characters, it merely
repeats one of the characters.

\Y\P$\4\X83:Procedures and functions for input scanning\X\mathrel{+}\S$\6
\4\&{procedure}\1\  \37$\\{scan\_identifier}(\\{char1},\39\\{char2},\39%
\\{char3}:\\{ASCII\_code})$;\2\6
\&{begin} \37$\\{buf\_ptr1}\K\\{buf\_ptr2}$;\6
\&{if} $(\\{lex\_class}[\\{scan\_char}]\I\\{numeric})$ \1\&{then}\C{scan until
end-of-line or an \\{illegal\_id\_char}}\6
\&{while} $((\\{id\_class}[\\{scan\_char}]=\\{legal\_id\_char})\W(\\{buf%
\_ptr2}<\\{last}))$ \1\&{do}\5
$\\{incr}(\\{buf\_ptr2})$;\2\2\6
\&{if} $(\\{token\_len}=0)$ \1\&{then}\5
$\\{scan\_result}\K\\{id\_null}$\6
\4\&{else} \&{if} $((\\{lex\_class}[\\{scan\_char}]=\\{white\_space})\V(\\{buf%
\_ptr2}=\\{last}))$ \1\&{then}\5
$\\{scan\_result}\K\\{white\_adjacent}$\6
\4\&{else} \&{if} $((\\{scan\_char}=\\{char1})\V(\\{scan\_char}=\\{char2})\V(%
\\{scan\_char}=\\{char3}))$ \1\&{then}\5
$\\{scan\_result}\K\\{specified\_char\_adjacent}$\6
\4\&{else} $\\{scan\_result}\K\\{other\_char\_adjacent}$;\2\2\2\6
\&{end};\par
\fi

\M91.
The next two procedures scan for an integer, setting the global
variable \\{token\_value} to the corresponding integer.

\Y\P\D \37$\\{char\_value}\S(\\{scan\_char}-\.{"0"})$\C{the value of the digit
being scanned}\par
\Y\P$\4\X16:Globals in the outer block\X\mathrel{+}\S$\6
\4\\{token\_value}: \37\\{integer};\C{the numeric value of the current token}%
\par
\fi

\M92.
This function scans for a nonnegative integer, stopping at the first
nondigit; it sets the value of \\{token\_value} accordingly.  It returns
\\{true} if the token was a legal nonnegative integer (i.e., consisted
of one or more digits).

\Y\P$\4\X83:Procedures and functions for input scanning\X\mathrel{+}\S$\6
\4\&{function}\1\  \37\\{scan\_nonneg\_integer}: \37\\{boolean};\2\6
\&{begin} \37$\\{buf\_ptr1}\K\\{buf\_ptr2}$;\5
$\\{token\_value}\K0$;\C{scan until end-of-line or a nondigit}\6
\&{while} $((\\{lex\_class}[\\{scan\_char}]=\\{numeric})\W(\\{buf\_ptr2}<%
\\{last}))$ \1\&{do}\6
\&{begin} \37$\\{token\_value}\K\\{token\_value}\ast10+\\{char\_value}$;\5
$\\{incr}(\\{buf\_ptr2})$;\6
\&{end};\2\6
\&{if} $(\\{token\_len}=0)$ \1\&{then}\C{there were no digits}\6
$\\{scan\_nonneg\_integer}\K\\{false}$\6
\4\&{else} $\\{scan\_nonneg\_integer}\K\\{true}$;\2\6
\&{end};\par
\fi

\M93.
This procedure scans for an integer, stopping at the first nondigit;
it sets the value of \\{token\_value} accordingly.  It returns \\{true} if
the token was a legal integer (i.e., consisted of an optional
\\{minus\_sign} followed by one or more digits).

\Y\P\D \37$\\{negative}\S(\\{sign\_length}=1)$\C{if this integer is negative}%
\par
\Y\P$\4\X83:Procedures and functions for input scanning\X\mathrel{+}\S$\6
\4\&{function}\1\  \37\\{scan\_integer}: \37\\{boolean};\6
\4\&{var} \37\\{sign\_length}: \37$0\to1$;\C{1 if there's a \\{minus\_sign}, 0
if not}\2\6
\&{begin} \37$\\{buf\_ptr1}\K\\{buf\_ptr2}$;\6
\&{if} $(\\{scan\_char}=\\{minus\_sign})$ \1\&{then}\C{it's a negative number}\6
\&{begin} \37$\\{sign\_length}\K1$;\5
$\\{incr}(\\{buf\_ptr2})$;\C{skip over the \\{minus\_sign}}\6
\&{end}\6
\4\&{else} $\\{sign\_length}\K0$;\2\6
$\\{token\_value}\K0$;\C{scan until end-of-line or a nondigit}\6
\&{while} $((\\{lex\_class}[\\{scan\_char}]=\\{numeric})\W(\\{buf\_ptr2}<%
\\{last}))$ \1\&{do}\6
\&{begin} \37$\\{token\_value}\K\\{token\_value}\ast10+\\{char\_value}$;\5
$\\{incr}(\\{buf\_ptr2})$;\6
\&{end};\2\6
\&{if} $(\\{negative})$ \1\&{then}\5
$\\{token\_value}\K-\\{token\_value}$;\2\6
\&{if} $(\\{token\_len}=\\{sign\_length})$ \1\&{then}\C{there were no digits}\6
$\\{scan\_integer}\K\\{false}$\6
\4\&{else} $\\{scan\_integer}\K\\{true}$;\2\6
\&{end};\par
\fi

\M94.
This function scans over \\{white\_space} characters, stopping either at
the first nonwhite character or the end of the line, respectively
returning \\{true} or \\{false}.

\Y\P$\4\X83:Procedures and functions for input scanning\X\mathrel{+}\S$\6
\4\&{function}\1\  \37\\{scan\_white\_space}: \37\\{boolean};\2\6
\&{begin} \37\C{scan until end-of-line or a nonwhite}\6
\&{while} $((\\{lex\_class}[\\{scan\_char}]=\\{white\_space})\W(\\{buf\_ptr2}<%
\\{last}))$ \1\&{do}\5
$\\{incr}(\\{buf\_ptr2})$;\2\6
\&{if} $(\\{buf\_ptr2}<\\{last})$ \1\&{then}\5
$\\{scan\_white\_space}\K\\{true}$\6
\4\&{else} $\\{scan\_white\_space}\K\\{false}$;\2\6
\&{end};\par
\fi

\M95.
The \\{print\_bad\_input\_line} procedure prints the current input line,
splitting it at the character being scanned: It prints $\\{buffer}[0]$,
$\\{buffer}[1]$, \dots, $\\{buffer}[\\{buf\_ptr2}-1]$ on one line and
$\\{buffer}[\\{buf\_ptr2}]$, \dots, $\\{buffer}[\\{last}-1]$ on the next (and
both
lines start with a colon between two \\{space}s).  Each \\{white\_space}
character is printed as a \\{space}.

\Y\P$\4\X3:Procedures and functions for all file I/O, error messages, and such%
\X\mathrel{+}\S$\6
\4\&{procedure}\1\  \37\\{print\_bad\_input\_line};\6
\4\&{var} \37\\{bf\_ptr}: \37\\{buf\_pointer};\2\6
\&{begin} \37$\\{print}(\.{\'\ :\ \'})$;\5
$\\{bf\_ptr}\K0$;\6
\&{while} $(\\{bf\_ptr}<\\{buf\_ptr2})$ \1\&{do}\6
\&{begin} \37\&{if} $(\\{lex\_class}[\\{buffer}[\\{bf\_ptr}]]=\\{white%
\_space})$ \1\&{then}\5
$\\{print}(\\{xchr}[\\{space}])$\6
\4\&{else} $\\{print}(\\{xchr}[\\{buffer}[\\{bf\_ptr}]])$;\2\6
$\\{incr}(\\{bf\_ptr})$;\6
\&{end};\2\6
\\{print\_newline};\5
$\\{print}(\.{\'\ :\ \'})$;\5
$\\{bf\_ptr}\K0$;\6
\&{while} $(\\{bf\_ptr}<\\{buf\_ptr2})$ \1\&{do}\6
\&{begin} \37$\\{print}(\\{xchr}[\\{space}])$;\5
$\\{incr}(\\{bf\_ptr})$;\6
\&{end};\2\6
$\\{bf\_ptr}\K\\{buf\_ptr2}$;\6
\&{while} $(\\{bf\_ptr}<\\{last})$ \1\&{do}\6
\&{begin} \37\&{if} $(\\{lex\_class}[\\{buffer}[\\{bf\_ptr}]]=\\{white%
\_space})$ \1\&{then}\5
$\\{print}(\\{xchr}[\\{space}])$\6
\4\&{else} $\\{print}(\\{xchr}[\\{buffer}[\\{bf\_ptr}]])$;\2\6
$\\{incr}(\\{bf\_ptr})$;\6
\&{end};\2\6
\\{print\_newline};\6
$\\{bf\_ptr}\K0$;\6
\&{while} $((\\{bf\_ptr}<\\{buf\_ptr2})\W(\\{lex\_class}[\\{buffer}[\\{bf%
\_ptr}]]=\\{white\_space}))$ \1\&{do}\5
$\\{incr}(\\{bf\_ptr})$;\2\6
\&{if} $(\\{bf\_ptr}=\\{buf\_ptr2})$ \1\&{then}\5
$\\{print\_ln}(\.{\'(Error\ may\ have\ been\ on\ previous\ line)\'})$;\2\6
\\{mark\_error};\6
\&{end};\par
\fi

\M96.
This little procedure exists because it's used by at least two other
procedures and thus saves some space.

\Y\P$\4\X3:Procedures and functions for all file I/O, error messages, and such%
\X\mathrel{+}\S$\6
\4\&{procedure}\1\  \37\\{print\_skipping\_whatever\_remains};\2\6
\&{begin} \37$\\{print}(\.{\'I\'}\.{\'m\ skipping\ whatever\ remains\ of\ this\
\'})$;\6
\&{end};\par
\fi

\N97.  Getting the top-level auxiliary file name.
These modules read the name of the top-level \.{.aux} file.  Some
systems will try to find this on the command line; if it's not there
it will come from the user's terminal.  In either case, the name goes
into the \\{char} array \\{name\_of\_file}, and the files relevant to this
name are opened.

\Y\P\D \37$\\{aux\_found}=41$\C{go here when the \.{.aux} name is legit}\par
\P\D \37$\\{aux\_not\_found}=46$\C{go here when it's not}\par
\Y\P$\4\X16:Globals in the outer block\X\mathrel{+}\S$\6
\4\\{aux\_name\_length}: \37$0\to\\{file\_name\_size}+1$;\C{\.{.aux} name sans
extension}\par
\fi

\M98.
I mean, this is truly disgraceful.  A user has to type something in to
the terminal just once during the entire run.  And it's not some
complicated string where you have to get every last punctuation mark
just right, and it's not some fancy list where you get nervous because
if you forget one item you have to type the whole thing again; it's
just a simple, ordinary, file name.  Now you'd think a five-year-old
could do it; you'd think it's so simple a user should be able to do it
in his sleep.  But noooooooooo.  He had to sit there droning on and on
about who knows what until he exceeded the bounds of common sense, and
he probably didn't even realize it.  Just pitiful.  What's this world
coming to?  We should probably just delete all his files and be done
with him.  Note: The \\{term\_out} file is system dependent.

\Y\P\D \37$\\{sam\_you\_made\_the\_file\_name\_too\_long}\S$\1\6
\&{begin} \37\\{sam\_too\_long\_file\_name\_print};\5
\&{goto} \37\\{aux\_not\_found};\6
\&{end}\2\par
\Y\P$\4\X3:Procedures and functions for all file I/O, error messages, and such%
\X\mathrel{+}\S$\6
\4\&{procedure}\1\  \37\\{sam\_too\_long\_file\_name\_print};\2\6
\&{begin} \37$\\{write}(\\{term\_out},\39\.{\'File\ name\ \`\'})$;\5
$\\{name\_ptr}\K1$;\6
\&{while} $(\\{name\_ptr}\L\\{aux\_name\_length})$ \1\&{do}\6
\&{begin} \37$\\{write}(\\{term\_out},\39\\{name\_of\_file}[\\{name\_ptr}])$;\5
$\\{incr}(\\{name\_ptr})$;\6
\&{end};\2\6
$\\{write\_ln}(\\{term\_out},\39\.{\'\'}\.{\'\ is\ too\ long\'})$;\6
\&{end};\par
\fi

\M99.
We've abused the user enough for one section; suffice it to
say here that most of what we said last module still applies.
Note: The \\{term\_out} file is system dependent.

\Y\P\D \37$\\{sam\_you\_made\_the\_file\_name\_wrong}\S$\1\6
\&{begin} \37\\{sam\_wrong\_file\_name\_print};\5
\&{goto} \37\\{aux\_not\_found};\6
\&{end}\2\par
\Y\P$\4\X3:Procedures and functions for all file I/O, error messages, and such%
\X\mathrel{+}\S$\6
\4\&{procedure}\1\  \37\\{sam\_wrong\_file\_name\_print};\2\6
\&{begin} \37$\\{write}(\\{term\_out},\39\.{\'I\ couldn\'}\.{\'t\ open\ file\
name\ \`\'})$;\5
$\\{name\_ptr}\K1$;\6
\&{while} $(\\{name\_ptr}\L\\{name\_length})$ \1\&{do}\6
\&{begin} \37$\\{write}(\\{term\_out},\39\\{name\_of\_file}[\\{name\_ptr}])$;\5
$\\{incr}(\\{name\_ptr})$;\6
\&{end};\2\6
$\\{write\_ln}(\\{term\_out},\39\.{\'\'}\.{\'\'})$;\6
\&{end};\par
\fi

\M100.
This procedure consists of a loop that reads and processes a (nonnull)
\.{.aux} file name.  It's this module and the next two that must be
changed on those systems using command-line arguments.  Note: The
\\{term\_out} and \\{term\_in} files are system dependent.

\Y\P$\4\X100:Procedures and functions for the reading and processing of input
files\X\S$\6
\4\&{procedure}\1\  \37\\{get\_the\_top\_level\_aux\_file\_name};\6
\4\&{label} \37$\\{aux\_found},\39\\{aux\_not\_found}$;\6
\4\&{var} \37\X101:Variables for possible command-line processing\X\2\6
\&{begin} \37$\\{check\_cmnd\_line}\K\\{false}$;\C{many systems will change
this}\6
\~ \1\&{loop}\6
\&{begin} \37\&{if} $(\\{check\_cmnd\_line})$ \1\&{then}\5
\X102:Process a possible command line\X\6
\4\&{else} \&{begin} \37$\\{write}(\\{term\_out},\39\.{\'Please\ type\ input\
file\ name\ (no\ extension)--\'})$;\6
\&{if} $(\\{eoln}(\\{term\_in}))$ \1\&{then}\C{so the first \\{read} works}\6
$\\{read\_ln}(\\{term\_in})$;\2\6
$\\{aux\_name\_length}\K0$;\6
\&{while} $(\R\\{eoln}(\\{term\_in}))$ \1\&{do}\6
\&{begin} \37\&{if} $(\\{aux\_name\_length}=\\{file\_name\_size})$ \1\&{then}\6
\&{begin} \37\&{while} $(\R\\{eoln}(\\{term\_in}))$ \1\&{do}\C{discard the rest
of the line}\6
$\\{get}(\\{term\_in})$;\2\6
\\{sam\_you\_made\_the\_file\_name\_too\_long};\6
\&{end};\2\6
$\\{incr}(\\{aux\_name\_length})$;\5
$\\{name\_of\_file}[\\{aux\_name\_length}]\K\\{term\_in}\^$;\5
$\\{get}(\\{term\_in})$;\6
\&{end};\2\6
\&{end};\2\6
\X103:Handle this \.{.aux} name\X;\6
\4\\{aux\_not\_found}: \37$\\{check\_cmnd\_line}\K\\{false}$;\6
\&{end};\2\6
\4\\{aux\_found}: \37\C{now we're ready to read the \.{.aux} file}\6
\&{end};\par
\As120, 126, 132, 139, 142, 143, 145, 170, 177, 178, 180, 201, 203, 205, 210,
211, 212, 214, 215\ETs217.
\U12.\fi

\M101.
The switch \\{check\_cmnd\_line} tells us whether we're to check for a
possible command-line argument.

\Y\P$\4\X101:Variables for possible command-line processing\X\S$\6
\4\\{check\_cmnd\_line}: \37\\{boolean};\C{\\{true} if we're to check the
command line}\par
\U100.\fi

\M102.
Here's where we do the real command-line work.  Those systems needing
more than a single module to handle the task should add the extras to
the ``System-dependent changes'' section.

\Y\P$\4\X102:Process a possible command line\X\S$\6
\&{begin} \37\\{do\_nothing};\C{the ``default system'' doesn't use the command
line}\6
\&{end}\par
\U100.\fi

\M103.
Here we orchestrate this \.{.aux} name's handling: we add the various
extensions, try to open the files with the resulting name, and
store the name strings we'll need later.

\Y\P$\4\X103:Handle this \.{.aux} name\X\S$\6
\&{begin} \37\&{if} $((\\{aux\_name\_length}+\\{length}(\\{s\_aux\_extension})>%
\\{file\_name\_size})\V\30(\\{aux\_name\_length}+\\{length}(\\{s\_log%
\_extension})>\\{file\_name\_size})\V\30(\\{aux\_name\_length}+\\{length}(\\{s%
\_bbl\_extension})>\\{file\_name\_size}))$ \1\&{then}\5
\\{sam\_you\_made\_the\_file\_name\_too\_long};\2\6
\X106:Add extensions and open files\X;\6
\X107:Put this name into the hash table\X;\6
\&{goto} \37\\{aux\_found};\6
\&{end}\par
\U100.\fi

\M104.
Here we set up definitions and declarations for files opened in this
section.  Each element in \\{aux\_list} (except for
$\\{aux\_list}[\\{aux\_stack\_size}]$, which is always unused) is a pointer to
the appropriate \\{str\_pool} string representing the \.{.aux} file name.
The array \\{aux\_file} contains the corresponding \PASCAL\ \&{file}
variables.

\Y\P\D \37$\\{cur\_aux\_str}\S\\{aux\_list}[\\{aux\_ptr}]$\C{shorthand for the
current \.{.aux} file}\par
\P\D \37$\\{cur\_aux\_file}\S\\{aux\_file}[\\{aux\_ptr}]$\C{shorthand for the
current \\{aux\_file}}\par
\P\D \37$\\{cur\_aux\_line}\S\\{aux\_ln\_stack}[\\{aux\_ptr}]$\C{line number of
current \.{.aux} file}\par
\Y\P$\4\X16:Globals in the outer block\X\mathrel{+}\S$\6
\4\\{aux\_file}: \37\&{array} $[\\{aux\_number}]$ \1\&{of}\5
\\{alpha\_file};\C{open \.{.aux} \&{file}  variables}\2\6
\4\\{aux\_list}: \37\&{array} $[\\{aux\_number}]$ \1\&{of}\5
\\{str\_number};\C{the open \.{.aux} file list}\2\6
\4\\{aux\_ptr}: \37\\{aux\_number};\C{points to the currently open \.{.aux}
file}\6
\4\\{aux\_ln\_stack}: \37\&{array} $[\\{aux\_number}]$ \1\&{of}\5
\\{integer};\C{open \.{.aux} line numbers}\2\6
\4\\{top\_lev\_str}: \37\\{str\_number};\C{the top-level \.{.aux} file's name}\7
\4\\{log\_file}: \37\\{alpha\_file};\C{the \&{file}  variable for the \.{.blg}
file}\6
\4\\{bbl\_file}: \37\\{alpha\_file};\C{the \&{file}  variable for the \.{.bbl}
file}\par
\fi

\M105.
Where \\{aux\_number} is the obvious.

\Y\P$\4\X22:Types in the outer block\X\mathrel{+}\S$\6
$\\{aux\_number}=0\to\\{aux\_stack\_size}$;\C{gives the \\{aux\_list} range}\par
\fi

\M106.
We must make sure the (top-level) \.{.aux}, \.{.blg}, and \.{.bbl}
files can be opened.

\Y\P$\4\X106:Add extensions and open files\X\S$\6
\&{begin} \37$\\{name\_length}\K\\{aux\_name\_length}$;\C{set to last used
position}\6
$\\{add\_extension}(\\{s\_aux\_extension})$;\C{this also sets \\{name\_length}}%
\6
$\\{aux\_ptr}\K0$;\C{initialize the \.{.aux} file stack}\6
\&{if} $(\R\\{a\_open\_in}(\\{cur\_aux\_file}))$ \1\&{then}\5
\\{sam\_you\_made\_the\_file\_name\_wrong};\2\6
$\\{name\_length}\K\\{aux\_name\_length}$;\5
$\\{add\_extension}(\\{s\_log\_extension})$;\C{this also sets \\{name\_length}}%
\6
\&{if} $(\R\\{a\_open\_out}(\\{log\_file}))$ \1\&{then}\5
\\{sam\_you\_made\_the\_file\_name\_wrong};\2\6
$\\{name\_length}\K\\{aux\_name\_length}$;\5
$\\{add\_extension}(\\{s\_bbl\_extension})$;\C{this also sets \\{name\_length}}%
\6
\&{if} $(\R\\{a\_open\_out}(\\{bbl\_file}))$ \1\&{then}\5
\\{sam\_you\_made\_the\_file\_name\_wrong};\2\6
\&{end}\par
\U103.\fi

\M107.
This code puts the \.{.aux} file name, both with and without the
extension, into the hash table, and it initializes \\{aux\_list}.  Note
that all previous top-level \.{.aux}-file stuff must have been
successful.

\Y\P$\4\X107:Put this name into the hash table\X\S$\6
\&{begin} \37$\\{name\_length}\K\\{aux\_name\_length}$;\5
$\\{add\_extension}(\\{s\_aux\_extension})$;\C{this also sets \\{name\_length}}%
\6
$\\{name\_ptr}\K1$;\6
\&{while} $(\\{name\_ptr}\L\\{name\_length})$ \1\&{do}\6
\&{begin} \37$\\{buffer}[\\{name\_ptr}]\K\\{xord}[\\{name\_of\_file}[\\{name%
\_ptr}]]$;\5
$\\{incr}(\\{name\_ptr})$;\6
\&{end};\2\6
$\\{top\_lev\_str}\K\\{hash\_text}[\\{str\_lookup}(\\{buffer},\391,\39\\{aux%
\_name\_length},\39\\{text\_ilk},\39\\{do\_insert})]$;\5
$\\{cur\_aux\_str}\K\\{hash\_text}[\\{str\_lookup}(\\{buffer},\391,\39\\{name%
\_length},\39\\{aux\_file\_ilk},\39\\{do\_insert})]$;\C{note that this has
initialized \\{aux\_list}}\6
\&{if} $(\\{hash\_found})$ \1\&{then}\6
\&{begin} \37\&{trace} \37\\{print\_aux\_name};\6
\&{ecart}\6
$\\{confusion}(\.{\'Already\ encountered\ auxiliary\ file\'})$;\6
\&{end};\2\6
$\\{cur\_aux\_line}\K0$;\C{this finishes initializing the top-level \.{.aux}
file}\6
\&{end}\par
\U103.\fi

\M108.
Print the name of the current \.{.aux} file, followed by a \\{newline}.

\Y\P$\4\X3:Procedures and functions for all file I/O, error messages, and such%
\X\mathrel{+}\S$\6
\4\&{procedure}\1\  \37\\{print\_aux\_name};\2\6
\&{begin} \37$\\{print\_pool\_str}(\\{cur\_aux\_str})$;\5
\\{print\_newline};\6
\&{end};\par
\fi

\N109.  Reading the auxiliary file(s).
Now it's time to read the \.{.aux} file.  The only commands we handle
are \.{\\citation} (there can be arbitrarily many, each having
arbitrarily many arguments), \.{\\bibdata} (there can be just one, but
it can have arbitrarily many arguments), \.{\\bibstyle} (there can be
just one, and it can have just one argument), and \.{\\@input} (there
can be arbitrarily many, each with one argument, and they can be
nested to a depth of \\{aux\_stack\_size}).  Each of these commands is
assumed to be on just a single line.  The rest of the \.{.aux} file is
ignored.

\Y\P\D \37$\\{aux\_done}=31$\C{go here when finished with the \.{.aux} files}%
\par
\Y\P$\4\X109:Labels in the outer block\X\S$\6
$,\39\\{aux\_done}$\par
\A146.
\U10.\fi

\M110.
We keep reading and processing input lines until none left.  This is
part of the main program; hence, because of the \\{aux\_done} label,
there's no conventional  \&{begin} -  \&{end}  pair surrounding the entire
module.

\Y\P$\4\X110:Read the \.{.aux} file\X\S$\6
$\\{print}(\.{\'The\ top-level\ auxiliary\ file:\ \'})$;\5
\\{print\_aux\_name};\6
\~ \1\&{loop}\6
\&{begin} \37\C{\\{pop\_the\_aux\_stack} will exit the loop}\6
$\\{incr}(\\{cur\_aux\_line})$;\6
\&{if} $(\R\\{input\_ln}(\\{cur\_aux\_file}))$ \1\&{then}\C{end of current %
\.{.aux} file}\6
\\{pop\_the\_aux\_stack}\6
\4\&{else} \\{get\_aux\_command\_and\_process};\2\6
\&{end};\2\6
\&{trace} \37$\\{trace\_pr\_ln}(\.{\'Finished\ reading\ the\ auxiliary\ file(s)%
\'})$;\6
\&{ecart}\6
\4\\{aux\_done}: \37\\{last\_check\_for\_aux\_errors};\par
\U10.\fi

\M111.
When we find a bug, we print a message and flush the rest of the line.
This macro must be called from within a procedure that has an \\{exit}
label.

\Y\P\D \37$\\{aux\_err\_return}\S$\1\6
\&{begin} \37\\{aux\_err\_print};\5
\&{return};\C{flush this input line}\6
\&{end}\2\par
\P\D \37$\\{aux\_err}(\#)\S$\1\6
\&{begin} \37$\\{print}(\#)$;\5
\\{aux\_err\_return};\6
\&{end}\2\par
\Y\P$\4\X3:Procedures and functions for all file I/O, error messages, and such%
\X\mathrel{+}\S$\6
\4\&{procedure}\1\  \37\\{aux\_err\_print};\2\6
\&{begin} \37$\\{print}(\.{\'---line\ \'},\39\\{cur\_aux\_line}:0,\39\.{\'\ of\
file\ \'})$;\5
\\{print\_aux\_name};\6
\\{print\_bad\_input\_line};\C{this call does the \\{mark\_error}}\6
\\{print\_skipping\_whatever\_remains};\5
$\\{print\_ln}(\.{\'command\'})$\6
\&{end};\par
\fi

\M112.
Here are a bunch of macros whose print statements are used at least
twice.  Thus we save space by making the statements procedures.  This
macro complains when there's a repeated command that's to be used just
once.

\Y\P\D \37$\\{aux\_err\_illegal\_another}(\#)\S$\1\6
\&{begin} \37$\\{aux\_err\_illegal\_another\_print}(\#)$;\5
\\{aux\_err\_return};\6
\&{end}\2\par
\Y\P$\4\X3:Procedures and functions for all file I/O, error messages, and such%
\X\mathrel{+}\S$\6
\4\&{procedure}\1\  \37$\\{aux\_err\_illegal\_another\_print}(\\{cmd\_num}:%
\\{integer})$;\2\6
\&{begin} \37$\\{print}(\.{\'Illegal,\ another\ \\bib\'})$;\6
\&{case} $(\\{cmd\_num})$ \1\&{of}\6
\4\\{n\_aux\_bibdata}: \37$\\{print}(\.{\'data\'})$;\6
\4\\{n\_aux\_bibstyle}: \37$\\{print}(\.{\'style\'})$;\6
\4\&{othercases} \37$\\{confusion}(\.{\'Illegal\ auxiliary-file\ command\'})$\2%
\6
\&{endcases};\5
$\\{print}(\.{\'\ command\'})$;\6
\&{end};\par
\fi

\M113.
This one complains when a command is missing its \\{right\_brace}.

\Y\P\D \37$\\{aux\_err\_no\_right\_brace}\S$\1\6
\&{begin} \37\\{aux\_err\_no\_right\_brace\_print};\5
\\{aux\_err\_return};\6
\&{end}\2\par
\Y\P$\4\X3:Procedures and functions for all file I/O, error messages, and such%
\X\mathrel{+}\S$\6
\4\&{procedure}\1\  \37\\{aux\_err\_no\_right\_brace\_print};\2\6
\&{begin} \37$\\{print}(\.{\'No\ "\'},\39\\{xchr}[\\{right\_brace}],\39\.{\'"%
\'})$;\6
\&{end};\par
\fi

\M114.
This one complains when a command has stuff after its \\{right\_brace}.

\Y\P\D \37$\\{aux\_err\_stuff\_after\_right\_brace}\S$\1\6
\&{begin} \37\\{aux\_err\_stuff\_after\_right\_brace\_print};\5
\\{aux\_err\_return};\6
\&{end}\2\par
\Y\P$\4\X3:Procedures and functions for all file I/O, error messages, and such%
\X\mathrel{+}\S$\6
\4\&{procedure}\1\  \37\\{aux\_err\_stuff\_after\_right\_brace\_print};\2\6
\&{begin} \37$\\{print}(\.{\'Stuff\ after\ "\'},\39\\{xchr}[\\{right\_brace}],%
\39\.{\'"\'})$;\6
\&{end};\par
\fi

\M115.
And this one complains when a command has \\{white\_space} in its
argument.

\Y\P\D \37$\\{aux\_err\_white\_space\_in\_argument}\S$\1\6
\&{begin} \37\\{aux\_err\_white\_space\_in\_argument\_print};\5
\\{aux\_err\_return};\6
\&{end}\2\par
\Y\P$\4\X3:Procedures and functions for all file I/O, error messages, and such%
\X\mathrel{+}\S$\6
\4\&{procedure}\1\  \37\\{aux\_err\_white\_space\_in\_argument\_print};\2\6
\&{begin} \37$\\{print}(\.{\'White\ space\ in\ argument\'})$;\6
\&{end};\par
\fi

\M116.
We're not at the end of an \.{.aux} file, so we see if the current
line might be a command of interest.  A command of interest will be a
line without blanks, consisting of a command name, a \\{left\_brace}, one
or more arguments separated by commas, and a \\{right\_brace}.

\Y\P$\4\X116:Scan for and process an \.{.aux} command\X\S$\6
\4\&{procedure}\1\  \37\\{get\_aux\_command\_and\_process};\6
\4\&{label} \37\\{exit};\2\6
\&{begin} \37$\\{buf\_ptr2}\K0$;\C{mark the beginning of the next token}\6
\&{if} $(\R\\{scan1}(\\{left\_brace}))$ \1\&{then}\C{no \\{left\_brace}---flush
line}\6
\&{return};\2\6
$\\{command\_num}\K\\{ilk\_info}[\\{str\_lookup}(\\{buffer},\39\\{buf\_ptr1},%
\39\\{token\_len},\39\\{aux\_command\_ilk},\39\\{dont\_insert})]$;\6
\&{if} $(\\{hash\_found})$ \1\&{then}\6
\&{case} $(\\{command\_num})$ \1\&{of}\6
\4\\{n\_aux\_bibdata}: \37\\{aux\_bib\_data\_command};\6
\4\\{n\_aux\_bibstyle}: \37\\{aux\_bib\_style\_command};\6
\4\\{n\_aux\_citation}: \37\\{aux\_citation\_command};\6
\4\\{n\_aux\_input}: \37\\{aux\_input\_command};\6
\4\&{othercases} \37$\\{confusion}(\.{\'Unknown\ auxiliary-file\ command\'})$\2%
\6
\&{endcases};\2\6
\4\\{exit}: \37\&{end};\par
\U143.\fi

\M117.
Here we introduce some variables for processing a \.{\\bibdata}
command.  Each element in \\{bib\_list} (except for
$\\{bib\_list}[\\{max\_bib\_files}]$, which is always unused) is a pointer to
the
appropriate \\{str\_pool} string representing the \.{.bib} file name.
The array \\{bib\_file} contains the corresponding \PASCAL\ \&{file}
variables.

\Y\P\D \37$\\{cur\_bib\_str}\S\\{bib\_list}[\\{bib\_ptr}]$\C{shorthand for
current \.{.bib} file}\par
\P\D \37$\\{cur\_bib\_file}\S\\{bib\_file}[\\{bib\_ptr}]$\C{shorthand for
current \\{bib\_file}}\par
\Y\P$\4\X16:Globals in the outer block\X\mathrel{+}\S$\6
\4\\{bib\_list}: \37\&{array} $[\\{bib\_number}]$ \1\&{of}\5
\\{str\_number};\C{the \.{.bib} file list}\2\6
\4\\{bib\_ptr}: \37\\{bib\_number};\C{pointer for the current \.{.bib} file}\6
\4\\{num\_bib\_files}: \37\\{bib\_number};\C{the total number of \.{.bib}
files}\6
\4\\{bib\_seen}: \37\\{boolean};\C{\\{true} if we've already seen a \.{%
\\bibdata} command}\6
\4\\{bib\_file}: \37\&{array} $[\\{bib\_number}]$ \1\&{of}\5
\\{alpha\_file};\C{corresponding \&{file}  variables}\2\par
\fi

\M118.
Where \\{bib\_number} is the obvious.

\Y\P$\4\X22:Types in the outer block\X\mathrel{+}\S$\6
$\\{bib\_number}=0\to\\{max\_bib\_files}$;\C{gives the \\{bib\_list} range}\par
\fi

\M119.
\Y\P$\4\X20:Set initial values of key variables\X\mathrel{+}\S$\6
$\\{bib\_ptr}\K0$;\C{this makes \\{bib\_list} empty}\6
$\\{bib\_seen}\K\\{false}$;\C{we haven't seen a \.{\\bibdata} command yet}\par
\fi

\M120.
A \.{\\bibdata} command will have its arguments between braces and
separated by commas.  There must be exactly one such command in the
\.{.aux} file(s).  All upper-case letters are converted to lower case.

\Y\P$\4\X100:Procedures and functions for the reading and processing of input
files\X\mathrel{+}\S$\6
\4\&{procedure}\1\  \37\\{aux\_bib\_data\_command};\6
\4\&{label} \37\\{exit};\2\6
\&{begin} \37\&{if} $(\\{bib\_seen})$ \1\&{then}\5
$\\{aux\_err\_illegal\_another}(\\{n\_aux\_bibdata})$;\2\6
$\\{bib\_seen}\K\\{true}$;\C{now we've seen a \.{\\bibdata} command}\6
\&{while} $(\\{scan\_char}\I\\{right\_brace})$ \1\&{do}\6
\&{begin} \37$\\{incr}(\\{buf\_ptr2})$;\C{skip over the previous
stop-character}\6
\&{if} $(\R\\{scan2\_white}(\\{right\_brace},\39\\{comma}))$ \1\&{then}\5
\\{aux\_err\_no\_right\_brace};\2\6
\&{if} $(\\{lex\_class}[\\{scan\_char}]=\\{white\_space})$ \1\&{then}\5
\\{aux\_err\_white\_space\_in\_argument};\2\6
\&{if} $((\\{last}>\\{buf\_ptr2}+1)\W(\\{scan\_char}=\\{right\_brace}))$ \1%
\&{then}\5
\\{aux\_err\_stuff\_after\_right\_brace};\2\6
\X123:Open a \.{.bib} file\X;\6
\&{end};\2\6
\4\\{exit}: \37\&{end};\par
\fi

\M121.
Here's a procedure we'll need shortly.  It prints the name of the
current \.{.bib} file, followed by a \\{newline}.

\Y\P$\4\X3:Procedures and functions for all file I/O, error messages, and such%
\X\mathrel{+}\S$\6
\4\&{procedure}\1\  \37\\{print\_bib\_name};\2\6
\&{begin} \37$\\{print\_pool\_str}(\\{cur\_bib\_str})$;\5
$\\{print\_pool\_str}(\\{s\_bib\_extension})$;\5
\\{print\_newline};\6
\&{end};\par
\fi

\M122.
This macro is similar to \\{aux\_err} but it complains specifically about
opening a file for a \.{\\bibdata} command.

\Y\P\D \37$\\{open\_bibdata\_aux\_err}(\#)\S$\1\6
\&{begin} \37$\\{print}(\#)$;\5
\\{print\_bib\_name};\5
\\{aux\_err\_return};\C{this does the \\{mark\_error}}\6
\&{end}\2\par
\fi

\M123.
Now we add the just-found argument to \\{bib\_list} if it hasn't already
been encountered as a \.{\\bibdata} argument and if, after appending
the \\{s\_bib\_extension} string, the resulting file name can be opened.

\Y\P$\4\X123:Open a \.{.bib} file\X\S$\6
\&{begin} \37\&{if} $(\\{bib\_ptr}=\\{max\_bib\_files})$ \1\&{then}\5
$\\{overflow}(\.{\'number\ of\ database\ files\ \'},\39\\{max\_bib\_files})$;\2%
\6
$\\{cur\_bib\_str}\K\\{hash\_text}[\\{str\_lookup}(\\{buffer},\39\\{buf\_ptr1},%
\39\\{token\_len},\39\\{bib\_file\_ilk},\39\\{do\_insert})]$;\6
\&{if} $(\\{hash\_found})$ \1\&{then}\C{already encountered this as a \.{%
\\bibdata} argument}\6
$\\{open\_bibdata\_aux\_err}(\.{\'This\ database\ file\ appears\ more\ than\
once:\ \'})$;\2\6
$\\{start\_name}(\\{cur\_bib\_str})$;\5
$\\{add\_extension}(\\{s\_bib\_extension})$;\6
\&{if} $(\R\\{a\_open\_in}(\\{cur\_bib\_file}))$ \1\&{then}\6
\&{begin} \37$\\{add\_area}(\\{s\_bib\_area})$;\6
\&{if} $(\R\\{a\_open\_in}(\\{cur\_bib\_file}))$ \1\&{then}\5
$\\{open\_bibdata\_aux\_err}(\.{\'I\ couldn\'}\.{\'t\ open\ database\ file\ %
\'})$;\2\6
\&{end};\2\6
\&{trace} \37$\\{trace\_pr\_pool\_str}(\\{cur\_bib\_str})$;\5
$\\{trace\_pr\_pool\_str}(\\{s\_bib\_extension})$;\5
$\\{trace\_pr\_ln}(\.{\'\ is\ a\ bibdata\ file\'})$;\6
\&{ecart}\6
$\\{incr}(\\{bib\_ptr})$;\6
\&{end}\par
\U120.\fi

\M124.
Here we introduce some variables for processing a \.{\\bibstyle}
command.

\Y\P$\4\X16:Globals in the outer block\X\mathrel{+}\S$\6
\4\\{bst\_seen}: \37\\{boolean};\C{\\{true} if we've already seen a \.{%
\\bibstyle} command}\6
\4\\{bst\_str}: \37\\{str\_number};\C{the string number for the \.{.bst} file}\6
\4\\{bst\_file}: \37\\{alpha\_file};\C{the corresponding \&{file}  variable}\par
\fi

\M125.
And we initialize.

\Y\P$\4\X20:Set initial values of key variables\X\mathrel{+}\S$\6
$\\{bst\_str}\K0$;\C{mark \\{bst\_str} as unused}\6
$\\{bst\_seen}\K\\{false}$;\C{we haven't seen a \.{\\bibstyle} command yet}\par
\fi

\M126.
A \.{\\bibstyle} command will have exactly one argument, and it will
be between braces.  There must be exactly one such command in the
\.{.aux} file(s).  All upper-case letters are converted to lower case.

\Y\P$\4\X100:Procedures and functions for the reading and processing of input
files\X\mathrel{+}\S$\6
\4\&{procedure}\1\  \37\\{aux\_bib\_style\_command};\6
\4\&{label} \37\\{exit};\2\6
\&{begin} \37\&{if} $(\\{bst\_seen})$ \1\&{then}\5
$\\{aux\_err\_illegal\_another}(\\{n\_aux\_bibstyle})$;\2\6
$\\{bst\_seen}\K\\{true}$;\C{now we've seen a \.{\\bibstyle} command}\6
$\\{incr}(\\{buf\_ptr2})$;\C{skip over the \\{left\_brace}}\6
\&{if} $(\R\\{scan1\_white}(\\{right\_brace}))$ \1\&{then}\5
\\{aux\_err\_no\_right\_brace};\2\6
\&{if} $(\\{lex\_class}[\\{scan\_char}]=\\{white\_space})$ \1\&{then}\5
\\{aux\_err\_white\_space\_in\_argument};\2\6
\&{if} $(\\{last}>\\{buf\_ptr2}+1)$ \1\&{then}\5
\\{aux\_err\_stuff\_after\_right\_brace};\2\6
\X127:Open the \.{.bst} file\X;\6
\4\\{exit}: \37\&{end};\par
\fi

\M127.
Now we open the file whose name is the just-found argument appended
with the \\{s\_bst\_extension} string, if possible.

\Y\P$\4\X127:Open the \.{.bst} file\X\S$\6
\&{begin} \37$\\{bst\_str}\K\\{hash\_text}[\\{str\_lookup}(\\{buffer},\39\\{buf%
\_ptr1},\39\\{token\_len},\39\\{bst\_file\_ilk},\39\\{do\_insert})]$;\6
\&{if} $(\\{hash\_found})$ \1\&{then}\6
\&{begin} \37\&{trace} \37\\{print\_bst\_name};\6
\&{ecart}\6
$\\{confusion}(\.{\'Already\ encountered\ style\ file\'})$;\6
\&{end};\2\6
$\\{start\_name}(\\{bst\_str})$;\5
$\\{add\_extension}(\\{s\_bst\_extension})$;\6
\&{if} $(\R\\{a\_open\_in}(\\{bst\_file}))$ \1\&{then}\6
\&{begin} \37$\\{add\_area}(\\{s\_bst\_area})$;\6
\&{if} $(\R\\{a\_open\_in}(\\{bst\_file}))$ \1\&{then}\6
\&{begin} \37$\\{print}(\.{\'I\ couldn\'}\.{\'t\ open\ style\ file\ \'})$;\5
\\{print\_bst\_name};\6
$\\{bst\_str}\K0$;\C{mark as unused again}\6
\\{aux\_err\_return};\6
\&{end};\2\6
\&{end};\2\6
$\\{print}(\.{\'The\ style\ file:\ \'})$;\5
\\{print\_bst\_name};\6
\&{end}\par
\U126.\fi

\M128.
Print the name of the \.{.bst} file, followed by a \\{newline}.

\Y\P$\4\X3:Procedures and functions for all file I/O, error messages, and such%
\X\mathrel{+}\S$\6
\4\&{procedure}\1\  \37\\{print\_bst\_name};\2\6
\&{begin} \37$\\{print\_pool\_str}(\\{bst\_str})$;\5
$\\{print\_pool\_str}(\\{s\_bst\_extension})$;\5
\\{print\_newline};\6
\&{end};\par
\fi

\M129.
Here we introduce some variables for processing a \.{\\citation}
command.  Each element in \\{cite\_list} (except for
$\\{cite\_list}[\\{max\_cites}]$, which is always unused) is a pointer to the
appropriate \\{str\_pool} string.  The cite-key list is kept in order of
occurrence with duplicates removed.

\Y\P\D \37$\\{cur\_cite\_str}\S\\{cite\_list}[\\{cite\_ptr}]$\C{shorthand for
the current cite key}\par
\Y\P$\4\X16:Globals in the outer block\X\mathrel{+}\S$\6
\4\\{cite\_list}: \37\&{packed} \37\&{array} $[\\{cite\_number}]$ \1\&{of}\5
\\{str\_number};\C{the cite-key list}\2\6
\4\\{cite\_ptr}: \37\\{cite\_number};\C{pointer for the current cite key}\6
\4\\{entry\_cite\_ptr}: \37\\{cite\_number};\C{cite pointer for the current
entry}\6
\4\\{num\_cites}: \37\\{cite\_number};\C{the total number of distinct cite
keys}\6
\4\\{old\_num\_cites}: \37\\{cite\_number};\C{set to a previous \\{num\_cites}
value}\6
\4\\{citation\_seen}: \37\\{boolean};\C{\\{true} if we've seen a \.{\\citation}
command}\6
\4\\{cite\_loc}: \37\\{hash\_loc};\C{the hash-table location of a cite key}\6
\4\\{lc\_cite\_loc}: \37\\{hash\_loc};\C{and of its lower-case equivalent}\6
\4\\{lc\_xcite\_loc}: \37\\{hash\_loc};\C{a second \\{lc\_cite\_loc} variable}\6
\4\\{cite\_found}: \37\\{boolean};\C{\\{true} if we've already seen this cite
key}\6
\4\\{all\_entries}: \37\\{boolean};\C{\\{true} if we're to use the entire
database}\6
\4\\{all\_marker}: \37\\{cite\_number};\C{we put the other entries in \\{cite%
\_list} here}\par
\fi

\M130.
Where \\{cite\_number} is the obvious.

\Y\P$\4\X22:Types in the outer block\X\mathrel{+}\S$\6
$\\{cite\_number}=0\to\\{max\_cites}$;\C{gives the \\{cite\_list} range}\par
\fi

\M131.
\Y\P$\4\X20:Set initial values of key variables\X\mathrel{+}\S$\6
$\\{cite\_ptr}\K0$;\C{this makes \\{cite\_list} empty}\6
$\\{citation\_seen}\K\\{false}$;\C{we haven't seen a \.{\\citation} command
yet}\6
$\\{all\_entries}\K\\{false}$;\C{by default, use just the entries explicitly
named}\par
\fi

\M132.
A \.{\\citation} command will have its arguments between braces and
separated by commas.  Upper/lower cases are considered to be different
for \.{\\citation} arguments, which is the same as the rest of \LaTeX\
but different from the rest of \BibTeX.  A cite key needn't exactly
case-match its corresponding database key to work, although two cite
keys that are case-mismatched will produce an error message.
(A {\sl case mismatch\/} is a mismatch, but only because of a case
difference.)

A \.{\\citation} command having \.{*} as an argument indicates that
the entire database will be included (almost as if a \.{\\nocite}
command that listed every cite key in the database, in order, had been
given at the corresponding spot in the \.{.tex} file).

\Y\P\D \37$\\{next\_cite}=23$\C{read the next argument}\par
\Y\P$\4\X100:Procedures and functions for the reading and processing of input
files\X\mathrel{+}\S$\6
\4\&{procedure}\1\  \37\\{aux\_citation\_command};\6
\4\&{label} \37$\\{next\_cite},\39\\{exit}$;\2\6
\&{begin} \37$\\{citation\_seen}\K\\{true}$;\C{now we've seen a \.{\\citation}
command}\6
\&{while} $(\\{scan\_char}\I\\{right\_brace})$ \1\&{do}\6
\&{begin} \37$\\{incr}(\\{buf\_ptr2})$;\C{skip over the previous
stop-character}\6
\&{if} $(\R\\{scan2\_white}(\\{right\_brace},\39\\{comma}))$ \1\&{then}\5
\\{aux\_err\_no\_right\_brace};\2\6
\&{if} $(\\{lex\_class}[\\{scan\_char}]=\\{white\_space})$ \1\&{then}\5
\\{aux\_err\_white\_space\_in\_argument};\2\6
\&{if} $((\\{last}>\\{buf\_ptr2}+1)\W(\\{scan\_char}=\\{right\_brace}))$ \1%
\&{then}\5
\\{aux\_err\_stuff\_after\_right\_brace};\2\6
\X133:Check the cite key\X;\6
\4\\{next\_cite}: \37\&{end};\2\6
\4\\{exit}: \37\&{end};\par
\fi

\M133.
We must check if (the lower-case version of) this cite key has been
previously encountered, and proceed accordingly.  The alias kludge
helps make the stack space not overflow on some machines.

\Y\P\D \37$\\{ex\_buf1}\S\\{ex\_buf}$\C{an alias, used only in this module}\par
\Y\P$\4\X133:Check the cite key\X\S$\6
\&{begin} \37\&{trace} \37\\{trace\_pr\_token};\5
$\\{trace\_pr}(\.{\'\ cite\ key\ encountered\'})$;\6
\&{ecart}\6
\X134:Check for entire database inclusion (and thus skip this cite key)\X;\6
$\\{tmp\_ptr}\K\\{buf\_ptr1}$;\6
\&{while} $(\\{tmp\_ptr}<\\{buf\_ptr2})$ \1\&{do}\6
\&{begin} \37$\\{ex\_buf1}[\\{tmp\_ptr}]\K\\{buffer}[\\{tmp\_ptr}]$;\5
$\\{incr}(\\{tmp\_ptr})$;\6
\&{end};\2\6
$\\{lower\_case}(\\{ex\_buf1},\39\\{buf\_ptr1},\39\\{token\_len})$;\C{convert
to `canonical' form}\6
$\\{lc\_cite\_loc}\K\\{str\_lookup}(\\{ex\_buf1},\39\\{buf\_ptr1},\39\\{token%
\_len},\39\\{lc\_cite\_ilk},\39\\{do\_insert})$;\6
\&{if} $(\\{hash\_found})$ \1\&{then}\C{already encountered this as a \.{%
\\citation} argument}\6
\X135:Cite seen, don't add a cite key\X\6
\4\&{else} \X136:Cite unseen, add a cite key\X;\C{it's a new cite key---add it
to \\{cite\_list}}\2\6
\&{end}\par
\U132.\fi

\M134.
Here we check for a \.{\\citation} command having \.{*} as an
argument, indicating that the entire database will be included.

\Y\P$\4\X134:Check for entire database inclusion (and thus skip this cite key)%
\X\S$\6
\&{begin} \37\&{if} $(\\{token\_len}=1)$ \1\&{then}\6
\&{if} $(\\{buffer}[\\{buf\_ptr1}]=\\{star})$ \1\&{then}\6
\&{begin} \37\&{trace} \37$\\{trace\_pr\_ln}(\.{\'---entire\ database\ to\ be\
included\'})$;\6
\&{ecart}\6
\&{if} $(\\{all\_entries})$ \1\&{then}\6
\&{begin} \37$\\{print\_ln}(\.{\'Multiple\ inclusions\ of\ entire\ database%
\'})$;\5
\\{aux\_err\_return};\6
\&{end}\6
\4\&{else} \&{begin} \37$\\{all\_entries}\K\\{true}$;\5
$\\{all\_marker}\K\\{cite\_ptr}$;\5
\&{goto} \37\\{next\_cite};\6
\&{end};\2\6
\&{end};\2\2\6
\&{end}\par
\U133.\fi

\M135.
We've previously encountered the lower-case version, so we check that
the actual version exactly matches the actual version of the
previously-encountered cite key(s).

\Y\P$\4\X135:Cite seen, don't add a cite key\X\S$\6
\&{begin} \37\&{trace} \37$\\{trace\_pr\_ln}(\.{\'\ previously\'})$;\6
\&{ecart}\6
$\\{dummy\_loc}\K\\{str\_lookup}(\\{buffer},\39\\{buf\_ptr1},\39\\{token\_len},%
\39\\{cite\_ilk},\39\\{dont\_insert})$;\6
\&{if} $(\R\\{hash\_found})$ \1\&{then}\C{case mismatch error}\6
\&{begin} \37$\\{print}(\.{\'Case\ mismatch\ error\ between\ cite\ keys\ \'})$;%
\5
\\{print\_token};\5
$\\{print}(\.{\'\ and\ \'})$;\5
$\\{print\_pool\_str}(\\{cite\_list}[\\{ilk\_info}[\\{ilk\_info}[\\{lc\_cite%
\_loc}]]])$;\5
\\{print\_newline};\5
\\{aux\_err\_return};\6
\&{end};\2\6
\&{end}\par
\U133.\fi

\M136.
Now we add the just-found argument to \\{cite\_list} if there isn't
anything funny happening.

\Y\P$\4\X136:Cite unseen, add a cite key\X\S$\6
\&{begin} \37\&{trace} \37\\{trace\_pr\_newline};\6
\&{ecart}\6
$\\{cite\_loc}\K\\{str\_lookup}(\\{buffer},\39\\{buf\_ptr1},\39\\{token\_len},%
\39\\{cite\_ilk},\39\\{do\_insert})$;\6
\&{if} $(\\{hash\_found})$ \1\&{then}\5
\\{hash\_cite\_confusion};\2\6
$\\{check\_cite\_overflow}(\\{cite\_ptr})$;\5
$\\{cur\_cite\_str}\K\\{hash\_text}[\\{cite\_loc}]$;\5
$\\{ilk\_info}[\\{cite\_loc}]\K\\{cite\_ptr}$;\5
$\\{ilk\_info}[\\{lc\_cite\_loc}]\K\\{cite\_loc}$;\5
$\\{incr}(\\{cite\_ptr})$;\6
\&{end}\par
\U133.\fi

\M137.
Here's a serious complaint (that is, a bug) concerning hash problems.
This is the first of several similar bug-procedures that exist only
because they save space.

\Y\P$\4\X3:Procedures and functions for all file I/O, error messages, and such%
\X\mathrel{+}\S$\6
\4\&{procedure}\1\  \37\\{hash\_cite\_confusion};\2\6
\&{begin} \37$\\{confusion}(\.{\'Cite\ hash\ error\'})$;\6
\&{end};\par
\fi

\M138.
Complain if somebody's got a cite fetish.  This procedure is called
when were about to add another cite key to \\{cite\_list}.  It assumes
that \\{cite\_loc} gives the potential cite key's hash table location.

\Y\P$\4\X3:Procedures and functions for all file I/O, error messages, and such%
\X\mathrel{+}\S$\6
\4\&{procedure}\1\  \37$\\{check\_cite\_overflow}(\\{last\_cite}:\\{cite%
\_number})$;\2\6
\&{begin} \37\&{if} $(\\{last\_cite}=\\{max\_cites})$ \1\&{then}\6
\&{begin} \37$\\{print\_pool\_str}(\\{hash\_text}[\\{cite\_loc}])$;\5
$\\{print\_ln}(\.{\'\ is\ the\ key:\'})$;\5
$\\{overflow}(\.{\'number\ of\ cite\ keys\ \'},\39\\{max\_cites})$;\6
\&{end};\2\6
\&{end};\par
\fi

\M139.
An \.{\\@input} command will have exactly one argument, it will
be between braces, and it must have the \\{s\_aux\_extension}.
All upper-case letters are converted to lower case.

\Y\P$\4\X100:Procedures and functions for the reading and processing of input
files\X\mathrel{+}\S$\6
\4\&{procedure}\1\  \37\\{aux\_input\_command};\6
\4\&{label} \37\\{exit};\6
\4\&{var} \37\\{aux\_extension\_ok}: \37\\{boolean};\C{to check for a correct
file extension}\2\6
\&{begin} \37$\\{incr}(\\{buf\_ptr2})$;\C{skip over the \\{left\_brace}}\6
\&{if} $(\R\\{scan1\_white}(\\{right\_brace}))$ \1\&{then}\5
\\{aux\_err\_no\_right\_brace};\2\6
\&{if} $(\\{lex\_class}[\\{scan\_char}]=\\{white\_space})$ \1\&{then}\5
\\{aux\_err\_white\_space\_in\_argument};\2\6
\&{if} $(\\{last}>\\{buf\_ptr2}+1)$ \1\&{then}\5
\\{aux\_err\_stuff\_after\_right\_brace};\2\6
\X140:Push the \.{.aux} stack\X;\6
\4\\{exit}: \37\&{end};\par
\fi

\M140.
We must check that this potential \.{.aux} file won't overflow the
stack, that it has the correct extension, that we haven't encountered
it before (to prevent, among other things, an infinite loop).

\Y\P$\4\X140:Push the \.{.aux} stack\X\S$\6
\&{begin} \37$\\{incr}(\\{aux\_ptr})$;\6
\&{if} $(\\{aux\_ptr}=\\{aux\_stack\_size})$ \1\&{then}\6
\&{begin} \37\\{print\_token};\5
$\\{print}(\.{\':\ \'})$;\5
$\\{overflow}(\.{\'auxiliary\ file\ depth\ \'},\39\\{aux\_stack\_size})$;\6
\&{end};\2\6
$\\{aux\_extension\_ok}\K\\{true}$;\6
\&{if} $(\\{token\_len}<\\{length}(\\{s\_aux\_extension}))$ \1\&{then}\6
$\\{aux\_extension\_ok}\K\\{false}$\C{else \\{str\_eq\_buf} might bomb the
program}\6
\4\&{else} \&{if} $(\R\\{str\_eq\_buf}(\\{s\_aux\_extension},\39\\{buffer},\39%
\\{buf\_ptr2}-\\{length}(\\{s\_aux\_extension}),\39\\{length}(\\{s\_aux%
\_extension})))$ \1\&{then}\5
$\\{aux\_extension\_ok}\K\\{false}$;\2\2\6
\&{if} $(\R\\{aux\_extension\_ok})$ \1\&{then}\6
\&{begin} \37\\{print\_token};\5
$\\{print}(\.{\'\ has\ a\ wrong\ extension\'})$;\5
$\\{decr}(\\{aux\_ptr})$;\5
\\{aux\_err\_return};\6
\&{end};\2\6
$\\{cur\_aux\_str}\K\\{hash\_text}[\\{str\_lookup}(\\{buffer},\39\\{buf\_ptr1},%
\39\\{token\_len},\39\\{aux\_file\_ilk},\39\\{do\_insert})]$;\6
\&{if} $(\\{hash\_found})$ \1\&{then}\6
\&{begin} \37$\\{print}(\.{\'Already\ encountered\ file\ \'})$;\5
\\{print\_aux\_name};\5
$\\{decr}(\\{aux\_ptr})$;\5
\\{aux\_err\_return};\6
\&{end};\2\6
\X141:Open this \.{.aux} file\X;\6
\&{end}\par
\U139.\fi

\M141.
We check that this \.{.aux} file can actually be opened, and then open it.

\Y\P$\4\X141:Open this \.{.aux} file\X\S$\6
\&{begin} \37$\\{start\_name}(\\{cur\_aux\_str})$;\C{extension already there
for \.{.aux} files}\6
$\\{name\_ptr}\K\\{name\_length}+1$;\6
\&{while} $(\\{name\_ptr}\L\\{file\_name\_size})$ \1\&{do}\C{pad with blanks}\6
\&{begin} \37$\\{name\_of\_file}[\\{name\_ptr}]\K\.{\'\ \'}$;\5
$\\{incr}(\\{name\_ptr})$;\6
\&{end};\2\6
\&{if} $(\R\\{a\_open\_in}(\\{cur\_aux\_file}))$ \1\&{then}\6
\&{begin} \37$\\{print}(\.{\'I\ couldn\'}\.{\'t\ open\ auxiliary\ file\ \'})$;\5
\\{print\_aux\_name};\5
$\\{decr}(\\{aux\_ptr})$;\5
\\{aux\_err\_return};\6
\&{end};\2\6
$\\{print}(\.{\'A\ level-\'},\39\\{aux\_ptr}:0,\39\.{\'\ auxiliary\ file:\ %
\'})$;\5
\\{print\_aux\_name};\5
$\\{cur\_aux\_line}\K0$;\6
\&{end}\par
\U140.\fi

\M142.
Here we close the current-level \.{.aux} file and go back up a level,
if possible, by decrementing \\{aux\_ptr}.

\Y\P$\4\X100:Procedures and functions for the reading and processing of input
files\X\mathrel{+}\S$\6
\4\&{procedure}\1\  \37\\{pop\_the\_aux\_stack};\2\6
\&{begin} \37$\\{a\_close}(\\{cur\_aux\_file})$;\6
\&{if} $(\\{aux\_ptr}=0)$ \1\&{then}\5
\&{goto} \37\\{aux\_done}\6
\4\&{else} $\\{decr}(\\{aux\_ptr})$;\2\6
\&{end};\par
\fi

\M143.
That's it for processing \.{.aux} commands, except for finishing the
procedural gymnastics.

\Y\P$\4\X100:Procedures and functions for the reading and processing of input
files\X\mathrel{+}\S$\6
\X116:Scan for and process an \.{.aux} command\X\par
\fi

\M144.
We must complain if anything's amiss.

\Y\P\D \37$\\{aux\_end\_err}(\#)\S$\1\6
\&{begin} \37\\{aux\_end1\_err\_print};\5
$\\{print}(\#)$;\5
\\{aux\_end2\_err\_print};\6
\&{end}\2\par
\Y\P$\4\X3:Procedures and functions for all file I/O, error messages, and such%
\X\mathrel{+}\S$\6
\4\&{procedure}\1\  \37\\{aux\_end1\_err\_print};\2\6
\&{begin} \37$\\{print}(\.{\'I\ found\ no\ \'})$;\6
\&{end};\7
\4\&{procedure}\1\  \37\\{aux\_end2\_err\_print};\2\6
\&{begin} \37$\\{print}(\.{\'---while\ reading\ file\ \'})$;\5
\\{print\_aux\_name};\5
\\{mark\_error};\6
\&{end};\par
\fi

\M145.
Before proceeding, we see if we have any complaints.

\Y\P$\4\X100:Procedures and functions for the reading and processing of input
files\X\mathrel{+}\S$\6
\4\&{procedure}\1\  \37\\{last\_check\_for\_aux\_errors};\2\6
\&{begin} \37$\\{num\_cites}\K\\{cite\_ptr}$;\C{record the number of distinct
cite keys}\6
$\\{num\_bib\_files}\K\\{bib\_ptr}$;\C{and the number of \.{.bib} files}\6
\&{if} $(\R\\{citation\_seen})$ \1\&{then}\5
$\\{aux\_end\_err}(\.{\'\\citation\ commands\'})$\6
\4\&{else} \&{if} $((\\{num\_cites}=0)\W(\R\\{all\_entries}))$ \1\&{then}\5
$\\{aux\_end\_err}(\.{\'cite\ keys\'})$;\2\2\6
\&{if} $(\R\\{bib\_seen})$ \1\&{then}\5
$\\{aux\_end\_err}(\.{\'\\bibdata\ command\'})$\6
\4\&{else} \&{if} $(\\{num\_bib\_files}=0)$ \1\&{then}\5
$\\{aux\_end\_err}(\.{\'database\ files\'})$;\2\2\6
\&{if} $(\R\\{bst\_seen})$ \1\&{then}\5
$\\{aux\_end\_err}(\.{\'\\bibstyle\ command\'})$\6
\4\&{else} \&{if} $(\\{bst\_str}=0)$ \1\&{then}\5
$\\{aux\_end\_err}(\.{\'style\ file\'})$;\2\2\6
\&{end};\par
\fi

\N146.  Reading the style file.
This part of the program reads the \.{.bst} file, which consists of a
sequence of commands.  Each \.{.bst} command consists of a name (for
which case differences are ignored) followed by zero or more
arguments, each enclosed in braces.

\Y\P\D \37$\\{bst\_done}=32$\C{go here when finished with the \.{.bst} file}\par
\P\D \37$\\{no\_bst\_file}=9932$\C{go here when skipping the \.{.bst} file}\par
\Y\P$\4\X109:Labels in the outer block\X\mathrel{+}\S$\6
$,\39\\{bst\_done},\39\\{no\_bst\_file}$\par
\fi

\M147.
The \\{bbl\_line\_num} gets initialized along with the \\{bst\_line\_num}, so
it's declared here too.

\Y\P$\4\X16:Globals in the outer block\X\mathrel{+}\S$\6
\4\\{bbl\_line\_num}: \37\\{integer};\C{line number of the \.{.bbl} (output)
file}\6
\4\\{bst\_line\_num}: \37\\{integer};\C{line number of the \.{.bst} file}\par
\fi

\M148.
This little procedure exists because it's used by at least two other
procedures and thus saves some space.

\Y\P$\4\X3:Procedures and functions for all file I/O, error messages, and such%
\X\mathrel{+}\S$\6
\4\&{procedure}\1\  \37\\{bst\_ln\_num\_print};\2\6
\&{begin} \37$\\{print}(\.{\'--line\ \'},\39\\{bst\_line\_num}:0,\39\.{\'\ of\
file\ \'})$;\5
\\{print\_bst\_name};\6
\&{end};\par
\fi

\M149.
When there's a serious error parsing the \.{.bst} file, we flush the
rest of the current command; a blank line is assumed to mark the end
of a command (but for the purposes of error recovery only).  Thus,
error recovery will be better if style designers leave blank lines
between \.{.bst} commands.  This macro must be called from within a
procedure that has an \\{exit} label.

\Y\P\D \37$\\{bst\_err\_print\_and\_look\_for\_blank\_line\_return}\S$\1\6
\&{begin} \37\\{bst\_err\_print\_and\_look\_for\_blank\_line};\5
\&{return};\6
\&{end}\2\par
\P\D \37$\\{bst\_err}(\#)\S$\1\6
\&{begin} \37\C{serious error during \.{.bst} parsing}\6
$\\{print}(\#)$;\5
\\{bst\_err\_print\_and\_look\_for\_blank\_line\_return};\6
\&{end}\2\par
\Y\P$\4\X3:Procedures and functions for all file I/O, error messages, and such%
\X\mathrel{+}\S$\6
\4\&{procedure}\1\  \37\\{bst\_err\_print\_and\_look\_for\_blank\_line};\2\6
\&{begin} \37$\\{print}(\.{\'-\'})$;\5
\\{bst\_ln\_num\_print};\5
\\{print\_bad\_input\_line};\C{this call does the \\{mark\_error}}\6
\&{while} $(\\{last}\I0)$ \1\&{do}\C{look for a blank input line}\6
\&{if} $(\R\\{input\_ln}(\\{bst\_file}))$ \1\&{then}\C{or the end of the file}\6
\&{goto} \37\\{bst\_done}\6
\4\&{else} $\\{incr}(\\{bst\_line\_num})$;\2\2\6
$\\{buf\_ptr2}\K\\{last}$;\C{to input the next line}\6
\&{end};\par
\fi

\M150.
When there's a harmless error parsing the \.{.bst} file (harmless
syntactically, at least) we give just a \\{warning\_message}.

\Y\P\D \37$\\{bst\_warn}(\#)\S$\1\6
\&{begin} \37\C{non-serious error during \.{.bst} parsing}\6
$\\{print}(\#)$;\5
\\{bst\_warn\_print};\6
\&{end}\2\par
\Y\P$\4\X3:Procedures and functions for all file I/O, error messages, and such%
\X\mathrel{+}\S$\6
\4\&{procedure}\1\  \37\\{bst\_warn\_print};\2\6
\&{begin} \37\\{bst\_ln\_num\_print};\5
\\{mark\_warning};\6
\&{end};\par
\fi

\M151.
Here's the outer loop for reading the \.{.bst} file---it keeps reading
and processing \.{.bst} commands until none left.  This is part of the
main program; hence, because of the \\{bst\_done} label, there's no
conventional  \&{begin} -  \&{end}  pair surrounding the entire module.

\Y\P$\4\X151:Read and execute the \.{.bst} file\X\S$\6
\&{if} $(\\{bst\_str}=0)$ \1\&{then}\C{there's no \.{.bst} file to read}\6
\&{goto} \37\\{no\_bst\_file};\C{this is a \&{goto}  so that \\{bst\_done} is
not in a block}\2\6
$\\{bst\_line\_num}\K0$;\C{initialize things}\6
$\\{bbl\_line\_num}\K1$;\C{best spot to initialize the output line number}\6
$\\{buf\_ptr2}\K\\{last}$;\C{to get the first input line}\6
\~ \1\&{loop}\6
\&{begin} \37\&{if} $(\R\\{eat\_bst\_white\_space})$ \1\&{then}\C{the end of
the \.{.bst} file}\6
\&{goto} \37\\{bst\_done};\2\6
\\{get\_bst\_command\_and\_process};\6
\&{end};\2\6
\4\\{bst\_done}: \37$\\{a\_close}(\\{bst\_file})$;\6
\4\\{no\_bst\_file}: \37$\\{a\_close}(\\{bbl\_file})$;\par
\U10.\fi

\M152.
This \.{.bst}-specific scanning function skips over \\{white\_space}
characters (and comments) until hitting a nonwhite character or the
end of the file, respectively returning \\{true} or \\{false}.  It also
updates \\{bst\_line\_num}, the line counter.

\Y\P$\4\X83:Procedures and functions for input scanning\X\mathrel{+}\S$\6
\4\&{function}\1\  \37\\{eat\_bst\_white\_space}: \37\\{boolean};\6
\4\&{label} \37\\{exit};\2\6
\&{begin} \37\~ \1\&{loop}\6
\&{begin} \37\&{if} $(\\{scan\_white\_space})$ \1\&{then}\C{hit a nonwhite
character on this line}\6
\&{if} $(\\{scan\_char}\I\\{comment})$ \1\&{then}\C{it's not a comment
character; return}\6
\&{begin} \37$\\{eat\_bst\_white\_space}\K\\{true}$;\5
\&{return};\6
\&{end};\2\2\6
\&{if} $(\R\\{input\_ln}(\\{bst\_file}))$ \1\&{then}\C{end-of-file; return %
\\{false}}\6
\&{begin} \37$\\{eat\_bst\_white\_space}\K\\{false}$;\5
\&{return};\6
\&{end};\2\6
$\\{incr}(\\{bst\_line\_num})$;\5
$\\{buf\_ptr2}\K0$;\6
\&{end};\2\6
\4\\{exit}: \37\&{end};\par
\fi

\M153.
It's often illegal to end a \.{.bst} command in certain places, and
this is where we come to check.

\Y\P\D \37$\\{eat\_bst\_white\_and\_eof\_check}(\#)\S$\1\6
\&{begin} \37\&{if} $(\R\\{eat\_bst\_white\_space})$ \1\&{then}\6
\&{begin} \37\\{eat\_bst\_print};\5
$\\{bst\_err}(\#)$;\6
\&{end};\2\6
\&{end}\2\par
\Y\P$\4\X3:Procedures and functions for all file I/O, error messages, and such%
\X\mathrel{+}\S$\6
\4\&{procedure}\1\  \37\\{eat\_bst\_print};\2\6
\&{begin} \37$\\{print}(\.{\'Illegal\ end\ of\ style\ file\ in\ command:\ %
\'})$;\6
\&{end};\par
\fi

\M154.
We must attend to a few details before getting to work on this
\.{.bst} command.

\Y\P$\4\X154:Scan for and process a \.{.bst} command\X\S$\6
\4\&{procedure}\1\  \37\\{get\_bst\_command\_and\_process};\6
\4\&{label} \37\\{exit};\2\6
\&{begin} \37\&{if} $(\R\\{scan\_alpha})$ \1\&{then}\5
$\\{bst\_err}(\.{\'"\'},\39\\{xchr}[\\{scan\_char}],\39\.{\'"\ can\'}\.{\'t\
start\ a\ style-file\ command\'})$;\2\6
$\\{lower\_case}(\\{buffer},\39\\{buf\_ptr1},\39\\{token\_len})$;\C{ignore case
differences}\6
$\\{command\_num}\K\\{ilk\_info}[\\{str\_lookup}(\\{buffer},\39\\{buf\_ptr1},%
\39\\{token\_len},\39\\{bst\_command\_ilk},\39\\{dont\_insert})]$;\6
\&{if} $(\R\\{hash\_found})$ \1\&{then}\6
\&{begin} \37\\{print\_token};\5
$\\{bst\_err}(\.{\'\ is\ an\ illegal\ style-file\ command\'})$;\6
\&{end};\2\6
\X155:Process the appropriate \.{.bst} command\X;\6
\4\\{exit}: \37\&{end};\par
\U217.\fi

\M155.
Here we determine which \.{.bst} command we're about to process, and
then go to it.

\Y\P$\4\X155:Process the appropriate \.{.bst} command\X\S$\6
\&{case} $(\\{command\_num})$ \1\&{of}\6
\4\\{n\_bst\_entry}: \37\\{bst\_entry\_command};\6
\4\\{n\_bst\_execute}: \37\\{bst\_execute\_command};\6
\4\\{n\_bst\_function}: \37\\{bst\_function\_command};\6
\4\\{n\_bst\_integers}: \37\\{bst\_integers\_command};\6
\4\\{n\_bst\_iterate}: \37\\{bst\_iterate\_command};\6
\4\\{n\_bst\_macro}: \37\\{bst\_macro\_command};\6
\4\\{n\_bst\_read}: \37\\{bst\_read\_command};\6
\4\\{n\_bst\_reverse}: \37\\{bst\_reverse\_command};\6
\4\\{n\_bst\_sort}: \37\\{bst\_sort\_command};\6
\4\\{n\_bst\_strings}: \37\\{bst\_strings\_command};\6
\4\&{othercases} \37$\\{confusion}(\.{\'Unknown\ style-file\ command\'})$\2\6
\&{endcases}\par
\U154.\fi

\M156.
We need data structures for the function definitions, the entry
variables, the global variables, and the actual entries corresponding
to the cite-key list.  First we define the classes of `function's
used.  Functions in all classes are of \\{bst\_fn\_ilk} except for
\\{int\_literal}s, which are of \\{integer\_ilk}; and \\{str\_literal}s, which
are of \\{text\_ilk}.

\Y\P\D \37$\\{built\_in}=0$\C{the `primitive' functions}\par
\P\D \37$\\{wiz\_defined}=1$\C{defined in the \.{.bst} file}\par
\P\D \37$\\{int\_literal}=2$\C{integer `constants'}\par
\P\D \37$\\{str\_literal}=3$\C{string `constants'}\par
\P\D \37$\\{field}=4$\C{things like `author' and `title'}\par
\P\D \37$\\{int\_entry\_var}=5$\C{integer entry variable}\par
\P\D \37$\\{str\_entry\_var}=6$\C{string entry variable}\par
\P\D \37$\\{int\_global\_var}=7$\C{integer global variable}\par
\P\D \37$\\{str\_global\_var}=8$\C{string global variable}\par
\P\D \37$\\{last\_fn\_class}=8$\C{the same number as on the line above}\par
\fi

\M157.
Here's another bug report.

\Y\P$\4\X3:Procedures and functions for all file I/O, error messages, and such%
\X\mathrel{+}\S$\6
\4\&{procedure}\1\  \37\\{unknwn\_function\_class\_confusion};\2\6
\&{begin} \37$\\{confusion}(\.{\'Unknown\ function\ class\'})$;\6
\&{end};\par
\fi

\M158.
Occasionally we'll want to \\{print} the name of one of these function
classes.

\Y\P$\4\X3:Procedures and functions for all file I/O, error messages, and such%
\X\mathrel{+}\S$\6
\4\&{procedure}\1\  \37$\\{print\_fn\_class}(\\{fn\_loc}:\\{hash\_loc})$;\2\6
\&{begin} \37\&{case} $(\\{fn\_type}[\\{fn\_loc}])$ \1\&{of}\6
\4\\{built\_in}: \37$\\{print}(\.{\'built-in\'})$;\6
\4\\{wiz\_defined}: \37$\\{print}(\.{\'wizard-defined\'})$;\6
\4\\{int\_literal}: \37$\\{print}(\.{\'integer-literal\'})$;\6
\4\\{str\_literal}: \37$\\{print}(\.{\'string-literal\'})$;\6
\4\\{field}: \37$\\{print}(\.{\'field\'})$;\6
\4\\{int\_entry\_var}: \37$\\{print}(\.{\'integer-entry-variable\'})$;\6
\4\\{str\_entry\_var}: \37$\\{print}(\.{\'string-entry-variable\'})$;\6
\4\\{int\_global\_var}: \37$\\{print}(\.{\'integer-global-variable\'})$;\6
\4\\{str\_global\_var}: \37$\\{print}(\.{\'string-global-variable\'})$;\6
\4\&{othercases} \37\\{unknwn\_function\_class\_confusion}\2\6
\&{endcases};\6
\&{end};\par
\fi

\M159.
This version is for printing when in  \&{trace}  mode.

\Y\P$\4\X3:Procedures and functions for all file I/O, error messages, and such%
\X\mathrel{+}\S$\6
\&{trace} \37\&{procedure}\1\  \37$\\{trace\_pr\_fn\_class}(\\{fn\_loc}:\\{hash%
\_loc})$;\2\6
\&{begin} \37\&{case} $(\\{fn\_type}[\\{fn\_loc}])$ \1\&{of}\6
\4\\{built\_in}: \37$\\{trace\_pr}(\.{\'built-in\'})$;\6
\4\\{wiz\_defined}: \37$\\{trace\_pr}(\.{\'wizard-defined\'})$;\6
\4\\{int\_literal}: \37$\\{trace\_pr}(\.{\'integer-literal\'})$;\6
\4\\{str\_literal}: \37$\\{trace\_pr}(\.{\'string-literal\'})$;\6
\4\\{field}: \37$\\{trace\_pr}(\.{\'field\'})$;\6
\4\\{int\_entry\_var}: \37$\\{trace\_pr}(\.{\'integer-entry-variable\'})$;\6
\4\\{str\_entry\_var}: \37$\\{trace\_pr}(\.{\'string-entry-variable\'})$;\6
\4\\{int\_global\_var}: \37$\\{trace\_pr}(\.{\'integer-global-variable\'})$;\6
\4\\{str\_global\_var}: \37$\\{trace\_pr}(\.{\'string-global-variable\'})$;\6
\4\&{othercases} \37\\{unknwn\_function\_class\_confusion}\2\6
\&{endcases};\6
\&{end};\6
\&{ecart}\par
\fi

\M160.
Besides the function classes, we have types based on \BibTeX's
capacity limitations and one based on what can go into the array
\\{wiz\_functions} explained below.

\Y\P\D \37$\\{quote\_next\_fn}=\\{hash\_base}-1$\C{special marker used in
defining functions}\par
\P\D \37$\\{end\_of\_def}=\\{hash\_max}+1$\C{another such special marker}\par
\Y\P$\4\X22:Types in the outer block\X\mathrel{+}\S$\6
$\\{fn\_class}=0\to\\{last\_fn\_class}$;\C{the \.{.bst} function classes}\6
$\\{wiz\_fn\_loc}=0\to\\{wiz\_fn\_space}$;\C{\\{wiz\_defined}-function storage
locations}\6
$\\{int\_ent\_loc}=0\to\\{max\_ent\_ints}$;\C{\\{int\_entry\_var} storage
locations}\6
$\\{str\_ent\_loc}=0\to\\{max\_ent\_strs}$;\C{\\{str\_entry\_var} storage
locations}\6
$\\{str\_glob\_loc}=0\to\\{max\_glb\_str\_minus\_1}$;\C{\\{str\_global\_var}
storage locations}\6
$\\{field\_loc}=0\to\\{max\_fields}$;\C{individual field storage locations}\6
$\\{hash\_ptr2}=\\{quote\_next\_fn}\to\\{end\_of\_def}$;\C{a special marker or
a \\{hash\_loc}}\par
\fi

\M161.
We store information about the \.{.bst} functions in arrays the same
size as the hash-table arrays and in locations corresponding to their
hash-table locations.  The two arrays \\{fn\_info} (an alias of
\\{ilk\_info} described earlier) and \\{fn\_type} accomplish this: \\{fn\_type}
specifies one of the above classes, and \\{fn\_info} gives information
dependent on the class.

Six other arrays give the contents of functions: The array
\\{wiz\_functions} holds definitions for \\{wiz\_defined} functions---each
such function consists of a sequence of pointers to hash-table
locations of other functions (with the two special-marker exceptions
above); the array \\{entry\_ints} contains the current values of
\\{int\_entry\_var}s; the array \\{entry\_strs} contains the current values
of \\{str\_entry\_var}s; an element of the array \\{global\_strs} contains
the current value of a \\{str\_global\_var} if the corresponding
\\{glb\_str\_ptr} entry is empty, otherwise the nonempty entry is a
pointer to the string; and the array \\{field\_info}, for each field of
each entry, contains either a pointer to the string or the special
value \\{missing}.

The array \\{global\_strs} isn't packed (that is, it isn't \&{array}  \dots\
\&{of} \&{packed} \&{array}  \dots$\,$) to increase speed on some systems;
however, on systems that are byte-addressable and that have a good
compiler, packing \\{global\_strs} would save lots of space without much
loss of speed.

\Y\P\D \37$\\{fn\_info}\S\\{ilk\_info}$\C{an alias used with functions}\Y\par
\P\D \37$\\{missing}=\\{empty}$\C{a special pointer for missing fields}\par
\Y\P$\4\X16:Globals in the outer block\X\mathrel{+}\S$\6
\4\\{fn\_loc}: \37\\{hash\_loc};\C{the hash-table location of a function}\6
\4\\{wiz\_loc}: \37\\{hash\_loc};\C{the hash-table location of a wizard
function}\6
\4\\{literal\_loc}: \37\\{hash\_loc};\C{the hash-table location of a literal
function}\6
\4\\{macro\_name\_loc}: \37\\{hash\_loc};\C{the hash-table location of a macro
name}\6
\4\\{macro\_def\_loc}: \37\\{hash\_loc};\C{the hash-table location of a macro
definition}\6
\4\\{fn\_type}: \37\&{packed} \37\&{array} $[\\{hash\_loc}]$ \1\&{of}\5
\\{fn\_class};\2\6
\4\\{wiz\_def\_ptr}: \37\\{wiz\_fn\_loc};\C{storage location for the next
wizard function}\6
\4\\{wiz\_fn\_ptr}: \37\\{wiz\_fn\_loc};\C{general \\{wiz\_functions} location}%
\6
\4\\{wiz\_functions}: \37\&{packed} \37\&{array} $[\\{wiz\_fn\_loc}]$ \1\&{of}\5
\\{hash\_ptr2};\2\6
\4\\{int\_ent\_ptr}: \37\\{int\_ent\_loc};\C{general \\{int\_entry\_var}
location}\6
\4\\{entry\_ints}: \37\&{array} $[\\{int\_ent\_loc}]$ \1\&{of}\5
\\{integer};\2\6
\4\\{num\_ent\_ints}: \37\\{int\_ent\_loc};\C{the number of distinct \\{int%
\_entry\_var} names}\6
\4\\{str\_ent\_ptr}: \37\\{str\_ent\_loc};\C{general \\{str\_entry\_var}
location}\6
\4\\{entry\_strs}: \37\&{array} $[\\{str\_ent\_loc}]$ \1\&{of}\5
\&{packed} \37\&{array} $[0\to\\{ent\_str\_size}]$ \1\&{of}\5
\\{ASCII\_code};\2\2\6
\4\\{num\_ent\_strs}: \37\\{str\_ent\_loc};\C{the number of distinct \\{str%
\_entry\_var} names}\6
\4\\{str\_glb\_ptr}: \37$0\to\\{max\_glob\_strs}$;\C{general \\{str\_global%
\_var} location}\6
\4\\{glb\_str\_ptr}: \37\&{array} $[\\{str\_glob\_loc}]$ \1\&{of}\5
\\{str\_number};\2\6
\4\\{global\_strs}: \37\&{array} $[\\{str\_glob\_loc}]$ \1\&{of}\5
\&{array} $[0\to\\{glob\_str\_size}]$ \1\&{of}\5
\\{ASCII\_code};\2\2\6
\4\\{glb\_str\_end}: \37\&{array} $[\\{str\_glob\_loc}]$ \1\&{of}\5
$0\to\\{glob\_str\_size}$;\C{end markers}\2\6
\4\\{num\_glb\_strs}: \37$0\to\\{max\_glob\_strs}$;\C{number of distinct \\{str%
\_global\_var} names}\6
\4\\{field\_ptr}: \37\\{field\_loc};\C{general \\{field\_info} location}\6
\4$\\{field\_parent\_ptr},\39\\{field\_end\_ptr}$: \37\\{field\_loc};\C{two
more for doing cross-refs}\6
\4$\\{cite\_parent\_ptr},\39\\{cite\_xptr}$: \37\\{cite\_number};\C{two others
for doing cross-refs}\6
\4\\{field\_info}: \37\&{packed} \37\&{array} $[\\{field\_loc}]$ \1\&{of}\5
\\{str\_number};\2\6
\4\\{num\_fields}: \37\\{field\_loc};\C{the number of distinct field names}\6
\4\\{num\_pre\_defined\_fields}: \37\\{field\_loc};\C{so far, just one: %
\.{crossref}}\6
\4\\{crossref\_num}: \37\\{field\_loc};\C{the number given to \.{crossref}}\6
\4\\{no\_fields}: \37\\{boolean};\C{used for \\{tr\_print}ing entry
information}\par
\fi

\M162.
Now we initialize storage for the \\{wiz\_defined} functions and we
initialize variables so that the first \\{str\_entry\_var},
\\{int\_entry\_var}, \\{str\_global\_var}, and \\{field} name will be assigned
the number~0.  Note: The variables \\{num\_ent\_strs} and \\{num\_fields}
will also be set when pre-defining strings.

\Y\P$\4\X20:Set initial values of key variables\X\mathrel{+}\S$\6
$\\{wiz\_def\_ptr}\K0$;\5
$\\{num\_ent\_ints}\K0$;\5
$\\{num\_ent\_strs}\K0$;\5
$\\{num\_fields}\K0$;\5
$\\{str\_glb\_ptr}\K0$;\6
\&{while} $(\\{str\_glb\_ptr}<\\{max\_glob\_strs})$ \1\&{do}\C{make \\{str%
\_global\_var}s empty}\6
\&{begin} \37$\\{glb\_str\_ptr}[\\{str\_glb\_ptr}]\K0$;\5
$\\{glb\_str\_end}[\\{str\_glb\_ptr}]\K0$;\5
$\\{incr}(\\{str\_glb\_ptr})$;\6
\&{end};\2\6
$\\{num\_glb\_strs}\K0$;\par
\fi

\N163.  Style-file commands.
There are ten \.{.bst} commands: Five (\.{entry}, \.{function},
\.{integers}, \.{macro}, and \.{strings}) declare and define
functions, one (\.{read}) reads in the \.{.bib}-file entries, and four
(\.{execute}, \.{iterate}, \.{reverse}, and \.{sort})
manipulate the entries and produce output.

The boolean variables \\{entry\_seen} and \\{read\_seen} indicate whether
we've yet encountered an \.{entry} and a \.{read} command.  There must
be exactly one of each of these, and the \.{entry} command, as well as
any \.{macro} command, must precede the \.{read} command.
Furthermore, the \.{read} command must precede the four that
manipulate the entries and produce output.

\Y\P$\4\X16:Globals in the outer block\X\mathrel{+}\S$\6
\4\\{entry\_seen}: \37\\{boolean};\C{\\{true} if we've already seen an %
\.{entry} command}\6
\4\\{read\_seen}: \37\\{boolean};\C{\\{true} if we've already seen a \.{read}
command}\6
\4\\{read\_performed}: \37\\{boolean};\C{\\{true} if we started reading the
database file(s)}\6
\4\\{reading\_completed}: \37\\{boolean};\C{\\{true} if we made it all the way
through}\6
\4\\{read\_completed}: \37\\{boolean};\C{\\{true} if the database info didn't
bomb \BibTeX}\par
\fi

\M164.
And we initialize them.

\Y\P$\4\X20:Set initial values of key variables\X\mathrel{+}\S$\6
$\\{entry\_seen}\K\\{false}$;\5
$\\{read\_seen}\K\\{false}$;\5
$\\{read\_performed}\K\\{false}$;\5
$\\{reading\_completed}\K\\{false}$;\5
$\\{read\_completed}\K\\{false}$;\par
\fi

\M165.
Here's another bug.

\Y\P$\4\X3:Procedures and functions for all file I/O, error messages, and such%
\X\mathrel{+}\S$\6
\4\&{procedure}\1\  \37\\{id\_scanning\_confusion};\2\6
\&{begin} \37$\\{confusion}(\.{\'Identifier\ scanning\ error\'})$;\6
\&{end};\par
\fi

\M166.
This macro is used to scan all \.{.bst} identifiers.  The argument
supplies the \.{.bst} command name.  The associated procedure simply
prints an error message.

\Y\P\D \37$\\{bst\_identifier\_scan}(\#)\S$\1\6
\&{begin} \37$\\{scan\_identifier}(\\{right\_brace},\39\\{comment},\39%
\\{comment})$;\6
\&{if} $((\\{scan\_result}=\\{white\_adjacent})\V(\\{scan\_result}=\\{specified%
\_char\_adjacent}))$ \1\&{then}\5
\\{do\_nothing}\6
\4\&{else} \&{begin} \37\\{bst\_id\_print};\5
$\\{bst\_err}(\#)$;\6
\&{end};\2\6
\&{end}\2\par
\Y\P$\4\X3:Procedures and functions for all file I/O, error messages, and such%
\X\mathrel{+}\S$\6
\4\&{procedure}\1\  \37\\{bst\_id\_print};\2\6
\&{begin} \37\&{if} $(\\{scan\_result}=\\{id\_null})$ \1\&{then}\5
$\\{print}(\.{\'"\'},\39\\{xchr}[\\{scan\_char}],\39\.{\'"\ begins\ identifier,%
\ command:\ \'})$\6
\4\&{else} \&{if} $(\\{scan\_result}=\\{other\_char\_adjacent})$ \1\&{then}\5
$\\{print}(\.{\'"\'},\39\\{xchr}[\\{scan\_char}],\39\.{\'"\ immediately\
follows\ identifier,\ command:\ \'})$\6
\4\&{else} \\{id\_scanning\_confusion};\2\2\6
\&{end};\par
\fi

\M167.
This macro just makes sure we're at a \\{left\_brace}.

\Y\P\D \37$\\{bst\_get\_and\_check\_left\_brace}(\#)\S$\1\6
\&{begin} \37\&{if} $(\\{scan\_char}\I\\{left\_brace})$ \1\&{then}\6
\&{begin} \37\\{bst\_left\_brace\_print};\5
$\\{bst\_err}(\#)$;\6
\&{end};\2\6
$\\{incr}(\\{buf\_ptr2})$;\C{skip over the \\{left\_brace}}\6
\&{end}\2\par
\Y\P$\4\X3:Procedures and functions for all file I/O, error messages, and such%
\X\mathrel{+}\S$\6
\4\&{procedure}\1\  \37\\{bst\_left\_brace\_print};\2\6
\&{begin} \37$\\{print}(\.{\'"\'},\39\\{xchr}[\\{left\_brace}],\39\.{\'"\ is\
missing\ in\ command:\ \'})$;\6
\&{end};\par
\fi

\M168.
And this one, a \\{right\_brace}.

\Y\P\D \37$\\{bst\_get\_and\_check\_right\_brace}(\#)\S$\1\6
\&{begin} \37\&{if} $(\\{scan\_char}\I\\{right\_brace})$ \1\&{then}\6
\&{begin} \37\\{bst\_right\_brace\_print};\5
$\\{bst\_err}(\#)$;\6
\&{end};\2\6
$\\{incr}(\\{buf\_ptr2})$;\C{skip over the \\{right\_brace}}\6
\&{end}\2\par
\Y\P$\4\X3:Procedures and functions for all file I/O, error messages, and such%
\X\mathrel{+}\S$\6
\4\&{procedure}\1\  \37\\{bst\_right\_brace\_print};\2\6
\&{begin} \37$\\{print}(\.{\'"\'},\39\\{xchr}[\\{right\_brace}],\39\.{\'"\ is\
missing\ in\ command:\ \'})$;\6
\&{end};\par
\fi

\M169.
This macro complains if we've already encountered a function to be
inserted into the hash table.

\Y\P\D \37$\\{check\_for\_already\_seen\_function}(\#)\S$\1\6
\&{begin} \37\&{if} $(\\{hash\_found})$ \1\&{then}\C{already encountered this
as a \.{.bst} function}\6
\&{begin} \37$\\{already\_seen\_function\_print}(\#)$;\5
\&{return};\6
\&{end};\2\6
\&{end}\2\par
\Y\P$\4\X3:Procedures and functions for all file I/O, error messages, and such%
\X\mathrel{+}\S$\6
\4\&{procedure}\1\  \37$\\{already\_seen\_function\_print}(\\{seen\_fn\_loc}:%
\\{hash\_loc})$;\6
\4\&{label} \37\\{exit};\C{so the call to \\{bst\_err} works}\2\6
\&{begin} \37$\\{print\_pool\_str}(\\{hash\_text}[\\{seen\_fn\_loc}])$;\5
$\\{print}(\.{\'\ is\ already\ a\ type\ "\'})$;\5
$\\{print\_fn\_class}(\\{seen\_fn\_loc})$;\5
$\\{print\_ln}(\.{\'"\ function\ name\'})$;\5
\\{bst\_err\_print\_and\_look\_for\_blank\_line\_return};\6
\4\\{exit}: \37\&{end};\par
\fi

\M170.
An \.{entry} command has three arguments, each a (possibly empty) list
of function names between braces (the names are separated by one or
more \\{white\_space} characters).  All function names in this and other
commands must be legal \.{.bst} identifiers.  Upper/lower cases are
considered to be the same for function names in these lists---all
upper-case letters are converted to lower case.  These arguments give
lists of \\{field}s, \\{int\_entry\_var}s, and \\{str\_entry\_var}s.

\Y\P$\4\X100:Procedures and functions for the reading and processing of input
files\X\mathrel{+}\S$\6
\4\&{procedure}\1\  \37\\{bst\_entry\_command};\6
\4\&{label} \37\\{exit};\2\6
\&{begin} \37\&{if} $(\\{entry\_seen})$ \1\&{then}\5
$\\{bst\_err}(\.{\'Illegal,\ another\ entry\ command\'})$;\2\6
$\\{entry\_seen}\K\\{true}$;\C{now we've seen an \.{entry} command}\6
$\\{eat\_bst\_white\_and\_eof\_check}(\.{\'entry\'})$;\5
\X171:Scan the list of \\{field}s\X;\6
$\\{eat\_bst\_white\_and\_eof\_check}(\.{\'entry\'})$;\6
\&{if} $(\\{num\_fields}=\\{num\_pre\_defined\_fields})$ \1\&{then}\5
$\\{bst\_warn}(\.{\'Warning--I\ didn\'}\.{\'t\ find\ any\ fields\'})$;\2\6
\X173:Scan the list of \\{int\_entry\_var}s\X;\6
$\\{eat\_bst\_white\_and\_eof\_check}(\.{\'entry\'})$;\5
\X175:Scan the list of \\{str\_entry\_var}s\X;\6
\4\\{exit}: \37\&{end};\par
\fi

\M171.
This module reads a \\{left\_brace}, the list of \\{field}s, and a
\\{right\_brace}.  The \\{field}s are those like `author' and `title.'

\Y\P$\4\X171:Scan the list of \\{field}s\X\S$\6
\&{begin} \37$\\{bst\_get\_and\_check\_left\_brace}(\.{\'entry\'})$;\5
$\\{eat\_bst\_white\_and\_eof\_check}(\.{\'entry\'})$;\6
\&{while} $(\\{scan\_char}\I\\{right\_brace})$ \1\&{do}\6
\&{begin} \37$\\{bst\_identifier\_scan}(\.{\'entry\'})$;\5
\X172:Insert a \\{field} into the hash table\X;\6
$\\{eat\_bst\_white\_and\_eof\_check}(\.{\'entry\'})$;\6
\&{end};\2\6
$\\{incr}(\\{buf\_ptr2})$;\C{skip over the \\{right\_brace}}\6
\&{end}\par
\U170.\fi

\M172.
Here we insert the just found field name into the hash table, record
it as a \\{field}, and assign it a number to be used in indexing into
the \\{field\_info} array.

\Y\P$\4\X172:Insert a \\{field} into the hash table\X\S$\6
\&{begin} \37\&{trace} \37\\{trace\_pr\_token};\5
$\\{trace\_pr\_ln}(\.{\'\ is\ a\ field\'})$;\6
\&{ecart}\6
$\\{lower\_case}(\\{buffer},\39\\{buf\_ptr1},\39\\{token\_len})$;\C{ignore case
differences}\6
$\\{fn\_loc}\K\\{str\_lookup}(\\{buffer},\39\\{buf\_ptr1},\39\\{token\_len},\39%
\\{bst\_fn\_ilk},\39\\{do\_insert})$;\5
$\\{check\_for\_already\_seen\_function}(\\{fn\_loc})$;\5
$\\{fn\_type}[\\{fn\_loc}]\K\\{field}$;\6
$\\{fn\_info}[\\{fn\_loc}]\K\\{num\_fields}$;\C{give this field a number (take
away its name)}\6
$\\{incr}(\\{num\_fields})$;\6
\&{end}\par
\U171.\fi

\M173.
This module reads a \\{left\_brace}, the list of \\{int\_entry\_var}s,
and a \\{right\_brace}.

\Y\P$\4\X173:Scan the list of \\{int\_entry\_var}s\X\S$\6
\&{begin} \37$\\{bst\_get\_and\_check\_left\_brace}(\.{\'entry\'})$;\5
$\\{eat\_bst\_white\_and\_eof\_check}(\.{\'entry\'})$;\6
\&{while} $(\\{scan\_char}\I\\{right\_brace})$ \1\&{do}\6
\&{begin} \37$\\{bst\_identifier\_scan}(\.{\'entry\'})$;\5
\X174:Insert an \\{int\_entry\_var} into the hash table\X;\6
$\\{eat\_bst\_white\_and\_eof\_check}(\.{\'entry\'})$;\6
\&{end};\2\6
$\\{incr}(\\{buf\_ptr2})$;\C{skip over the \\{right\_brace}}\6
\&{end}\par
\U170.\fi

\M174.
Here we insert the just found \\{int\_entry\_var} name into the hash table
and record it as an \\{int\_entry\_var}.  An \\{int\_entry\_var} is one that
the style designer wants a separate copy of for each entry.

\Y\P$\4\X174:Insert an \\{int\_entry\_var} into the hash table\X\S$\6
\&{begin} \37\&{trace} \37\\{trace\_pr\_token};\5
$\\{trace\_pr\_ln}(\.{\'\ is\ an\ integer\ entry-variable\'})$;\6
\&{ecart}\6
$\\{lower\_case}(\\{buffer},\39\\{buf\_ptr1},\39\\{token\_len})$;\C{ignore case
differences}\6
$\\{fn\_loc}\K\\{str\_lookup}(\\{buffer},\39\\{buf\_ptr1},\39\\{token\_len},\39%
\\{bst\_fn\_ilk},\39\\{do\_insert})$;\5
$\\{check\_for\_already\_seen\_function}(\\{fn\_loc})$;\5
$\\{fn\_type}[\\{fn\_loc}]\K\\{int\_entry\_var}$;\6
$\\{fn\_info}[\\{fn\_loc}]\K\\{num\_ent\_ints}$;\C{give this \\{int\_entry%
\_var} a number}\6
$\\{incr}(\\{num\_ent\_ints})$;\6
\&{end}\par
\U173.\fi

\M175.
This module reads a \\{left\_brace}, the list of \\{str\_entry\_var}s, and a
\\{right\_brace}.  A \\{str\_entry\_var} is one that the style designer wants
a separate copy of for each entry.

\Y\P$\4\X175:Scan the list of \\{str\_entry\_var}s\X\S$\6
\&{begin} \37$\\{bst\_get\_and\_check\_left\_brace}(\.{\'entry\'})$;\5
$\\{eat\_bst\_white\_and\_eof\_check}(\.{\'entry\'})$;\6
\&{while} $(\\{scan\_char}\I\\{right\_brace})$ \1\&{do}\6
\&{begin} \37$\\{bst\_identifier\_scan}(\.{\'entry\'})$;\5
\X176:Insert a \\{str\_entry\_var} into the hash table\X;\6
$\\{eat\_bst\_white\_and\_eof\_check}(\.{\'entry\'})$;\6
\&{end};\2\6
$\\{incr}(\\{buf\_ptr2})$;\C{skip over the \\{right\_brace}}\6
\&{end}\par
\U170.\fi

\M176.
Here we insert the just found \\{str\_entry\_var} name into the hash
table, record it as a \\{str\_entry\_var}, and set its pointer into
\\{entry\_strs}.

\Y\P$\4\X176:Insert a \\{str\_entry\_var} into the hash table\X\S$\6
\&{begin} \37\&{trace} \37\\{trace\_pr\_token};\5
$\\{trace\_pr\_ln}(\.{\'\ is\ a\ string\ entry-variable\'})$;\6
\&{ecart}\6
$\\{lower\_case}(\\{buffer},\39\\{buf\_ptr1},\39\\{token\_len})$;\C{ignore case
differences}\6
$\\{fn\_loc}\K\\{str\_lookup}(\\{buffer},\39\\{buf\_ptr1},\39\\{token\_len},\39%
\\{bst\_fn\_ilk},\39\\{do\_insert})$;\5
$\\{check\_for\_already\_seen\_function}(\\{fn\_loc})$;\5
$\\{fn\_type}[\\{fn\_loc}]\K\\{str\_entry\_var}$;\6
$\\{fn\_info}[\\{fn\_loc}]\K\\{num\_ent\_strs}$;\C{give this \\{str\_entry%
\_var} a number}\6
$\\{incr}(\\{num\_ent\_strs})$;\6
\&{end}\par
\U175.\fi

\M177.
A legal argument for an \.{execute}, \.{iterate}, or \.{reverse}
command must exist and be \\{built\_in} or \\{wiz\_defined}.
Here's where we check, returning \\{true} if the argument is illegal.

\Y\P$\4\X100:Procedures and functions for the reading and processing of input
files\X\mathrel{+}\S$\6
\4\&{function}\1\  \37\\{bad\_argument\_token}: \37\\{boolean};\6
\4\&{label} \37\\{exit};\2\6
\&{begin} \37$\\{bad\_argument\_token}\K\\{true}$;\C{now it's easy to exit if
necessary}\6
$\\{lower\_case}(\\{buffer},\39\\{buf\_ptr1},\39\\{token\_len})$;\C{ignore case
differences}\6
$\\{fn\_loc}\K\\{str\_lookup}(\\{buffer},\39\\{buf\_ptr1},\39\\{token\_len},\39%
\\{bst\_fn\_ilk},\39\\{dont\_insert})$;\6
\&{if} $(\R\\{hash\_found})$ \1\&{then}\C{unknown \.{.bst} function}\6
\&{begin} \37\\{print\_token};\5
$\\{bst\_err}(\.{\'\ is\ an\ unknown\ function\'})$;\6
\&{end}\6
\4\&{else} \&{if} $((\\{fn\_type}[\\{fn\_loc}]\I\\{built\_in})\W(\\{fn\_type}[%
\\{fn\_loc}]\I\\{wiz\_defined}))$ \1\&{then}\6
\&{begin} \37\\{print\_token};\5
$\\{print}(\.{\'\ has\ bad\ function\ type\ \'})$;\5
$\\{print\_fn\_class}(\\{fn\_loc})$;\5
\\{bst\_err\_print\_and\_look\_for\_blank\_line\_return};\6
\&{end};\2\2\6
$\\{bad\_argument\_token}\K\\{false}$;\6
\4\\{exit}: \37\&{end};\par
\fi

\M178.
An \.{execute} command has one argument, a single \\{built\_in} or
\\{wiz\_defined} function name between braces.  Upper/lower cases are
considered to be the same---all upper-case letters are converted to
lower case.  Also, we must make sure we've already seen a \.{read}
command.

This module reads a \\{left\_brace}, a single function to be executed,
and a \\{right\_brace}.

\Y\P$\4\X100:Procedures and functions for the reading and processing of input
files\X\mathrel{+}\S$\6
\4\&{procedure}\1\  \37\\{bst\_execute\_command};\6
\4\&{label} \37\\{exit};\2\6
\&{begin} \37\&{if} $(\R\\{read\_seen})$ \1\&{then}\5
$\\{bst\_err}(\.{\'Illegal,\ execute\ command\ before\ read\ command\'})$;\2\6
$\\{eat\_bst\_white\_and\_eof\_check}(\.{\'execute\'})$;\5
$\\{bst\_get\_and\_check\_left\_brace}(\.{\'execute\'})$;\5
$\\{eat\_bst\_white\_and\_eof\_check}(\.{\'execute\'})$;\5
$\\{bst\_identifier\_scan}(\.{\'execute\'})$;\5
\X179:Check the \.{execute}-command argument token\X;\6
$\\{eat\_bst\_white\_and\_eof\_check}(\.{\'execute\'})$;\5
$\\{bst\_get\_and\_check\_right\_brace}(\.{\'execute\'})$;\5
\X296:Perform an \.{execute} command\X;\6
\4\\{exit}: \37\&{end};\par
\fi

\M179.
Before executing the function, we must make sure it's a legal one.  It
must exist and be \\{built\_in} or \\{wiz\_defined}.

\Y\P$\4\X179:Check the \.{execute}-command argument token\X\S$\6
\&{begin} \37\&{trace} \37\\{trace\_pr\_token};\5
$\\{trace\_pr\_ln}(\.{\'\ is\ a\ to\ be\ executed\ function\'})$;\6
\&{ecart}\6
\&{if} $(\\{bad\_argument\_token})$ \1\&{then}\5
\&{return};\2\6
\&{end}\par
\U178.\fi

\M180.
A \.{function} command has two arguments; the first is a
\\{wiz\_defined} function name between braces.  Upper/lower cases are
considered to be the same---all upper-case letters are converted to
lower case.  The second argument defines this function.  It consists
of a sequence of functions, between braces, separated by \\{white\_space}
characters.  Upper/lower cases are considered to be the same for
function names but not for \\{str\_literal}s.

\Y\P$\4\X100:Procedures and functions for the reading and processing of input
files\X\mathrel{+}\S$\6
\4\&{procedure}\1\  \37\\{bst\_function\_command};\6
\4\&{label} \37\\{exit};\2\6
\&{begin} \37$\\{eat\_bst\_white\_and\_eof\_check}(\.{\'function\'})$;\5
\X181:Scan the \\{wiz\_defined} function name\X;\6
$\\{eat\_bst\_white\_and\_eof\_check}(\.{\'function\'})$;\5
$\\{bst\_get\_and\_check\_left\_brace}(\.{\'function\'})$;\5
$\\{scan\_fn\_def}(\\{wiz\_loc})$;\C{this scans the function definition}\6
\4\\{exit}: \37\&{end};\par
\fi

\M181.
This module reads a \\{left\_brace}, a \\{wiz\_defined} function name, and
a \\{right\_brace}.

\Y\P$\4\X181:Scan the \\{wiz\_defined} function name\X\S$\6
\&{begin} \37$\\{bst\_get\_and\_check\_left\_brace}(\.{\'function\'})$;\5
$\\{eat\_bst\_white\_and\_eof\_check}(\.{\'function\'})$;\5
$\\{bst\_identifier\_scan}(\.{\'function\'})$;\5
\X182:Check the \\{wiz\_defined} function name\X;\6
$\\{eat\_bst\_white\_and\_eof\_check}(\.{\'function\'})$;\5
$\\{bst\_get\_and\_check\_right\_brace}(\.{\'function\'})$;\6
\&{end}\par
\U180.\fi

\M182.
The function name must exist and be a new one; we mark it as
\\{wiz\_defined}.  Also, see if it's the default entry-type function.

\Y\P$\4\X182:Check the \\{wiz\_defined} function name\X\S$\6
\&{begin} \37\&{trace} \37\\{trace\_pr\_token};\5
$\\{trace\_pr\_ln}(\.{\'\ is\ a\ wizard-defined\ function\'})$;\6
\&{ecart}\6
$\\{lower\_case}(\\{buffer},\39\\{buf\_ptr1},\39\\{token\_len})$;\C{ignore case
differences}\6
$\\{wiz\_loc}\K\\{str\_lookup}(\\{buffer},\39\\{buf\_ptr1},\39\\{token\_len},%
\39\\{bst\_fn\_ilk},\39\\{do\_insert})$;\5
$\\{check\_for\_already\_seen\_function}(\\{wiz\_loc})$;\5
$\\{fn\_type}[\\{wiz\_loc}]\K\\{wiz\_defined}$;\6
\&{if} $(\\{hash\_text}[\\{wiz\_loc}]=\\{s\_default})$ \1\&{then}\C{we've found
the default entry-type}\6
$\\{b\_default}\K\\{wiz\_loc}$;\C{see the \\{built\_in} functions for \\{b%
\_default}}\2\6
\&{end}\par
\U181.\fi

\M183.
We're about to start scanning tokens in a function definition.  When a
function token is illegal, we skip until it ends; a \\{white\_space}
character, an end-of-line, a \\{right\_brace}, or a \\{comment} marks the
end of the current token.

\Y\P\D \37$\\{next\_token}=25$\C{a bad function token; go read the next one}\Y%
\par
\P\D \37$\\{skip\_token}(\#)\S$\1\6
\&{begin} \37\C{not-so-serious error during \.{.bst} parsing}\6
$\\{print}(\#)$;\5
\\{skip\_token\_print};\C{also, skip to the current token's end}\6
\&{goto} \37\\{next\_token};\6
\&{end}\2\par
\Y\P$\4\X83:Procedures and functions for input scanning\X\mathrel{+}\S$\6
\4\&{procedure}\1\  \37\\{skip\_token\_print};\2\6
\&{begin} \37$\\{print}(\.{\'-\'})$;\5
\\{bst\_ln\_num\_print};\5
\\{mark\_error};\6
\&{if} $(\\{scan2\_white}(\\{right\_brace},\39\\{comment}))$ \1\&{then}\C{ok if
token ends line}\6
\\{do\_nothing};\2\6
\&{end};\par
\fi

\M184.
This macro is similar to the last one but is specifically for
recursion in a \\{wiz\_defined} function, which is illegal; it helps save
space.

\Y\P\D \37$\\{skip\_recursive\_token}\S$\1\6
\&{begin} \37\\{print\_recursion\_illegal};\5
\&{goto} \37\\{next\_token};\6
\&{end}\2\par
\Y\P$\4\X83:Procedures and functions for input scanning\X\mathrel{+}\S$\6
\4\&{procedure}\1\  \37\\{print\_recursion\_illegal};\2\6
\&{begin} \37\&{trace} \37\\{trace\_pr\_newline};\6
\&{ecart}\6
$\\{print\_ln}(\.{\'Curse\ you,\ wizard,\ before\ you\ recurse\ me:\'})$;\5
$\\{print}(\.{\'function\ \'})$;\5
\\{print\_token};\5
$\\{print\_ln}(\.{\'\ is\ illegal\ in\ its\ own\ definition\'})$;\5
$\B\\{print\_recursion\_illegal}$;\5
$\T$\6
\\{skip\_token\_print};\C{also, skip to the current token's end}\6
\&{end};\par
\fi

\M185.
Here's another macro for saving some space when there's a problem with
a token.

\Y\P\D \37$\\{skip\_token\_unknown\_function}\S$\1\6
\&{begin} \37\\{skp\_token\_unknown\_function\_print};\5
\&{goto} \37\\{next\_token};\6
\&{end}\2\par
\Y\P$\4\X83:Procedures and functions for input scanning\X\mathrel{+}\S$\6
\4\&{procedure}\1\  \37\\{skp\_token\_unknown\_function\_print};\2\6
\&{begin} \37\\{print\_token};\5
$\\{print}(\.{\'\ is\ an\ unknown\ function\'})$;\5
\\{skip\_token\_print};\C{also, skip to the current token's end}\6
\&{end};\par
\fi

\M186.
And another.

\Y\P\D \37$\\{skip\_token\_illegal\_stuff\_after\_literal}\S$\1\6
\&{begin} \37\\{skip\_illegal\_stuff\_after\_token\_print};\5
\&{goto} \37\\{next\_token};\6
\&{end}\2\par
\Y\P$\4\X83:Procedures and functions for input scanning\X\mathrel{+}\S$\6
\4\&{procedure}\1\  \37\\{skip\_illegal\_stuff\_after\_token\_print};\2\6
\&{begin} \37$\\{print}(\.{\'"\'},\39\\{xchr}[\\{scan\_char}],\39\.{\'"\ can\'}%
\.{\'t\ follow\ a\ literal\'})$;\5
\\{skip\_token\_print};\C{also, skip to the current token's end}\6
\&{end};\par
\fi

\M187.
This recursive function reads and stores the list of functions
(separated by \\{white\_space} characters or ends-of-line) that define
this new function, and reads a \\{right\_brace}.

\Y\P$\4\X83:Procedures and functions for input scanning\X\mathrel{+}\S$\6
\4\&{procedure}\1\  \37$\\{scan\_fn\_def}(\\{fn\_hash\_loc}:\\{hash\_loc})$;\6
\4\&{label} \37$\\{next\_token},\39\\{exit}$;\6
\4\&{type} \37$\\{fn\_def\_loc}=0\to\\{single\_fn\_space}$;\C{for a single %
\\{wiz\_defined}-function}\6
\4\&{var} \37\\{singl\_function}: \37\&{packed} \37\&{array} $[\\{fn\_def%
\_loc}]$ \1\&{of}\5
\\{hash\_ptr2};\2\6
\\{single\_ptr}: \37\\{fn\_def\_loc};\C{next storage location for this
definition}\6
\\{copy\_ptr}: \37\\{fn\_def\_loc};\C{dummy variable}\6
\\{end\_of\_num}: \37\\{buf\_pointer};\C{the end of an implicit function's
name}\6
\\{impl\_fn\_loc}: \37\\{hash\_loc};\C{an implicit function's hash-table
location}\2\6
\&{begin} \37$\\{eat\_bst\_white\_and\_eof\_check}(\.{\'function\'})$;\5
$\\{single\_ptr}\K0$;\6
\&{while} $(\\{scan\_char}\I\\{right\_brace})$ \1\&{do}\6
\&{begin} \37\X189:Get the next function of the definition\X;\6
\4\\{next\_token}: \37$\\{eat\_bst\_white\_and\_eof\_check}(\.{\'function\'})$;%
\6
\&{end};\2\6
\X200:Complete this function's definition\X;\6
$\\{incr}(\\{buf\_ptr2})$;\C{skip over the \\{right\_brace}}\6
\4\\{exit}: \37\&{end};\par
\fi

\M188.
This macro inserts a hash-table location (or one of the two
special markers \\{quote\_next\_fn} and \\{end\_of\_def}) into the
\\{singl\_function} array, which will later be copied into the
\\{wiz\_functions} array.

\Y\P\D \37$\\{insert\_fn\_loc}(\#)\S$\1\6
\&{begin} \37$\\{singl\_function}[\\{single\_ptr}]\K\#$;\6
\&{if} $(\\{single\_ptr}=\\{single\_fn\_space})$ \1\&{then}\5
\\{singl\_fn\_overflow};\2\6
$\\{incr}(\\{single\_ptr})$;\6
\&{end}\2\par
\Y\P$\4\X3:Procedures and functions for all file I/O, error messages, and such%
\X\mathrel{+}\S$\6
\4\&{procedure}\1\  \37\\{singl\_fn\_overflow};\2\6
\&{begin} \37$\\{overflow}(\.{\'single\ function\ space\ \'},\39\\{single\_fn%
\_space})$;\6
\&{end};\par
\fi

\M189.
There are five possibilities for the first character of the token
representing the next function of the definition: If it's a
\\{number\_sign}, the token is an \\{int\_literal}; if it's a
\\{double\_quote}, the token is a \\{str\_literal}; if it's a
\\{single\_quote}, the token is a quoted function; if it's a
\\{left\_brace}, the token isn't really a token, but rather the start of
another function definition (which will result in a recursive call to
\\{scan\_fn\_def}); if it's anything else, the token is the name of an
already-defined function.  Note: To prevent the wizard from using
recursion, we have to check that neither a quoted function nor an
already-defined-function is actually the currently-being-defined
function (which is stored at \\{wiz\_loc}).

\Y\P$\4\X189:Get the next function of the definition\X\S$\6
\&{case} $(\\{scan\_char})$ \1\&{of}\6
\4\\{number\_sign}: \37\X190:Scan an \\{int\_literal}\X;\6
\4\\{double\_quote}: \37\X191:Scan a \\{str\_literal}\X;\6
\4\\{single\_quote}: \37\X192:Scan a quoted function\X;\6
\4\\{left\_brace}: \37\X194:Start a new function definition\X;\6
\4\&{othercases} \37\X199:Scan an already-defined function\X\2\6
\&{endcases}\par
\U187.\fi

\M190.
An \\{int\_literal} is preceded by a \\{number\_sign}, consists of an
integer (i.e., an optional \\{minus\_sign} followed by one or more
\\{numeric} characters), and is followed either by a \\{white\_space}
character, an end-of-line, or a \\{right\_brace}.  The array \\{fn\_info}
contains the value of the integer for \\{int\_literal}s.

\Y\P$\4\X190:Scan an \\{int\_literal}\X\S$\6
\&{begin} \37$\\{incr}(\\{buf\_ptr2})$;\C{skip over the \\{number\_sign}}\6
\&{if} $(\R\\{scan\_integer})$ \1\&{then}\5
$\\{skip\_token}(\.{\'Illegal\ integer\ in\ integer\ literal\'})$;\2\6
\&{trace} \37$\\{trace\_pr}(\.{\'\#\'})$;\5
\\{trace\_pr\_token};\5
$\\{trace\_pr\_ln}(\.{\'\ is\ an\ integer\ literal\ with\ value\ \'},\39%
\\{token\_value}:0)$;\6
\&{ecart}\6
$\\{literal\_loc}\K\\{str\_lookup}(\\{buffer},\39\\{buf\_ptr1},\39\\{token%
\_len},\39\\{integer\_ilk},\39\\{do\_insert})$;\6
\&{if} $(\R\\{hash\_found})$ \1\&{then}\6
\&{begin} \37$\\{fn\_type}[\\{literal\_loc}]\K\\{int\_literal}$;\C{set the %
\\{fn\_class}}\6
$\\{fn\_info}[\\{literal\_loc}]\K\\{token\_value}$;\C{the value of this
integer}\6
\&{end};\2\6
\&{if} $((\\{lex\_class}[\\{scan\_char}]\I\\{white\_space})\W(\\{buf\_ptr2}<%
\\{last})\W(\\{scan\_char}\I\\{right\_brace})\W\30(\\{scan\_char}\I%
\\{comment}))$ \1\&{then}\5
\\{skip\_token\_illegal\_stuff\_after\_literal};\2\6
$\\{insert\_fn\_loc}(\\{literal\_loc})$;\C{add this function to \\{wiz%
\_functions}}\6
\&{end}\par
\U189.\fi

\M191.
A \\{str\_literal} is preceded by a \\{double\_quote} and consists of all
characters on this line up to the next \\{double\_quote}.  Also, there
must be either a \\{white\_space} character, an end-of-line, a
\\{right\_brace}, or a \\{comment} following (since functions in the
definition must be separated by \\{white\_space}).  The array \\{fn\_info}
contains nothing for \\{str\_literal}s.

\Y\P$\4\X191:Scan a \\{str\_literal}\X\S$\6
\&{begin} \37$\\{incr}(\\{buf\_ptr2})$;\C{skip over the \\{double\_quote}}\6
\&{if} $(\R\\{scan1}(\\{double\_quote}))$ \1\&{then}\5
$\\{skip\_token}(\.{\'No\ \`\'},\39\\{xchr}[\\{double\_quote}],\39\.{\'\'}\.{\'%
\ to\ end\ string\ literal\'})$;\2\6
\&{trace} \37$\\{trace\_pr}(\.{\'"\'})$;\5
\\{trace\_pr\_token};\5
$\\{trace\_pr}(\.{\'"\'})$;\5
$\\{trace\_pr\_ln}(\.{\'\ is\ a\ string\ literal\'})$;\6
\&{ecart}\6
$\\{literal\_loc}\K\\{str\_lookup}(\\{buffer},\39\\{buf\_ptr1},\39\\{token%
\_len},\39\\{text\_ilk},\39\\{do\_insert})$;\6
$\\{fn\_type}[\\{literal\_loc}]\K\\{str\_literal}$;\C{set the \\{fn\_class}}\6
$\\{incr}(\\{buf\_ptr2})$;\C{skip over the \\{double\_quote}}\6
\&{if} $((\\{lex\_class}[\\{scan\_char}]\I\\{white\_space})\W(\\{buf\_ptr2}<%
\\{last})\W(\\{scan\_char}\I\\{right\_brace})\W\30(\\{scan\_char}\I%
\\{comment}))$ \1\&{then}\5
\\{skip\_token\_illegal\_stuff\_after\_literal};\2\6
$\\{insert\_fn\_loc}(\\{literal\_loc})$;\C{add this function to \\{wiz%
\_functions}}\6
\&{end}\par
\U189.\fi

\M192.
A quoted function is preceded by a \\{single\_quote} and consists of all
characters up to the next \\{white\_space} character, end-of-line,
\\{right\_brace}, or \\{comment}.

\Y\P$\4\X192:Scan a quoted function\X\S$\6
\&{begin} \37$\\{incr}(\\{buf\_ptr2})$;\C{skip over the \\{single\_quote}}\6
\&{if} $(\\{scan2\_white}(\\{right\_brace},\39\\{comment}))$ \1\&{then}\C{ok if
token ends line}\6
\\{do\_nothing};\2\6
\&{trace} \37$\\{trace\_pr}(\.{\'\'}\.{\'\'})$;\5
\\{trace\_pr\_token};\5
$\\{trace\_pr}(\.{\'\ is\ a\ quoted\ function\ \'})$;\6
\&{ecart}\6
$\\{lower\_case}(\\{buffer},\39\\{buf\_ptr1},\39\\{token\_len})$;\C{ignore case
differences}\6
$\\{fn\_loc}\K\\{str\_lookup}(\\{buffer},\39\\{buf\_ptr1},\39\\{token\_len},\39%
\\{bst\_fn\_ilk},\39\\{dont\_insert})$;\6
\&{if} $(\R\\{hash\_found})$ \1\&{then}\C{unknown \.{.bst} function}\6
\\{skip\_token\_unknown\_function}\6
\4\&{else} \X193:Check and insert the quoted function\X;\2\6
\&{end}\par
\U189.\fi

\M193.
Here we check that this quoted function is a legal one---the function
name must already exist, but it mustn't be the currently-being-defined
function (which is stored at \\{wiz\_loc}).

\Y\P$\4\X193:Check and insert the quoted function\X\S$\6
\&{begin} \37\&{if} $(\\{fn\_loc}=\\{wiz\_loc})$ \1\&{then}\5
\\{skip\_recursive\_token}\6
\4\&{else} \&{begin} \37\&{trace} \37$\\{trace\_pr}(\.{\'of\ type\ \'})$;\5
$\\{trace\_pr\_fn\_class}(\\{fn\_loc})$;\5
\\{trace\_pr\_newline};\6
\&{ecart}\6
$\\{insert\_fn\_loc}(\\{quote\_next\_fn})$;\C{add special marker together with}%
\6
$\\{insert\_fn\_loc}(\\{fn\_loc})$;\C{this function to \\{wiz\_functions}}\6
\&{end}\2\6
\&{end}\par
\U192.\fi

\M194.
This module marks the implicit function as being quoted, generates a
name, and stores it in the hash table.  This name is strictly internal
to this program, starts with a \\{single\_quote} (since that will make
this function name unique), and ends with the variable \\{impl\_fn\_num}
converted to ASCII.  The alias kludge helps make the stack space not
overflow on some machines.

\Y\P\D \37$\\{ex\_buf2}\S\\{ex\_buf}$\C{an alias, used only in this module}\par
\Y\P$\4\X194:Start a new function definition\X\S$\6
\&{begin} \37$\\{ex\_buf2}[0]\K\\{single\_quote}$;\5
$\\{int\_to\_ASCII}(\\{impl\_fn\_num},\39\\{ex\_buf2},\391,\39\\{end\_of%
\_num})$;\5
$\\{impl\_fn\_loc}\K\\{str\_lookup}(\\{ex\_buf2},\390,\39\\{end\_of\_num},\39%
\\{bst\_fn\_ilk},\39\\{do\_insert})$;\6
\&{if} $(\\{hash\_found})$ \1\&{then}\5
$\\{confusion}(\.{\'Already\ encountered\ implicit\ function\'})$;\2\6
\&{trace} \37$\\{trace\_pr\_pool\_str}(\\{hash\_text}[\\{impl\_fn\_loc}])$;\5
$\\{trace\_pr\_ln}(\.{\'\ is\ an\ implicit\ function\'})$;\6
\&{ecart}\6
$\\{incr}(\\{impl\_fn\_num})$;\5
$\\{fn\_type}[\\{impl\_fn\_loc}]\K\\{wiz\_defined}$;\6
$\\{insert\_fn\_loc}(\\{quote\_next\_fn})$;\C{all implicit functions are
quoted}\6
$\\{insert\_fn\_loc}(\\{impl\_fn\_loc})$;\C{add it to \\{wiz\_functions}}\6
$\\{incr}(\\{buf\_ptr2})$;\C{skip over the \\{left\_brace}}\6
$\\{scan\_fn\_def}(\\{impl\_fn\_loc})$;\C{this is the recursive call}\6
\&{end}\par
\U189.\fi

\M195.
The variable \\{impl\_fn\_num} counts the number of implicit functions
seen in the \.{.bst} file.

\Y\P$\4\X16:Globals in the outer block\X\mathrel{+}\S$\6
\4\\{impl\_fn\_num}: \37\\{integer};\C{the number of implicit functions seen so
far}\par
\fi

\M196.
Now we initialize it.

\Y\P$\4\X20:Set initial values of key variables\X\mathrel{+}\S$\6
$\\{impl\_fn\_num}\K0$;\par
\fi

\M197.
This module appends a character to \\{int\_buf} after checking to make
sure it will fit; for use in \\{int\_to\_ASCII}.

\Y\P\D \37$\\{append\_int\_char}(\#)\S$\1\6
\&{begin} \37\&{if} $(\\{int\_ptr}=\\{buf\_size})$ \1\&{then}\5
\\{buffer\_overflow};\2\6
$\\{int\_buf}[\\{int\_ptr}]\K\#$;\5
$\\{incr}(\\{int\_ptr})$;\6
\&{end}\2\par
\fi

\M198.
This procedure takes the integer \\{int}, copies the appropriate
\\{ASCII\_code} string into \\{int\_buf} starting at \\{int\_begin}, and sets
the  \&{var}  parameter \\{int\_end} to the first unused \\{int\_buf} location.
The ASCII string will consist of decimal digits, the first of which
will be not be a~0 if the integer is nonzero, with a prepended minus
sign if the integer is negative.

\Y\P$\4\X54:Procedures and functions for handling numbers, characters, and
strings\X\mathrel{+}\S$\6
\4\&{procedure}\1\  \37$\\{int\_to\_ASCII}(\\{int}:\\{integer};\,\35\mathop{%
\&{var}}\\{int\_buf}:\\{buf\_type};\,\35\\{int\_begin}:\\{buf\_pointer};\,\35%
\mathop{\&{var}}\\{int\_end}:\\{buf\_pointer})$;\6
\4\&{var} \37$\\{int\_ptr},\39\\{int\_xptr}$: \37\\{buf\_pointer};\C{pointers
into \\{int\_buf}}\6
\\{int\_tmp\_val}: \37\\{ASCII\_code};\C{the temporary element in an exchange}%
\2\6
\&{begin} \37$\\{int\_ptr}\K\\{int\_begin}$;\6
\&{if} $(\\{int}<0)$ \1\&{then}\C{add the \\{minus\_sign} and use the absolute
value}\6
\&{begin} \37$\\{append\_int\_char}(\\{minus\_sign})$;\5
$\\{int}\K-\\{int}$;\6
\&{end};\2\6
$\\{int\_xptr}\K\\{int\_ptr}$;\6
\1\&{repeat} \37\C{copy digits into \\{int\_buf}}\6
$\\{append\_int\_char}(\.{"0"}+(\\{int}\mathbin{\&{mod}}10))$;\5
$\\{int}\K\\{int}\mathbin{\&{div}}10$;\6
\4\&{until}\5
$(\\{int}=0)$;\2\6
$\\{int\_end}\K\\{int\_ptr}$;\C{set the string length}\6
$\\{decr}(\\{int\_ptr})$;\6
\&{while} $(\\{int\_xptr}<\\{int\_ptr})$ \1\&{do}\C{and reorder (flip) the
digits}\6
\&{begin} \37$\\{int\_tmp\_val}\K\\{int\_buf}[\\{int\_xptr}]$;\5
$\\{int\_buf}[\\{int\_xptr}]\K\\{int\_buf}[\\{int\_ptr}]$;\5
$\\{int\_buf}[\\{int\_ptr}]\K\\{int\_tmp\_val}$;\5
$\\{decr}(\\{int\_ptr})$;\5
$\\{incr}(\\{int\_xptr})$;\6
\&{end}\2\6
\&{end};\par
\fi

\M199.
An already-defined function consists of all characters up to the next
\\{white\_space} character, end-of-line, \\{right\_brace}, or \\{comment}.
This function name must already exist, but it mustn't be the
currently-being-defined function (which is stored at \\{wiz\_loc}).

\Y\P$\4\X199:Scan an already-defined function\X\S$\6
\&{begin} \37\&{if} $(\\{scan2\_white}(\\{right\_brace},\39\\{comment}))$ \1%
\&{then}\C{ok if token ends line}\6
\\{do\_nothing};\2\6
\&{trace} \37\\{trace\_pr\_token};\5
$\\{trace\_pr}(\.{\'\ is\ a\ function\ \'})$;\6
\&{ecart}\6
$\\{lower\_case}(\\{buffer},\39\\{buf\_ptr1},\39\\{token\_len})$;\C{ignore case
differences}\6
$\\{fn\_loc}\K\\{str\_lookup}(\\{buffer},\39\\{buf\_ptr1},\39\\{token\_len},\39%
\\{bst\_fn\_ilk},\39\\{dont\_insert})$;\6
\&{if} $(\R\\{hash\_found})$ \1\&{then}\C{unknown \.{.bst} function}\6
\\{skip\_token\_unknown\_function}\6
\4\&{else} \&{if} $(\\{fn\_loc}=\\{wiz\_loc})$ \1\&{then}\5
\\{skip\_recursive\_token}\6
\4\&{else} \&{begin} \37\&{trace} \37$\\{trace\_pr}(\.{\'of\ type\ \'})$;\5
$\\{trace\_pr\_fn\_class}(\\{fn\_loc})$;\5
\\{trace\_pr\_newline};\6
\&{ecart}\6
$\\{insert\_fn\_loc}(\\{fn\_loc})$;\C{add this function to \\{wiz\_functions}}\6
\&{end};\2\2\6
\&{end}\par
\U189.\fi

\M200.
Now we add the \\{end\_of\_def} special marker, make sure this function will
fit into \\{wiz\_functions}, and put it there.

\Y\P$\4\X200:Complete this function's definition\X\S$\6
\&{begin} \37$\\{insert\_fn\_loc}(\\{end\_of\_def})$;\C{add special marker
ending the definition}\6
\&{if} $(\\{single\_ptr}+\\{wiz\_def\_ptr}>\\{wiz\_fn\_space})$ \1\&{then}\6
\&{begin} \37$\\{print}(\\{single\_ptr}+\\{wiz\_def\_ptr}:0,\39\.{\':\ \'})$;\5
$\\{overflow}(\.{\'wizard-defined\ function\ space\ \'},\39\\{wiz\_fn%
\_space})$;\6
\&{end};\2\6
$\\{fn\_info}[\\{fn\_hash\_loc}]\K\\{wiz\_def\_ptr}$;\C{pointer into \\{wiz%
\_functions}}\6
$\\{copy\_ptr}\K0$;\6
\&{while} $(\\{copy\_ptr}<\\{single\_ptr})$ \1\&{do}\C{make this function
official}\6
\&{begin} \37$\\{wiz\_functions}[\\{wiz\_def\_ptr}]\K\\{singl\_function}[%
\\{copy\_ptr}]$;\5
$\\{incr}(\\{copy\_ptr})$;\5
$\\{incr}(\\{wiz\_def\_ptr})$;\6
\&{end};\2\6
\&{end}\par
\U187.\fi

\M201.
An \.{integers} command has one argument, a list of function names
between braces (the names are separated by one or more \\{white\_space}
characters).  Upper/lower cases are considered to be the same for
function names in these lists---all upper-case letters are converted to
lower case.  Each name in this list specifies an \\{int\_global\_var}.
There may be several \.{integers} commands in the \.{.bst} file.

This module reads a \\{left\_brace}, a list of \\{int\_global\_var}s, and a
\\{right\_brace}.

\Y\P$\4\X100:Procedures and functions for the reading and processing of input
files\X\mathrel{+}\S$\6
\4\&{procedure}\1\  \37\\{bst\_integers\_command};\6
\4\&{label} \37\\{exit};\2\6
\&{begin} \37$\\{eat\_bst\_white\_and\_eof\_check}(\.{\'integers\'})$;\5
$\\{bst\_get\_and\_check\_left\_brace}(\.{\'integers\'})$;\5
$\\{eat\_bst\_white\_and\_eof\_check}(\.{\'integers\'})$;\6
\&{while} $(\\{scan\_char}\I\\{right\_brace})$ \1\&{do}\6
\&{begin} \37$\\{bst\_identifier\_scan}(\.{\'integers\'})$;\5
\X202:Insert an \\{int\_global\_var} into the hash table\X;\6
$\\{eat\_bst\_white\_and\_eof\_check}(\.{\'integers\'})$;\6
\&{end};\2\6
$\\{incr}(\\{buf\_ptr2})$;\C{skip over the \\{right\_brace}}\6
\4\\{exit}: \37\&{end};\par
\fi

\M202.
Here we insert the just found \\{int\_global\_var} name into the hash
table and record it as an \\{int\_global\_var}.  Also, we initialize it by
setting $\\{fn\_info}[\\{fn\_loc}]$ to 0.

\Y\P$\4\X202:Insert an \\{int\_global\_var} into the hash table\X\S$\6
\&{begin} \37\&{trace} \37\\{trace\_pr\_token};\5
$\\{trace\_pr\_ln}(\.{\'\ is\ an\ integer\ global-variable\'})$;\6
\&{ecart}\6
$\\{lower\_case}(\\{buffer},\39\\{buf\_ptr1},\39\\{token\_len})$;\C{ignore case
differences}\6
$\\{fn\_loc}\K\\{str\_lookup}(\\{buffer},\39\\{buf\_ptr1},\39\\{token\_len},\39%
\\{bst\_fn\_ilk},\39\\{do\_insert})$;\5
$\\{check\_for\_already\_seen\_function}(\\{fn\_loc})$;\5
$\\{fn\_type}[\\{fn\_loc}]\K\\{int\_global\_var}$;\6
$\\{fn\_info}[\\{fn\_loc}]\K0$;\C{initialize}\6
\&{end}\par
\U201.\fi

\M203.
An \.{iterate} command has one argument, a single \\{built\_in} or
\\{wiz\_defined} function name between braces.  Upper/lower cases are
considered to be the same---all upper-case letters are converted to
lower case.  Also, we must make sure we've already seen a \.{read}
command.

This module reads a \\{left\_brace}, a single function to be iterated,
and a \\{right\_brace}.

\Y\P$\4\X100:Procedures and functions for the reading and processing of input
files\X\mathrel{+}\S$\6
\4\&{procedure}\1\  \37\\{bst\_iterate\_command};\6
\4\&{label} \37\\{exit};\2\6
\&{begin} \37\&{if} $(\R\\{read\_seen})$ \1\&{then}\5
$\\{bst\_err}(\.{\'Illegal,\ iterate\ command\ before\ read\ command\'})$;\2\6
$\\{eat\_bst\_white\_and\_eof\_check}(\.{\'iterate\'})$;\5
$\\{bst\_get\_and\_check\_left\_brace}(\.{\'iterate\'})$;\5
$\\{eat\_bst\_white\_and\_eof\_check}(\.{\'iterate\'})$;\5
$\\{bst\_identifier\_scan}(\.{\'iterate\'})$;\5
\X204:Check the \.{iterate}-command argument token\X;\6
$\\{eat\_bst\_white\_and\_eof\_check}(\.{\'iterate\'})$;\5
$\\{bst\_get\_and\_check\_right\_brace}(\.{\'iterate\'})$;\5
\X297:Perform an \.{iterate} command\X;\6
\4\\{exit}: \37\&{end};\par
\fi

\M204.
Before iterating the function, we must make sure it's a legal one.  It
must exist and be \\{built\_in} or \\{wiz\_defined}.

\Y\P$\4\X204:Check the \.{iterate}-command argument token\X\S$\6
\&{begin} \37\&{trace} \37\\{trace\_pr\_token};\5
$\\{trace\_pr\_ln}(\.{\'\ is\ a\ to\ be\ iterated\ function\'})$;\6
\&{ecart}\6
\&{if} $(\\{bad\_argument\_token})$ \1\&{then}\5
\&{return};\2\6
\&{end}\par
\U203.\fi

\M205.
A \.{macro} command, like a \.{function} command, has two arguments;
the first is a macro name between braces.  The name must be a legal
\.{.bst} identifier.  Upper/lower cases are considered to be the
same---all upper-case letters are converted to lower case.  The second
argument defines this macro.  It consists of a
\\{double\_quote}-delimited string (which must be on a single line)
between braces, with optional \\{white\_space} characters between the
braces and the \\{double\_quote}s.  This \\{double\_quote}-delimited string
is parsed exactly as a \\{str\_literal} is for the \.{function} command.

\Y\P$\4\X100:Procedures and functions for the reading and processing of input
files\X\mathrel{+}\S$\6
\4\&{procedure}\1\  \37\\{bst\_macro\_command};\6
\4\&{label} \37\\{exit};\2\6
\&{begin} \37\&{if} $(\\{read\_seen})$ \1\&{then}\5
$\\{bst\_err}(\.{\'Illegal,\ macro\ command\ after\ read\ command\'})$;\2\6
$\\{eat\_bst\_white\_and\_eof\_check}(\.{\'macro\'})$;\5
\X206:Scan the macro name\X;\6
$\\{eat\_bst\_white\_and\_eof\_check}(\.{\'macro\'})$;\5
\X208:Scan the macro's definition\X;\6
\4\\{exit}: \37\&{end};\par
\fi

\M206.
This module reads a \\{left\_brace}, a macro name, and a \\{right\_brace}.

\Y\P$\4\X206:Scan the macro name\X\S$\6
\&{begin} \37$\\{bst\_get\_and\_check\_left\_brace}(\.{\'macro\'})$;\5
$\\{eat\_bst\_white\_and\_eof\_check}(\.{\'macro\'})$;\5
$\\{bst\_identifier\_scan}(\.{\'macro\'})$;\5
\X207:Check the macro name\X;\6
$\\{eat\_bst\_white\_and\_eof\_check}(\.{\'macro\'})$;\5
$\\{bst\_get\_and\_check\_right\_brace}(\.{\'macro\'})$;\6
\&{end}\par
\U205.\fi

\M207.
The macro name must be a new one; we mark it as \\{macro\_ilk}.

\Y\P$\4\X207:Check the macro name\X\S$\6
\&{begin} \37\&{trace} \37\\{trace\_pr\_token};\5
$\\{trace\_pr\_ln}(\.{\'\ is\ a\ macro\'})$;\6
\&{ecart}\6
$\\{lower\_case}(\\{buffer},\39\\{buf\_ptr1},\39\\{token\_len})$;\C{ignore case
differences}\6
$\\{macro\_name\_loc}\K\\{str\_lookup}(\\{buffer},\39\\{buf\_ptr1},\39\\{token%
\_len},\39\\{macro\_ilk},\39\\{do\_insert})$;\6
\&{if} $(\\{hash\_found})$ \1\&{then}\6
\&{begin} \37\\{print\_token};\5
$\\{bst\_err}(\.{\'\ is\ already\ defined\ as\ a\ macro\'})$;\6
\&{end};\2\6
$\\{ilk\_info}[\\{macro\_name\_loc}]\K\\{hash\_text}[\\{macro\_name\_loc}]$;%
\C{default in case of error}\6
\&{end}\par
\U206.\fi

\M208.
This module reads a \\{left\_brace}, the \\{double\_quote}-delimited string
that defines this macro, and a \\{right\_brace}.

\Y\P$\4\X208:Scan the macro's definition\X\S$\6
\&{begin} \37$\\{bst\_get\_and\_check\_left\_brace}(\.{\'macro\'})$;\5
$\\{eat\_bst\_white\_and\_eof\_check}(\.{\'macro\'})$;\6
\&{if} $(\\{scan\_char}\I\\{double\_quote})$ \1\&{then}\5
$\\{bst\_err}(\.{\'A\ macro\ definition\ must\ be\ \'},\39\\{xchr}[\\{double%
\_quote}],\39\.{\'-delimited\'})$;\2\6
\X209:Scan the macro definition-string\X;\6
$\\{eat\_bst\_white\_and\_eof\_check}(\.{\'macro\'})$;\5
$\\{bst\_get\_and\_check\_right\_brace}(\.{\'macro\'})$;\6
\&{end}\par
\U205.\fi

\M209.
A macro definition-string is preceded by a \\{double\_quote} and consists
of all characters on this line up to the next \\{double\_quote}.  The
array \\{ilk\_info} contains a pointer to this string for the macro name.

\Y\P$\4\X209:Scan the macro definition-string\X\S$\6
\&{begin} \37$\\{incr}(\\{buf\_ptr2})$;\C{skip over the \\{double\_quote}}\6
\&{if} $(\R\\{scan1}(\\{double\_quote}))$ \1\&{then}\5
$\\{bst\_err}(\.{\'There\'}\.{\'s\ no\ \`\'},\39\\{xchr}[\\{double\_quote}],\39%
\.{\'\'}\.{\'\ to\ end\ macro\ definition\'})$;\2\6
\&{trace} \37$\\{trace\_pr}(\.{\'"\'})$;\5
\\{trace\_pr\_token};\5
$\\{trace\_pr}(\.{\'"\'})$;\5
$\\{trace\_pr\_ln}(\.{\'\ is\ a\ macro\ string\'})$;\6
\&{ecart}\6
$\\{macro\_def\_loc}\K\\{str\_lookup}(\\{buffer},\39\\{buf\_ptr1},\39\\{token%
\_len},\39\\{text\_ilk},\39\\{do\_insert})$;\6
$\\{fn\_type}[\\{macro\_def\_loc}]\K\\{str\_literal}$;\C{set the \\{fn\_class}}%
\6
$\\{ilk\_info}[\\{macro\_name\_loc}]\K\\{hash\_text}[\\{macro\_def\_loc}]$;\5
$\\{incr}(\\{buf\_ptr2})$;\C{skip over the \\{double\_quote}}\6
\&{end}\par
\U208.\fi

\M210.
We need to include stuff for \.{.bib} reading here because that's done
by the \.{read} command.

\Y\P$\4\X100:Procedures and functions for the reading and processing of input
files\X\mathrel{+}\S$\6
\X236:Scan for and process a \.{.bib} command or database entry\X\par
\fi

\M211.
The \.{read} command has no arguments so there's no more parsing to
do.  We must make sure we haven't seen a \.{read} command before and
we've already seen an \.{entry} command.

\Y\P$\4\X100:Procedures and functions for the reading and processing of input
files\X\mathrel{+}\S$\6
\4\&{procedure}\1\  \37\\{bst\_read\_command};\6
\4\&{label} \37\\{exit};\2\6
\&{begin} \37\&{if} $(\\{read\_seen})$ \1\&{then}\5
$\\{bst\_err}(\.{\'Illegal,\ another\ read\ command\'})$;\2\6
$\\{read\_seen}\K\\{true}$;\C{now we've seen a \.{read} command}\6
\&{if} $(\R\\{entry\_seen})$ \1\&{then}\5
$\\{bst\_err}(\.{\'Illegal,\ read\ command\ before\ entry\ command\'})$;\2\6
$\\{sv\_ptr1}\K\\{buf\_ptr2}$;\C{save the contents of the \.{.bst} input line}\6
$\\{sv\_ptr2}\K\\{last}$;\5
$\\{tmp\_ptr}\K\\{sv\_ptr1}$;\6
\&{while} $(\\{tmp\_ptr}<\\{sv\_ptr2})$ \1\&{do}\6
\&{begin} \37$\\{sv\_buffer}[\\{tmp\_ptr}]\K\\{buffer}[\\{tmp\_ptr}]$;\5
$\\{incr}(\\{tmp\_ptr})$;\6
\&{end};\2\6
\X223:Read the \.{.bib} file(s)\X;\6
$\\{buf\_ptr2}\K\\{sv\_ptr1}$;\C{and restore}\6
$\\{last}\K\\{sv\_ptr2}$;\5
$\\{tmp\_ptr}\K\\{buf\_ptr2}$;\6
\&{while} $(\\{tmp\_ptr}<\\{last})$ \1\&{do}\6
\&{begin} \37$\\{buffer}[\\{tmp\_ptr}]\K\\{sv\_buffer}[\\{tmp\_ptr}]$;\5
$\\{incr}(\\{tmp\_ptr})$;\6
\&{end};\2\6
\4\\{exit}: \37\&{end};\par
\fi

\M212.
A \.{reverse} command has one argument, a single \\{built\_in} or
\\{wiz\_defined} function name between braces.  Upper/lower cases are
considered to be the same---all upper-case letters are converted to
lower case.  Also, we must make sure we've already seen a \.{read}
command.

This module reads a \\{left\_brace}, a single function to be iterated in
reverse, and a \\{right\_brace}.

\Y\P$\4\X100:Procedures and functions for the reading and processing of input
files\X\mathrel{+}\S$\6
\4\&{procedure}\1\  \37\\{bst\_reverse\_command};\6
\4\&{label} \37\\{exit};\2\6
\&{begin} \37\&{if} $(\R\\{read\_seen})$ \1\&{then}\5
$\\{bst\_err}(\.{\'Illegal,\ reverse\ command\ before\ read\ command\'})$;\2\6
$\\{eat\_bst\_white\_and\_eof\_check}(\.{\'reverse\'})$;\5
$\\{bst\_get\_and\_check\_left\_brace}(\.{\'reverse\'})$;\5
$\\{eat\_bst\_white\_and\_eof\_check}(\.{\'reverse\'})$;\5
$\\{bst\_identifier\_scan}(\.{\'reverse\'})$;\5
\X213:Check the \.{reverse}-command argument token\X;\6
$\\{eat\_bst\_white\_and\_eof\_check}(\.{\'reverse\'})$;\5
$\\{bst\_get\_and\_check\_right\_brace}(\.{\'reverse\'})$;\5
\X298:Perform a \.{reverse} command\X;\6
\4\\{exit}: \37\&{end};\par
\fi

\M213.
Before iterating the function in reverse, we must make sure it's a
legal one.  It must exist and be \\{built\_in} or \\{wiz\_defined}.

\Y\P$\4\X213:Check the \.{reverse}-command argument token\X\S$\6
\&{begin} \37\&{trace} \37\\{trace\_pr\_token};\5
$\\{trace\_pr\_ln}(\.{\'\ is\ a\ to\ be\ iterated\ in\ reverse\ function\'})$;\6
\&{ecart}\6
\&{if} $(\\{bad\_argument\_token})$ \1\&{then}\5
\&{return};\2\6
\&{end}\par
\U212.\fi

\M214.
The \.{sort} command has no arguments so there's no more parsing to
do, but we must make sure we've already seen a \.{read} command.

\Y\P$\4\X100:Procedures and functions for the reading and processing of input
files\X\mathrel{+}\S$\6
\4\&{procedure}\1\  \37\\{bst\_sort\_command};\6
\4\&{label} \37\\{exit};\2\6
\&{begin} \37\&{if} $(\R\\{read\_seen})$ \1\&{then}\5
$\\{bst\_err}(\.{\'Illegal,\ sort\ command\ before\ read\ command\'})$;\2\6
\X299:Perform a \.{sort} command\X;\6
\4\\{exit}: \37\&{end};\par
\fi

\M215.
A \.{strings} command has one argument, a list of function names
between braces (the names are separated by one or more \\{white\_space}
characters).  Upper/lower cases are considered to be the same for
function names in these lists---all upper-case letters are converted to
lower case.  Each name in this list specifies a \\{str\_global\_var}.
There may be several \.{strings} commands in the \.{.bst} file.

This module reads a \\{left\_brace}, a list of \\{str\_global\_var}s,
and a \\{right\_brace}.

\Y\P$\4\X100:Procedures and functions for the reading and processing of input
files\X\mathrel{+}\S$\6
\4\&{procedure}\1\  \37\\{bst\_strings\_command};\6
\4\&{label} \37\\{exit};\2\6
\&{begin} \37$\\{eat\_bst\_white\_and\_eof\_check}(\.{\'strings\'})$;\5
$\\{bst\_get\_and\_check\_left\_brace}(\.{\'strings\'})$;\5
$\\{eat\_bst\_white\_and\_eof\_check}(\.{\'strings\'})$;\6
\&{while} $(\\{scan\_char}\I\\{right\_brace})$ \1\&{do}\6
\&{begin} \37$\\{bst\_identifier\_scan}(\.{\'strings\'})$;\5
\X216:Insert a \\{str\_global\_var} into the hash table\X;\6
$\\{eat\_bst\_white\_and\_eof\_check}(\.{\'strings\'})$;\6
\&{end};\2\6
$\\{incr}(\\{buf\_ptr2})$;\C{skip over the \\{right\_brace}}\6
\4\\{exit}: \37\&{end};\par
\fi

\M216.
Here we insert the just found \\{str\_global\_var} name into the hash
table, record it as a \\{str\_global\_var}, set its pointer into
\\{global\_strs}, and initialize its value there to the null string.

\Y\P\D \37$\\{end\_of\_string}=\\{invalid\_code}$\C{this illegal \\{ASCII%
\_code} ends a string}\par
\Y\P$\4\X216:Insert a \\{str\_global\_var} into the hash table\X\S$\6
\&{begin} \37\&{trace} \37\\{trace\_pr\_token};\5
$\\{trace\_pr\_ln}(\.{\'\ is\ a\ string\ global-variable\'})$;\6
\&{ecart}\6
$\\{lower\_case}(\\{buffer},\39\\{buf\_ptr1},\39\\{token\_len})$;\C{ignore case
differences}\6
$\\{fn\_loc}\K\\{str\_lookup}(\\{buffer},\39\\{buf\_ptr1},\39\\{token\_len},\39%
\\{bst\_fn\_ilk},\39\\{do\_insert})$;\5
$\\{check\_for\_already\_seen\_function}(\\{fn\_loc})$;\5
$\\{fn\_type}[\\{fn\_loc}]\K\\{str\_global\_var}$;\6
$\\{fn\_info}[\\{fn\_loc}]\K\\{num\_glb\_strs}$;\C{pointer into \\{global%
\_strs}}\6
\&{if} $(\\{num\_glb\_strs}=\\{max\_glob\_strs})$ \1\&{then}\5
$\\{overflow}(\.{\'number\ of\ string\ global-variables\ \'},\39\\{max\_glob%
\_strs})$;\2\6
$\\{incr}(\\{num\_glb\_strs})$;\6
\&{end}\par
\U215.\fi

\M217.
That's it for processing \.{.bst} commands, except for finishing the
procedural gymnastics.  Note that this must topologically follow the
stuff for \.{.bib} reading, because that's done by the \.{.bst}'s
\.{read} command.

\Y\P$\4\X100:Procedures and functions for the reading and processing of input
files\X\mathrel{+}\S$\6
\X154:Scan for and process a \.{.bst} command\X\par
\fi

\N218.  Reading the database file(s).
This section reads the \.{.bib} file(s), each of which consists of a
sequence of entries (perhaps with a few \.{.bib} commands thrown in,
as explained later).  Each entry consists of an \\{at\_sign}, an entry
type, and, between braces or parentheses and separated by \\{comma}s, a
database key and a list of fields.  Each field consists of a field
name, an \\{equals\_sign}, and nonempty list of field tokens separated by
\\{concat\_char}s.  Each field token is either a nonnegative number, a
macro name (like `jan'), or a brace-balanced string delimited by
either \\{double\_quote}s or braces.  Finally, case differences are
ignored for all but delimited strings and database keys, and
\\{white\_space} characters and ends-of-line may appear in all reasonable
places (i.e., anywhere except within entry types, database keys, field
names, and macro names); furthermore, comments may appear anywhere
between entries (or before the first or after the last) as long as
they contain no \\{at\_sign}s.


\fi

\M219.
These global variables are used while reading the \.{.bib} file(s).
The elements of \\{type\_list}, which indicate an entry's type (book,
article, etc.), point either to a \\{hash\_loc} or are one of two special
markers: \\{empty}, from which $\\{hash\_base}=\\{empty}+1$ was defined,
means we haven't yet encountered the \.{.bib} entry corresponding to
this cite key; and \\{undefined} means we've encountered it but it had
an unknown entry type.  Thus the array \\{type\_list} is of type
\\{hash\_ptr2}, also defined earlier.  An element of the boolean array
\\{entry\_exists} whose corresponding entry in \\{cite\_list} gets
overwritten (which happens only when \\{all\_entries} is \\{true})
indicates whether we've encountered that entry of \\{cite\_list} while
reading the \.{.bib} file(s); this information is unused for entries
that aren't (or more precisely, that have no chance of being)
overwritten.  When we're reading the database file, the array
\\{cite\_info} contains auxiliary information for \\{cite\_list}.  Later,
\\{cite\_info} will become \\{sorted\_cites}, and this dual role imposes the
(not-very-imposing) restriction $\\{max\_strings}\G\\{max\_cites}$.

\Y\P\D \37$\\{undefined}=\\{hash\_max}+1$\C{a special marker used for \\{type%
\_list}}\par
\Y\P$\4\X16:Globals in the outer block\X\mathrel{+}\S$\6
\4\\{bib\_line\_num}: \37\\{integer};\C{line number of the \.{.bib} file}\6
\4\\{entry\_type\_loc}: \37\\{hash\_loc};\C{the hash-table location of an entry
type}\6
\4\\{type\_list}: \37\&{packed} \37\&{array} $[\\{cite\_number}]$ \1\&{of}\5
\\{hash\_ptr2};\2\6
\4\\{type\_exists}: \37\\{boolean};\C{\\{true} if this entry type is %
\.{.bst}-defined}\6
\4\\{entry\_exists}: \37\&{packed} \37\&{array} $[\\{cite\_number}]$ \1\&{of}\5
\\{boolean};\2\6
\4\\{store\_entry}: \37\\{boolean};\C{\\{true} if we're to store info for this
entry}\6
\4\\{field\_name\_loc}: \37\\{hash\_loc};\C{the hash-table location of a field
name}\6
\4\\{field\_val\_loc}: \37\\{hash\_loc};\C{the hash-table location of a field
value}\6
\4\\{store\_field}: \37\\{boolean};\C{\\{true} if we're to store info for this
field}\6
\4\\{store\_token}: \37\\{boolean};\C{\\{true} if we're to store this macro
token}\6
\4\\{right\_outer\_delim}: \37\\{ASCII\_code};\C{either a \\{right\_brace} or a
\\{right\_paren}}\6
\4\\{right\_str\_delim}: \37\\{ASCII\_code};\C{either a \\{right\_brace} or a %
\\{double\_quote}}\6
\4\\{at\_bib\_command}: \37\\{boolean};\C{\\{true} for a command, false for an
entry}\6
\4\\{cur\_macro\_loc}: \37\\{hash\_loc};\C{\\{macro\_loc} for a \.{string}
being defined}\6
\4\\{cite\_info}: \37\&{packed} \37\&{array} $[\\{cite\_number}]$ \1\&{of}\5
\\{str\_number};\C{extra \\{cite\_list} info}\2\6
\4\\{cite\_hash\_found}: \37\\{boolean};\C{set to a previous \\{hash\_found}
value}\6
\4\\{preamble\_ptr}: \37\\{bib\_number};\C{pointer into the \\{s\_preamble}
array}\6
\4\\{num\_preamble\_strings}: \37\\{bib\_number};\C{counts the \\{s\_preamble}
strings}\par
\fi

\M220.
This little procedure exists because it's used by at least two other
procedures and thus saves some space.

\Y\P$\4\X3:Procedures and functions for all file I/O, error messages, and such%
\X\mathrel{+}\S$\6
\4\&{procedure}\1\  \37\\{bib\_ln\_num\_print};\2\6
\&{begin} \37$\\{print}(\.{\'--line\ \'},\39\\{bib\_line\_num}:0,\39\.{\'\ of\
file\ \'})$;\5
\\{print\_bib\_name};\6
\&{end};\par
\fi

\M221.
When there's a serious error parsing a \.{.bib} file, we flush
everything up to the beginning of the next entry.

\Y\P\D \37$\\{bib\_err}(\#)\S$\1\6
\&{begin} \37\C{serious error during \.{.bib} parsing}\6
$\\{print}(\#)$;\5
\\{bib\_err\_print};\5
\&{return};\6
\&{end}\2\par
\Y\P$\4\X3:Procedures and functions for all file I/O, error messages, and such%
\X\mathrel{+}\S$\6
\4\&{procedure}\1\  \37\\{bib\_err\_print};\2\6
\&{begin} \37$\\{print}(\.{\'-\'})$;\5
\\{bib\_ln\_num\_print};\5
\\{print\_bad\_input\_line};\C{this call does the \\{mark\_error}}\6
\\{print\_skipping\_whatever\_remains};\6
\&{if} $(\\{at\_bib\_command})$ \1\&{then}\5
$\\{print\_ln}(\.{\'command\'})$\6
\4\&{else} $\\{print\_ln}(\.{\'entry\'})$;\2\6
\&{end};\par
\fi

\M222.
When there's a harmless error parsing a \.{.bib} file, we just give a
warning message.  This is always called after other stuff has been
printed out.

\Y\P\D \37$\\{bib\_warn}(\#)\S$\1\6
\&{begin} \37\C{non-serious error during \.{.bst} parsing}\6
$\\{print}(\#)$;\5
\\{bib\_warn\_print};\6
\&{end}\2\par
\P\D \37$\\{bib\_warn\_newline}(\#)\S$\1\6
\&{begin} \37\C{same as above but with a newline}\6
$\\{print\_ln}(\#)$;\5
\\{bib\_warn\_print};\6
\&{end}\2\par
\Y\P$\4\X3:Procedures and functions for all file I/O, error messages, and such%
\X\mathrel{+}\S$\6
\4\&{procedure}\1\  \37\\{bib\_warn\_print};\2\6
\&{begin} \37\\{bib\_ln\_num\_print};\5
\\{mark\_warning};\6
\&{end};\par
\fi

\M223.
For all \\{num\_bib\_files} database files, we keep reading and processing
\.{.bib} entries until none left.

\Y\P$\4\X223:Read the \.{.bib} file(s)\X\S$\6
\&{begin} \37\X224:Final initialization for \.{.bib} processing\X;\6
$\\{read\_performed}\K\\{true}$;\5
$\\{bib\_ptr}\K0$;\6
\&{while} $(\\{bib\_ptr}<\\{num\_bib\_files})$ \1\&{do}\6
\&{begin} \37$\\{print}(\.{\'Database\ file\ \#\'},\39\\{bib\_ptr}+1:0,\39\.{%
\':\ \'})$;\5
\\{print\_bib\_name};\6
$\\{bib\_line\_num}\K0$;\C{initialize to get the first input line}\6
$\\{buf\_ptr2}\K\\{last}$;\6
\&{while} $(\R\\{eof}(\\{cur\_bib\_file}))$ \1\&{do}\5
\\{get\_bib\_command\_or\_entry\_and\_process};\2\6
$\\{a\_close}(\\{cur\_bib\_file})$;\5
$\\{incr}(\\{bib\_ptr})$;\6
\&{end};\2\6
$\\{reading\_completed}\K\\{true}$;\6
\&{trace} \37$\\{trace\_pr\_ln}(\.{\'Finished\ reading\ the\ database\ file(s)%
\'})$;\6
\&{ecart}\6
\X276:Final initialization for processing the entries\X;\6
$\\{read\_completed}\K\\{true}$;\6
\&{end}\par
\U211.\fi

\M224.
We need to initialize the \\{field\_info} array, and also various things
associated with the \\{cite\_list} array (but not \\{cite\_list} itself).

\Y\P$\4\X224:Final initialization for \.{.bib} processing\X\S$\6
\&{begin} \37\X225:Initialize the \\{field\_info}\X;\6
\X227:Initialize things for the \\{cite\_list}\X;\6
\&{end}\par
\U223.\fi

\M225.
This module initializes all fields of all entries to \\{missing}, the
value to which all fields are initialized.

\Y\P$\4\X225:Initialize the \\{field\_info}\X\S$\6
\&{begin} \37$\\{check\_field\_overflow}(\\{num\_fields}\ast\\{num\_cites})$;\5
$\\{field\_ptr}\K0$;\6
\&{while} $(\\{field\_ptr}<\\{max\_fields})$ \1\&{do}\6
\&{begin} \37$\\{field\_info}[\\{field\_ptr}]\K\\{missing}$;\5
$\\{incr}(\\{field\_ptr})$;\6
\&{end};\2\6
\&{end}\par
\U224.\fi

\M226.
Complain if somebody's got a field fetish.

\Y\P$\4\X3:Procedures and functions for all file I/O, error messages, and such%
\X\mathrel{+}\S$\6
\4\&{procedure}\1\  \37$\\{check\_field\_overflow}(\\{total\_fields}:%
\\{integer})$;\2\6
\&{begin} \37\&{if} $(\\{total\_fields}>\\{max\_fields})$ \1\&{then}\6
\&{begin} \37$\\{print\_ln}(\\{total\_fields}:0,\39\.{\'\ fields:\'})$;\5
$\\{overflow}(\.{\'total\ number\ of\ fields\ \'},\39\\{max\_fields})$;\6
\&{end};\2\6
\&{end};\par
\fi

\M227.
We must initialize the \\{type\_list} array so that we can detect
duplicate (or missing) entries for cite keys on \\{cite\_list}.  Also,
when we're to include the entire database, we use the array
\\{entry\_exists} to detect those missing entries whose \\{cite\_list} info
will (or to be more precise, might) be overwritten; and we use the
array \\{cite\_info} to save the part of \\{cite\_list} that will (might) be
overwritten.  We also use \\{cite\_info} for counting cross~references
when it's appropriate---when an entry isn't otherwise to be included
on \\{cite\_list} (that is, the entry isn't \.{\\cite}d or
\.{\\nocite}d).  Such an entry is included on the final \\{cite\_list} if
it's cross~referenced at least \\{min\_crossrefs} times.

\Y\P$\4\X227:Initialize things for the \\{cite\_list}\X\S$\6
\&{begin} \37$\\{cite\_ptr}\K0$;\6
\&{while} $(\\{cite\_ptr}<\\{max\_cites})$ \1\&{do}\6
\&{begin} \37$\\{type\_list}[\\{cite\_ptr}]\K\\{empty}$;\6
$\\{cite\_info}[\\{cite\_ptr}]\K\\{any\_value}$;\C{to appeas \PASCAL's boolean
evaluation}\6
$\\{incr}(\\{cite\_ptr})$;\6
\&{end};\2\6
$\\{old\_num\_cites}\K\\{num\_cites}$;\6
\&{if} $(\\{all\_entries})$ \1\&{then}\6
\&{begin} \37$\\{cite\_ptr}\K\\{all\_marker}$;\6
\&{while} $(\\{cite\_ptr}<\\{old\_num\_cites})$ \1\&{do}\6
\&{begin} \37$\\{cite\_info}[\\{cite\_ptr}]\K\\{cite\_list}[\\{cite\_ptr}]$;\5
$\\{entry\_exists}[\\{cite\_ptr}]\K\\{false}$;\5
$\\{incr}(\\{cite\_ptr})$;\6
\&{end};\2\6
$\\{cite\_ptr}\K\\{all\_marker}$;\C{we insert the ``other'' entries here}\6
\&{end}\6
\4\&{else} \&{begin} \37$\\{cite\_ptr}\K\\{num\_cites}$;\C{we insert the
cross-referenced entries here}\6
$\\{all\_marker}\K\\{any\_value}$;\C{to appease \PASCAL's boolean evaluation}\6
\&{end};\2\6
\&{end}\par
\U224.\fi

\M228.
Before we actually start the code for reading a database file, we must
define this \.{.bib}-specific scanning function.  It skips over
\\{white\_space} characters until hitting a nonwhite character or the end
of the file, respectively returning \\{true} or \\{false}.  It also
updates \\{bib\_line\_num}, the line counter.

\Y\P$\4\X83:Procedures and functions for input scanning\X\mathrel{+}\S$\6
\4\&{function}\1\  \37\\{eat\_bib\_white\_space}: \37\\{boolean};\6
\4\&{label} \37\\{exit};\2\6
\&{begin} \37\&{while} $(\R\\{scan\_white\_space})$ \1\&{do}\C{no characters
left; read another line}\6
\&{begin} \37\&{if} $(\R\\{input\_ln}(\\{cur\_bib\_file}))$ \1\&{then}%
\C{end-of-file; return \\{false}}\6
\&{begin} \37$\\{eat\_bib\_white\_space}\K\\{false}$;\5
\&{return};\6
\&{end};\2\6
$\\{incr}(\\{bib\_line\_num})$;\5
$\\{buf\_ptr2}\K0$;\6
\&{end};\2\6
$\\{eat\_bib\_white\_space}\K\\{true}$;\6
\4\\{exit}: \37\&{end};\par
\fi

\M229.
It's often illegal to end a \.{.bib} command in certain places, and
this is where we come to check.

\Y\P\D \37$\\{eat\_bib\_white\_and\_eof\_check}\S$\1\6
\&{begin} \37\&{if} $(\R\\{eat\_bib\_white\_space})$ \1\&{then}\6
\&{begin} \37\\{eat\_bib\_print};\5
\&{return};\6
\&{end};\2\6
\&{end}\2\par
\Y\P$\4\X3:Procedures and functions for all file I/O, error messages, and such%
\X\mathrel{+}\S$\6
\4\&{procedure}\1\  \37\\{eat\_bib\_print};\6
\4\&{label} \37\\{exit};\C{so the call to \\{bib\_err} works}\2\6
\&{begin} \37$\\{bib\_err}(\.{\'Illegal\ end\ of\ database\ file\'})$;\6
\4\\{exit}: \37\&{end};\par
\fi

\M230.
And here are a bunch of error-message macros, each called more than
once, that thus save space as implemented.  This one is for when one
of two possible characters is expected while scanning.

\Y\P\D \37$\\{bib\_one\_of\_two\_expected\_err}(\#)\S$\1\6
\&{begin} \37$\\{bib\_one\_of\_two\_print}(\#)$;\5
\&{return};\6
\&{end}\2\par
\Y\P$\4\X3:Procedures and functions for all file I/O, error messages, and such%
\X\mathrel{+}\S$\6
\4\&{procedure}\1\  \37$\\{bib\_one\_of\_two\_print}(\\{char1},\39\\{char2}:%
\\{ASCII\_code})$;\6
\4\&{label} \37\\{exit};\C{so the call to \\{bib\_err} works}\2\6
\&{begin} \37$\\{bib\_err}(\.{\'I\ was\ expecting\ a\ \`\'},\39\\{xchr}[%
\\{char1}],\39\.{\'\'}\.{\'\ or\ a\ \`\'},\39\\{xchr}[\\{char2}],\39\.{\'\'}\.{%
\'\'})$;\6
\4\\{exit}: \37\&{end};\par
\fi

\M231.
This one's for an expected \\{equals\_sign}.

\Y\P\D \37$\\{bib\_equals\_sign\_expected\_err}\S$\1\6
\&{begin} \37\\{bib\_equals\_sign\_print};\5
\&{return};\6
\&{end}\2\par
\Y\P$\4\X3:Procedures and functions for all file I/O, error messages, and such%
\X\mathrel{+}\S$\6
\4\&{procedure}\1\  \37\\{bib\_equals\_sign\_print};\6
\4\&{label} \37\\{exit};\C{so the call to \\{bib\_err} works}\2\6
\&{begin} \37$\\{bib\_err}(\.{\'I\ was\ expecting\ an\ "\'},\39\\{xchr}[%
\\{equals\_sign}],\39\.{\'"\'})$;\6
\4\\{exit}: \37\&{end};\par
\fi

\M232.
This complains about unbalanced braces.

\Y\P\D \37$\\{bib\_unbalanced\_braces\_err}\S$\1\6
\&{begin} \37\\{bib\_unbalanced\_braces\_print};\5
\&{return};\6
\&{end}\2\par
\Y\P$\4\X3:Procedures and functions for all file I/O, error messages, and such%
\X\mathrel{+}\S$\6
\4\&{procedure}\1\  \37\\{bib\_unbalanced\_braces\_print};\6
\4\&{label} \37\\{exit};\C{so the call to \\{bib\_err} works}\2\6
\&{begin} \37$\\{bib\_err}(\.{\'Unbalanced\ braces\'})$;\6
\4\\{exit}: \37\&{end};\par
\fi

\M233.
And this one about an overly exuberant field.

\Y\P\D \37$\\{bib\_field\_too\_long\_err}\S$\1\6
\&{begin} \37\\{bib\_field\_too\_long\_print};\5
\&{return};\6
\&{end}\2\par
\Y\P$\4\X3:Procedures and functions for all file I/O, error messages, and such%
\X\mathrel{+}\S$\6
\4\&{procedure}\1\  \37\\{bib\_field\_too\_long\_print};\6
\4\&{label} \37\\{exit};\C{so the call to \\{bib\_err} works}\2\6
\&{begin} \37$\\{bib\_err}(\.{\'Your\ field\ is\ more\ than\ \'},\39\\{buf%
\_size}:0,\39\.{\'\ characters\'})$;\6
\4\\{exit}: \37\&{end};\par
\fi

\M234.
This one is just a warning, not an error.  It's for when something
isn't (or might not be) quite right with a macro name.

\Y\P\D \37$\\{macro\_name\_warning}(\#)\S$\1\6
\&{begin} \37\\{macro\_warn\_print};\5
$\\{bib\_warn\_newline}(\#)$;\6
\&{end}\2\par
\Y\P$\4\X3:Procedures and functions for all file I/O, error messages, and such%
\X\mathrel{+}\S$\6
\4\&{procedure}\1\  \37\\{macro\_warn\_print};\2\6
\&{begin} \37$\\{print}(\.{\'Warning--string\ name\ "\'})$;\5
\\{print\_token};\5
$\\{print}(\.{\'"\ is\ \'})$;\6
\&{end};\par
\fi

\M235.
This macro is used to scan all \.{.bib} identifiers.  The argument
tells what was happening at the time.  The associated procedure simply
prints an error message.

\Y\P\D \37$\\{bib\_identifier\_scan\_check}(\#)\S$\1\6
\&{begin} \37\&{if} $((\\{scan\_result}=\\{white\_adjacent})\V(\\{scan%
\_result}=\\{specified\_char\_adjacent}))$ \1\&{then}\5
\\{do\_nothing}\6
\4\&{else} \&{begin} \37\\{bib\_id\_print};\5
$\\{bib\_err}(\#)$;\6
\&{end};\2\6
\&{end}\2\par
\Y\P$\4\X3:Procedures and functions for all file I/O, error messages, and such%
\X\mathrel{+}\S$\6
\4\&{procedure}\1\  \37\\{bib\_id\_print};\2\6
\&{begin} \37\&{if} $(\\{scan\_result}=\\{id\_null})$ \1\&{then}\5
$\\{print}(\.{\'You\'}\.{\'re\ missing\ \'})$\6
\4\&{else} \&{if} $(\\{scan\_result}=\\{other\_char\_adjacent})$ \1\&{then}\5
$\\{print}(\.{\'"\'},\39\\{xchr}[\\{scan\_char}],\39\.{\'"\ immediately\
follows\ \'})$\6
\4\&{else} \\{id\_scanning\_confusion};\2\2\6
\&{end};\par
\fi

\M236.
This module either reads a database entry, whose three main components
are an entry type, a database key, and a list of fields, or it reads a
\.{.bib} command, whose structure is command dependent and explained
later.

\Y\P\D \37$\\{cite\_already\_set}=22$\C{this gets around \PASCAL\ limitations}%
\par
\P\D \37$\\{first\_time\_entry}=26$\C{for checking for repeated database
entries}\par
\Y\P$\4\X236:Scan for and process a \.{.bib} command or database entry\X\S$\6
\4\&{procedure}\1\  \37\\{get\_bib\_command\_or\_entry\_and\_process};\6
\4\&{label} \37$\\{cite\_already\_set},\39\\{first\_time\_entry},\39\\{loop%
\_exit},\39\\{exit}$;\2\6
\&{begin} \37$\\{at\_bib\_command}\K\\{false}$;\6
\X237:Skip to the next database entry or \.{.bib} command\X;\6
\X238:Scan the entry type or scan and process the \.{.bib} command\X;\6
\\{eat\_bib\_white\_and\_eof\_check};\5
\X266:Scan the entry's database key\X;\6
\\{eat\_bib\_white\_and\_eof\_check};\5
\X274:Scan the entry's list of fields\X;\6
\4\\{exit}: \37\&{end};\par
\U210.\fi

\M237.
This module skips over everything until hitting an \\{at\_sign} or the
end of the file.  It also updates \\{bib\_line\_num}, the line counter.

\Y\P$\4\X237:Skip to the next database entry or \.{.bib} command\X\S$\6
\&{while} $(\R\\{scan1}(\\{at\_sign}))$ \1\&{do}\C{no \\{at\_sign}; get next
line}\6
\&{begin} \37\&{if} $(\R\\{input\_ln}(\\{cur\_bib\_file}))$ \1\&{then}%
\C{end-of-file}\6
\&{return};\2\6
$\\{incr}(\\{bib\_line\_num})$;\5
$\\{buf\_ptr2}\K0$;\6
\&{end}\2\par
\U236.\fi

\M238.
This module reads an \\{at\_sign} and an entry type (like `book' or
`article') or a \.{.bib} command.  If it's an entry type, it must be
defined in the \.{.bst} file if this entry is to be included in the
reference list.

\Y\P$\4\X238:Scan the entry type or scan and process the \.{.bib} command\X\S$\6
\&{begin} \37\&{if} $(\\{scan\_char}\I\\{at\_sign})$ \1\&{then}\5
$\\{confusion}(\.{\'An\ "\'},\39\\{xchr}[\\{at\_sign}],\39\.{\'"\ disappeared%
\'})$;\2\6
$\\{incr}(\\{buf\_ptr2})$;\C{skip over the \\{at\_sign}}\6
\\{eat\_bib\_white\_and\_eof\_check};\5
$\\{scan\_identifier}(\\{left\_brace},\39\\{left\_paren},\39\\{left\_paren})$;\5
$\\{bib\_identifier\_scan\_check}(\.{\'an\ entry\ type\'})$;\6
\&{trace} \37\\{trace\_pr\_token};\5
$\\{trace\_pr\_ln}(\.{\'\ is\ an\ entry\ type\ or\ a\ database-file\ command%
\'})$;\6
\&{ecart}\6
$\\{lower\_case}(\\{buffer},\39\\{buf\_ptr1},\39\\{token\_len})$;\C{ignore case
differences}\6
$\\{command\_num}\K\\{ilk\_info}[\\{str\_lookup}(\\{buffer},\39\\{buf\_ptr1},%
\39\\{token\_len},\39\\{bib\_command\_ilk},\39\\{dont\_insert})]$;\6
\&{if} $(\\{hash\_found})$ \1\&{then}\5
\X239:Process a \.{.bib} command\X\6
\4\&{else} \&{begin} \37\C{process an entry type}\6
$\\{entry\_type\_loc}\K\\{str\_lookup}(\\{buffer},\39\\{buf\_ptr1},\39\\{token%
\_len},\39\\{bst\_fn\_ilk},\39\\{dont\_insert})$;\6
\&{if} $((\R\\{hash\_found})\V(\\{fn\_type}[\\{entry\_type\_loc}]\I\\{wiz%
\_defined}))$ \1\&{then}\6
$\\{type\_exists}\K\\{false}$\C{no such entry type defined in the \.{.bst}
file}\6
\4\&{else} $\\{type\_exists}\K\\{true}$;\2\6
\&{end};\2\6
\&{end}\par
\U236.\fi

\M239.
Here we determine which \.{.bib} command we're about to process, then
go to it.

\Y\P$\4\X239:Process a \.{.bib} command\X\S$\6
\&{begin} \37$\\{at\_bib\_command}\K\\{true}$;\6
\&{case} $(\\{command\_num})$ \1\&{of}\6
\4\\{n\_bib\_comment}: \37\X241:Process a \.{comment} command\X;\6
\4\\{n\_bib\_preamble}: \37\X242:Process a \.{preamble} command\X;\6
\4\\{n\_bib\_string}: \37\X243:Process a \.{string} command\X;\6
\4\&{othercases} \37\\{bib\_cmd\_confusion}\2\6
\&{endcases};\6
\&{end}\par
\U238.\fi

\M240.
Here's another bug.

\Y\P$\4\X3:Procedures and functions for all file I/O, error messages, and such%
\X\mathrel{+}\S$\6
\4\&{procedure}\1\  \37\\{bib\_cmd\_confusion};\2\6
\&{begin} \37$\\{confusion}(\.{\'Unknown\ database-file\ command\'})$;\6
\&{end};\par
\fi

\M241.
The \.{comment} command is implemented for SCRIBE compatibility.  It's
not really needed because \BibTeX\ treats (flushes) everything not
within an entry as a comment anyway.

\Y\P$\4\X241:Process a \.{comment} command\X\S$\6
\&{begin} \37\&{return};\C{flush comments}\6
\&{end}\par
\U239.\fi

\M242.
The \.{preamble} command lets a user have \TeX\ stuff inserted (by the
standard styles, at least) directly into the \.{.bbl} file.  It is
intended primarily for allowing \TeX\ macro definitions used within
the bibliography entries (for better sorting, for example).  One
\.{preamble} command per \.{.bib} file should suffice.

A \.{preamble} command has either braces or parentheses as outer
delimiters.  Inside is the preamble string, which has the same syntax
as a field value: a nonempty list of field tokens separated by
\\{concat\_char}s.  There are three types of field tokens---nonnegative
numbers, macro names, and delimited strings.

This module does all the scanning (that's not subcontracted), but the
\.{.bib}-specific scanning function
\\{scan\_and\_store\_the\_field\_value\_and\_eat\_white} actually stores the
value.

\Y\P$\4\X242:Process a \.{preamble} command\X\S$\6
\&{begin} \37\&{if} $(\\{preamble\_ptr}=\\{max\_bib\_files})$ \1\&{then}\5
$\\{bib\_err}(\.{\'You\'}\.{\'ve\ exceeded\ \'},\39\\{max\_bib\_files}:0,\39\.{%
\'\ preamble\ commands\'})$;\2\6
\\{eat\_bib\_white\_and\_eof\_check};\6
\&{if} $(\\{scan\_char}=\\{left\_brace})$ \1\&{then}\5
$\\{right\_outer\_delim}\K\\{right\_brace}$\6
\4\&{else} \&{if} $(\\{scan\_char}=\\{left\_paren})$ \1\&{then}\5
$\\{right\_outer\_delim}\K\\{right\_paren}$\6
\4\&{else} $\\{bib\_one\_of\_two\_expected\_err}(\\{left\_brace},\39\\{left%
\_paren})$;\2\2\6
$\\{incr}(\\{buf\_ptr2})$;\C{skip over the left-delimiter}\6
\\{eat\_bib\_white\_and\_eof\_check};\5
$\\{store\_field}\K\\{true}$;\6
\&{if} $(\R\\{scan\_and\_store\_the\_field\_value\_and\_eat\_white})$ \1%
\&{then}\5
\&{return};\2\6
\&{if} $(\\{scan\_char}\I\\{right\_outer\_delim})$ \1\&{then}\5
$\\{bib\_err}(\.{\'Missing\ "\'},\39\\{xchr}[\\{right\_outer\_delim}],\39\.{\'"%
\ in\ preamble\ command\'})$;\2\6
$\\{incr}(\\{buf\_ptr2})$;\C{skip over the \\{right\_outer\_delim}}\6
\&{return};\6
\&{end}\par
\U239.\fi

\M243.
The \.{string} command is implemented both for SCRIBE compatibility
and for allowing a user: to override a \.{.bst}-file \.{macro}
command, to define one that the \.{.bst} file doesn't, or to engage in
good, wholesome, typing laziness.

The \.{string} command does mostly the same thing as the
\.{.bst}-file's \.{macro} command (but the syntax is different and the
\.{string} command compresses \\{white\_space}).  In fact, later in this
program, the term ``macro'' refers to either a \.{.bst} ``macro'' or a
\.{.bib} ``string'' (when it's clear from the context that it's not
a \.{WEB} macro).

A \.{string} command has either braces or parentheses as outer
delimiters.  Inside is the string's name (it must be a legal
identifier, and case differences are ignored---all upper-case letters
are converted to lower case), then an \\{equals\_sign}, and the string's
definition, which has the same syntax as a field value: a nonempty
list of field tokens separated by \\{concat\_char}s.  There are three
types of field tokens---nonnegative numbers, macro names, and
delimited strings.

\Y\P$\4\X243:Process a \.{string} command\X\S$\6
\&{begin} \37\\{eat\_bib\_white\_and\_eof\_check};\5
\X244:Scan the string's name\X;\6
\\{eat\_bib\_white\_and\_eof\_check};\5
\X246:Scan the string's definition field\X;\6
\&{return};\6
\&{end}\par
\U239.\fi

\M244.
This module reads a left outer-delimiter and a string name.

\Y\P$\4\X244:Scan the string's name\X\S$\6
\&{begin} \37\&{if} $(\\{scan\_char}=\\{left\_brace})$ \1\&{then}\5
$\\{right\_outer\_delim}\K\\{right\_brace}$\6
\4\&{else} \&{if} $(\\{scan\_char}=\\{left\_paren})$ \1\&{then}\5
$\\{right\_outer\_delim}\K\\{right\_paren}$\6
\4\&{else} $\\{bib\_one\_of\_two\_expected\_err}(\\{left\_brace},\39\\{left%
\_paren})$;\2\2\6
$\\{incr}(\\{buf\_ptr2})$;\C{skip over the left-delimiter}\6
\\{eat\_bib\_white\_and\_eof\_check};\5
$\\{scan\_identifier}(\\{equals\_sign},\39\\{equals\_sign},\39\\{equals%
\_sign})$;\5
$\\{bib\_identifier\_scan\_check}(\.{\'a\ string\ name\'})$;\5
\X245:Store the string's name\X;\6
\&{end}\par
\U243.\fi

\M245.
This module marks this string as \\{macro\_ilk}; the commented-out code
will give a warning message when overwriting a previously defined
macro.

\Y\P$\4\X245:Store the string's name\X\S$\6
\&{begin} \37\&{trace} \37\\{trace\_pr\_token};\5
$\\{trace\_pr\_ln}(\.{\'\ is\ a\ database-defined\ macro\'})$;\6
\&{ecart}\6
$\\{lower\_case}(\\{buffer},\39\\{buf\_ptr1},\39\\{token\_len})$;\C{ignore case
differences}\6
$\\{cur\_macro\_loc}\K\\{str\_lookup}(\\{buffer},\39\\{buf\_ptr1},\39\\{token%
\_len},\39\\{macro\_ilk},\39\\{do\_insert})$;\5
$\\{ilk\_info}[\\{cur\_macro\_loc}]\K\\{hash\_text}[\\{cur\_macro\_loc}]$;%
\C{default in case of error}\6
$\B$\1\6
\&{if} $(\\{hash\_found})$ \1\&{then}\C{already seen macro}\6
$\\{macro\_name\_warning}(\.{\'having\ its\ definition\ overwritten\'})$;\2\2\6
$\T$\6
\&{end}\par
\U244.\fi

\M246.
This module skips over the \\{equals\_sign}, reads and stores the list of
field tokens that defines this macro (compressing \\{white\_space}), and
reads a \\{right\_outer\_delim}.

\Y\P$\4\X246:Scan the string's definition field\X\S$\6
\&{begin} \37\&{if} $(\\{scan\_char}\I\\{equals\_sign})$ \1\&{then}\5
\\{bib\_equals\_sign\_expected\_err};\2\6
$\\{incr}(\\{buf\_ptr2})$;\C{skip over the \\{equals\_sign}}\6
\\{eat\_bib\_white\_and\_eof\_check};\5
$\\{store\_field}\K\\{true}$;\6
\&{if} $(\R\\{scan\_and\_store\_the\_field\_value\_and\_eat\_white})$ \1%
\&{then}\5
\&{return};\2\6
\&{if} $(\\{scan\_char}\I\\{right\_outer\_delim})$ \1\&{then}\5
$\\{bib\_err}(\.{\'Missing\ "\'},\39\\{xchr}[\\{right\_outer\_delim}],\39\.{\'"%
\ in\ string\ command\'})$;\2\6
$\\{incr}(\\{buf\_ptr2})$;\C{skip over the \\{right\_outer\_delim}}\6
\&{end}\par
\U243.\fi

\M247.
The variables for the function
\\{scan\_and\_store\_the\_field\_value\_and\_eat\_white} must be global since
the functions it calls use them too.  The alias kludge helps make the
stack space not overflow on some machines.

\Y\P\D \37$\\{field\_vl\_str}\S\\{ex\_buf}$\C{aliases, used ``only'' for this
function}\par
\P\D \37$\\{field\_end}\S\\{ex\_buf\_ptr}$\C{the end marker for the field-value
string}\par
\P\D \37$\\{field\_start}\S\\{ex\_buf\_xptr}$\C{and the start marker}\par
\Y\P$\4\X16:Globals in the outer block\X\mathrel{+}\S$\6
\4\\{bib\_brace\_level}: \37\\{integer};\C{brace nesting depth (excluding %
\\{str\_delim}s)}\par
\fi

\M248.
Since the function \\{scan\_and\_store\_the\_field\_value\_and\_eat\_white}
calls several other yet-to-be-described functions (one directly and
two indirectly), we must perform some topological gymnastics.

\Y\P$\4\X83:Procedures and functions for input scanning\X\mathrel{+}\S$\6
\X252:The scanning function \\{compress\_bib\_white}\X\6
\X253:The scanning function \\{scan\_balanced\_braces}\X\6
\X250:The scanning function \\{scan\_a\_field\_token\_and\_eat\_white}\X\par
\fi

\M249.
This function scans the list of field tokens that define the field
value string.  If \\{store\_field} is \\{true} it accumulates (indirectly)
in \\{field\_vl\_str} the concatenation of all the field tokens,
compressing nonnull \\{white\_space} to a single \\{space} and, if the
field value is for a field (rather than a string definition), removing
any leading or trailing \\{white\_space}; when it's finished it puts the
string into the hash table.  It returns \\{false} if there was a serious
syntax error.

\Y\P$\4\X83:Procedures and functions for input scanning\X\mathrel{+}\S$\6
\4\&{function}\1\  \37\\{scan\_and\_store\_the\_field\_value\_and\_eat\_white}:
\37\\{boolean};\6
\4\&{label} \37\\{exit};\2\6
\&{begin} \37$\\{scan\_and\_store\_the\_field\_value\_and\_eat\_white}\K%
\\{false}$;\C{now it's easy to exit if necessary}\6
$\\{field\_end}\K0$;\6
\&{if} $(\R\\{scan\_a\_field\_token\_and\_eat\_white})$ \1\&{then}\5
\&{return};\2\6
\&{while} $(\\{scan\_char}=\\{concat\_char})$ \1\&{do}\C{scan remaining field
tokens}\6
\&{begin} \37$\\{incr}(\\{buf\_ptr2})$;\C{skip over the \\{concat\_char}}\6
\\{eat\_bib\_white\_and\_eof\_check};\6
\&{if} $(\R\\{scan\_a\_field\_token\_and\_eat\_white})$ \1\&{then}\5
\&{return};\2\6
\&{end};\2\6
\&{if} $(\\{store\_field})$ \1\&{then}\5
\X261:Store the field value string\X;\2\6
$\\{scan\_and\_store\_the\_field\_value\_and\_eat\_white}\K\\{true}$;\6
\4\\{exit}: \37\&{end};\par
\fi

\M250.
Each field token is either a nonnegative number, a macro name (like
`jan'), or a brace-balanced string delimited by either \\{double\_quote}s
or braces.  Thus there are four possibilities for the first character
of the field token: If it's a \\{left\_brace} or a \\{double\_quote}, the
token (with balanced braces, up to the matching \\{right\_str\_delim}) is
a string; if it's \\{numeric}, the token is a number; if it's anything
else, the token is a macro name (and should thus have been defined by
either the \.{.bst}-file's \.{macro} command or the \.{.bib}-file's
\.{string} command).  This function returns \\{false} if there was a
serious syntax error.

\Y\P$\4\X250:The scanning function \\{scan\_a\_field\_token\_and\_eat\_white}\X%
\S$\6
\4\&{function}\1\  \37\\{scan\_a\_field\_token\_and\_eat\_white}: \37%
\\{boolean};\6
\4\&{label} \37\\{exit};\2\6
\&{begin} \37$\\{scan\_a\_field\_token\_and\_eat\_white}\K\\{false}$;\C{now
it's easy to exit if necessary}\6
\&{case} $(\\{scan\_char})$ \1\&{of}\6
\4\\{left\_brace}: \37\&{begin} \37$\\{right\_str\_delim}\K\\{right\_brace}$;\6
\&{if} $(\R\\{scan\_balanced\_braces})$ \1\&{then}\5
\&{return};\2\6
\&{end};\6
\4\\{double\_quote}: \37\&{begin} \37$\\{right\_str\_delim}\K\\{double%
\_quote}$;\6
\&{if} $(\R\\{scan\_balanced\_braces})$ \1\&{then}\5
\&{return};\2\6
\&{end};\6
\4$\.{"0"},\39\.{"1"},\39\.{"2"},\39\.{"3"},\39\.{"4"},\39\.{"5"},\39\.{"6"},%
\39\.{"7"},\39\.{"8"},\39\.{"9"}$: \37\X258:Scan a number\X;\6
\4\&{othercases} \37\X259:Scan a macro name\X\2\6
\&{endcases};\5
\\{eat\_bib\_white\_and\_eof\_check};\5
$\\{scan\_a\_field\_token\_and\_eat\_white}\K\\{true}$;\6
\4\\{exit}: \37\&{end};\par
\U248.\fi

\M251.
Now we come to the stuff that actually accumulates the field value to
be stored.  This module copies a character into \\{field\_vl\_str} if it
will fit; since it's so low level, it's implemented as a macro.

\Y\P\D \37$\\{copy\_char}(\#)\S$\1\6
\&{begin} \37\&{if} $(\\{field\_end}=\\{buf\_size})$ \1\&{then}\5
\\{bib\_field\_too\_long\_err}\6
\4\&{else} \&{begin} \37$\\{field\_vl\_str}[\\{field\_end}]\K\#$;\5
$\\{incr}(\\{field\_end})$;\6
\&{end};\2\6
\&{end}\2\par
\fi

\M252.
The \.{.bib}-specific scanning function \\{compress\_bib\_white} skips
over \\{white\_space} characters within a string until hitting a nonwhite
character; in fact, it does everything \\{eat\_bib\_white\_space} does, but
it also adds a \\{space} to \\{field\_vl\_str}.  This function is never
called if there are no \\{white\_space} characters (or ends-of-line) to
be scanned (though the associated macro might be).  The function
returns \\{false} if there is a serious syntax error.

\Y\P\D \37$\\{check\_for\_and\_compress\_bib\_white\_space}\S$\1\6
\&{begin} \37\&{if} $((\\{lex\_class}[\\{scan\_char}]=\\{white\_space})\V(%
\\{buf\_ptr2}=\\{last}))$ \1\&{then}\6
\&{if} $(\R\\{compress\_bib\_white})$ \1\&{then}\5
\&{return};\2\2\6
\&{end}\2\par
\Y\P$\4\X252:The scanning function \\{compress\_bib\_white}\X\S$\6
\4\&{function}\1\  \37\\{compress\_bib\_white}: \37\\{boolean};\6
\4\&{label} \37\\{exit};\2\6
\&{begin} \37$\\{compress\_bib\_white}\K\\{false}$;\C{now it's easy to exit if
necessary}\6
$\\{copy\_char}(\\{space})$;\6
\&{while} $(\R\\{scan\_white\_space})$ \1\&{do}\C{no characters left; read
another line}\6
\&{begin} \37\&{if} $(\R\\{input\_ln}(\\{cur\_bib\_file}))$ \1\&{then}%
\C{end-of-file; complain}\6
\&{begin} \37\\{eat\_bib\_print};\5
\&{return};\6
\&{end};\2\6
$\\{incr}(\\{bib\_line\_num})$;\5
$\\{buf\_ptr2}\K0$;\6
\&{end};\2\6
$\\{compress\_bib\_white}\K\\{true}$;\6
\4\\{exit}: \37\&{end};\par
\U248.\fi

\M253.
This \.{.bib}-specific function scans a string with balanced braces,
stopping just past the matching \\{right\_str\_delim}.  How much work it
does depends on whether $\\{store\_field}=\\{true}$.  It returns \\{false} if
there was a serious syntax error.

\Y\P$\4\X253:The scanning function \\{scan\_balanced\_braces}\X\S$\6
\4\&{function}\1\  \37\\{scan\_balanced\_braces}: \37\\{boolean};\6
\4\&{label} \37$\\{loop\_exit},\39\\{exit}$;\2\6
\&{begin} \37$\\{scan\_balanced\_braces}\K\\{false}$;\C{now it's easy to exit
if necessary}\6
$\\{incr}(\\{buf\_ptr2})$;\C{skip over the left-delimiter}\6
\\{check\_for\_and\_compress\_bib\_white\_space};\6
\&{if} $(\\{field\_end}>1)$ \1\&{then}\6
\&{if} $(\\{field\_vl\_str}[\\{field\_end}-1]=\\{space})$ \1\&{then}\6
\&{if} $(\\{field\_vl\_str}[\\{field\_end}-2]=\\{space})$ \1\&{then}\C{remove
wrongly added \\{space}}\6
$\\{decr}(\\{field\_end})$;\2\2\2\6
$\\{bib\_brace\_level}\K0$;\C{and we're at a non\\{white\_space} character}\6
\&{if} $(\\{store\_field})$ \1\&{then}\5
\X256:Do a full brace-balanced scan\X\6
\4\&{else} \X254:Do a quick brace-balanced scan\X;\2\6
$\\{incr}(\\{buf\_ptr2})$;\C{skip over the \\{right\_str\_delim}}\6
$\\{scan\_balanced\_braces}\K\\{true}$;\6
\4\\{exit}: \37\&{end};\par
\U248.\fi

\M254.
This module scans over a brace-balanced string without keeping track
of anything but the brace level.  It starts with $\\{bib\_brace\_level}=0$
and at a non\\{white\_space} character.

\Y\P$\4\X254:Do a quick brace-balanced scan\X\S$\6
\&{begin} \37\&{while} $(\\{scan\_char}\I\\{right\_str\_delim})$ \1\&{do}%
\C{we're at $\\{bib\_brace\_level}=0$}\6
\&{if} $(\\{scan\_char}=\\{left\_brace})$ \1\&{then}\6
\&{begin} \37$\\{incr}(\\{bib\_brace\_level})$;\5
$\\{incr}(\\{buf\_ptr2})$;\C{skip over the \\{left\_brace}}\6
\\{eat\_bib\_white\_and\_eof\_check};\6
\&{while} $(\\{bib\_brace\_level}>0)$ \1\&{do}\5
\X255:Do a quick scan with $\\{bib\_brace\_level}>0$\X;\2\6
\&{end}\6
\4\&{else} \&{if} $(\\{scan\_char}=\\{right\_brace})$ \1\&{then}\5
\\{bib\_unbalanced\_braces\_err}\6
\4\&{else} \&{begin} \37$\\{incr}(\\{buf\_ptr2})$;\C{skip over some other
character}\6
\&{if} $(\R\\{scan3}(\\{right\_str\_delim},\39\\{left\_brace},\39\\{right%
\_brace}))$ \1\&{then}\5
\\{eat\_bib\_white\_and\_eof\_check};\2\6
\&{end}\2\2\2\6
\&{end}\par
\U253.\fi

\M255.
This module does the same as above but, because $\\{bib\_brace\_level}>0$, it
doesn't have to look for a \\{right\_str\_delim}.

\Y\P$\4\X255:Do a quick scan with $\\{bib\_brace\_level}>0$\X\S$\6
\&{begin} \37\C{top part of the  \&{while}  loop---we're always at a nonwhite
character}\6
\&{if} $(\\{scan\_char}=\\{right\_brace})$ \1\&{then}\6
\&{begin} \37$\\{decr}(\\{bib\_brace\_level})$;\5
$\\{incr}(\\{buf\_ptr2})$;\C{skip over the \\{right\_brace}}\6
\\{eat\_bib\_white\_and\_eof\_check};\6
\&{end}\6
\4\&{else} \&{if} $(\\{scan\_char}=\\{left\_brace})$ \1\&{then}\6
\&{begin} \37$\\{incr}(\\{bib\_brace\_level})$;\5
$\\{incr}(\\{buf\_ptr2})$;\C{skip over the \\{left\_brace}}\6
\\{eat\_bib\_white\_and\_eof\_check};\6
\&{end}\6
\4\&{else} \&{begin} \37$\\{incr}(\\{buf\_ptr2})$;\C{skip over some other
character}\6
\&{if} $(\R\\{scan2}(\\{right\_brace},\39\\{left\_brace}))$ \1\&{then}\5
\\{eat\_bib\_white\_and\_eof\_check};\2\6
\&{end}\2\2\6
\&{end}\par
\U254.\fi

\M256.
This module scans over a brace-balanced string, compressing multiple
\\{white\_space} characters into a single \\{space}.  It starts with
$\\{bib\_brace\_level}=0$ and starts at a non\\{white\_space} character.

\Y\P$\4\X256:Do a full brace-balanced scan\X\S$\6
\&{begin} \37\&{while} $(\\{scan\_char}\I\\{right\_str\_delim})$ \1\&{do}\6
\&{case} $(\\{scan\_char})$ \1\&{of}\6
\4\\{left\_brace}: \37\&{begin} \37$\\{incr}(\\{bib\_brace\_level})$;\5
$\\{copy\_char}(\\{left\_brace})$;\6
$\\{incr}(\\{buf\_ptr2})$;\C{skip over the \\{left\_brace}}\6
\\{check\_for\_and\_compress\_bib\_white\_space};\6
\X257:Do a full scan with $\\{bib\_brace\_level}>0$\X;\6
\&{end};\6
\4\\{right\_brace}: \37\\{bib\_unbalanced\_braces\_err};\6
\4\&{othercases} \37\&{begin} \37$\\{copy\_char}(\\{scan\_char})$;\5
$\\{incr}(\\{buf\_ptr2})$;\C{skip over some other character}\6
\\{check\_for\_and\_compress\_bib\_white\_space};\6
\&{end}\2\6
\&{endcases};\2\6
\&{end}\par
\U253.\fi

\M257.
This module is similar to the last but starts with $\\{bib\_brace\_level}>0$
(and, like the last, it starts at a non\\{white\_space} character).

\Y\P$\4\X257:Do a full scan with $\\{bib\_brace\_level}>0$\X\S$\6
\&{begin} \37\~ \1\&{loop}\6
\&{case} $(\\{scan\_char})$ \1\&{of}\6
\4\\{right\_brace}: \37\&{begin} \37$\\{decr}(\\{bib\_brace\_level})$;\5
$\\{copy\_char}(\\{right\_brace})$;\6
$\\{incr}(\\{buf\_ptr2})$;\C{skip over the \\{right\_brace}}\6
\\{check\_for\_and\_compress\_bib\_white\_space};\6
\&{if} $(\\{bib\_brace\_level}=0)$ \1\&{then}\5
\&{goto} \37\\{loop\_exit};\2\6
\&{end};\6
\4\\{left\_brace}: \37\&{begin} \37$\\{incr}(\\{bib\_brace\_level})$;\5
$\\{copy\_char}(\\{left\_brace})$;\6
$\\{incr}(\\{buf\_ptr2})$;\C{skip over the \\{left\_brace}}\6
\\{check\_for\_and\_compress\_bib\_white\_space};\6
\&{end};\6
\4\&{othercases} \37\&{begin} \37$\\{copy\_char}(\\{scan\_char})$;\5
$\\{incr}(\\{buf\_ptr2})$;\C{skip over some other character}\6
\\{check\_for\_and\_compress\_bib\_white\_space};\6
\&{end}\2\6
\&{endcases};\2\6
\4\\{loop\_exit}: \37\&{end}\par
\U256.\fi

\M258.
This module scans a nonnegative number and copies it to \\{field\_vl\_str}
if it's to store the field.

\Y\P$\4\X258:Scan a number\X\S$\6
\&{begin} \37\&{if} $(\R\\{scan\_nonneg\_integer})$ \1\&{then}\5
$\\{confusion}(\.{\'A\ digit\ disappeared\'})$;\2\6
\&{if} $(\\{store\_field})$ \1\&{then}\6
\&{begin} \37$\\{tmp\_ptr}\K\\{buf\_ptr1}$;\6
\&{while} $(\\{tmp\_ptr}<\\{buf\_ptr2})$ \1\&{do}\6
\&{begin} \37$\\{copy\_char}(\\{buffer}[\\{tmp\_ptr}])$;\5
$\\{incr}(\\{tmp\_ptr})$;\6
\&{end};\2\6
\&{end};\2\6
\&{end}\par
\U250.\fi

\M259.
This module scans a macro name and copies its string to \\{field\_vl\_str}
if it's to store the field, complaining if the macro is recursive or
undefined.

\Y\P$\4\X259:Scan a macro name\X\S$\6
\&{begin} \37$\\{scan\_identifier}(\\{comma},\39\\{right\_outer\_delim},\39%
\\{concat\_char})$;\5
$\\{bib\_identifier\_scan\_check}(\.{\'a\ field\ part\'})$;\6
\&{if} $(\\{store\_field})$ \1\&{then}\6
\&{begin} \37$\\{lower\_case}(\\{buffer},\39\\{buf\_ptr1},\39\\{token\_len})$;%
\C{ignore case differences}\6
$\\{macro\_name\_loc}\K\\{str\_lookup}(\\{buffer},\39\\{buf\_ptr1},\39\\{token%
\_len},\39\\{macro\_ilk},\39\\{dont\_insert})$;\5
$\\{store\_token}\K\\{true}$;\6
\&{if} $(\\{at\_bib\_command})$ \1\&{then}\6
\&{if} $(\\{command\_num}=\\{n\_bib\_string})$ \1\&{then}\6
\&{if} $(\\{macro\_name\_loc}=\\{cur\_macro\_loc})$ \1\&{then}\6
\&{begin} \37$\\{store\_token}\K\\{false}$;\5
$\\{macro\_name\_warning}(\.{\'used\ in\ its\ own\ definition\'})$;\6
\&{end};\2\2\2\6
\&{if} $(\R\\{hash\_found})$ \1\&{then}\6
\&{begin} \37$\\{store\_token}\K\\{false}$;\5
$\\{macro\_name\_warning}(\.{\'undefined\'})$;\6
\&{end};\2\6
\&{if} $(\\{store\_token})$ \1\&{then}\5
\X260:Copy the macro string to \\{field\_vl\_str}\X;\2\6
\&{end};\2\6
\&{end}\par
\U250.\fi

\M260.
The macro definition may have \\{white\_space} that needs compressing,
because it may have been defined in the \.{.bst} file.

\Y\P$\4\X260:Copy the macro string to \\{field\_vl\_str}\X\S$\6
\&{begin} \37$\\{tmp\_ptr}\K\\{str\_start}[\\{ilk\_info}[\\{macro\_name%
\_loc}]]$;\5
$\\{tmp\_end\_ptr}\K\\{str\_start}[\\{ilk\_info}[\\{macro\_name\_loc}]+1]$;\6
\&{if} $(\\{field\_end}=0)$ \1\&{then}\6
\&{if} $((\\{lex\_class}[\\{str\_pool}[\\{tmp\_ptr}]]=\\{white\_space})\W(%
\\{tmp\_ptr}<\\{tmp\_end\_ptr}))$ \1\&{then}\6
\&{begin} \37\C{compress leading \\{white\_space} of first nonnull token}\6
$\\{copy\_char}(\\{space})$;\5
$\\{incr}(\\{tmp\_ptr})$;\6
\&{while} $((\\{lex\_class}[\\{str\_pool}[\\{tmp\_ptr}]]=\\{white\_space})\W(%
\\{tmp\_ptr}<\\{tmp\_end\_ptr}))$ \1\&{do}\5
$\\{incr}(\\{tmp\_ptr})$;\2\6
\&{end};\C{the next remaining character is non\\{white\_space}}\2\2\6
\&{while} $(\\{tmp\_ptr}<\\{tmp\_end\_ptr})$ \1\&{do}\6
\&{begin} \37\&{if} $(\\{lex\_class}[\\{str\_pool}[\\{tmp\_ptr}]]\I\\{white%
\_space})$ \1\&{then}\5
$\\{copy\_char}(\\{str\_pool}[\\{tmp\_ptr}])$\6
\4\&{else} \&{if} $(\\{field\_vl\_str}[\\{field\_end}-1]\I\\{space})$ \1%
\&{then}\5
$\\{copy\_char}(\\{space})$;\2\2\6
$\\{incr}(\\{tmp\_ptr})$;\6
\&{end};\2\6
\&{end}\par
\U259.\fi

\M261.
Now it's time to store the field value in the hash table, and store an
appropriate pointer to it (depending on whether it's for a database
entry or command).  But first, if necessary, we remove a trailing
\\{space} and a leading \\{space} if these exist.  (Hey, if we had some
ham we could make ham-and-eggs if we had some eggs.)

\Y\P$\4\X261:Store the field value string\X\S$\6
\&{begin} \37\&{if} $(\R\\{at\_bib\_command})$ \1\&{then}\C{chop trailing %
\\{space} for a field}\6
\&{if} $(\\{field\_end}>0)$ \1\&{then}\6
\&{if} $(\\{field\_vl\_str}[\\{field\_end}-1]=\\{space})$ \1\&{then}\5
$\\{decr}(\\{field\_end})$;\2\2\2\6
\&{if} $((\R\\{at\_bib\_command})\W(\\{field\_vl\_str}[0]=\\{space})\W(\\{field%
\_end}>0))$ \1\&{then}\C{chop leading \\{space} for a field}\6
$\\{field\_start}\K1$\6
\4\&{else} $\\{field\_start}\K0$;\2\6
$\\{field\_val\_loc}\K\\{str\_lookup}(\\{field\_vl\_str},\39\\{field\_start},%
\39\\{field\_end}-\\{field\_start},\39\\{text\_ilk},\39\\{do\_insert})$;\5
$\\{fn\_type}[\\{field\_val\_loc}]\K\\{str\_literal}$;\C{set the \\{fn\_class}}%
\6
\&{trace} \37$\\{trace\_pr}(\.{\'"\'})$;\5
$\\{trace\_pr\_pool\_str}(\\{hash\_text}[\\{field\_val\_loc}])$;\5
$\\{trace\_pr\_ln}(\.{\'"\ is\ a\ field\ value\'})$;\6
\&{ecart}\6
\&{if} $(\\{at\_bib\_command})$ \1\&{then}\C{for a \.{preamble} or \.{string}
command}\6
\X262:Store the field value for a command\X\6
\4\&{else} \C{for a database entry}\2\6
\X263:Store the field value for a database entry\X;\6
\&{end}\par
\U249.\fi

\M262.
Here's where we store the goods when we're dealing with a command
rather than an entry.

\Y\P$\4\X262:Store the field value for a command\X\S$\6
\&{begin} \37\&{case} $(\\{command\_num})$ \1\&{of}\6
\4\\{n\_bib\_preamble}: \37\&{begin} \37$\\{s\_preamble}[\\{preamble\_ptr}]\K%
\\{hash\_text}[\\{field\_val\_loc}]$;\5
$\\{incr}(\\{preamble\_ptr})$;\6
\&{end};\6
\4\\{n\_bib\_string}: \37$\\{ilk\_info}[\\{cur\_macro\_loc}]\K\\{hash\_text}[%
\\{field\_val\_loc}]$;\6
\4\&{othercases} \37\\{bib\_cmd\_confusion}\2\6
\&{endcases};\6
\&{end}\par
\U261.\fi

\M263.
And here, an entry.

\Y\P$\4\X263:Store the field value for a database entry\X\S$\6
\&{begin} \37$\\{field\_ptr}\K\\{entry\_cite\_ptr}\ast\\{num\_fields}+\\{fn%
\_info}[\\{field\_name\_loc}]$;\6
\&{if} $(\\{field\_info}[\\{field\_ptr}]\I\\{missing})$ \1\&{then}\6
\&{begin} \37$\\{print}(\.{\'Warning--I\'}\.{\'m\ ignoring\ \'})$;\5
$\\{print\_pool\_str}(\\{cite\_list}[\\{entry\_cite\_ptr}])$;\5
$\\{print}(\.{\'\'}\.{\'s\ extra\ "\'})$;\5
$\\{print\_pool\_str}(\\{hash\_text}[\\{field\_name\_loc}])$;\5
$\\{bib\_warn\_newline}(\.{\'"\ field\'})$;\6
\&{end}\6
\4\&{else} \&{begin} \37\C{the field was empty, store its new value}\6
$\\{field\_info}[\\{field\_ptr}]\K\\{hash\_text}[\\{field\_val\_loc}]$;\6
\&{if} $((\\{fn\_info}[\\{field\_name\_loc}]=\\{crossref\_num})\W(\R\\{all%
\_entries}))$ \1\&{then}\5
\X264:Add or update a cross reference on \\{cite\_list} if necessary\X;\2\6
\&{end};\2\6
\&{end}\par
\U261.\fi

\M264.
If the cross-referenced entry isn't already on \\{cite\_list} we add it
(at least temporarily); if it is already on \\{cite\_list} we update the
cross-reference count, if necessary.  Note that \\{all\_entries} is
\\{false} here.  The alias kludge helps make the stack space not
overflow on some machines.

\Y\P\D \37$\\{extra\_buf}\S\\{out\_buf}$\C{an alias, used only in this module}%
\par
\Y\P$\4\X264:Add or update a cross reference on \\{cite\_list} if necessary\X%
\S$\6
\&{begin} \37$\\{tmp\_ptr}\K\\{field\_start}$;\6
\&{while} $(\\{tmp\_ptr}<\\{field\_end})$ \1\&{do}\6
\&{begin} \37$\\{extra\_buf}[\\{tmp\_ptr}]\K\\{field\_vl\_str}[\\{tmp\_ptr}]$;\5
$\\{incr}(\\{tmp\_ptr})$;\6
\&{end};\2\6
$\\{lower\_case}(\\{extra\_buf},\39\\{field\_start},\39\\{field\_end}-\\{field%
\_start})$;\C{convert to `canonical' form}\6
$\\{lc\_cite\_loc}\K\\{str\_lookup}(\\{extra\_buf},\39\\{field\_start},\39%
\\{field\_end}-\\{field\_start},\39\\{lc\_cite\_ilk},\39\\{do\_insert})$;\6
\&{if} $(\\{hash\_found})$ \1\&{then}\6
\&{begin} \37$\\{cite\_loc}\K\\{ilk\_info}[\\{lc\_cite\_loc}]$;\C{even if
there's a case mismatch}\6
\&{if} $(\\{ilk\_info}[\\{cite\_loc}]\G\\{old\_num\_cites})$ \1\&{then}\C{a
previous \.{crossref}}\6
$\\{incr}(\\{cite\_info}[\\{ilk\_info}[\\{cite\_loc}]])$;\2\6
\&{end}\6
\4\&{else} \&{begin} \37\C{it's a new \.{crossref}}\6
$\\{cite\_loc}\K\\{str\_lookup}(\\{field\_vl\_str},\39\\{field\_start},\39%
\\{field\_end}-\\{field\_start},\39\\{cite\_ilk},\39\\{do\_insert})$;\6
\&{if} $(\\{hash\_found})$ \1\&{then}\5
\\{hash\_cite\_confusion};\2\6
$\\{add\_database\_cite}(\\{cite\_ptr})$;\C{this increments \\{cite\_ptr}}\6
$\\{cite\_info}[\\{ilk\_info}[\\{cite\_loc}]]\K1$;\C{the first cross-ref for
this cite key}\6
\&{end};\2\6
\&{end}\par
\U263.\fi

\M265.
This procedure adds (or restores) to \\{cite\_list} a cite key; it is
called only when \\{all\_entries} is \\{true} or when adding
cross~references, and it assumes that \\{cite\_loc} and \\{lc\_cite\_loc} are
set.  It also increments its argument.

\Y\P$\4\X54:Procedures and functions for handling numbers, characters, and
strings\X\mathrel{+}\S$\6
\4\&{procedure}\1\  \37$\\{add\_database\_cite}(\mathop{\&{var}}\\{new\_cite}:%
\\{cite\_number})$;\2\6
\&{begin} \37$\\{check\_cite\_overflow}(\\{new\_cite})$;\C{make sure this cite
will fit}\6
$\\{check\_field\_overflow}(\\{num\_fields}\ast\\{new\_cite})$;\5
$\\{cite\_list}[\\{new\_cite}]\K\\{hash\_text}[\\{cite\_loc}]$;\5
$\\{ilk\_info}[\\{cite\_loc}]\K\\{new\_cite}$;\5
$\\{ilk\_info}[\\{lc\_cite\_loc}]\K\\{cite\_loc}$;\5
$\\{incr}(\\{new\_cite})$;\6
\&{end};\par
\fi

\M266.
And now, back to processing an entry (rather than a command).  This
module reads a left outer-delimiter and a database key.

\Y\P$\4\X266:Scan the entry's database key\X\S$\6
\&{begin} \37\&{if} $(\\{scan\_char}=\\{left\_brace})$ \1\&{then}\5
$\\{right\_outer\_delim}\K\\{right\_brace}$\6
\4\&{else} \&{if} $(\\{scan\_char}=\\{left\_paren})$ \1\&{then}\5
$\\{right\_outer\_delim}\K\\{right\_paren}$\6
\4\&{else} $\\{bib\_one\_of\_two\_expected\_err}(\\{left\_brace},\39\\{left%
\_paren})$;\2\2\6
$\\{incr}(\\{buf\_ptr2})$;\C{skip over the left-delimiter}\6
\\{eat\_bib\_white\_and\_eof\_check};\6
\&{if} $(\\{right\_outer\_delim}=\\{right\_paren})$ \1\&{then}\C{to allow it in
a database key}\6
\&{begin} \37\&{if} $(\\{scan1\_white}(\\{comma}))$ \1\&{then}\C{ok if database
key ends line}\6
\\{do\_nothing};\2\6
\&{end}\6
\4\&{else} \&{if} $(\\{scan2\_white}(\\{comma},\39\\{right\_brace}))$ \1%
\&{then}\C{$\\{right\_brace}=\\{right\_outer\_delim}$}\6
\\{do\_nothing};\2\2\6
\X267:Check for a database key of interest\X;\6
\&{end}\par
\U236.\fi

\M267.
The lower-case version of this database key must correspond to one in
\\{cite\_list}, or else \\{all\_entries} must be \\{true}, if this entry is to
be included in the reference list.  Accordingly, this module sets
\\{store\_entry}, which determines whether the relevant information for
this entry is stored.  The alias kludge helps make the stack space not
overflow on some machines.

\Y\P\D \37$\\{ex\_buf3}\S\\{ex\_buf}$\C{an alias, used only in this module}\par
\Y\P$\4\X267:Check for a database key of interest\X\S$\6
\&{begin} \37\&{trace} \37\\{trace\_pr\_token};\5
$\\{trace\_pr\_ln}(\.{\'\ is\ a\ database\ key\'})$;\6
\&{ecart}\6
$\\{tmp\_ptr}\K\\{buf\_ptr1}$;\6
\&{while} $(\\{tmp\_ptr}<\\{buf\_ptr2})$ \1\&{do}\6
\&{begin} \37$\\{ex\_buf3}[\\{tmp\_ptr}]\K\\{buffer}[\\{tmp\_ptr}]$;\5
$\\{incr}(\\{tmp\_ptr})$;\6
\&{end};\2\6
$\\{lower\_case}(\\{ex\_buf3},\39\\{buf\_ptr1},\39\\{token\_len})$;\C{convert
to `canonical' form}\6
\&{if} $(\\{all\_entries})$ \1\&{then}\5
$\\{lc\_cite\_loc}\K\\{str\_lookup}(\\{ex\_buf3},\39\\{buf\_ptr1},\39\\{token%
\_len},\39\\{lc\_cite\_ilk},\39\\{do\_insert})$\6
\4\&{else} $\\{lc\_cite\_loc}\K\\{str\_lookup}(\\{ex\_buf3},\39\\{buf\_ptr1},%
\39\\{token\_len},\39\\{lc\_cite\_ilk},\39\\{dont\_insert})$;\2\6
\&{if} $(\\{hash\_found})$ \1\&{then}\6
\&{begin} \37$\\{entry\_cite\_ptr}\K\\{ilk\_info}[\\{ilk\_info}[\\{lc\_cite%
\_loc}]]$;\5
\X268:Check for a duplicate or \.{crossref}-matching database key\X;\6
\&{end};\2\6
$\\{store\_entry}\K\\{true}$;\C{unless $(\R\\{hash\_found})\W(\R\\{all%
\_entries})$}\6
\&{if} $(\\{all\_entries})$ \1\&{then}\5
\X272:Put this cite key in its place\X\6
\4\&{else} \&{if} $(\R\\{hash\_found})$ \1\&{then}\5
$\\{store\_entry}\K\\{false}$;\C{no such cite key exists on \\{cite\_list}}\2\2%
\6
\&{if} $(\\{store\_entry})$ \1\&{then}\5
\X273:Make sure this entry is ok before proceeding\X;\2\6
\&{end}\par
\U266.\fi

\M268.
It's illegal to have two (or more) entries with the same database key
(even if there are case differrences), and we skip the rest of the
entry for such a repeat occurrence.  Also, we make this entry's
database key the official \\{cite\_list} key if it's on \\{cite\_list} only
because of cross references.

\Y\P$\4\X268:Check for a duplicate or \.{crossref}-matching database key\X\S$\6
\&{begin} \37\&{if} $((\R\\{all\_entries})\V(\\{entry\_cite\_ptr}<\\{all%
\_marker})\V(\\{entry\_cite\_ptr}\G\\{old\_num\_cites}))$ \1\&{then}\6
\&{begin} \37\&{if} $(\\{type\_list}[\\{entry\_cite\_ptr}]=\\{empty})$ \1%
\&{then}\6
\&{begin} \37\X269:Make sure this entry's database key is on \\{cite\_list}\X;\6
\&{goto} \37\\{first\_time\_entry};\6
\&{end};\2\6
\&{end}\6
\4\&{else} \&{if} $(\R\\{entry\_exists}[\\{entry\_cite\_ptr}])$ \1\&{then}\6
\&{begin} \37\X270:Find the lower-case equivalent of the \\{cite\_info} key\X;\6
\&{if} $(\\{lc\_xcite\_loc}=\\{lc\_cite\_loc})$ \1\&{then}\5
\&{goto} \37\\{first\_time\_entry};\2\6
\&{end};\6
\C{oops---repeated entry---issue a reprimand}\2\2\6
\&{if} $(\\{type\_list}[\\{entry\_cite\_ptr}]=\\{empty})$ \1\&{then}\5
$\\{confusion}(\.{\'The\ cite\ list\ is\ messed\ up\'})$;\2\6
$\\{bib\_err}(\.{\'Repeated\ entry\'})$;\6
\4\\{first\_time\_entry}: \37\C{note that when we leave normally, \\{hash%
\_found} is \\{true}}\6
\&{end}\par
\U267.\fi

\M269.
An entry that's on \\{cite\_list} only because of cross referencing must
have its database key (rather than one of the \.{crossref} keys) as
the official \\{cite\_list} string.  Here's where we assure that.  The
variable \\{hash\_found} is \\{true} upon entrance to and exit from this
module.

\Y\P$\4\X269:Make sure this entry's database key is on \\{cite\_list}\X\S$\6
\&{begin} \37\&{if} $((\R\\{all\_entries})\W(\\{entry\_cite\_ptr}\G\\{old\_num%
\_cites}))$ \1\&{then}\6
\&{begin} \37$\\{cite\_loc}\K\\{str\_lookup}(\\{buffer},\39\\{buf\_ptr1},\39%
\\{token\_len},\39\\{cite\_ilk},\39\\{do\_insert})$;\6
\&{if} $(\R\\{hash\_found})$ \1\&{then}\6
\&{begin} \37\C{it's not on \\{cite\_list}---put it there}\6
$\\{ilk\_info}[\\{lc\_cite\_loc}]\K\\{cite\_loc}$;\5
$\\{ilk\_info}[\\{cite\_loc}]\K\\{entry\_cite\_ptr}$;\5
$\\{cite\_list}[\\{entry\_cite\_ptr}]\K\\{hash\_text}[\\{cite\_loc}]$;\6
$\\{hash\_found}\K\\{true}$;\C{restore this value for later use}\6
\&{end};\2\6
\&{end};\2\6
\&{end}\par
\U268.\fi

\M270.
This module, a simpler version of the
\\{find\_cite\_locs\_for\_this\_cite\_key} function, exists primarily to
compute \\{lc\_xcite\_loc}.  When this code is executed we have
$(\\{all\_entries})\W(\\{entry\_cite\_ptr}\G\\{all\_marker})\W(\R\\{entry%
\_exists}[\\{entry\_cite\_ptr}])$.  The alias kludge helps make the stack
space not overflow on some machines.

\Y\P\D \37$\\{ex\_buf4}\S\\{ex\_buf}$\C{aliases, used only}\par
\P\D \37$\\{ex\_buf4\_ptr}\S\\{ex\_buf\_ptr}$\C{in this module}\par
\Y\P$\4\X270:Find the lower-case equivalent of the \\{cite\_info} key\X\S$\6
\&{begin} \37$\\{ex\_buf4\_ptr}\K0$;\5
$\\{tmp\_ptr}\K\\{str\_start}[\\{cite\_info}[\\{entry\_cite\_ptr}]]$;\5
$\\{tmp\_end\_ptr}\K\\{str\_start}[\\{cite\_info}[\\{entry\_cite\_ptr}]+1]$;\6
\&{while} $(\\{tmp\_ptr}<\\{tmp\_end\_ptr})$ \1\&{do}\6
\&{begin} \37$\\{ex\_buf4}[\\{ex\_buf4\_ptr}]\K\\{str\_pool}[\\{tmp\_ptr}]$;\5
$\\{incr}(\\{ex\_buf4\_ptr})$;\5
$\\{incr}(\\{tmp\_ptr})$;\6
\&{end};\2\6
$\\{lower\_case}(\\{ex\_buf4},\390,\39\\{length}(\\{cite\_info}[\\{entry\_cite%
\_ptr}]))$;\C{convert to `canonical' form}\6
$\\{lc\_xcite\_loc}\K\\{str\_lookup}(\\{ex\_buf4},\390,\39\\{length}(\\{cite%
\_info}[\\{entry\_cite\_ptr}]),\39\\{lc\_cite\_ilk},\39\\{dont\_insert})$;\6
\&{if} $(\R\\{hash\_found})$ \1\&{then}\5
\\{cite\_key\_disappeared\_confusion};\2\6
\&{end}\par
\U268.\fi

\M271.
Here's another bug complaint.

\Y\P$\4\X3:Procedures and functions for all file I/O, error messages, and such%
\X\mathrel{+}\S$\6
\4\&{procedure}\1\  \37\\{cite\_key\_disappeared\_confusion};\2\6
\&{begin} \37$\\{confusion}(\.{\'A\ cite\ key\ disappeared\'})$;\6
\&{end};\par
\fi

\M272.
This module, which gets executed only when \\{all\_entries} is \\{true},
does one of three things, depending on whether or not, and where, the
cite key appears on \\{cite\_list}: If it's on \\{cite\_list} before
\\{all\_marker}, there's nothing to be done; if it's after \\{all\_marker},
it must be reinserted (at the current place) and we must note that its
corresponding entry exists; and if it's not on \\{cite\_list} at all, it
must be inserted for the first time.  The \&{goto}  construct must stay
as is, partly because some \PASCAL\ compilers might complain if
``$\W$'' were to connect the two boolean expressions (since
\\{entry\_cite\_ptr} could be uninitialized when \\{hash\_found} is \\{false}).

\Y\P$\4\X272:Put this cite key in its place\X\S$\6
\&{begin} \37\&{if} $(\\{hash\_found})$ \1\&{then}\6
\&{begin} \37\&{if} $(\\{entry\_cite\_ptr}<\\{all\_marker})$ \1\&{then}\5
\&{goto} \37\\{cite\_already\_set}\C{that is, do nothing}\6
\4\&{else} \&{begin} \37$\\{entry\_exists}[\\{entry\_cite\_ptr}]\K\\{true}$;\5
$\\{cite\_loc}\K\\{ilk\_info}[\\{lc\_cite\_loc}]$;\6
\&{end};\2\6
\&{end}\6
\4\&{else} \&{begin} \37\C{this is a new key}\6
$\\{cite\_loc}\K\\{str\_lookup}(\\{buffer},\39\\{buf\_ptr1},\39\\{token\_len},%
\39\\{cite\_ilk},\39\\{do\_insert})$;\6
\&{if} $(\\{hash\_found})$ \1\&{then}\5
\\{hash\_cite\_confusion};\2\6
\&{end};\2\6
$\\{entry\_cite\_ptr}\K\\{cite\_ptr}$;\5
$\\{add\_database\_cite}(\\{cite\_ptr})$;\C{this increments \\{cite\_ptr}}\6
\4\\{cite\_already\_set}: \37\&{end}\par
\U267.\fi

\M273.
We must give a warning if this entry~type doesn't exist.  Also, we
point the appropriate entry of \\{type\_list} to the entry type just read
above.

For SCRIBE compatibility, the code to give a warning for a case
mismatch between a cite key and a database key has been commented out.
In fact, SCRIBE is the reason that it doesn't produce an error message
outright.  (Note: Case mismatches between two cite keys produce
full-blown errors.)

\Y\P$\4\X273:Make sure this entry is ok before proceeding\X\S$\6
\&{begin} \37$\B\\{dummy\_loc}\K\\{str\_lookup}(\\{buffer},\39\\{buf\_ptr1},\39%
\\{token\_len},\39\\{cite\_ilk},\39\\{dont\_insert})$;\6
\&{if} $(\R\\{hash\_found})$ \1\&{then}\C{give a warning if there is a case
difference}\6
\&{begin} \37$\\{print}(\.{\'Warning--case\ mismatch,\ database\ key\ "\'})$;\5
\\{print\_token};\5
$\\{print}(\.{\'",\ cite\ key\ "\'})$;\5
$\\{print\_pool\_str}(\\{cite\_list}[\\{entry\_cite\_ptr}])$;\5
$\\{bib\_warn\_newline}(\.{\'"\'})$;\6
\&{end};\2\6
$\T$\6
\&{if} $(\\{type\_exists})$ \1\&{then}\5
$\\{type\_list}[\\{entry\_cite\_ptr}]\K\\{entry\_type\_loc}$\6
\4\&{else} \&{begin} \37$\\{type\_list}[\\{entry\_cite\_ptr}]\K\\{undefined}$;\5
$\\{print}(\.{\'Warning--entry\ type\ for\ "\'})$;\5
\\{print\_token};\5
$\\{bib\_warn\_newline}(\.{\'"\ isn\'}\.{\'t\ style-file\ defined\'})$;\6
\&{end};\2\6
\&{end}\par
\U267.\fi

\M274.
This module reads a \\{comma} and a field as many times as it can, and
then reads a \\{right\_outer\_delim}, ending the current entry.

\Y\P$\4\X274:Scan the entry's list of fields\X\S$\6
\&{begin} \37\&{while} $(\\{scan\_char}\I\\{right\_outer\_delim})$ \1\&{do}\6
\&{begin} \37\&{if} $(\\{scan\_char}\I\\{comma})$ \1\&{then}\5
$\\{bib\_one\_of\_two\_expected\_err}(\\{comma},\39\\{right\_outer\_delim})$;\2%
\6
$\\{incr}(\\{buf\_ptr2})$;\C{skip over the \\{comma}}\6
\\{eat\_bib\_white\_and\_eof\_check};\6
\&{if} $(\\{scan\_char}=\\{right\_outer\_delim})$ \1\&{then}\5
\&{goto} \37\\{loop\_exit};\2\6
\X275:Get the next field name\X;\6
\\{eat\_bib\_white\_and\_eof\_check};\6
\&{if} $(\R\\{scan\_and\_store\_the\_field\_value\_and\_eat\_white})$ \1%
\&{then}\5
\&{return};\2\6
\&{end};\2\6
\4\\{loop\_exit}: \37$\\{incr}(\\{buf\_ptr2})$;\C{skip over the \\{right\_outer%
\_delim}}\6
\&{end}\par
\U236.\fi

\M275.
This module reads a field name; its contents won't be stored unless it
was declared in the \.{.bst} file and $\\{store\_entry}=\\{true}$.

\Y\P$\4\X275:Get the next field name\X\S$\6
\&{begin} \37$\\{scan\_identifier}(\\{equals\_sign},\39\\{equals\_sign},\39%
\\{equals\_sign})$;\5
$\\{bib\_identifier\_scan\_check}(\.{\'a\ field\ name\'})$;\6
\&{trace} \37\\{trace\_pr\_token};\5
$\\{trace\_pr\_ln}(\.{\'\ is\ a\ field\ name\'})$;\6
\&{ecart}\6
$\\{store\_field}\K\\{false}$;\6
\&{if} $(\\{store\_entry})$ \1\&{then}\6
\&{begin} \37$\\{lower\_case}(\\{buffer},\39\\{buf\_ptr1},\39\\{token\_len})$;%
\C{ignore case differences}\6
$\\{field\_name\_loc}\K\\{str\_lookup}(\\{buffer},\39\\{buf\_ptr1},\39\\{token%
\_len},\39\\{bst\_fn\_ilk},\39\\{dont\_insert})$;\6
\&{if} $(\\{hash\_found})$ \1\&{then}\6
\&{if} $(\\{fn\_type}[\\{field\_name\_loc}]=\\{field})$ \1\&{then}\6
$\\{store\_field}\K\\{true}$;\C{field name was pre-defined or %
\.{.bst}-declared}\2\2\6
\&{end};\2\6
\\{eat\_bib\_white\_and\_eof\_check};\6
\&{if} $(\\{scan\_char}\I\\{equals\_sign})$ \1\&{then}\5
\\{bib\_equals\_sign\_expected\_err};\2\6
$\\{incr}(\\{buf\_ptr2})$;\C{skip over the \\{equals\_sign}}\6
\&{end}\par
\U274.\fi

\M276.
This gets things ready for further \.{.bst} processing.

\Y\P$\4\X276:Final initialization for processing the entries\X\S$\6
\&{begin} \37$\\{num\_cites}\K\\{cite\_ptr}$;\C{to include database and %
\.{crossref} cite keys, too}\6
$\\{num\_preamble\_strings}\K\\{preamble\_ptr}$;\C{number of \.{preamble}
commands seen}\6
\X277:Add cross-reference information\X;\6
\X279:Subtract cross-reference information\X;\6
\X283:Remove missing entries or those cross referenced too few times\X;\6
\X287:Initialize the \\{int\_entry\_var}s\X;\6
\X288:Initialize the \\{str\_entry\_var}s\X;\6
\X289:Initialize the \\{sorted\_cites}\X;\6
\&{end}\par
\U223.\fi

\M277.
Now we update any entry (here called a {\it child\/} entry) that
cross~referenced another (here called a {\it parent\/} entry); this
cross~referencing occurs when the child's \.{crossref} field (value)
consists of the parent's database key.  To do the update, we replace
the child's \\{missing} fields by the corresponding fields of the
parent.  Also, we make sure the \.{crossref} field contains the
case-correct version.  Finally, although it is technically illegal to
nest cross~references, and although we give a warning (a few modules
hence) when someone tries, we do what we can to accommodate the
attempt.

\Y\P$\4\X277:Add cross-reference information\X\S$\6
\&{begin} \37$\\{cite\_ptr}\K0$;\6
\&{while} $(\\{cite\_ptr}<\\{num\_cites})$ \1\&{do}\6
\&{begin} \37$\\{field\_ptr}\K\\{cite\_ptr}\ast\\{num\_fields}+\\{crossref%
\_num}$;\6
\&{if} $(\\{field\_info}[\\{field\_ptr}]\I\\{missing})$ \1\&{then}\6
\&{if} $(\\{find\_cite\_locs\_for\_this\_cite\_key}(\\{field\_info}[\\{field%
\_ptr}]))$ \1\&{then}\6
\&{begin} \37$\\{cite\_loc}\K\\{ilk\_info}[\\{lc\_cite\_loc}]$;\5
$\\{field\_info}[\\{field\_ptr}]\K\\{hash\_text}[\\{cite\_loc}]$;\5
$\\{cite\_parent\_ptr}\K\\{ilk\_info}[\\{cite\_loc}]$;\5
$\\{field\_ptr}\K\\{cite\_ptr}\ast\\{num\_fields}+\\{num\_pre\_defined%
\_fields}$;\5
$\\{field\_end\_ptr}\K\\{field\_ptr}-\\{num\_pre\_defined\_fields}+\\{num%
\_fields}$;\5
$\\{field\_parent\_ptr}\K\\{cite\_parent\_ptr}\ast\\{num\_fields}+\\{num\_pre%
\_defined\_fields}$;\6
\&{while} $(\\{field\_ptr}<\\{field\_end\_ptr})$ \1\&{do}\6
\&{begin} \37\&{if} $(\\{field\_info}[\\{field\_ptr}]=\\{missing})$ \1\&{then}\5
$\\{field\_info}[\\{field\_ptr}]\K\\{field\_info}[\\{field\_parent\_ptr}]$;\2\6
$\\{incr}(\\{field\_ptr})$;\5
$\\{incr}(\\{field\_parent\_ptr})$;\6
\&{end};\2\6
\&{end};\2\2\6
$\\{incr}(\\{cite\_ptr})$;\6
\&{end};\2\6
\&{end}\par
\U276.\fi

\M278.
Occasionally we need to figure out the hash-table location of a given
cite-key string and its lower-case equivalent.  This function does
that.  To perform the task it needs to borrow a buffer, a need that
gives rise to the alias kludge---it helps make the stack space not
overflow on some machines (and while it's at it, it'll borrow a
pointer, too).  Finally, the function returns \\{true} if the cite key
exists on \\{cite\_list}, and its sets \\{cite\_hash\_found} according to
whether or not it found the actual version (before \\{lower\_case}ing) of
the cite key; however, its {\sl raison d'\^$\mkern-8mu$etre\/}
(literally, ``to eat a raisin'') is to compute \\{cite\_loc} and
\\{lc\_cite\_loc}.

\Y\P\D \37$\\{ex\_buf5}\S\\{ex\_buf}$\C{aliases, used only}\par
\P\D \37$\\{ex\_buf5\_ptr}\S\\{ex\_buf\_ptr}$\C{in this module}\par
\Y\P$\4\X54:Procedures and functions for handling numbers, characters, and
strings\X\mathrel{+}\S$\6
\4\&{function}\1\  \37$\\{find\_cite\_locs\_for\_this\_cite\_key}(\\{cite%
\_str}:\\{str\_number})$: \37\\{boolean};\2\6
\&{begin} \37$\\{ex\_buf5\_ptr}\K0$;\5
$\\{tmp\_ptr}\K\\{str\_start}[\\{cite\_str}]$;\5
$\\{tmp\_end\_ptr}\K\\{str\_start}[\\{cite\_str}+1]$;\6
\&{while} $(\\{tmp\_ptr}<\\{tmp\_end\_ptr})$ \1\&{do}\6
\&{begin} \37$\\{ex\_buf5}[\\{ex\_buf5\_ptr}]\K\\{str\_pool}[\\{tmp\_ptr}]$;\5
$\\{incr}(\\{ex\_buf5\_ptr})$;\5
$\\{incr}(\\{tmp\_ptr})$;\6
\&{end};\2\6
$\\{cite\_loc}\K\\{str\_lookup}(\\{ex\_buf5},\390,\39\\{length}(\\{cite\_str}),%
\39\\{cite\_ilk},\39\\{dont\_insert})$;\5
$\\{cite\_hash\_found}\K\\{hash\_found}$;\5
$\\{lower\_case}(\\{ex\_buf5},\390,\39\\{length}(\\{cite\_str}))$;\C{convert to
`canonical' form}\6
$\\{lc\_cite\_loc}\K\\{str\_lookup}(\\{ex\_buf5},\390,\39\\{length}(\\{cite%
\_str}),\39\\{lc\_cite\_ilk},\39\\{dont\_insert})$;\6
\&{if} $(\\{hash\_found})$ \1\&{then}\5
$\\{find\_cite\_locs\_for\_this\_cite\_key}\K\\{true}$\6
\4\&{else} $\\{find\_cite\_locs\_for\_this\_cite\_key}\K\\{false}$;\2\6
\&{end};\par
\fi

\M279.
Here we remove the \.{crossref} field value for each child whose
parent was cross~referenced too few times.  We also issue any
necessary warnings arising from a bad cross~reference.

\Y\P$\4\X279:Subtract cross-reference information\X\S$\6
\&{begin} \37$\\{cite\_ptr}\K0$;\6
\&{while} $(\\{cite\_ptr}<\\{num\_cites})$ \1\&{do}\6
\&{begin} \37$\\{field\_ptr}\K\\{cite\_ptr}\ast\\{num\_fields}+\\{crossref%
\_num}$;\6
\&{if} $(\\{field\_info}[\\{field\_ptr}]\I\\{missing})$ \1\&{then}\6
\&{if} $(\R\\{find\_cite\_locs\_for\_this\_cite\_key}(\\{field\_info}[\\{field%
\_ptr}]))$ \1\&{then}\6
\&{begin} \37\C{the parent is not on \\{cite\_list}}\6
\&{if} $(\\{cite\_hash\_found})$ \1\&{then}\5
\\{hash\_cite\_confusion};\2\6
\\{nonexistent\_cross\_reference\_error};\5
$\\{field\_info}[\\{field\_ptr}]\K\\{missing}$;\C{remove the \.{crossref} ptr}\6
\&{end}\6
\4\&{else} \&{begin} \37\C{the parent exists on \\{cite\_list}}\6
\&{if} $(\\{cite\_loc}\I\\{ilk\_info}[\\{lc\_cite\_loc}])$ \1\&{then}\5
\\{hash\_cite\_confusion};\2\6
$\\{cite\_parent\_ptr}\K\\{ilk\_info}[\\{cite\_loc}]$;\6
\&{if} $(\\{type\_list}[\\{cite\_parent\_ptr}]=\\{empty})$ \1\&{then}\6
\&{begin} \37\\{nonexistent\_cross\_reference\_error};\6
$\\{field\_info}[\\{field\_ptr}]\K\\{missing}$;\C{remove the \.{crossref} ptr}\6
\&{end}\6
\4\&{else} \&{begin} \37\C{the parent exists in the database too}\6
$\\{field\_parent\_ptr}\K\\{cite\_parent\_ptr}\ast\\{num\_fields}+\\{crossref%
\_num}$;\6
\&{if} $(\\{field\_info}[\\{field\_parent\_ptr}]\I\\{missing})$ \1\&{then}\5
\X282:Complain about a nested cross reference\X;\2\6
\&{if} $((\R\\{all\_entries})\W(\\{cite\_parent\_ptr}\G\\{old\_num\_cites})\W(%
\\{cite\_info}[\\{cite\_parent\_ptr}]<\\{min\_crossrefs}))$ \1\&{then}\6
$\\{field\_info}[\\{field\_ptr}]\K\\{missing}$;\C{remove the \.{crossref} ptr}%
\2\6
\&{end};\2\6
\&{end};\2\2\6
$\\{incr}(\\{cite\_ptr})$;\6
\&{end};\2\6
\&{end}\par
\U276.\fi

\M280.
This procedure exists to save space, since it's used twice---once for
each of the two succeeding modules.

\Y\P$\4\X3:Procedures and functions for all file I/O, error messages, and such%
\X\mathrel{+}\S$\6
\4\&{procedure}\1\  \37$\\{bad\_cross\_reference\_print}(\|s:\\{str\_number})$;%
\2\6
\&{begin} \37$\\{print}(\.{\'--entry\ "\'})$;\5
$\\{print\_pool\_str}(\\{cur\_cite\_str})$;\5
$\\{print\_ln}(\.{\'"\'})$;\5
$\\{print}(\.{\'refers\ to\ entry\ "\'})$;\5
$\\{print\_pool\_str}(\|s)$;\6
\&{end};\par
\fi

\M281.
When an entry being cross referenced doesn't exist on \\{cite\_list}, we
complain.

\Y\P$\4\X3:Procedures and functions for all file I/O, error messages, and such%
\X\mathrel{+}\S$\6
\4\&{procedure}\1\  \37\\{nonexistent\_cross\_reference\_error};\2\6
\&{begin} \37$\\{print}(\.{\'A\ bad\ cross\ reference-\'})$;\5
$\\{bad\_cross\_reference\_print}(\\{field\_info}[\\{field\_ptr}])$;\5
$\\{print\_ln}(\.{\'",\ which\ doesn\'}\.{\'t\ exist\'})$;\5
\\{mark\_error};\6
\&{end};\par
\fi

\M282.
We also complain when an entry being cross referenced has a
non\\{missing} \.{crossref} field itself, but this one is just a
warning, not a full-blown error.

\Y\P$\4\X282:Complain about a nested cross reference\X\S$\6
\&{begin} \37$\\{print}(\.{\'Warning--you\'}\.{\'ve\ nested\ cross\ references%
\'})$;\5
$\\{bad\_cross\_reference\_print}(\\{cite\_list}[\\{cite\_parent\_ptr}])$;\5
$\\{print\_ln}(\.{\'",\ which\ also\ refers\ to\ something\'})$;\5
\\{mark\_warning};\6
\&{end}\par
\U279.\fi

\M283.
We remove (and give a warning for) each cite key on the original
\\{cite\_list} without a corresponding database entry.  And we remove any
entry that was included on \\{cite\_list} only because it was
cross~referenced, yet was cross~referenced fewer than \\{min\_crossrefs}
times.  Throughout this module, \\{cite\_ptr} points to the next cite key
to be checked and \\{cite\_xptr} points to the next permanent spot on
\\{cite\_list}.

\Y\P$\4\X283:Remove missing entries or those cross referenced too few times\X%
\S$\6
\&{begin} \37$\\{cite\_ptr}\K0$;\6
\&{while} $(\\{cite\_ptr}<\\{num\_cites})$ \1\&{do}\6
\&{begin} \37\&{if} $(\\{type\_list}[\\{cite\_ptr}]=\\{empty})$ \1\&{then}\5
$\\{print\_missing\_entry}(\\{cur\_cite\_str})$\6
\4\&{else} \&{if} $((\\{all\_entries})\V(\\{cite\_ptr}<\\{old\_num\_cites})\V(%
\\{cite\_info}[\\{cite\_ptr}]\G\\{min\_crossrefs}))$ \1\&{then}\6
\&{begin} \37\&{if} $(\\{cite\_ptr}>\\{cite\_xptr})$ \1\&{then}\5
\X285:Slide this cite key down to its permanent spot\X;\2\6
$\\{incr}(\\{cite\_xptr})$;\6
\&{end};\2\2\6
$\\{incr}(\\{cite\_ptr})$;\6
\&{end};\2\6
$\\{num\_cites}\K\\{cite\_xptr}$;\6
\&{if} $(\\{all\_entries})$ \1\&{then}\5
\X286:Complain about missing entries whose cite keys got overwritten\X;\2\6
\&{end}\par
\U276.\fi

\M284.
When a cite key on the original \\{cite\_list} (or added to \\{cite\_list}
because of cross~referencing) didn't appear in the database, complain.

\Y\P$\4\X3:Procedures and functions for all file I/O, error messages, and such%
\X\mathrel{+}\S$\6
\4\&{procedure}\1\  \37$\\{print\_missing\_entry}(\|s:\\{str\_number})$;\2\6
\&{begin} \37$\\{print}(\.{\'Warning--I\ didn\'}\.{\'t\ find\ a\ database\
entry\ for\ "\'})$;\5
$\\{print\_pool\_str}(\|s)$;\5
$\\{print\_ln}(\.{\'"\'})$;\5
\\{mark\_warning};\6
\&{end};\par
\fi

\M285.
We have to move to its final resting place all the entry information
associated with the exact location in \\{cite\_list} of this cite key.

\Y\P$\4\X285:Slide this cite key down to its permanent spot\X\S$\6
\&{begin} \37$\\{cite\_list}[\\{cite\_xptr}]\K\\{cite\_list}[\\{cite\_ptr}]$;\5
$\\{type\_list}[\\{cite\_xptr}]\K\\{type\_list}[\\{cite\_ptr}]$;\6
\&{if} $(\R\\{find\_cite\_locs\_for\_this\_cite\_key}(\\{cite\_list}[\\{cite%
\_ptr}]))$ \1\&{then}\5
\\{cite\_key\_disappeared\_confusion};\2\6
\&{if} $((\R\\{cite\_hash\_found})\V(\\{cite\_loc}\I\\{ilk\_info}[\\{lc\_cite%
\_loc}]))$ \1\&{then}\5
\\{hash\_cite\_confusion};\2\6
$\\{ilk\_info}[\\{cite\_loc}]\K\\{cite\_xptr}$;\6
$\\{field\_ptr}\K\\{cite\_xptr}\ast\\{num\_fields}$;\5
$\\{field\_end\_ptr}\K\\{field\_ptr}+\\{num\_fields}$;\5
$\\{tmp\_ptr}\K\\{cite\_ptr}\ast\\{num\_fields}$;\6
\&{while} $(\\{field\_ptr}<\\{field\_end\_ptr})$ \1\&{do}\6
\&{begin} \37$\\{field\_info}[\\{field\_ptr}]\K\\{field\_info}[\\{tmp\_ptr}]$;\5
$\\{incr}(\\{field\_ptr})$;\5
$\\{incr}(\\{tmp\_ptr})$;\6
\&{end};\2\6
\&{end}\par
\U283.\fi

\M286.
We need this module only when we're including the whole database.
It's for missing entries whose cite key originally resided in
\\{cite\_list} at a spot that another cite key (might have) claimed.

\Y\P$\4\X286:Complain about missing entries whose cite keys got overwritten\X%
\S$\6
\&{begin} \37$\\{cite\_ptr}\K\\{all\_marker}$;\6
\&{while} $(\\{cite\_ptr}<\\{old\_num\_cites})$ \1\&{do}\6
\&{begin} \37\&{if} $(\R\\{entry\_exists}[\\{cite\_ptr}])$ \1\&{then}\5
$\\{print\_missing\_entry}(\\{cite\_info}[\\{cite\_ptr}])$;\2\6
$\\{incr}(\\{cite\_ptr})$;\6
\&{end};\2\6
\&{end}\par
\U283.\fi

\M287.
This module initializes all \\{int\_entry\_var}s of all entries to 0, the
value to which all integers are initialized.

\Y\P$\4\X287:Initialize the \\{int\_entry\_var}s\X\S$\6
\&{begin} \37\&{if} $(\\{num\_ent\_ints}\ast\\{num\_cites}>\\{max\_ent\_ints})$
\1\&{then}\6
\&{begin} \37$\\{print}(\\{num\_ent\_ints}\ast\\{num\_cites},\39\.{\':\ \'})$;\5
$\\{overflow}(\.{\'total\ number\ of\ integer\ entry-variables\ \'},\39\\{max%
\_ent\_ints})$;\6
\&{end};\2\6
$\\{int\_ent\_ptr}\K0$;\6
\&{while} $(\\{int\_ent\_ptr}<\\{num\_ent\_ints}\ast\\{num\_cites})$ \1\&{do}\6
\&{begin} \37$\\{entry\_ints}[\\{int\_ent\_ptr}]\K0$;\5
$\\{incr}(\\{int\_ent\_ptr})$;\6
\&{end};\2\6
\&{end}\par
\U276.\fi

\M288.
This module initializes all \\{str\_entry\_var}s of all entries to the
null string, the value to which all strings are initialized.

\Y\P$\4\X288:Initialize the \\{str\_entry\_var}s\X\S$\6
\&{begin} \37\&{if} $(\\{num\_ent\_strs}\ast\\{num\_cites}>\\{max\_ent\_strs})$
\1\&{then}\6
\&{begin} \37$\\{print}(\\{num\_ent\_strs}\ast\\{num\_cites},\39\.{\':\ \'})$;\5
$\\{overflow}(\.{\'total\ number\ of\ string\ entry-variables\ \'},\39\\{max%
\_ent\_strs})$;\6
\&{end};\2\6
$\\{str\_ent\_ptr}\K0$;\6
\&{while} $(\\{str\_ent\_ptr}<\\{num\_ent\_strs}\ast\\{num\_cites})$ \1\&{do}\6
\&{begin} \37$\\{entry\_strs}[\\{str\_ent\_ptr}][0]\K\\{end\_of\_string}$;\5
$\\{incr}(\\{str\_ent\_ptr})$;\6
\&{end};\2\6
\&{end}\par
\U276.\fi

\M289.
The array \\{sorted\_cites} initially specifies that the entries are to
be processed in order of cite-key occurrence.  The \.{sort} command
may change this to whatever it likes (which, we hope, is whatever the
style-designer instructs it to like).  We make \\{sorted\_cites} an alias
to save space; this works fine because we're done with \\{cite\_info}.

\Y\P\D \37$\\{sorted\_cites}\S\\{cite\_info}$\C{an alias used for the rest of
the program}\par
\Y\P$\4\X289:Initialize the \\{sorted\_cites}\X\S$\6
\&{begin} \37$\\{cite\_ptr}\K0$;\6
\&{while} $(\\{cite\_ptr}<\\{num\_cites})$ \1\&{do}\6
\&{begin} \37$\\{sorted\_cites}[\\{cite\_ptr}]\K\\{cite\_ptr}$;\5
$\\{incr}(\\{cite\_ptr})$;\6
\&{end};\2\6
\&{end}\par
\U276.\fi

\N290.  Executing the style file.
This part of the program produces the output by executing the
\.{.bst}-file commands \.{execute}, \.{iterate}, \.{reverse}, and
\.{sort}.  To do this it uses a stack (consisting of the two arrays
\\{lit\_stack} and \\{lit\_stk\_type}) for storing literals, a buffer
\\{ex\_buf} for manipulating strings, and an array \\{sorted\_cites}
for holding pointers to the sorted cite keys (\\{sorted\_cites} is an
alias of \\{cite\_info}).

\Y\P$\4\X16:Globals in the outer block\X\mathrel{+}\S$\6
\4\\{lit\_stack}: \37\&{array} $[\\{lit\_stk\_loc}]$ \1\&{of}\5
\\{integer};\C{the literal function stack}\2\6
\4\\{lit\_stk\_type}: \37\&{array} $[\\{lit\_stk\_loc}]$ \1\&{of}\5
\\{stk\_type};\C{their corresponding types}\2\6
\4\\{lit\_stk\_ptr}: \37\\{lit\_stk\_loc};\C{points just above the top of the
stack}\6
\4\\{cmd\_str\_ptr}: \37\\{str\_number};\C{stores value of \\{str\_ptr} during
execution}\6
\4\\{ent\_chr\_ptr}: \37$0\to\\{ent\_str\_size}$;\C{points at a \\{str\_entry%
\_var} character}\6
\4\\{glob\_chr\_ptr}: \37$0\to\\{glob\_str\_size}$;\C{points at a \\{str%
\_global\_var} character}\6
\4\\{ex\_buf}: \37\\{buf\_type};\C{a buffer for manipulating strings}\6
\4\\{ex\_buf\_ptr}: \37\\{buf\_pointer};\C{general \\{ex\_buf} location}\6
\4\\{ex\_buf\_length}: \37\\{buf\_pointer};\C{the length of the current string
in \\{ex\_buf}}\6
\4\\{out\_buf}: \37\\{buf\_type};\C{the \.{.bbl} output buffer}\6
\4\\{out\_buf\_ptr}: \37\\{buf\_pointer};\C{general \\{out\_buf} location}\6
\4\\{out\_buf\_length}: \37\\{buf\_pointer};\C{the length of the current string
in \\{out\_buf}}\6
\4\\{mess\_with\_entries}: \37\\{boolean};\C{\\{true} if functions can use
entry info}\6
\4\\{sort\_cite\_ptr}: \37\\{cite\_number};\C{a loop index for the sorted cite
keys}\6
\4\\{sort\_key\_num}: \37\\{str\_ent\_loc};\C{index for the \\{str\_entry\_var}
\.{sort.key\$}}\6
\4\\{brace\_level}: \37\\{integer};\C{the brace nesting depth within a string}%
\par
\fi

\M291.
Where \\{lit\_stk\_loc} is a stack location, and where \\{stk\_type} gives
one of the three types of literals (an integer, a string, or a
function) or a special marker.  If a \\{lit\_stk\_type} element is a
\\{stk\_int} then the corresponding \\{lit\_stack} element is an integer; if
a \\{stk\_str}, then a pointer to a \\{str\_pool} string; and if a \\{stk\_fn},
then a pointer to the function's hash-table location.  However, if the
literal should have been a \\{stk\_str} that was the value of a field
that happened to be \\{missing}, then the special value
\\{stk\_field\_missing} goes on the stack instead; its corresponding
\\{lit\_stack} element is a pointer to the field-name's string.  Finally,
\\{stk\_empty} is the type of a literal popped from an empty stack.

\Y\P\D \37$\\{stk\_int}=0$\C{an integer literal}\par
\P\D \37$\\{stk\_str}=1$\C{a string literal}\par
\P\D \37$\\{stk\_fn}=2$\C{a function literal}\par
\P\D \37$\\{stk\_field\_missing}=3$\C{a special marker: a field value was
missing}\par
\P\D \37$\\{stk\_empty}=4$\C{another: the stack was empty when this was popped}%
\par
\P\D \37$\\{last\_lit\_type}=4$\C{the same number as on the line above}\par
\Y\P$\4\X22:Types in the outer block\X\mathrel{+}\S$\6
$\\{lit\_stk\_loc}=0\to\\{lit\_stk\_size}$;\C{the stack range}\6
$\\{stk\_type}=0\to\\{last\_lit\_type}$;\C{the literal types}\par
\fi

\M292.
And the first output line requires this initialization.

\Y\P$\4\X20:Set initial values of key variables\X\mathrel{+}\S$\6
$\\{out\_buf\_length}\K0$;\par
\fi

\M293.
When there's an error while executing \.{.bst} functions, what we do
depends on whether the function is messing with the entries.
Furthermore this error is serious enough to classify as an
\\{error\_message} instead of a \\{warning\_message}.  These messages (that
is, from \\{bst\_ex\_warn}) are meant both for the user and for the style
designer while debugging.

\Y\P\D \37$\\{bst\_ex\_warn}(\#)\S$\1\6
\&{begin} \37\C{error while executing some function}\6
$\\{print}(\#)$;\5
\\{bst\_ex\_warn\_print};\6
\&{end}\2\par
\Y\P$\4\X3:Procedures and functions for all file I/O, error messages, and such%
\X\mathrel{+}\S$\6
\4\&{procedure}\1\  \37\\{bst\_ex\_warn\_print};\2\6
\&{begin} \37\&{if} $(\\{mess\_with\_entries})$ \1\&{then}\6
\&{begin} \37$\\{print}(\.{\'\ for\ entry\ \'})$;\5
$\\{print\_pool\_str}(\\{cur\_cite\_str})$;\6
\&{end};\2\6
\\{print\_newline};\5
$\\{print}(\.{\'while\ executing-\'})$;\5
\\{bst\_ln\_num\_print};\5
\\{mark\_error};\6
\&{end};\par
\fi

\M294.
When an error is so harmless, we print a \\{warning\_message} instead of
an \\{error\_message}.

\Y\P\D \37$\\{bst\_mild\_ex\_warn}(\#)\S$\1\6
\&{begin} \37\C{error while executing some function}\6
$\\{print}(\#)$;\5
\\{bst\_mild\_ex\_warn\_print};\6
\&{end}\2\par
\Y\P$\4\X3:Procedures and functions for all file I/O, error messages, and such%
\X\mathrel{+}\S$\6
\4\&{procedure}\1\  \37\\{bst\_mild\_ex\_warn\_print};\2\6
\&{begin} \37\&{if} $(\\{mess\_with\_entries})$ \1\&{then}\6
\&{begin} \37$\\{print}(\.{\'\ for\ entry\ \'})$;\5
$\\{print\_pool\_str}(\\{cur\_cite\_str})$;\6
\&{end};\2\6
\\{print\_newline};\5
$\\{bst\_warn}(\.{\'while\ executing\'})$;\C{This does the \\{mark\_warning}}\6
\&{end};\par
\fi

\M295.
It's illegal to mess with the entry information at certain times;
here's a complaint for these times.

\Y\P$\4\X3:Procedures and functions for all file I/O, error messages, and such%
\X\mathrel{+}\S$\6
\4\&{procedure}\1\  \37\\{bst\_cant\_mess\_with\_entries\_print};\2\6
\&{begin} \37$\\{bst\_ex\_warn}(\.{\'You\ can\'}\.{\'t\ mess\ with\ entries\
here\'})$;\6
\&{end};\par
\fi

\M296.
This module executes a single specified function once.  It can't do
anything with the entries.

\Y\P$\4\X296:Perform an \.{execute} command\X\S$\6
\&{begin} \37\\{init\_command\_execution};\5
$\\{mess\_with\_entries}\K\\{false}$;\5
$\\{execute\_fn}(\\{fn\_loc})$;\5
\\{check\_command\_execution};\6
\&{end}\par
\U178.\fi

\M297.
This module iterates a single specified function for all entries
specified by \\{cite\_list}.

\Y\P$\4\X297:Perform an \.{iterate} command\X\S$\6
\&{begin} \37\\{init\_command\_execution};\5
$\\{mess\_with\_entries}\K\\{true}$;\5
$\\{sort\_cite\_ptr}\K0$;\6
\&{while} $(\\{sort\_cite\_ptr}<\\{num\_cites})$ \1\&{do}\6
\&{begin} \37$\\{cite\_ptr}\K\\{sorted\_cites}[\\{sort\_cite\_ptr}]$;\6
\&{trace} \37$\\{trace\_pr\_pool\_str}(\\{hash\_text}[\\{fn\_loc}])$;\5
$\\{trace\_pr}(\.{\'\ to\ be\ iterated\ on\ \'})$;\5
$\\{trace\_pr\_pool\_str}(\\{cur\_cite\_str})$;\5
\\{trace\_pr\_newline};\6
\&{ecart}\6
$\\{execute\_fn}(\\{fn\_loc})$;\5
\\{check\_command\_execution};\5
$\\{incr}(\\{sort\_cite\_ptr})$;\6
\&{end};\2\6
\&{end}\par
\U203.\fi

\M298.
This module iterates a single specified function for all entries
specified by \\{cite\_list}, but does it in reverse order.

\Y\P$\4\X298:Perform a \.{reverse} command\X\S$\6
\&{begin} \37\\{init\_command\_execution};\5
$\\{mess\_with\_entries}\K\\{true}$;\6
\&{if} $(\\{num\_cites}>0)$ \1\&{then}\6
\&{begin} \37$\\{sort\_cite\_ptr}\K\\{num\_cites}$;\6
\1\&{repeat} \37$\\{decr}(\\{sort\_cite\_ptr})$;\5
$\\{cite\_ptr}\K\\{sorted\_cites}[\\{sort\_cite\_ptr}]$;\6
\&{trace} \37$\\{trace\_pr\_pool\_str}(\\{hash\_text}[\\{fn\_loc}])$;\5
$\\{trace\_pr}(\.{\'\ to\ be\ iterated\ in\ reverse\ on\ \'})$;\5
$\\{trace\_pr\_pool\_str}(\\{cur\_cite\_str})$;\5
\\{trace\_pr\_newline};\6
\&{ecart}\6
$\\{execute\_fn}(\\{fn\_loc})$;\5
\\{check\_command\_execution};\6
\4\&{until}\5
$(\\{sort\_cite\_ptr}=0)$;\2\6
\&{end};\2\6
\&{end}\par
\U212.\fi

\M299.
This module sorts the entries based on \.{sort.key\$}; it is a stable
sort.

\Y\P$\4\X299:Perform a \.{sort} command\X\S$\6
\&{begin} \37\&{trace} \37$\\{trace\_pr\_ln}(\.{\'Sorting\ the\ entries\'})$;\6
\&{ecart}\6
\&{if} $(\\{num\_cites}>1)$ \1\&{then}\5
$\\{quick\_sort}(0,\39\\{num\_cites}-1)$;\2\6
\&{trace} \37$\\{trace\_pr\_ln}(\.{\'Done\ sorting\'})$;\6
\&{ecart}\6
\&{end}\par
\U214.\fi

\M300.
These next two procedures (actually, one procedures and one function,
but who's counting) are subroutines for \\{quick\_sort}, which follows.
The \\{swap} procedure exchanges the two elements its arguments point
to.

\Y\P$\4\X54:Procedures and functions for handling numbers, characters, and
strings\X\mathrel{+}\S$\6
\4\&{procedure}\1\  \37$\\{swap}(\\{swap1},\39\\{swap2}:\\{cite\_number})$;\6
\4\&{var} \37\\{innocent\_bystander}: \37\\{cite\_number};\C{the temporary
element in an exchange}\2\6
\&{begin} \37$\\{innocent\_bystander}\K\\{sorted\_cites}[\\{swap2}]$;\5
$\\{sorted\_cites}[\\{swap2}]\K\\{sorted\_cites}[\\{swap1}]$;\5
$\\{sorted\_cites}[\\{swap1}]\K\\{innocent\_bystander}$;\6
\&{end};\par
\fi

\M301.
The function \\{less\_than} compares the two \.{sort.key\$}s indirectly
pointed to by its arguments and returns \\{true} if the first argument's
\.{sort.key\$} is lexicographically less than the second's (that is,
alphabetically earlier).  In case of ties the function compares the
indices \\{arg1} and \\{arg2}, which are assumed to be different, and
returns \\{true} if the first is smaller.  This function uses
\\{ASCII\_code}s to compare, so it might give ``interesting'' results
when handling nonletters.

\Y\P\D \37$\\{compare\_return}(\#)\S$\1\6
\&{begin} \37\C{the compare is finished}\6
$\\{less\_than}\K\#$;\5
\&{return};\6
\&{end}\2\par
\Y\P$\4\X54:Procedures and functions for handling numbers, characters, and
strings\X\mathrel{+}\S$\6
\4\&{function}\1\  \37$\\{less\_than}(\\{arg1},\39\\{arg2}:\\{cite\_number})$: %
\37\\{boolean};\6
\4\&{label} \37\\{exit};\6
\4\&{var} \37\\{char\_ptr}: \37$0\to\\{ent\_str\_size}$;\C{character index into
compared strings}\6
$\\{ptr1},\39\\{ptr2}$: \37\\{str\_ent\_loc};\C{the two \.{sort.key\$}
pointers}\6
$\\{char1},\39\\{char2}$: \37\\{ASCII\_code};\C{the two characters being
compared}\2\6
\&{begin} \37$\\{ptr1}\K\\{arg1}\ast\\{num\_ent\_strs}+\\{sort\_key\_num}$;\5
$\\{ptr2}\K\\{arg2}\ast\\{num\_ent\_strs}+\\{sort\_key\_num}$;\5
$\\{char\_ptr}\K0$;\6
\~ \1\&{loop}\6
\&{begin} \37$\\{char1}\K\\{entry\_strs}[\\{ptr1}][\\{char\_ptr}]$;\5
$\\{char2}\K\\{entry\_strs}[\\{ptr2}][\\{char\_ptr}]$;\6
\&{if} $(\\{char1}=\\{end\_of\_string})$ \1\&{then}\6
\&{if} $(\\{char2}=\\{end\_of\_string})$ \1\&{then}\6
\&{if} $(\\{arg1}<\\{arg2})$ \1\&{then}\5
$\\{compare\_return}(\\{true})$\6
\4\&{else} \&{if} $(\\{arg1}>\\{arg2})$ \1\&{then}\5
$\\{compare\_return}(\\{false})$\6
\4\&{else} \C{$\\{arg1}=\\{arg2}$}\2\2\2\2\6
$\\{confusion}(\.{\'Duplicate\ sort\ key\'})$\6
\4\&{else} \37\C{$\\{char2}\I\\{end\_of\_string}$}\6
$\\{compare\_return}(\\{true})$\6
\4\&{else} \37\C{$\\{char1}\I\\{end\_of\_string}$}\6
\&{if} $(\\{char2}=\\{end\_of\_string})$ \1\&{then}\5
$\\{compare\_return}(\\{false})$\6
\4\&{else} \&{if} $(\\{char1}<\\{char2})$ \1\&{then}\5
$\\{compare\_return}(\\{true})$\6
\4\&{else} \&{if} $(\\{char1}>\\{char2})$ \1\&{then}\5
$\\{compare\_return}(\\{false})$;\2\2\2\6
$\\{incr}(\\{char\_ptr})$;\6
\&{end};\2\6
\4\\{exit}: \37\&{end};\par
\fi

\M302.
The recursive procedure \\{quick\_sort} sorts the entries indirectly
pointed to by the \\{sorted\_cites} elements between \\{left\_end} and
\\{right\_end}, inclusive, based on the value of the \\{str\_entry\_var}
\.{sort.key\$}.  It's a fairly standard quicksort (for example, see
Algorithm 5.2.2Q in {\sl The Art of Computer Programming}), but uses
the median-of-three method to choose the partition element just in
case the entries are already sorted (or nearly sorted---humans and
ASCII might have different ideas on lexicographic ordering); it is a
stable sort.  This code generally prefers clarity to assembler-type
execution-time efficiency since \\{cite\_list}s will rarely be huge.

The value \\{short\_list}, which must be at least $2\ast\\{end\_offset}+2$ for
this code to work, tells us the list-length at which the list is small
enough to warrant switching over to straight insertion sort from the
recursive quicksort.  The values here come from modest empirical tests
aimed at minimizing, for large \\{cite\_list}s (five hundred or so), the
number of comparisons (between keys) plus the number of calls to
\\{quick\_sort}.  The value \\{end\_offset} must be positive; this helps
avoid $n^2$ behavior observed when the list starts out nearly, but not
completely, sorted (and fairly frequently large \\{cite\_list}s come from
entire databases, which fairly frequently are nearly sorted).

\Y\P\D \37$\\{short\_list}=10$\C{use straight insertion sort at or below this
length}\par
\P\D \37$\\{end\_offset}=4$\C{the index end-offsets for choosing a
median-of-three}\par
\Y\P$\4\X17:Check the ``constant'' values for consistency\X\mathrel{+}\S$\6
\&{if} $(\\{short\_list}<2\ast\\{end\_offset}+2)$ \1\&{then}\5
$\\{bad}\K100\ast\\{bad}+22$;\2\par
\fi

\M303.
Here's the actual procedure.

\Y\P\D \37$\\{next\_insert}=24$\C{now insert the next element}\par
\Y\P$\4\X54:Procedures and functions for handling numbers, characters, and
strings\X\mathrel{+}\S$\6
\4\&{procedure}\1\  \37$\\{quick\_sort}(\\{left\_end},\39\\{right\_end}:\\{cite%
\_number})$;\6
\4\&{label} \37\\{next\_insert};\6
\4\&{var} \37$\\{left},\39\\{right}$: \37\\{cite\_number};\C{two general %
\\{sorted\_cites} pointers}\6
\\{insert\_ptr}: \37\\{cite\_number};\C{the to-be-(straight)-inserted element}\6
\\{middle}: \37\\{cite\_number};\C{the $(\\{left\_end}+\\{right\_end})\mathbin{%
\&{div}}2$ element}\6
\\{partition}: \37\\{cite\_number};\C{the median-of-three partition element}\2\6
\&{begin} \37\&{trace} \37$\\{trace\_pr\_ln}(\.{\'Sorting\ \'},\39\\{left%
\_end}:0,\39\.{\'\ through\ \'},\39\\{right\_end}:0)$;\6
\&{ecart}\6
\&{if} $(\\{right\_end}-\\{left\_end}<\\{short\_list})$ \1\&{then}\5
\X304:Do a straight insertion sort\X\6
\4\&{else} \&{begin} \37\X305:Draw out the median-of-three partition element\X;%
\6
\X306:Do the partitioning and the recursive calls\X;\6
\&{end};\2\6
\&{end};\par
\fi

\M304.
This code sorts the entries between \\{left\_end} and \\{right\_end} when
the difference is less than \\{short\_list}.  Each iteration of the outer
loop inserts the element indicated by \\{insert\_ptr} into its proper
place among the (sorted) elements from \\{left\_end} through
$\\{insert\_ptr}-1$.

\Y\P$\4\X304:Do a straight insertion sort\X\S$\6
\&{begin} \37\&{for} $\\{insert\_ptr}\K\\{left\_end}+1\mathrel{\&{to}}\\{right%
\_end}$ \1\&{do}\6
\&{begin} \37\&{for} $\\{right}\K\\{insert\_ptr}\mathrel{\&{downto}}\\{left%
\_end}+1$ \1\&{do}\6
\&{begin} \37\&{if} $(\\{less\_than}(\\{sorted\_cites}[\\{right}-1],\39%
\\{sorted\_cites}[\\{right}]))$ \1\&{then}\5
\&{goto} \37\\{next\_insert};\2\6
$\\{swap}(\\{right}-1,\39\\{right})$;\6
\&{end};\2\6
\4\\{next\_insert}: \37\&{end};\2\6
\&{end}\par
\U303.\fi

\M305.
Now we find the median of the three \.{sort.key\$}s to which the three
elements $\\{sorted\_cites}[\\{left\_end}+\\{end\_offset}]$,
$\\{sorted\_cites}[\\{right\_end}]-\\{end\_offset}$, and
$\\{sorted\_cites}[(\\{left\_end}+\\{right\_end})\mathbin{\&{div}}2]$ point (a
nonzero
\\{end\_offset} avoids using as the leftmost of the three elements the
one that was swapped there when the old partition element was swapped
into its final spot; this turns out to avoid $n^2$ behavior when the
list is nearly sorted to start with).  This code determines which of
the six possible permutations we're dealing with and moves the median
element to \\{left\_end}.  The comments next to the \\{swap} actions give
the known orderings of the corresponding elements of \\{sorted\_cites}
before the action.

\Y\P$\4\X305:Draw out the median-of-three partition element\X\S$\6
\&{begin} \37$\\{left}\K\\{left\_end}+\\{end\_offset}$;\5
$\\{middle}\K(\\{left\_end}+\\{right\_end})\mathbin{\&{div}}2$;\5
$\\{right}\K\\{right\_end}-\\{end\_offset}$;\6
\&{if} $(\\{less\_than}(\\{sorted\_cites}[\\{left}],\39\\{sorted\_cites}[%
\\{middle}]))$ \1\&{then}\6
\&{if} $(\\{less\_than}(\\{sorted\_cites}[\\{middle}],\39\\{sorted\_cites}[%
\\{right}]))$ \1\&{then}\C{$\\{left}<\\{middle}<\\{right}$}\6
$\\{swap}(\\{left\_end},\39\\{middle})$\6
\4\&{else} \&{if} $(\\{less\_than}(\\{sorted\_cites}[\\{left}],\39\\{sorted%
\_cites}[\\{right}]))$ \1\&{then}\C{$\\{left}<\\{right}<\\{middle}$}\6
$\\{swap}(\\{left\_end},\39\\{right})$\6
\4\&{else} \C{$\\{right}<\\{left}<\\{middle}$}\2\2\2\6
$\\{swap}(\\{left\_end},\39\\{left})$\6
\4\&{else} \37\&{if} $(\\{less\_than}(\\{sorted\_cites}[\\{right}],\39\\{sorted%
\_cites}[\\{middle}]))$ \1\&{then}\C{$\\{right}<\\{middle}<\\{left}$}\6
$\\{swap}(\\{left\_end},\39\\{middle})$\6
\4\&{else} \&{if} $(\\{less\_than}(\\{sorted\_cites}[\\{right}],\39\\{sorted%
\_cites}[\\{left}]))$ \1\&{then}\C{$\\{middle}<\\{right}<\\{left}$}\6
$\\{swap}(\\{left\_end},\39\\{right})$\6
\4\&{else} \C{$\\{middle}<\\{left}<\\{right}$}\2\2\6
$\\{swap}(\\{left\_end},\39\\{left})$;\6
\&{end}\par
\U303.\fi

\M306.
This module uses the median-of-three computed above to partition the
elements into those less than and those greater than the median.
Equal \.{sort.key\$}s are sorted by order of occurrence (in
\\{cite\_list}).

\Y\P$\4\X306:Do the partitioning and the recursive calls\X\S$\6
\&{begin} \37$\\{partition}\K\\{sorted\_cites}[\\{left\_end}]$;\5
$\\{left}\K\\{left\_end}+1$;\5
$\\{right}\K\\{right\_end}$;\6
\1\&{repeat} \37\&{while} $(\\{less\_than}(\\{sorted\_cites}[\\{left}],\39%
\\{partition}))$ \1\&{do}\5
$\\{incr}(\\{left})$;\2\6
\&{while} $(\\{less\_than}(\\{partition},\39\\{sorted\_cites}[\\{right}]))$ \1%
\&{do}\5
$\\{decr}(\\{right})$;\C{now $\\{sorted\_cites}[\\{right}]<\\{partition}<%
\\{sorted\_cites}[\\{left}]$}\2\6
\&{if} $(\\{left}<\\{right})$ \1\&{then}\6
\&{begin} \37$\\{swap}(\\{left},\39\\{right})$;\5
$\\{incr}(\\{left})$;\5
$\\{decr}(\\{right})$;\6
\&{end};\2\6
\4\&{until}\5
$(\\{left}=\\{right}+1)$;\C{pointers have crossed}\2\6
$\\{swap}(\\{left\_end},\39\\{right})$;\C{restoring the partition element to
its \\{right}ful place}\6
$\\{quick\_sort}(\\{left\_end},\39\\{right}-1)$;\5
$\\{quick\_sort}(\\{left},\39\\{right\_end})$;\6
\&{end}\par
\U303.\fi

\M307.
Ok, that's it for sorting; now we'll play with the literal stack.
This procedure pushes a literal onto the stack, checking for stack
overflow.

\Y\P$\4\X307:Procedures and functions for style-file function execution\X\S$\6
\4\&{procedure}\1\  \37$\\{push\_lit\_stk}(\\{push\_lt}:\\{integer};\,\35%
\\{push\_type}:\\{stk\_type})$; \6
\&{trace}  \6
\4\&{var} \37\\{dum\_ptr}: \37\\{lit\_stk\_loc};\C{used just as an index
variable}\6
\&{ecart} \6
\&{begin} \37$\\{lit\_stack}[\\{lit\_stk\_ptr}]\K\\{push\_lt}$;\5
$\\{lit\_stk\_type}[\\{lit\_stk\_ptr}]\K\\{push\_type}$;\6
\&{trace} \37\&{for} $\\{dum\_ptr}\K0\mathrel{\&{to}}\\{lit\_stk\_ptr}$ \1%
\&{do}\5
$\\{trace\_pr}(\.{\'\ \ \'})$;\2\6
$\\{trace\_pr}(\.{\'Pushing\ \'})$;\6
\&{case} $(\\{lit\_stk\_type}[\\{lit\_stk\_ptr}])$ \1\&{of}\6
\4\\{stk\_int}: \37$\\{trace\_pr\_ln}(\\{lit\_stack}[\\{lit\_stk\_ptr}]:0)$;\6
\4\\{stk\_str}: \37\&{begin} \37$\\{trace\_pr}(\.{\'"\'})$;\5
$\\{trace\_pr\_pool\_str}(\\{lit\_stack}[\\{lit\_stk\_ptr}])$;\5
$\\{trace\_pr\_ln}(\.{\'"\'})$;\6
\&{end};\6
\4\\{stk\_fn}: \37\&{begin} \37$\\{trace\_pr}(\.{\'\`\'})$;\5
$\\{trace\_pr\_pool\_str}(\\{hash\_text}[\\{lit\_stack}[\\{lit\_stk\_ptr}]])$;\5
$\\{trace\_pr\_ln}(\.{\'\'}\.{\'\'})$;\6
\&{end};\6
\4\\{stk\_field\_missing}: \37\&{begin} \37$\\{trace\_pr}(\.{\'missing\ field\ %
\`\'})$;\5
$\\{trace\_pr\_pool\_str}(\\{lit\_stack}[\\{lit\_stk\_ptr}])$;\5
$\\{trace\_pr\_ln}(\.{\'\'}\.{\'\'})$;\6
\&{end};\6
\4\\{stk\_empty}: \37$\\{trace\_pr\_ln}(\.{\'a\ bad\ literal--popped\ from\ an\
empty\ stack\'})$;\6
\4\&{othercases} \37\\{unknwn\_literal\_confusion}\2\6
\&{endcases};\6
\&{ecart}\6
\&{if} $(\\{lit\_stk\_ptr}=\\{lit\_stk\_size})$ \1\&{then}\5
$\\{overflow}(\.{\'literal-stack\ size\ \'},\39\\{lit\_stk\_size})$;\2\6
$\\{incr}(\\{lit\_stk\_ptr})$;\6
\&{end};\par
\As309, 312, 314, 315, 316, 317, 318, 320, 322\ETs342.
\U12.\fi

\M308.
This macro pushes the last thing, necessarily a string, that was
popped.  And this module, along with others that push the literal
stack without explicitly calling \\{push\_lit\_stack}, have an index entry
under ``push the literal stack''; these implicit pushes collectively
speed up the program by about ten percent.

\Y\P\D \37$\\{repush\_string}\S$\1\6
\&{begin} \37\&{if} $(\\{lit\_stack}[\\{lit\_stk\_ptr}]\G\\{cmd\_str\_ptr})$ \1%
\&{then}\5
\\{unflush\_string};\2\6
$\\{incr}(\\{lit\_stk\_ptr})$;\6
\&{end}\2\par
\fi

\M309.
This procedure pops the stack, checking for, and trying to recover
from, stack underflow.  (Actually, this procedure is really a
function, since it returns the two values through its  \&{var}
parameters.)  Also, if the literal being popped is a \\{stk\_str} that's
been created during the execution of the current \.{.bst} command, pop
it from \\{str\_pool} as well (it will be the string corresponding to
$\\{str\_ptr}-1$).  Note that when this happens, the string is no longer
`officially' available so that it must be used before anything else is
added to \\{str\_pool}.

\Y\P$\4\X307:Procedures and functions for style-file function execution\X%
\mathrel{+}\S$\6
\4\&{procedure}\1\  \37$\\{pop\_lit\_stk}(\mathop{\&{var}}\\{pop\_lit}:%
\\{integer};\,\35\mathop{\&{var}}\\{pop\_type}:\\{stk\_type})$;\2\6
\&{begin} \37\&{if} $(\\{lit\_stk\_ptr}=0)$ \1\&{then}\6
\&{begin} \37$\\{bst\_ex\_warn}(\.{\'You\ can\'}\.{\'t\ pop\ an\ empty\ literal%
\ stack\'})$;\6
$\\{pop\_type}\K\\{stk\_empty}$;\C{this is an error recovery attempt}\6
\&{end}\6
\4\&{else} \&{begin} \37$\\{decr}(\\{lit\_stk\_ptr})$;\5
$\\{pop\_lit}\K\\{lit\_stack}[\\{lit\_stk\_ptr}]$;\5
$\\{pop\_type}\K\\{lit\_stk\_type}[\\{lit\_stk\_ptr}]$;\6
\&{if} $(\\{pop\_type}=\\{stk\_str})$ \1\&{then}\6
\&{if} $(\\{pop\_lit}\G\\{cmd\_str\_ptr})$ \1\&{then}\6
\&{begin} \37\&{if} $(\\{pop\_lit}\I\\{str\_ptr}-1)$ \1\&{then}\5
$\\{confusion}(\.{\'Nontop\ top\ of\ string\ stack\'})$;\2\6
\\{flush\_string};\6
\&{end};\2\2\6
\&{end};\2\6
\&{end};\par
\fi

\M310.
More bug complaints, this time about bad literals.

\Y\P$\4\X3:Procedures and functions for all file I/O, error messages, and such%
\X\mathrel{+}\S$\6
\4\&{procedure}\1\  \37\\{illegl\_literal\_confusion};\2\6
\&{begin} \37$\\{confusion}(\.{\'Illegal\ literal\ type\'})$;\6
\&{end};\7
\4\&{procedure}\1\  \37\\{unknwn\_literal\_confusion};\2\6
\&{begin} \37$\\{confusion}(\.{\'Unknown\ literal\ type\'})$;\6
\&{end};\par
\fi

\M311.
Occasionally we'll want to know what's on the literal stack.  Here we
print out a stack literal, giving its type.  This procedure should
never be called after popping an empty stack.

\Y\P$\4\X3:Procedures and functions for all file I/O, error messages, and such%
\X\mathrel{+}\S$\6
\4\&{procedure}\1\  \37$\\{print\_stk\_lit}(\\{stk\_lt}:\\{integer};\,\35\\{stk%
\_tp}:\\{stk\_type})$;\2\6
\&{begin} \37\&{case} $(\\{stk\_tp})$ \1\&{of}\6
\4\\{stk\_int}: \37$\\{print}(\\{stk\_lt}:0,\39\.{\'\ is\ an\ integer\ literal%
\'})$;\6
\4\\{stk\_str}: \37\&{begin} \37$\\{print}(\.{\'"\'})$;\5
$\\{print\_pool\_str}(\\{stk\_lt})$;\5
$\\{print}(\.{\'"\ is\ a\ string\ literal\'})$;\6
\&{end};\6
\4\\{stk\_fn}: \37\&{begin} \37$\\{print}(\.{\'\`\'})$;\5
$\\{print\_pool\_str}(\\{hash\_text}[\\{stk\_lt}])$;\5
$\\{print}(\.{\'\'}\.{\'\ is\ a\ function\ literal\'})$;\6
\&{end};\6
\4\\{stk\_field\_missing}: \37\&{begin} \37$\\{print}(\.{\'\`\'})$;\5
$\\{print\_pool\_str}(\\{stk\_lt})$;\5
$\\{print}(\.{\'\'}\.{\'\ is\ a\ missing\ field\'})$;\6
\&{end};\6
\4\\{stk\_empty}: \37\\{illegl\_literal\_confusion};\6
\4\&{othercases} \37\\{unknwn\_literal\_confusion}\2\6
\&{endcases};\6
\&{end};\par
\fi

\M312.
This procedure appropriately chastises the style designer; however, if
the wrong literal came from popping an empty stack, the procedure
\\{pop\_lit\_stack} will have already done the chastising (because this
procedure is called only after popping the stack) so there's no need
for more.

\Y\P$\4\X307:Procedures and functions for style-file function execution\X%
\mathrel{+}\S$\6
\4\&{procedure}\1\  \37$\\{print\_wrong\_stk\_lit}(\\{stk\_lt}:\\{integer};\,%
\35\\{stk\_tp1},\39\\{stk\_tp2}:\\{stk\_type})$;\2\6
\&{begin} \37\&{if} $(\\{stk\_tp1}\I\\{stk\_empty})$ \1\&{then}\6
\&{begin} \37$\\{print\_stk\_lit}(\\{stk\_lt},\39\\{stk\_tp1})$;\6
\&{case} $(\\{stk\_tp2})$ \1\&{of}\6
\4\\{stk\_int}: \37$\\{print}(\.{\',\ not\ an\ integer,\'})$;\6
\4\\{stk\_str}: \37$\\{print}(\.{\',\ not\ a\ string,\'})$;\6
\4\\{stk\_fn}: \37$\\{print}(\.{\',\ not\ a\ function,\'})$;\6
\4$\\{stk\_field\_missing},\39\\{stk\_empty}$: \37\\{illegl\_literal%
\_confusion};\6
\4\&{othercases} \37\\{unknwn\_literal\_confusion}\2\6
\&{endcases};\5
\\{bst\_ex\_warn\_print};\6
\&{end};\2\6
\&{end};\par
\fi

\M313.
This is similar to \\{print\_stk\_lit}, but here we don't give the
literal's type, and here we end with a new line.  This procedure
should never be called after popping an empty stack.

\Y\P$\4\X3:Procedures and functions for all file I/O, error messages, and such%
\X\mathrel{+}\S$\6
\4\&{procedure}\1\  \37$\\{print\_lit}(\\{stk\_lt}:\\{integer};\,\35\\{stk%
\_tp}:\\{stk\_type})$;\2\6
\&{begin} \37\&{case} $(\\{stk\_tp})$ \1\&{of}\6
\4\\{stk\_int}: \37$\\{print\_ln}(\\{stk\_lt}:0)$;\6
\4\\{stk\_str}: \37\&{begin} \37$\\{print\_pool\_str}(\\{stk\_lt})$;\5
\\{print\_newline};\6
\&{end};\6
\4\\{stk\_fn}: \37\&{begin} \37$\\{print\_pool\_str}(\\{hash\_text}[\\{stk%
\_lt}])$;\5
\\{print\_newline};\6
\&{end};\6
\4\\{stk\_field\_missing}: \37\&{begin} \37$\\{print\_pool\_str}(\\{stk\_lt})$;%
\5
\\{print\_newline};\6
\&{end};\6
\4\\{stk\_empty}: \37\\{illegl\_literal\_confusion};\6
\4\&{othercases} \37\\{unknwn\_literal\_confusion}\2\6
\&{endcases};\6
\&{end};\par
\fi

\M314.
This procedure pops and prints the top of the stack; when the stack is
empty the procedure \\{pop\_lit\_stk} complains.

\Y\P$\4\X307:Procedures and functions for style-file function execution\X%
\mathrel{+}\S$\6
\4\&{procedure}\1\  \37\\{pop\_top\_and\_print};\6
\4\&{var} \37\\{stk\_lt}: \37\\{integer};\5
\\{stk\_tp}: \37\\{stk\_type};\2\6
\&{begin} \37$\\{pop\_lit\_stk}(\\{stk\_lt},\39\\{stk\_tp})$;\6
\&{if} $(\\{stk\_tp}=\\{stk\_empty})$ \1\&{then}\5
$\\{print\_ln}(\.{\'Empty\ literal\'})$\6
\4\&{else} $\\{print\_lit}(\\{stk\_lt},\39\\{stk\_tp})$;\2\6
\&{end};\par
\fi

\M315.
This procedure pops and prints the whole stack.

\Y\P$\4\X307:Procedures and functions for style-file function execution\X%
\mathrel{+}\S$\6
\4\&{procedure}\1\  \37\\{pop\_whole\_stack};\2\6
\&{begin} \37\&{while} $(\\{lit\_stk\_ptr}>0)$ \1\&{do}\5
\\{pop\_top\_and\_print};\2\6
\&{end};\par
\fi

\M316.
At the beginning of a \.{.bst}-command execution we make the stack
empty and record how much of \\{str\_pool} has been used.

\Y\P$\4\X307:Procedures and functions for style-file function execution\X%
\mathrel{+}\S$\6
\4\&{procedure}\1\  \37\\{init\_command\_execution};\2\6
\&{begin} \37$\\{lit\_stk\_ptr}\K0$;\C{make the stack empty}\6
$\\{cmd\_str\_ptr}\K\\{str\_ptr}$;\C{we'll check this when we finish command
execution}\6
\&{end};\par
\fi

\M317.
At the end of a \.{.bst} command-execution we check that the stack and
\\{str\_pool} are still in good shape.

\Y\P$\4\X307:Procedures and functions for style-file function execution\X%
\mathrel{+}\S$\6
\4\&{procedure}\1\  \37\\{check\_command\_execution};\2\6
\&{begin} \37\&{if} $(\\{lit\_stk\_ptr}\I0)$ \1\&{then}\6
\&{begin} \37$\\{print\_ln}(\.{\'ptr=\'},\39\\{lit\_stk\_ptr}:0,\39\.{\',\
stack=\'})$;\5
\\{pop\_whole\_stack};\5
$\\{bst\_ex\_warn}(\.{\'---the\ literal\ stack\ isn\'}\.{\'t\ empty\'})$;\6
\&{end};\2\6
\&{if} $(\\{cmd\_str\_ptr}\I\\{str\_ptr})$ \1\&{then}\6
\&{begin} \37\&{trace} \37$\\{print\_ln}(\.{\'Pointer\ is\ \'},\39\\{str%
\_ptr}:0,\39\.{\'\ but\ should\ be\ \'},\39\\{cmd\_str\_ptr}:0)$;\6
\&{ecart}\6
$\\{confusion}(\.{\'Nonempty\ empty\ string\ stack\'})$;\6
\&{end};\2\6
\&{end};\par
\fi

\M318.
This procedure adds to \\{str\_pool} the string from $\\{ex\_buf}[0]$ through
$\\{ex\_buf}[\\{ex\_buf\_length}-1]$ if it will fit.  It assumes the global
variable \\{ex\_buf\_length} gives the length of the current string in
\\{ex\_buf}.  It then pushes this string onto the literal stack.

\Y\P$\4\X307:Procedures and functions for style-file function execution\X%
\mathrel{+}\S$\6
\4\&{procedure}\1\  \37\\{add\_pool\_buf\_and\_push};\2\6
\&{begin} \37$\\{str\_room}(\\{ex\_buf\_length})$;\C{make sure this string will
fit}\6
$\\{ex\_buf\_ptr}\K0$;\6
\&{while} $(\\{ex\_buf\_ptr}<\\{ex\_buf\_length})$ \1\&{do}\6
\&{begin} \37$\\{append\_char}(\\{ex\_buf}[\\{ex\_buf\_ptr}])$;\5
$\\{incr}(\\{ex\_buf\_ptr})$;\6
\&{end};\2\6
$\\{push\_lit\_stk}(\\{make\_string},\39\\{stk\_str})$;\C{and push it onto the
stack}\6
\&{end};\par
\fi

\M319.
These macros append a character to \\{ex\_buf}.  Which is called depends
on whether the character is known to fit.

\Y\P\D \37$\\{append\_ex\_buf\_char}(\#)\S$\1\6
\&{begin} \37$\\{ex\_buf}[\\{ex\_buf\_ptr}]\K\#$;\5
$\\{incr}(\\{ex\_buf\_ptr})$;\6
\&{end}\2\par
\P\D \37$\\{append\_ex\_buf\_char\_and\_check}(\#)\S$\1\6
\&{begin} \37\&{if} $(\\{ex\_buf\_ptr}=\\{buf\_size})$ \1\&{then}\5
\\{buffer\_overflow};\2\6
$\\{append\_ex\_buf\_char}(\#)$;\6
\&{end}\2\par
\fi

\M320.
This procedure adds to the execution buffer the given string in
\\{str\_pool} if it will fit.  It assumes the global variable
\\{ex\_buf\_length} gives the length of the current string in \\{ex\_buf},
and thus also gives the location of the next character.

\Y\P$\4\X307:Procedures and functions for style-file function execution\X%
\mathrel{+}\S$\6
\4\&{procedure}\1\  \37$\\{add\_buf\_pool}(\\{p\_str}:\\{str\_number})$;\2\6
\&{begin} \37$\\{p\_ptr1}\K\\{str\_start}[\\{p\_str}]$;\5
$\\{p\_ptr2}\K\\{str\_start}[\\{p\_str}+1]$;\6
\&{if} $(\\{ex\_buf\_length}+(\\{p\_ptr2}-\\{p\_ptr1})>\\{buf\_size})$ \1%
\&{then}\5
\\{buffer\_overflow};\2\6
$\\{ex\_buf\_ptr}\K\\{ex\_buf\_length}$;\6
\&{while} $(\\{p\_ptr1}<\\{p\_ptr2})$ \1\&{do}\6
\&{begin} \37\C{copy characters into the buffer}\6
$\\{append\_ex\_buf\_char}(\\{str\_pool}[\\{p\_ptr1}])$;\5
$\\{incr}(\\{p\_ptr1})$;\6
\&{end};\2\6
$\\{ex\_buf\_length}\K\\{ex\_buf\_ptr}$;\6
\&{end};\par
\fi

\M321.
This procedure actually writes onto the \.{.bbl}~file a line of output
(the characters from $\\{out\_buf}[0]$ to $\\{out\_buf}[\\{out\_buf%
\_length}-1]$,
after removing trailing \\{white\_space} characters).  It also updates
\\{bbl\_line\_num}, the line counter.  It writes a blank line if and only
if \\{out\_buf} is empty.  The program uses this procedure in such a way
that \\{out\_buf} will be nonempty if there have been characters put in
it since the most recent \.{newline\$}.

\Y\P$\4\X3:Procedures and functions for all file I/O, error messages, and such%
\X\mathrel{+}\S$\6
\4\&{procedure}\1\  \37\\{output\_bbl\_line};\6
\4\&{label} \37$\\{loop\_exit},\39\\{exit}$;\2\6
\&{begin} \37\&{if} $(\\{out\_buf\_length}\I0)$ \1\&{then}\C{the buffer's not
empty}\6
\&{begin} \37\&{while} $(\\{out\_buf\_length}>0)$ \1\&{do}\C{remove trailing %
\\{white\_space}}\6
\&{if} $(\\{lex\_class}[\\{out\_buf}[\\{out\_buf\_length}-1]]=\\{white%
\_space})$ \1\&{then}\5
$\\{decr}(\\{out\_buf\_length})$\6
\4\&{else} \&{goto} \37\\{loop\_exit};\2\2\6
\4\\{loop\_exit}: \37\&{if} $(\\{out\_buf\_length}=0)$ \1\&{then}\C{ignore a
line of just \\{white\_space}}\6
\&{return};\2\6
$\\{out\_buf\_ptr}\K0$;\6
\&{while} $(\\{out\_buf\_ptr}<\\{out\_buf\_length})$ \1\&{do}\6
\&{begin} \37$\\{write}(\\{bbl\_file},\39\\{xchr}[\\{out\_buf}[\\{out\_buf%
\_ptr}]])$;\5
$\\{incr}(\\{out\_buf\_ptr})$;\6
\&{end};\2\6
\&{end};\2\6
$\\{write\_ln}(\\{bbl\_file})$;\5
$\\{incr}(\\{bbl\_line\_num})$;\C{update line number}\6
$\\{out\_buf\_length}\K0$;\C{make the next line empty}\6
\4\\{exit}: \37\&{end};\par
\fi

\M322.
This procedure adds to the output buffer the given string in
\\{str\_pool}.  It assumes the global variable \\{out\_buf\_length} gives the
length of the current string in \\{out\_buf}, and thus also gives the
location for the next character.  If there are enough characters
present in the output buffer, it writes one or more lines out to the
\.{.bbl} file.  It breaks a line only at a \\{white\_space} character,
and when it does, it adds two \\{space}s to the next output line.

\Y\P$\4\X307:Procedures and functions for style-file function execution\X%
\mathrel{+}\S$\6
\4\&{procedure}\1\  \37$\\{add\_out\_pool}(\\{p\_str}:\\{str\_number})$;\6
\4\&{label} \37$\\{loop1\_exit},\39\\{loop2\_exit}$;\6
\4\&{var} \37\\{break\_ptr}: \37\\{buf\_pointer};\C{the first character
following the line break}\6
\\{end\_ptr}: \37\\{buf\_pointer};\C{temporary end-of-buffer pointer}\6
\\{break\_pt\_found}: \37\\{boolean};\C{a suitable \\{white\_space} character}\6
\\{unbreakable\_tail}: \37\\{boolean};\C{as it contains no \\{white\_space}
character}\2\6
\&{begin} \37$\\{p\_ptr1}\K\\{str\_start}[\\{p\_str}]$;\5
$\\{p\_ptr2}\K\\{str\_start}[\\{p\_str}+1]$;\6
\&{if} $(\\{out\_buf\_length}+(\\{p\_ptr2}-\\{p\_ptr1})>\\{buf\_size})$ \1%
\&{then}\5
$\\{overflow}(\.{\'output\ buffer\ size\ \'},\39\\{buf\_size})$;\2\6
$\\{out\_buf\_ptr}\K\\{out\_buf\_length}$;\6
\&{while} $(\\{p\_ptr1}<\\{p\_ptr2})$ \1\&{do}\6
\&{begin} \37\C{copy characters into the buffer}\6
$\\{out\_buf}[\\{out\_buf\_ptr}]\K\\{str\_pool}[\\{p\_ptr1}]$;\5
$\\{incr}(\\{p\_ptr1})$;\5
$\\{incr}(\\{out\_buf\_ptr})$;\6
\&{end};\2\6
$\\{out\_buf\_length}\K\\{out\_buf\_ptr}$;\5
$\\{unbreakable\_tail}\K\\{false}$;\6
\&{while} $((\\{out\_buf\_length}>\\{max\_print\_line})\W(\R\\{unbreakable%
\_tail}))$ \1\&{do}\5
\X323:Break that line\X;\2\6
\&{end};\par
\fi

\M323.
Here we break the line by looking for a \\{white\_space} character,
backwards from $\\{out\_buf}[\\{max\_print\_line}]$ until
$\\{out\_buf}[\\{min\_print\_line}]$; we break at the \\{white\_space} and
indent
the next line two \\{space}s.  The next module handles things when
there's no \\{white\_space} character to break at.  (It seems that the
annoyances to the average user of a warning message when there's an
output line longer than \\{max\_print\_line} outweigh the benefits, so we
don't issue such warnings in the current code.)

\Y\P$\4\X323:Break that line\X\S$\6
\&{begin} \37$\\{end\_ptr}\K\\{out\_buf\_length}$;\5
$\\{out\_buf\_ptr}\K\\{max\_print\_line}$;\5
$\\{break\_pt\_found}\K\\{false}$;\6
\&{while} $((\\{lex\_class}[\\{out\_buf}[\\{out\_buf\_ptr}]]\I\\{white\_space})%
\W(\\{out\_buf\_ptr}\G\\{min\_print\_line}))$ \1\&{do}\5
$\\{decr}(\\{out\_buf\_ptr})$;\2\6
\&{if} $(\\{out\_buf\_ptr}=\\{min\_print\_line}-1)$ \1\&{then}\C{no \\{white%
\_space} character}\6
\X324:Break that unbreakably long line\X\C{(if \\{white\_space} follows)}\6
\4\&{else} $\\{break\_pt\_found}\K\\{true}$;\C{hit a \\{white\_space}
character}\2\6
\&{if} $(\\{break\_pt\_found})$ \1\&{then}\6
\&{begin} \37$\\{out\_buf\_length}\K\\{out\_buf\_ptr}$;\5
$\\{break\_ptr}\K\\{out\_buf\_length}+1$;\5
\\{output\_bbl\_line};\C{output what we can}\6
$\\{out\_buf}[0]\K\\{space}$;\5
$\\{out\_buf}[1]\K\\{space}$;\C{start the next line with two \\{space}s}\6
$\\{out\_buf\_ptr}\K2$;\5
$\\{tmp\_ptr}\K\\{break\_ptr}$;\6
\&{while} $(\\{tmp\_ptr}<\\{end\_ptr})$ \1\&{do}\C{and slide the rest down}\6
\&{begin} \37$\\{out\_buf}[\\{out\_buf\_ptr}]\K\\{out\_buf}[\\{tmp\_ptr}]$;\5
$\\{incr}(\\{out\_buf\_ptr})$;\5
$\\{incr}(\\{tmp\_ptr})$;\6
\&{end};\2\6
$\\{out\_buf\_length}\K\\{end\_ptr}-\\{break\_ptr}+2$;\6
\&{end};\2\6
\&{end}\par
\U322.\fi

\M324.
If there's no \\{white\_space} character up through
$\\{out\_buf}[\\{max\_print\_line}]$, we instead break the line at the first
following \\{white\_space} character, if one exists.  And if, starting
with that \\{white\_space} character, there are multiple consecutive
\\{white\_space} characters, \\{out\_buf\_ptr} points to the last of them.
If no \\{white\_space} character exists, we haven't found a viable break
point, so we don't break the line (yet).

\Y\P$\4\X324:Break that unbreakably long line\X\S$\6
\&{begin} \37$\\{out\_buf\_ptr}\K\\{max\_print\_line}+1$;\C{\\{break\_pt%
\_found} is still \\{false}}\6
\&{while} $(\\{out\_buf\_ptr}<\\{end\_ptr})$ \1\&{do}\6
\&{if} $(\\{lex\_class}[\\{out\_buf}[\\{out\_buf\_ptr}]]\I\\{white\_space})$ \1%
\&{then}\5
$\\{incr}(\\{out\_buf\_ptr})$\6
\4\&{else} \&{goto} \37\\{loop1\_exit};\2\2\6
\4\\{loop1\_exit}: \37\&{if} $(\\{out\_buf\_ptr}=\\{end\_ptr})$ \1\&{then}\5
$\\{unbreakable\_tail}\K\\{true}$\C{because no \\{white\_space} character}\6
\4\&{else} \C{at \\{white\_space}, and $\\{out\_buf\_ptr}<\\{end\_ptr}$}\2\6
\&{begin} \37$\\{break\_pt\_found}\K\\{true}$;\6
\&{while} $(\\{out\_buf\_ptr}+1<\\{end\_ptr})$ \1\&{do}\C{look for more %
\\{white\_space}}\6
\&{if} $(\\{lex\_class}[\\{out\_buf}[\\{out\_buf\_ptr}+1]]=\\{white\_space})$ %
\1\&{then}\5
$\\{incr}(\\{out\_buf\_ptr})$\C{which then points to \\{white\_space}}\6
\4\&{else} \&{goto} \37\\{loop2\_exit};\2\2\6
\4\\{loop2\_exit}: \37\&{end};\6
\&{end}\par
\U323.\fi

\M325.
This procedure executes a single specified function; it is the single
execution-primitive that does everything (except windows, and it takes
Tuesdays off).

\Y\P$\4\X325:\\{execute\_fn} itself\X\S$\6
\4\&{procedure}\1\  \37$\\{execute\_fn}(\\{ex\_fn\_loc}:\\{hash\_loc})$;\6
\4\X343:Declarations for executing \\{built\_in} functions\X\\{wiz\_ptr}: \37%
\\{wiz\_fn\_loc};\C{general \\{wiz\_functions} location}\2\6
\&{begin} \37\&{trace} \37$\\{trace\_pr}(\.{\'execute\_fn\ \`\'})$;\5
$\\{trace\_pr\_pool\_str}(\\{hash\_text}[\\{ex\_fn\_loc}])$;\5
$\\{trace\_pr\_ln}(\.{\'\'}\.{\'\'})$;\6
\&{ecart}\6
\&{case} $(\\{fn\_type}[\\{ex\_fn\_loc}])$ \1\&{of}\6
\4\\{built\_in}: \37\X341:Execute a \\{built\_in} function\X;\6
\4\\{wiz\_defined}: \37\X326:Execute a \\{wiz\_defined} function\X;\6
\4\\{int\_literal}: \37$\\{push\_lit\_stk}(\\{fn\_info}[\\{ex\_fn\_loc}],\39%
\\{stk\_int})$;\6
\4\\{str\_literal}: \37$\\{push\_lit\_stk}(\\{hash\_text}[\\{ex\_fn\_loc}],\39%
\\{stk\_str})$;\6
\4\\{field}: \37\X327:Execute a field\X;\6
\4\\{int\_entry\_var}: \37\X328:Execute an \\{int\_entry\_var}\X;\6
\4\\{str\_entry\_var}: \37\X329:Execute a \\{str\_entry\_var}\X;\6
\4\\{int\_global\_var}: \37$\\{push\_lit\_stk}(\\{fn\_info}[\\{ex\_fn\_loc}],%
\39\\{stk\_int})$;\6
\4\\{str\_global\_var}: \37\X330:Execute a \\{str\_global\_var}\X;\6
\4\&{othercases} \37\\{unknwn\_function\_class\_confusion}\2\6
\&{endcases};\6
\&{end};\par
\U342.\fi

\M326.
To execute a \\{wiz\_defined} function, we just execute all those
functions in its definition, except that the special marker
\\{quote\_next\_fn} means we push the next function onto the stack.

\Y\P$\4\X326:Execute a \\{wiz\_defined} function\X\S$\6
\&{begin} \37$\\{wiz\_ptr}\K\\{fn\_info}[\\{ex\_fn\_loc}]$;\6
\&{while} $(\\{wiz\_functions}[\\{wiz\_ptr}]\I\\{end\_of\_def})$ \1\&{do}\6
\&{begin} \37\&{if} $(\\{wiz\_functions}[\\{wiz\_ptr}]\I\\{quote\_next\_fn})$ %
\1\&{then}\5
$\\{execute\_fn}(\\{wiz\_functions}[\\{wiz\_ptr}])$\6
\4\&{else} \&{begin} \37$\\{incr}(\\{wiz\_ptr})$;\5
$\\{push\_lit\_stk}(\\{wiz\_functions}[\\{wiz\_ptr}],\39\\{stk\_fn})$;\6
\&{end};\2\6
$\\{incr}(\\{wiz\_ptr})$;\6
\&{end};\2\6
\&{end}\par
\U325.\fi

\M327.
This module pushes the string given by the field onto the literal
stack unless it's \\{missing}, in which case it pushes a special value
onto the stack.

\Y\P$\4\X327:Execute a field\X\S$\6
\&{begin} \37\&{if} $(\R\\{mess\_with\_entries})$ \1\&{then}\5
\\{bst\_cant\_mess\_with\_entries\_print}\6
\4\&{else} \&{begin} \37$\\{field\_ptr}\K\\{cite\_ptr}\ast\\{num\_fields}+\\{fn%
\_info}[\\{ex\_fn\_loc}]$;\6
\&{if} $(\\{field\_info}[\\{field\_ptr}]=\\{missing})$ \1\&{then}\5
$\\{push\_lit\_stk}(\\{hash\_text}[\\{ex\_fn\_loc}],\39\\{stk\_field%
\_missing})$\6
\4\&{else} $\\{push\_lit\_stk}(\\{field\_info}[\\{field\_ptr}],\39\\{stk%
\_str})$;\2\6
\&{end}\2\6
\&{end}\par
\U325.\fi

\M328.
This module pushes the integer given by an \\{int\_entry\_var} onto the
literal stack.

\Y\P$\4\X328:Execute an \\{int\_entry\_var}\X\S$\6
\&{begin} \37\&{if} $(\R\\{mess\_with\_entries})$ \1\&{then}\5
\\{bst\_cant\_mess\_with\_entries\_print}\6
\4\&{else} $\\{push\_lit\_stk}(\\{entry\_ints}[\\{cite\_ptr}\ast\\{num\_ent%
\_ints}+\\{fn\_info}[\\{ex\_fn\_loc}]],\39\\{stk\_int})$;\2\6
\&{end}\par
\U325.\fi

\M329.
This module adds the string given by a \\{str\_entry\_var} to \\{str\_pool}
via the execution buffer and pushes it onto the literal stack.

\Y\P$\4\X329:Execute a \\{str\_entry\_var}\X\S$\6
\&{begin} \37\&{if} $(\R\\{mess\_with\_entries})$ \1\&{then}\5
\\{bst\_cant\_mess\_with\_entries\_print}\6
\4\&{else} \&{begin} \37$\\{str\_ent\_ptr}\K\\{cite\_ptr}\ast\\{num\_ent%
\_strs}+\\{fn\_info}[\\{ex\_fn\_loc}]$;\6
$\\{ex\_buf\_ptr}\K0$;\C{also serves as \\{ent\_chr\_ptr}}\6
\&{while} $(\\{entry\_strs}[\\{str\_ent\_ptr}][\\{ex\_buf\_ptr}]\I\\{end\_of%
\_string})$ \1\&{do}\C{copy characters into the buffer}\6
$\\{append\_ex\_buf\_char}(\\{entry\_strs}[\\{str\_ent\_ptr}][\\{ex\_buf%
\_ptr}])$;\2\6
$\\{ex\_buf\_length}\K\\{ex\_buf\_ptr}$;\5
\\{add\_pool\_buf\_and\_push};\C{push this string onto the stack}\6
\&{end};\2\6
\&{end}\par
\U325.\fi

\M330.
This module pushes the string given by a \\{str\_global\_var} onto the
literal stack, but it copies the string to \\{str\_pool} (character by
character) only if it has to---it {\it doesn't\/} have to if the
string is static (that is, if the string isn't at the top, temporary
part of the string pool).

\Y\P$\4\X330:Execute a \\{str\_global\_var}\X\S$\6
\&{begin} \37$\\{str\_glb\_ptr}\K\\{fn\_info}[\\{ex\_fn\_loc}]$;\6
\&{if} $(\\{glb\_str\_ptr}[\\{str\_glb\_ptr}]>0)$ \1\&{then}\C{we're dealing
with a static string}\6
$\\{push\_lit\_stk}(\\{glb\_str\_ptr}[\\{str\_glb\_ptr}],\39\\{stk\_str})$\6
\4\&{else} \&{begin} \37$\\{str\_room}(\\{glb\_str\_end}[\\{str\_glb\_ptr}])$;\5
$\\{glob\_chr\_ptr}\K0$;\6
\&{while} $(\\{glob\_chr\_ptr}<\\{glb\_str\_end}[\\{str\_glb\_ptr}])$ \1\&{do}%
\C{copy the string}\6
\&{begin} \37$\\{append\_char}(\\{global\_strs}[\\{str\_glb\_ptr}][\\{glob\_chr%
\_ptr}])$;\5
$\\{incr}(\\{glob\_chr\_ptr})$;\6
\&{end};\2\6
$\\{push\_lit\_stk}(\\{make\_string},\39\\{stk\_str})$;\C{and push it onto the
stack}\6
\&{end};\2\6
\&{end}\par
\U325.\fi

\N331.  The built-in functions.
This section gives the all the code for all the built-in functions
(including pre-defined \\{field}s, \\{str\_entry\_var}s, and
\\{int\_global\_var}s, which technically aren't classified as \\{built\_in}).
To modify or add one, we needn't go anywhere else (with one exception:
The constant \\{max\_pop}, which gives the maximum number of literals
that any of these functions pops off the stack, is defined earlier
because it's needed earlier; thus, if we need to update it, which will
happen if some new \\{built\_in} functions uses more than \\{max\_pop}
literals from the stack, we'll have to go outside this section).
Adding a \\{built\_in} function entails modifying (at least four of) the
five modules marked by ``add a built-in function'' in the index, in
addition to adding the code to execute the function.

These variables all begin with \\{b\_} and specify the hash-table
locations of the \\{built\_in} functions, except that \\{b\_default} is
pseudo-\\{built\_in}---either it will point to the no-op \.{skip\$} or to
the \.{.bst}-defined function \.{default.type}; it's used when an
entry has a type that's not defined in the \.{.bst} file.

\Y\P$\4\X16:Globals in the outer block\X\mathrel{+}\S$\6
\4\\{b\_equals}: \37\\{hash\_loc};\C{\.{=}}\6
\4\\{b\_greater\_than}: \37\\{hash\_loc};\C{\.{>}}\6
\4\\{b\_less\_than}: \37\\{hash\_loc};\C{\.{<}}\6
\4\\{b\_plus}: \37\\{hash\_loc};\C{\.{+} (this may be changed to an \\{a%
\_minus})}\6
\4\\{b\_minus}: \37\\{hash\_loc};\C{\.{-}}\6
\4\\{b\_concatenate}: \37\\{hash\_loc};\C{\.{*}}\6
\4\\{b\_gets}: \37\\{hash\_loc};\C{\.{:=} (formerly, \\{b\_gat})}\6
\4\\{b\_add\_period}: \37\\{hash\_loc};\C{\.{add.period\$}}\6
\4\\{b\_call\_type}: \37\\{hash\_loc};\C{\.{call.type\$}}\6
\4\\{b\_change\_case}: \37\\{hash\_loc};\C{\.{change.case\$}}\6
\4\\{b\_chr\_to\_int}: \37\\{hash\_loc};\C{\.{chr.to.int\$}}\6
\4\\{b\_cite}: \37\\{hash\_loc};\C{\.{cite\$}}\6
\4\\{b\_duplicate}: \37\\{hash\_loc};\C{\.{duplicate\$}}\6
\4\\{b\_empty}: \37\\{hash\_loc};\C{\.{empty\$}}\6
\4\\{b\_format\_name}: \37\\{hash\_loc};\C{\.{format.name\$}}\6
\4\\{b\_if}: \37\\{hash\_loc};\C{\.{if\$}}\6
\4\\{b\_int\_to\_chr}: \37\\{hash\_loc};\C{\.{int.to.chr\$}}\6
\4\\{b\_int\_to\_str}: \37\\{hash\_loc};\C{\.{int.to.str\$}}\6
\4\\{b\_missing}: \37\\{hash\_loc};\C{\.{missing\$}}\6
\4\\{b\_newline}: \37\\{hash\_loc};\C{\.{newline\$}}\6
\4\\{b\_num\_names}: \37\\{hash\_loc};\C{\.{num.names\$}}\6
\4\\{b\_pop}: \37\\{hash\_loc};\C{\.{pop\$}}\6
\4\\{b\_preamble}: \37\\{hash\_loc};\C{\.{preamble\$}}\6
\4\\{b\_purify}: \37\\{hash\_loc};\C{\.{purify\$}}\6
\4\\{b\_quote}: \37\\{hash\_loc};\C{\.{quote\$}}\6
\4\\{b\_skip}: \37\\{hash\_loc};\C{\.{skip\$}}\6
\4\\{b\_stack}: \37\\{hash\_loc};\C{\.{stack\$}}\6
\4\\{b\_substring}: \37\\{hash\_loc};\C{\.{substring\$}}\6
\4\\{b\_swap}: \37\\{hash\_loc};\C{\.{swap\$}}\6
\4\\{b\_text\_length}: \37\\{hash\_loc};\C{\.{text.length\$}}\6
\4\\{b\_text\_prefix}: \37\\{hash\_loc};\C{\.{text.prefix\$}}\6
\4\\{b\_top\_stack}: \37\\{hash\_loc};\C{\.{top\$}}\6
\4\\{b\_type}: \37\\{hash\_loc};\C{\.{type\$}}\6
\4\\{b\_warning}: \37\\{hash\_loc};\C{\.{warning\$}}\6
\4\\{b\_while}: \37\\{hash\_loc};\C{\.{while\$}}\6
\4\\{b\_width}: \37\\{hash\_loc};\C{\.{width\$}}\6
\4\\{b\_write}: \37\\{hash\_loc};\C{\.{write\$}}\6
\4\\{b\_default}: \37\\{hash\_loc};\C{either \.{skip\$} or \.{default.type}}\7
\&{stat} \37\\{blt\_in\_loc}: \37\&{array} $[\\{blt\_in\_range}]$ \1\&{of}\5
\\{hash\_loc};\C{for execution counts}\2\6
\4\\{execution\_count}: \37\&{array} $[\\{blt\_in\_range}]$ \1\&{of}\5
\\{integer};\C{the same}\2\6
\4\\{total\_ex\_count}: \37\\{integer};\C{the sum of all \\{execution\_count}s}%
\6
\4\\{blt\_in\_ptr}: \37\\{blt\_in\_range};\C{a pointer into \\{blt\_in\_loc}}\6
\&{tats}\par
\fi

\M332.
Where \\{blt\_in\_range} gives the legal \\{built\_in} function numbers.

\Y\P$\4\X22:Types in the outer block\X\mathrel{+}\S$\6
$\\{blt\_in\_range}=0\to\\{num\_blt\_in\_fns}$;\par
\fi

\M333.
These constants all begin with \\{n\_} and are used for the   \&{case}
statement that determines which \\{built\_in} function to execute.

\Y\P\D \37$\\{n\_equals}=0$\C{\.{=}}\par
\P\D \37$\\{n\_greater\_than}=1$\C{\.{>}}\par
\P\D \37$\\{n\_less\_than}=2$\C{\.{<}}\par
\P\D \37$\\{n\_plus}=3$\C{\.{+}}\par
\P\D \37$\\{n\_minus}=4$\C{\.{-}}\par
\P\D \37$\\{n\_concatenate}=5$\C{\.{*}}\par
\P\D \37$\\{n\_gets}=6$\C{\.{:=}}\par
\P\D \37$\\{n\_add\_period}=7$\C{\.{add.period\$}}\par
\P\D \37$\\{n\_call\_type}=8$\C{\.{call.type\$}}\par
\P\D \37$\\{n\_change\_case}=9$\C{\.{change.case\$}}\par
\P\D \37$\\{n\_chr\_to\_int}=10$\C{\.{chr.to.int\$}}\par
\P\D \37$\\{n\_cite}=11$\C{\.{cite\$} (this may start a riot)}\par
\P\D \37$\\{n\_duplicate}=12$\C{\.{duplicate\$}}\par
\P\D \37$\\{n\_empty}=13$\C{\.{empty\$}}\par
\P\D \37$\\{n\_format\_name}=14$\C{\.{format.name\$}}\par
\P\D \37$\\{n\_if}=15$\C{\.{if\$}}\par
\P\D \37$\\{n\_int\_to\_chr}=16$\C{\.{int.to.chr\$}}\par
\P\D \37$\\{n\_int\_to\_str}=17$\C{\.{int.to.str\$}}\par
\P\D \37$\\{n\_missing}=18$\C{\.{missing\$}}\par
\P\D \37$\\{n\_newline}=19$\C{\.{newline\$}}\par
\P\D \37$\\{n\_num\_names}=20$\C{\.{num.names\$}}\par
\P\D \37$\\{n\_pop}=21$\C{\.{pop\$}}\par
\P\D \37$\\{n\_preamble}=22$\C{\.{preamble\$}}\par
\P\D \37$\\{n\_purify}=23$\C{\.{purify\$}}\par
\P\D \37$\\{n\_quote}=24$\C{\.{quote\$}}\par
\P\D \37$\\{n\_skip}=25$\C{\.{skip\$}}\par
\P\D \37$\\{n\_stack}=26$\C{\.{stack\$}}\par
\P\D \37$\\{n\_substring}=27$\C{\.{substring\$}}\par
\P\D \37$\\{n\_swap}=28$\C{\.{swap\$}}\par
\P\D \37$\\{n\_text\_length}=29$\C{\.{text.length\$}}\par
\P\D \37$\\{n\_text\_prefix}=30$\C{\.{text.prefix\$}}\par
\P\D \37$\\{n\_top\_stack}=31$\C{\.{top\$}}\par
\P\D \37$\\{n\_type}=32$\C{\.{type\$}}\par
\P\D \37$\\{n\_warning}=33$\C{\.{warning\$}}\par
\P\D \37$\\{n\_while}=34$\C{\.{while\$}}\par
\P\D \37$\\{n\_width}=35$\C{\.{width\$}}\par
\P\D \37$\\{n\_write}=36$\C{\.{write\$}}\par
\Y\P$\4\X14:Constants in the outer block\X\mathrel{+}\S$\6
$\\{num\_blt\_in\_fns}=37$;\C{one more than the previous number}\par
\fi

\M334.
It's time for us to insert more pre-defined strings into \\{str\_pool}
(and thus the hash table) and to insert the \\{built\_in} functions into
the hash table.  The strings corresponding to these functions should
contain no upper-case letters, and they must all be exactly
\\{longest\_pds} characters long.  The \\{build\_in} routine (to appear
shortly) does the work.

Important note: These pre-definitions must not have any glitches or the
program may bomb because the \\{log\_file} hasn't been opened yet.

\Y\P$\4\X75:Pre-define certain strings\X\mathrel{+}\S$\6
$\\{build\_in}(\.{\'=\ \ \ \ \ \ \ \ \ \ \ \'},\391,\39\\{b\_equals},\39\\{n%
\_equals})$;\5
$\\{build\_in}(\.{\'>\ \ \ \ \ \ \ \ \ \ \ \'},\391,\39\\{b\_greater\_than},\39%
\\{n\_greater\_than})$;\5
$\\{build\_in}(\.{\'<\ \ \ \ \ \ \ \ \ \ \ \'},\391,\39\\{b\_less\_than},\39%
\\{n\_less\_than})$;\5
$\\{build\_in}(\.{\'+\ \ \ \ \ \ \ \ \ \ \ \'},\391,\39\\{b\_plus},\39\\{n%
\_plus})$;\5
$\\{build\_in}(\.{\'-\ \ \ \ \ \ \ \ \ \ \ \'},\391,\39\\{b\_minus},\39\\{n%
\_minus})$;\5
$\\{build\_in}(\.{\'*\ \ \ \ \ \ \ \ \ \ \ \'},\391,\39\\{b\_concatenate},\39%
\\{n\_concatenate})$;\5
$\\{build\_in}(\.{\':=\ \ \ \ \ \ \ \ \ \ \'},\392,\39\\{b\_gets},\39\\{n%
\_gets})$;\5
$\\{build\_in}(\.{\'add.period\$\ \'},\3911,\39\\{b\_add\_period},\39\\{n\_add%
\_period})$;\5
$\\{build\_in}(\.{\'call.type\$\ \ \'},\3910,\39\\{b\_call\_type},\39\\{n\_call%
\_type})$;\5
$\\{build\_in}(\.{\'change.case\$\'},\3912,\39\\{b\_change\_case},\39\\{n%
\_change\_case})$;\5
$\\{build\_in}(\.{\'chr.to.int\$\ \'},\3911,\39\\{b\_chr\_to\_int},\39\\{n\_chr%
\_to\_int})$;\5
$\\{build\_in}(\.{\'cite\$\ \ \ \ \ \ \ \'},\395,\39\\{b\_cite},\39\\{n%
\_cite})$;\5
$\\{build\_in}(\.{\'duplicate\$\ \ \'},\3910,\39\\{b\_duplicate},\39\\{n%
\_duplicate})$;\5
$\\{build\_in}(\.{\'empty\$\ \ \ \ \ \ \'},\396,\39\\{b\_empty},\39\\{n%
\_empty})$;\5
$\\{build\_in}(\.{\'format.name\$\'},\3912,\39\\{b\_format\_name},\39\\{n%
\_format\_name})$;\5
$\\{build\_in}(\.{\'if\$\ \ \ \ \ \ \ \ \ \'},\393,\39\\{b\_if},\39\\{n\_if})$;%
\5
$\\{build\_in}(\.{\'int.to.chr\$\ \'},\3911,\39\\{b\_int\_to\_chr},\39\\{n\_int%
\_to\_chr})$;\5
$\\{build\_in}(\.{\'int.to.str\$\ \'},\3911,\39\\{b\_int\_to\_str},\39\\{n\_int%
\_to\_str})$;\5
$\\{build\_in}(\.{\'missing\$\ \ \ \ \'},\398,\39\\{b\_missing},\39\\{n%
\_missing})$;\5
$\\{build\_in}(\.{\'newline\$\ \ \ \ \'},\398,\39\\{b\_newline},\39\\{n%
\_newline})$;\5
$\\{build\_in}(\.{\'num.names\$\ \ \'},\3910,\39\\{b\_num\_names},\39\\{n\_num%
\_names})$;\5
$\\{build\_in}(\.{\'pop\$\ \ \ \ \ \ \ \ \'},\394,\39\\{b\_pop},\39\\{n%
\_pop})$;\5
$\\{build\_in}(\.{\'preamble\$\ \ \ \'},\399,\39\\{b\_preamble},\39\\{n%
\_preamble})$;\5
$\\{build\_in}(\.{\'purify\$\ \ \ \ \ \'},\397,\39\\{b\_purify},\39\\{n%
\_purify})$;\5
$\\{build\_in}(\.{\'quote\$\ \ \ \ \ \ \'},\396,\39\\{b\_quote},\39\\{n%
\_quote})$;\5
$\\{build\_in}(\.{\'skip\$\ \ \ \ \ \ \ \'},\395,\39\\{b\_skip},\39\\{n%
\_skip})$;\5
$\\{build\_in}(\.{\'stack\$\ \ \ \ \ \ \'},\396,\39\\{b\_stack},\39\\{n%
\_stack})$;\5
$\\{build\_in}(\.{\'substring\$\ \ \'},\3910,\39\\{b\_substring},\39\\{n%
\_substring})$;\5
$\\{build\_in}(\.{\'swap\$\ \ \ \ \ \ \ \'},\395,\39\\{b\_swap},\39\\{n%
\_swap})$;\5
$\\{build\_in}(\.{\'text.length\$\'},\3912,\39\\{b\_text\_length},\39\\{n\_text%
\_length})$;\5
$\\{build\_in}(\.{\'text.prefix\$\'},\3912,\39\\{b\_text\_prefix},\39\\{n\_text%
\_prefix})$;\5
$\\{build\_in}(\.{\'top\$\ \ \ \ \ \ \ \ \'},\394,\39\\{b\_top\_stack},\39\\{n%
\_top\_stack})$;\5
$\\{build\_in}(\.{\'type\$\ \ \ \ \ \ \ \'},\395,\39\\{b\_type},\39\\{n%
\_type})$;\5
$\\{build\_in}(\.{\'warning\$\ \ \ \ \'},\398,\39\\{b\_warning},\39\\{n%
\_warning})$;\5
$\\{build\_in}(\.{\'width\$\ \ \ \ \ \ \'},\396,\39\\{b\_width},\39\\{n%
\_width})$;\5
$\\{build\_in}(\.{\'while\$\ \ \ \ \ \ \'},\396,\39\\{b\_while},\39\\{n%
\_while})$;\5
$\\{build\_in}(\.{\'width\$\ \ \ \ \ \ \'},\396,\39\\{b\_width},\39\\{n%
\_width})$;\5
$\\{build\_in}(\.{\'write\$\ \ \ \ \ \ \'},\396,\39\\{b\_write},\39\\{n%
\_write})$;\par
\fi

\M335.
This procedure inserts a \\{built\_in} function into the hash table and
initializes the corresponding pre-defined string (of length at most
\\{longest\_pds}).  The array \\{fn\_info} contains a number from 0 through
the number of \\{built\_in} functions minus 1 (i.e., $\\{num\_blt\_in\_fns}-1$
if we're keeping statistics); this number is used by a   \&{case}
statement to execute this function and is used for keeping execution
counts when keeping statistics.

\Y\P$\4\X54:Procedures and functions for handling numbers, characters, and
strings\X\mathrel{+}\S$\6
\4\&{procedure}\1\  \37$\\{build\_in}(\\{pds}:\\{pds\_type};\,\35\\{len}:\\{pds%
\_len};\,\35\mathop{\&{var}}\\{fn\_hash\_loc}:\\{hash\_loc};\,\35\\{blt\_in%
\_num}:\\{blt\_in\_range})$;\2\6
\&{begin} \37$\\{pre\_define}(\\{pds},\39\\{len},\39\\{bst\_fn\_ilk})$;\6
$\\{fn\_hash\_loc}\K\\{pre\_def\_loc}$;\C{the \\{pre\_define} routine sets %
\\{pre\_def\_loc}}\6
$\\{fn\_type}[\\{fn\_hash\_loc}]\K\\{built\_in}$;\5
$\\{fn\_info}[\\{fn\_hash\_loc}]\K\\{blt\_in\_num}$;\6
\&{stat} \37$\\{blt\_in\_loc}[\\{blt\_in\_num}]\K\\{fn\_hash\_loc}$;\6
$\\{execution\_count}[\\{blt\_in\_num}]\K0$;\C{initialize the
function-execution count}\6
\&{tats}\6
\&{end};\par
\fi

\M336.
This is a procedure so that \\{initialize} is smaller.

\Y\P$\4\X54:Procedures and functions for handling numbers, characters, and
strings\X\mathrel{+}\S$\6
\4\&{procedure}\1\  \37\\{pre\_def\_certain\_strings};\2\6
\&{begin} \37\X75:Pre-define certain strings\X\6
\&{end};\par
\fi

\M337.
These variables all begin with \\{s\_} and specify the locations in
\\{str\_pool} of certain often-used strings that the \.{.bst} commands
need.  The \\{s\_preamble} array is big enough to allow an average of one
\.{preamble\$} command per \.{.bib} file.

\Y\P$\4\X16:Globals in the outer block\X\mathrel{+}\S$\6
\4\\{s\_null}: \37\\{str\_number};\C{the null string}\6
\4\\{s\_default}: \37\\{str\_number};\C{\.{default.type}, for unknown entry
types}\6
\4\\{s\_t}: \37\\{str\_number};\C{\.{t}, for \\{title\_lowers} case conversion}%
\6
\4\\{s\_l}: \37\\{str\_number};\C{\.{l}, for \\{all\_lowers} case conversion}\6
\4\\{s\_u}: \37\\{str\_number};\C{\.{u}, for \\{all\_uppers} case conversion}\6
\4\\{s\_preamble}: \37\&{array} $[\\{bib\_number}]$ \1\&{of}\5
\\{str\_number};\C{for the \.{preamble\$} \\{built\_in} function}\2\par
\fi

\M338.
These constants all begin with \\{n\_} and are used for the   \&{case}
statement that determines which, if any, control sequence we're
dealing with; a control sequence of interest will be either one of the
undotted characters `\.{\\i}' or `\.{\\j}' or one of the foreign
characters in Table~3.2 of the \LaTeX\ manual.

\Y\P\D \37$\\{n\_i}=0$\C{\.{i}, for the undotted character \.{\\i}}\par
\P\D \37$\\{n\_j}=1$\C{\.{j}, for the undotted character \.{\\j}}\par
\P\D \37$\\{n\_oe}=2$\C{\.{oe}, for the foreign character \.{\\oe}}\par
\P\D \37$\\{n\_oe\_upper}=3$\C{\.{OE}, for the foreign character \.{\\OE}}\par
\P\D \37$\\{n\_ae}=4$\C{\.{ae}, for the foreign character \.{\\ae}}\par
\P\D \37$\\{n\_ae\_upper}=5$\C{\.{AE}, for the foreign character \.{\\AE}}\par
\P\D \37$\\{n\_aa}=6$\C{\.{aa}, for the foreign character \.{\\aa}}\par
\P\D \37$\\{n\_aa\_upper}=7$\C{\.{AA}, for the foreign character \.{\\AA}}\par
\P\D \37$\\{n\_o}=8$\C{\.{o}, for the foreign character \.{\\o}}\par
\P\D \37$\\{n\_o\_upper}=9$\C{\.{O}, for the foreign character \.{\\O}}\par
\P\D \37$\\{n\_l}=10$\C{\.{l}, for the foreign character \.{\\l}}\par
\P\D \37$\\{n\_l\_upper}=11$\C{\.{L}, for the foreign character \.{\\L}}\par
\P\D \37$\\{n\_ss}=12$\C{\.{ss}, for the foreign character \.{\\ss}}\par
\fi

\M339.
Here we pre-define a few strings used in executing the \.{.bst} file:
the null string, which is sometimes pushed onto the stack; a string
used for default entry types; and some control sequences used to spot
foreign characters.  We also initialize the \\{s\_preamble} array to
empty.  These pre-defined strings must all be exactly \\{longest\_pds}
characters long.

Important note: These pre-definitions must not have any glitches or
the program may bomb because the \\{log\_file} hasn't been opened yet,
and \\{text\_ilk}s should be pre-defined here, not earlier, for
\.{.bst}-function-execution purposes.

\Y\P$\4\X75:Pre-define certain strings\X\mathrel{+}\S$\6
$\\{pre\_define}(\.{\'\ \ \ \ \ \ \ \ \ \ \ \ \'},\390,\39\\{text\_ilk})$;\5
$\\{s\_null}\K\\{hash\_text}[\\{pre\_def\_loc}]$;\5
$\\{fn\_type}[\\{pre\_def\_loc}]\K\\{str\_literal}$;\6
$\\{pre\_define}(\.{\'default.type\'},\3912,\39\\{text\_ilk})$;\5
$\\{s\_default}\K\\{hash\_text}[\\{pre\_def\_loc}]$;\5
$\\{fn\_type}[\\{pre\_def\_loc}]\K\\{str\_literal}$;\6
$\\{b\_default}\K\\{b\_skip}$;\C{this may be changed to the \.{default.type}
function}\6
$\\{preamble\_ptr}\K0$;\C{initialize the \\{s\_preamble} array}\6
$\\{pre\_define}(\.{\'i\ \ \ \ \ \ \ \ \ \ \ \'},\391,\39\\{control\_seq%
\_ilk})$;\5
$\\{ilk\_info}[\\{pre\_def\_loc}]\K\\{n\_i}$;\5
$\\{pre\_define}(\.{\'j\ \ \ \ \ \ \ \ \ \ \ \'},\391,\39\\{control\_seq%
\_ilk})$;\5
$\\{ilk\_info}[\\{pre\_def\_loc}]\K\\{n\_j}$;\5
$\\{pre\_define}(\.{\'oe\ \ \ \ \ \ \ \ \ \ \'},\392,\39\\{control\_seq%
\_ilk})$;\5
$\\{ilk\_info}[\\{pre\_def\_loc}]\K\\{n\_oe}$;\5
$\\{pre\_define}(\.{\'OE\ \ \ \ \ \ \ \ \ \ \'},\392,\39\\{control\_seq%
\_ilk})$;\5
$\\{ilk\_info}[\\{pre\_def\_loc}]\K\\{n\_oe\_upper}$;\5
$\\{pre\_define}(\.{\'ae\ \ \ \ \ \ \ \ \ \ \'},\392,\39\\{control\_seq%
\_ilk})$;\5
$\\{ilk\_info}[\\{pre\_def\_loc}]\K\\{n\_ae}$;\5
$\\{pre\_define}(\.{\'AE\ \ \ \ \ \ \ \ \ \ \'},\392,\39\\{control\_seq%
\_ilk})$;\5
$\\{ilk\_info}[\\{pre\_def\_loc}]\K\\{n\_ae\_upper}$;\5
$\\{pre\_define}(\.{\'aa\ \ \ \ \ \ \ \ \ \ \'},\392,\39\\{control\_seq%
\_ilk})$;\5
$\\{ilk\_info}[\\{pre\_def\_loc}]\K\\{n\_aa}$;\5
$\\{pre\_define}(\.{\'AA\ \ \ \ \ \ \ \ \ \ \'},\392,\39\\{control\_seq%
\_ilk})$;\5
$\\{ilk\_info}[\\{pre\_def\_loc}]\K\\{n\_aa\_upper}$;\5
$\\{pre\_define}(\.{\'o\ \ \ \ \ \ \ \ \ \ \ \'},\391,\39\\{control\_seq%
\_ilk})$;\5
$\\{ilk\_info}[\\{pre\_def\_loc}]\K\\{n\_o}$;\5
$\\{pre\_define}(\.{\'O\ \ \ \ \ \ \ \ \ \ \ \'},\391,\39\\{control\_seq%
\_ilk})$;\5
$\\{ilk\_info}[\\{pre\_def\_loc}]\K\\{n\_o\_upper}$;\5
$\\{pre\_define}(\.{\'l\ \ \ \ \ \ \ \ \ \ \ \'},\391,\39\\{control\_seq%
\_ilk})$;\5
$\\{ilk\_info}[\\{pre\_def\_loc}]\K\\{n\_l}$;\5
$\\{pre\_define}(\.{\'L\ \ \ \ \ \ \ \ \ \ \ \'},\391,\39\\{control\_seq%
\_ilk})$;\5
$\\{ilk\_info}[\\{pre\_def\_loc}]\K\\{n\_l\_upper}$;\5
$\\{pre\_define}(\.{\'ss\ \ \ \ \ \ \ \ \ \ \'},\392,\39\\{control\_seq%
\_ilk})$;\5
$\\{ilk\_info}[\\{pre\_def\_loc}]\K\\{n\_ss}$;\par
\fi

\M340.
Now we pre-define any built-in \\{field}s, \\{str\_entry\_var}s, and
\\{int\_global\_var}s; these strings must all be exactly \\{longest\_pds}
characters long.  Note that although these are built-in functions, we
classify them (in the \\{fn\_type} array) otherwise.

Important note: These pre-definitions must not have any glitches or
the program may bomb because the \\{log\_file} hasn't been opened yet.

\Y\P$\4\X75:Pre-define certain strings\X\mathrel{+}\S$\6
$\\{pre\_define}(\.{\'crossref\ \ \ \ \'},\398,\39\\{bst\_fn\_ilk})$;\5
$\\{fn\_type}[\\{pre\_def\_loc}]\K\\{field}$;\6
$\\{fn\_info}[\\{pre\_def\_loc}]\K\\{num\_fields}$;\C{give this \\{field} a
number}\6
$\\{crossref\_num}\K\\{num\_fields}$;\5
$\\{incr}(\\{num\_fields})$;\6
$\\{num\_pre\_defined\_fields}\K\\{num\_fields}$;\C{that's it for pre-defined %
\\{field}s}\6
$\\{pre\_define}(\.{\'sort.key\$\ \ \ \'},\399,\39\\{bst\_fn\_ilk})$;\5
$\\{fn\_type}[\\{pre\_def\_loc}]\K\\{str\_entry\_var}$;\5
$\\{fn\_info}[\\{pre\_def\_loc}]\K\\{num\_ent\_strs}$;\C{give this \\{str%
\_entry\_var} a number}\6
$\\{sort\_key\_num}\K\\{num\_ent\_strs}$;\5
$\\{incr}(\\{num\_ent\_strs})$;\6
$\\{pre\_define}(\.{\'entry.max\$\ \ \'},\3910,\39\\{bst\_fn\_ilk})$;\5
$\\{fn\_type}[\\{pre\_def\_loc}]\K\\{int\_global\_var}$;\5
$\\{fn\_info}[\\{pre\_def\_loc}]\K\\{ent\_str\_size}$;\C{initialize this \\{int%
\_global\_var}}\6
$\\{pre\_define}(\.{\'global.max\$\ \'},\3911,\39\\{bst\_fn\_ilk})$;\5
$\\{fn\_type}[\\{pre\_def\_loc}]\K\\{int\_global\_var}$;\5
$\\{fn\_info}[\\{pre\_def\_loc}]\K\\{glob\_str\_size}$;\C{initialize this %
\\{int\_global\_var}}\par
\fi

\M341.
This module branches to the code for the appropriate \\{built\_in}
function.  Only three---{\.{call.type\$}}, {\.{if\$}}, and
{\.{while\$}}---do a recursive call.

\Y\P$\4\X341:Execute a \\{built\_in} function\X\S$\6
\&{begin} \37\&{stat} \37\C{update this function's execution count}\6
$\\{incr}(\\{execution\_count}[\\{fn\_info}[\\{ex\_fn\_loc}]])$;\6
\&{tats}\6
\&{case} $(\\{fn\_info}[\\{ex\_fn\_loc}])$ \1\&{of}\6
\4\\{n\_equals}: \37\\{x\_equals};\6
\4\\{n\_greater\_than}: \37\\{x\_greater\_than};\6
\4\\{n\_less\_than}: \37\\{x\_less\_than};\6
\4\\{n\_plus}: \37\\{x\_plus};\6
\4\\{n\_minus}: \37\\{x\_minus};\6
\4\\{n\_concatenate}: \37\\{x\_concatenate};\6
\4\\{n\_gets}: \37\\{x\_gets};\6
\4\\{n\_add\_period}: \37\\{x\_add\_period};\6
\4\\{n\_call\_type}: \37\X363:\\{execute\_fn}({\.{call.type\$}})\X;\6
\4\\{n\_change\_case}: \37\\{x\_change\_case};\6
\4\\{n\_chr\_to\_int}: \37\\{x\_chr\_to\_int};\6
\4\\{n\_cite}: \37\\{x\_cite};\6
\4\\{n\_duplicate}: \37\\{x\_duplicate};\6
\4\\{n\_empty}: \37\\{x\_empty};\6
\4\\{n\_format\_name}: \37\\{x\_format\_name};\6
\4\\{n\_if}: \37\X421:\\{execute\_fn}({\.{if\$}})\X;\6
\4\\{n\_int\_to\_chr}: \37\\{x\_int\_to\_chr};\6
\4\\{n\_int\_to\_str}: \37\\{x\_int\_to\_str};\6
\4\\{n\_missing}: \37\\{x\_missing};\6
\4\\{n\_newline}: \37\X425:\\{execute\_fn}({\.{newline\$}})\X;\6
\4\\{n\_num\_names}: \37\\{x\_num\_names};\6
\4\\{n\_pop}: \37\X428:\\{execute\_fn}({\.{pop\$}})\X;\6
\4\\{n\_preamble}: \37\\{x\_preamble};\6
\4\\{n\_purify}: \37\\{x\_purify};\6
\4\\{n\_quote}: \37\\{x\_quote};\6
\4\\{n\_skip}: \37\X435:\\{execute\_fn}({\.{skip\$}})\X;\6
\4\\{n\_stack}: \37\X436:\\{execute\_fn}({\.{stack\$}})\X;\6
\4\\{n\_substring}: \37\\{x\_substring};\6
\4\\{n\_swap}: \37\\{x\_swap};\6
\4\\{n\_text\_length}: \37\\{x\_text\_length};\6
\4\\{n\_text\_prefix}: \37\\{x\_text\_prefix};\6
\4\\{n\_top\_stack}: \37\X446:\\{execute\_fn}({\.{top\$}})\X;\6
\4\\{n\_type}: \37\\{x\_type};\6
\4\\{n\_warning}: \37\\{x\_warning};\6
\4\\{n\_while}: \37\X449:\\{execute\_fn}({\.{while\$}})\X;\6
\4\\{n\_width}: \37\\{x\_width};\6
\4\\{n\_write}: \37\\{x\_write};\6
\4\&{othercases} \37$\\{confusion}(\.{\'Unknown\ built-in\ function\'})$\2\6
\&{endcases};\6
\&{end}\par
\U325.\fi

\M342.
This extra level of module-pointing allows a uniformity of module
names for the \\{built\_in} functions, regardless of whether they do a
recursive call to \\{execute\_fn} or are trivial (a single statement).
Those that do a recursive call are left as part of \\{execute\_fn},
avoiding \PASCAL's forward procedure mechanism, and those that don't
(except for the single-statement ones) are made into procedures so
that \\{execute\_fn} doesn't get too large.

\Y\P$\4\X307:Procedures and functions for style-file function execution\X%
\mathrel{+}\S$\6
\X345:\\{execute\_fn}({\.{=}})\X\6
\X346:\\{execute\_fn}({\.{>}})\X\6
\X347:\\{execute\_fn}({\.{<}})\X\6
\X348:\\{execute\_fn}({\.{+}})\X\6
\X349:\\{execute\_fn}({\.{-}})\X\6
\X350:\\{execute\_fn}({\.{*}})\X\6
\X354:\\{execute\_fn}({\.{:=}})\X\6
\X360:\\{execute\_fn}({\.{add.period\$}})\X\6
\X364:\\{execute\_fn}({\.{change.case\$}})\X\6
\X377:\\{execute\_fn}({\.{chr.to.int\$}})\X\6
\X378:\\{execute\_fn}({\.{cite\$}})\X\6
\X379:\\{execute\_fn}({\.{duplicate\$}})\X\6
\X380:\\{execute\_fn}({\.{empty\$}})\X\6
\X382:\\{execute\_fn}({\.{format.name\$}})\X\6
\X422:\\{execute\_fn}({\.{int.to.chr\$}})\X\6
\X423:\\{execute\_fn}({\.{int.to.str\$}})\X\6
\X424:\\{execute\_fn}({\.{missing\$}})\X\6
\X426:\\{execute\_fn}({\.{num.names\$}})\X\6
\X429:\\{execute\_fn}({\.{preamble\$}})\X\6
\X430:\\{execute\_fn}({\.{purify\$}})\X\6
\X434:\\{execute\_fn}({\.{quote\$}})\X\6
\X437:\\{execute\_fn}({\.{substring\$}})\X\6
\X439:\\{execute\_fn}({\.{swap\$}})\X\6
\X441:\\{execute\_fn}({\.{text.length\$}})\X\6
\X443:\\{execute\_fn}({\.{text.prefix\$}})\X\6
\X447:\\{execute\_fn}({\.{type\$}})\X\6
\X448:\\{execute\_fn}({\.{warning\$}})\X\6
\X450:\\{execute\_fn}({\.{width\$}})\X\6
\X454:\\{execute\_fn}({\.{write\$}})\X\6
\X325:\\{execute\_fn} itself\X\par
\fi

\M343.
Now it's time to declare some things for executing \\{built\_in}
functions only.  These (and only these) variables are used
recursively, so they can't be global.

\Y\P\D \37$\\{end\_while}=51$\C{stop executing the \.{while\$} function}\par
\Y\P$\4\X343:Declarations for executing \\{built\_in} functions\X\S$\6
\4\&{label} \37\\{end\_while}; \6
\4\&{var} \37$\\{r\_pop\_lt1},\39\\{r\_pop\_lt2}$: \37\\{integer};\C{stack
literals for \.{while\$}}\6
$\\{r\_pop\_tp1},\39\\{r\_pop\_tp2}$: \37\\{stk\_type};\C{stack types for %
\.{while\$}}\par
\U325.\fi

\M344.
These are nonrecursive variables that \\{execute\_fn} uses.  Declaring
them here (instead of in the previous module) saves execution time and
stack space on most machines.

\Y\P\D \37$\\{name\_buf}\S\\{sv\_buffer}$\C{an alias, a buffer for manipulating
names}\par
\Y\P$\4\X16:Globals in the outer block\X\mathrel{+}\S$\6
\4$\\{pop\_lit1},\39\\{pop\_lit2},\39\\{pop\_lit3}$: \37\\{integer};\C{stack
literals}\6
\4$\\{pop\_typ1},\39\\{pop\_typ2},\39\\{pop\_typ3}$: \37\\{stk\_type};\C{stack
types}\6
\4\\{sp\_ptr}: \37\\{pool\_pointer};\C{for manipulating \\{str\_pool} strings}\6
\4$\\{sp\_xptr1},\39\\{sp\_xptr2}$: \37\\{pool\_pointer};\C{more of the same}\6
\4\\{sp\_end}: \37\\{pool\_pointer};\C{marks the end of a \\{str\_pool} string}%
\6
\4$\\{sp\_length},\39\\{sp2\_length}$: \37\\{pool\_pointer};\C{lengths of %
\\{str\_pool} strings}\6
\4\\{sp\_brace\_level}: \37\\{integer};\C{for scanning \\{str\_pool} strings}\6
\4$\\{ex\_buf\_xptr},\39\\{ex\_buf\_yptr}$: \37\\{buf\_pointer};\C{extra \\{ex%
\_buf} locations}\6
\4\\{control\_seq\_loc}: \37\\{hash\_loc};\C{hash-table loc of a control
sequence}\6
\4\\{preceding\_white}: \37\\{boolean};\C{used in scanning strings}\6
\4\\{and\_found}: \37\\{boolean};\C{to stop the loop that looks for an ``and''}%
\6
\4\\{num\_names}: \37\\{integer};\C{for counting names}\6
\4\\{name\_bf\_ptr}: \37\\{buf\_pointer};\C{general \\{name\_buf} location}\6
\4$\\{name\_bf\_xptr},\39\\{name\_bf\_yptr}$: \37\\{buf\_pointer};\C{and two
more}\6
\4\\{nm\_brace\_level}: \37\\{integer};\C{for scanning \\{name\_buf} strings}\6
\4\\{name\_tok}: \37\&{packed} \37\&{array} $[\\{buf\_pointer}]$ \1\&{of}\5
\\{buf\_pointer};\C{name-token ptr list}\2\6
\4\\{name\_sep\_char}: \37\&{packed} \37\&{array} $[\\{buf\_pointer}]$ \1\&{of}%
\5
\\{ASCII\_code};\C{token-ending chars}\2\6
\4\\{num\_tokens}: \37\\{buf\_pointer};\C{this counts name tokens}\6
\4\\{token\_starting}: \37\\{boolean};\C{used in scanning name tokens}\6
\4\\{alpha\_found}: \37\\{boolean};\C{used in scanning the format string}\6
\4$\\{double\_letter},\39\\{end\_of\_group},\39\\{to\_be\_written}$: \37%
\\{boolean};\C{the same}\6
\4\\{first\_start}: \37\\{buf\_pointer};\C{start-ptr into \\{name\_tok} for the
first name}\6
\4\\{first\_end}: \37\\{buf\_pointer};\C{end-ptr into \\{name\_tok} for the
first name}\6
\4\\{last\_end}: \37\\{buf\_pointer};\C{end-ptr into \\{name\_tok} for the last
name}\6
\4\\{von\_start}: \37\\{buf\_pointer};\C{start-ptr into \\{name\_tok} for the
von name}\6
\4\\{von\_end}: \37\\{buf\_pointer};\C{end-ptr into \\{name\_tok} for the von
name}\6
\4\\{jr\_end}: \37\\{buf\_pointer};\C{end-ptr into \\{name\_tok} for the jr
name}\6
\4$\\{cur\_token},\39\\{last\_token}$: \37\\{buf\_pointer};\C{\\{name\_tok}
ptrs for outputting tokens}\6
\4\\{use\_default}: \37\\{boolean};\C{for the inter-token intra-name part
string}\6
\4\\{num\_commas}: \37\\{buf\_pointer};\C{used to determine the name syntax}\6
\4$\\{comma1},\39\\{comma2}$: \37\\{buf\_pointer};\C{ptrs into \\{name\_tok}}\6
\4\\{num\_text\_chars}: \37\\{buf\_pointer};\C{special characters count as one}%
\par
\fi

\M345.
The \\{built\_in} function {\.{=}} pops the top two (integer or string)
literals, compares them, and pushes the integer 1 if they're equal, 0
otherwise.  If they're not either both string or both integer, it
complains and pushes the integer 0.

\Y\P$\4\X345:\\{execute\_fn}({\.{=}})\X\S$\6
\4\&{procedure}\1\  \37\\{x\_equals};\2\6
\&{begin} \37$\\{pop\_lit\_stk}(\\{pop\_lit1},\39\\{pop\_typ1})$;\5
$\\{pop\_lit\_stk}(\\{pop\_lit2},\39\\{pop\_typ2})$;\6
\&{if} $(\\{pop\_typ1}\I\\{pop\_typ2})$ \1\&{then}\6
\&{begin} \37\&{if} $((\\{pop\_typ1}\I\\{stk\_empty})\W(\\{pop\_typ2}\I\\{stk%
\_empty}))$ \1\&{then}\6
\&{begin} \37$\\{print\_stk\_lit}(\\{pop\_lit1},\39\\{pop\_typ1})$;\5
$\\{print}(\.{\',\ \'})$;\5
$\\{print\_stk\_lit}(\\{pop\_lit2},\39\\{pop\_typ2})$;\5
\\{print\_newline};\5
$\\{bst\_ex\_warn}(\.{\'---they\ aren\'}\.{\'t\ the\ same\ literal\ types\'})$;%
\6
\&{end};\2\6
$\\{push\_lit\_stk}(0,\39\\{stk\_int})$;\6
\&{end}\6
\4\&{else} \&{if} $((\\{pop\_typ1}\I\\{stk\_int})\W(\\{pop\_typ1}\I\\{stk%
\_str}))$ \1\&{then}\6
\&{begin} \37\&{if} $(\\{pop\_typ1}\I\\{stk\_empty})$ \1\&{then}\6
\&{begin} \37$\\{print\_stk\_lit}(\\{pop\_lit1},\39\\{pop\_typ1})$;\5
$\\{bst\_ex\_warn}(\.{\',\ not\ an\ integer\ or\ a\ string,\'})$;\6
\&{end};\2\6
$\\{push\_lit\_stk}(0,\39\\{stk\_int})$;\6
\&{end}\6
\4\&{else} \&{if} $(\\{pop\_typ1}=\\{stk\_int})$ \1\&{then}\6
\&{if} $(\\{pop\_lit2}=\\{pop\_lit1})$ \1\&{then}\5
$\\{push\_lit\_stk}(1,\39\\{stk\_int})$\6
\4\&{else} $\\{push\_lit\_stk}(0,\39\\{stk\_int})$\2\6
\4\&{else} \&{if} $(\\{str\_eq\_str}(\\{pop\_lit2},\39\\{pop\_lit1}))$ \1%
\&{then}\5
$\\{push\_lit\_stk}(1,\39\\{stk\_int})$\6
\4\&{else} $\\{push\_lit\_stk}(0,\39\\{stk\_int})$;\2\2\2\2\6
\&{end};\par
\U342.\fi

\M346.
The \\{built\_in} function {\.{>}} pops the top two (integer) literals,
compares them, and pushes the integer 1 if the second is greater than
the first, 0 otherwise.  If either isn't an integer literal, it
complains and pushes the integer 0.

\Y\P$\4\X346:\\{execute\_fn}({\.{>}})\X\S$\6
\4\&{procedure}\1\  \37\\{x\_greater\_than};\2\6
\&{begin} \37$\\{pop\_lit\_stk}(\\{pop\_lit1},\39\\{pop\_typ1})$;\5
$\\{pop\_lit\_stk}(\\{pop\_lit2},\39\\{pop\_typ2})$;\6
\&{if} $(\\{pop\_typ1}\I\\{stk\_int})$ \1\&{then}\6
\&{begin} \37$\\{print\_wrong\_stk\_lit}(\\{pop\_lit1},\39\\{pop\_typ1},\39%
\\{stk\_int})$;\5
$\\{push\_lit\_stk}(0,\39\\{stk\_int})$;\6
\&{end}\6
\4\&{else} \&{if} $(\\{pop\_typ2}\I\\{stk\_int})$ \1\&{then}\6
\&{begin} \37$\\{print\_wrong\_stk\_lit}(\\{pop\_lit2},\39\\{pop\_typ2},\39%
\\{stk\_int})$;\5
$\\{push\_lit\_stk}(0,\39\\{stk\_int})$;\6
\&{end}\6
\4\&{else} \&{if} $(\\{pop\_lit2}>\\{pop\_lit1})$ \1\&{then}\5
$\\{push\_lit\_stk}(1,\39\\{stk\_int})$\6
\4\&{else} $\\{push\_lit\_stk}(0,\39\\{stk\_int})$;\2\2\2\6
\&{end};\par
\U342.\fi

\M347.
The \\{built\_in} function {\.{<}} pops the top two (integer) literals,
compares them, and pushes the integer 1 if the second is less than the
first, 0 otherwise.  If either isn't an integer literal, it complains
and pushes the integer 0.

\Y\P$\4\X347:\\{execute\_fn}({\.{<}})\X\S$\6
\4\&{procedure}\1\  \37\\{x\_less\_than};\2\6
\&{begin} \37$\\{pop\_lit\_stk}(\\{pop\_lit1},\39\\{pop\_typ1})$;\5
$\\{pop\_lit\_stk}(\\{pop\_lit2},\39\\{pop\_typ2})$;\6
\&{if} $(\\{pop\_typ1}\I\\{stk\_int})$ \1\&{then}\6
\&{begin} \37$\\{print\_wrong\_stk\_lit}(\\{pop\_lit1},\39\\{pop\_typ1},\39%
\\{stk\_int})$;\5
$\\{push\_lit\_stk}(0,\39\\{stk\_int})$;\6
\&{end}\6
\4\&{else} \&{if} $(\\{pop\_typ2}\I\\{stk\_int})$ \1\&{then}\6
\&{begin} \37$\\{print\_wrong\_stk\_lit}(\\{pop\_lit2},\39\\{pop\_typ2},\39%
\\{stk\_int})$;\5
$\\{push\_lit\_stk}(0,\39\\{stk\_int})$;\6
\&{end}\6
\4\&{else} \&{if} $(\\{pop\_lit2}<\\{pop\_lit1})$ \1\&{then}\5
$\\{push\_lit\_stk}(1,\39\\{stk\_int})$\6
\4\&{else} $\\{push\_lit\_stk}(0,\39\\{stk\_int})$;\2\2\2\6
\&{end};\par
\U342.\fi

\M348.
The \\{built\_in} function {\.{+}} pops the top two (integer) literals
and pushes their sum.  If either isn't an integer literal, it
complains and pushes the integer 0.

\Y\P$\4\X348:\\{execute\_fn}({\.{+}})\X\S$\6
\4\&{procedure}\1\  \37\\{x\_plus};\2\6
\&{begin} \37$\\{pop\_lit\_stk}(\\{pop\_lit1},\39\\{pop\_typ1})$;\5
$\\{pop\_lit\_stk}(\\{pop\_lit2},\39\\{pop\_typ2})$;\6
\&{if} $(\\{pop\_typ1}\I\\{stk\_int})$ \1\&{then}\6
\&{begin} \37$\\{print\_wrong\_stk\_lit}(\\{pop\_lit1},\39\\{pop\_typ1},\39%
\\{stk\_int})$;\5
$\\{push\_lit\_stk}(0,\39\\{stk\_int})$;\6
\&{end}\6
\4\&{else} \&{if} $(\\{pop\_typ2}\I\\{stk\_int})$ \1\&{then}\6
\&{begin} \37$\\{print\_wrong\_stk\_lit}(\\{pop\_lit2},\39\\{pop\_typ2},\39%
\\{stk\_int})$;\5
$\\{push\_lit\_stk}(0,\39\\{stk\_int})$;\6
\&{end}\6
\4\&{else} $\\{push\_lit\_stk}(\\{pop\_lit2}+\\{pop\_lit1},\39\\{stk\_int})$;\2%
\2\6
\&{end};\par
\U342.\fi

\M349.
The \\{built\_in} function {\.{-}} pops the top two (integer) literals
and pushes their difference (the first subtracted from the second).
If either isn't an integer literal, it complains and pushes the
integer 0.

\Y\P$\4\X349:\\{execute\_fn}({\.{-}})\X\S$\6
\4\&{procedure}\1\  \37\\{x\_minus};\2\6
\&{begin} \37$\\{pop\_lit\_stk}(\\{pop\_lit1},\39\\{pop\_typ1})$;\5
$\\{pop\_lit\_stk}(\\{pop\_lit2},\39\\{pop\_typ2})$;\6
\&{if} $(\\{pop\_typ1}\I\\{stk\_int})$ \1\&{then}\6
\&{begin} \37$\\{print\_wrong\_stk\_lit}(\\{pop\_lit1},\39\\{pop\_typ1},\39%
\\{stk\_int})$;\5
$\\{push\_lit\_stk}(0,\39\\{stk\_int})$;\6
\&{end}\6
\4\&{else} \&{if} $(\\{pop\_typ2}\I\\{stk\_int})$ \1\&{then}\6
\&{begin} \37$\\{print\_wrong\_stk\_lit}(\\{pop\_lit2},\39\\{pop\_typ2},\39%
\\{stk\_int})$;\5
$\\{push\_lit\_stk}(0,\39\\{stk\_int})$;\6
\&{end}\6
\4\&{else} $\\{push\_lit\_stk}(\\{pop\_lit2}-\\{pop\_lit1},\39\\{stk\_int})$;\2%
\2\6
\&{end};\par
\U342.\fi

\M350.
The \\{built\_in} function {\.{*}} pops the top two (string) literals,
concatenates them (in reverse order, that is, the order in which
pushed), and pushes the resulting string back onto the stack.  If
either isn't a string literal, it complains and pushes the null
string.

\Y\P$\4\X350:\\{execute\_fn}({\.{*}})\X\S$\6
\4\&{procedure}\1\  \37\\{x\_concatenate};\2\6
\&{begin} \37$\\{pop\_lit\_stk}(\\{pop\_lit1},\39\\{pop\_typ1})$;\5
$\\{pop\_lit\_stk}(\\{pop\_lit2},\39\\{pop\_typ2})$;\6
\&{if} $(\\{pop\_typ1}\I\\{stk\_str})$ \1\&{then}\6
\&{begin} \37$\\{print\_wrong\_stk\_lit}(\\{pop\_lit1},\39\\{pop\_typ1},\39%
\\{stk\_str})$;\5
$\\{push\_lit\_stk}(\\{s\_null},\39\\{stk\_str})$;\6
\&{end}\6
\4\&{else} \&{if} $(\\{pop\_typ2}\I\\{stk\_str})$ \1\&{then}\6
\&{begin} \37$\\{print\_wrong\_stk\_lit}(\\{pop\_lit2},\39\\{pop\_typ2},\39%
\\{stk\_str})$;\5
$\\{push\_lit\_stk}(\\{s\_null},\39\\{stk\_str})$;\6
\&{end}\6
\4\&{else} \X351:Concatenate the two strings and push\X;\2\2\6
\&{end};\par
\U342.\fi

\M351.
Often both strings will be at the top of the string pool, in which
case we just move some pointers.  Furthermore, it's worth doing some
special stuff in case either string is null, since empirically this
seems to happen about $20\%$ of the time.  In any case, we don't need
the execution buffer---we simple move the strings around in the string
pool when necessary.

\Y\P$\4\X351:Concatenate the two strings and push\X\S$\6
\&{begin} \37\&{if} $(\\{pop\_lit2}\G\\{cmd\_str\_ptr})$ \1\&{then}\6
\&{if} $(\\{pop\_lit1}\G\\{cmd\_str\_ptr})$ \1\&{then}\6
\&{begin} \37$\\{str\_start}[\\{pop\_lit1}]\K\\{str\_start}[\\{pop\_lit1}+1]$;\5
\\{unflush\_string};\5
$\\{incr}(\\{lit\_stk\_ptr})$;\6
\&{end}\6
\4\&{else} \&{if} $(\\{length}(\\{pop\_lit2})=0)$ \1\&{then}\5
$\\{push\_lit\_stk}(\\{pop\_lit1},\39\\{stk\_str})$\6
\4\&{else} \C{\\{pop\_lit2} is nonnull, only \\{pop\_lit1} is below \\{cmd\_str%
\_ptr}}\2\2\2\6
\&{begin} \37$\\{pool\_ptr}\K\\{str\_start}[\\{pop\_lit2}+1]$;\5
$\\{str\_room}(\\{length}(\\{pop\_lit1}))$;\5
$\\{sp\_ptr}\K\\{str\_start}[\\{pop\_lit1}]$;\5
$\\{sp\_end}\K\\{str\_start}[\\{pop\_lit1}+1]$;\6
\&{while} $(\\{sp\_ptr}<\\{sp\_end})$ \1\&{do}\6
\&{begin} \37$\\{append\_char}(\\{str\_pool}[\\{sp\_ptr}])$;\5
$\\{incr}(\\{sp\_ptr})$;\6
\&{end};\2\6
$\\{push\_lit\_stk}(\\{make\_string},\39\\{stk\_str})$;\C{and push it onto the
stack}\6
\&{end}\6
\4\&{else} \37\X352:Concatenate them and push when $\\{pop\_lit2}<\\{cmd\_str%
\_ptr}$\X;\6
\&{end}\par
\U350.\fi

\M352.
We simply continue the previous module.

\Y\P$\4\X352:Concatenate them and push when $\\{pop\_lit2}<\\{cmd\_str\_ptr}$\X%
\S$\6
\&{begin} \37\&{if} $(\\{pop\_lit1}\G\\{cmd\_str\_ptr})$ \1\&{then}\6
\&{if} $(\\{length}(\\{pop\_lit2})=0)$ \1\&{then}\6
\&{begin} \37\\{unflush\_string};\5
$\\{lit\_stack}[\\{lit\_stk\_ptr}]\K\\{pop\_lit1}$;\5
$\\{incr}(\\{lit\_stk\_ptr})$;\6
\&{end}\6
\4\&{else} \&{if} $(\\{length}(\\{pop\_lit1})=0)$ \1\&{then}\5
$\\{incr}(\\{lit\_stk\_ptr})$\6
\4\&{else} \C{both strings nonnull, only \\{pop\_lit2} is below \\{cmd\_str%
\_ptr}}\2\2\2\6
\&{begin} \37$\\{sp\_length}\K\\{length}(\\{pop\_lit1})$;\5
$\\{sp2\_length}\K\\{length}(\\{pop\_lit2})$;\5
$\\{str\_room}(\\{sp\_length}+\\{sp2\_length})$;\5
$\\{sp\_ptr}\K\\{str\_start}[\\{pop\_lit1}+1]$;\5
$\\{sp\_end}\K\\{str\_start}[\\{pop\_lit1}]$;\5
$\\{sp\_xptr1}\K\\{sp\_ptr}+\\{sp2\_length}$;\6
\&{while} $(\\{sp\_ptr}>\\{sp\_end})$ \1\&{do}\C{slide up \\{pop\_lit1}}\6
\&{begin} \37$\\{decr}(\\{sp\_ptr})$;\5
$\\{decr}(\\{sp\_xptr1})$;\5
$\\{str\_pool}[\\{sp\_xptr1}]\K\\{str\_pool}[\\{sp\_ptr}]$;\6
\&{end};\2\6
$\\{sp\_ptr}\K\\{str\_start}[\\{pop\_lit2}]$;\5
$\\{sp\_end}\K\\{str\_start}[\\{pop\_lit2}+1]$;\6
\&{while} $(\\{sp\_ptr}<\\{sp\_end})$ \1\&{do}\C{slide up \\{pop\_lit2}}\6
\&{begin} \37$\\{append\_char}(\\{str\_pool}[\\{sp\_ptr}])$;\5
$\\{incr}(\\{sp\_ptr})$;\6
\&{end};\2\6
$\\{pool\_ptr}\K\\{pool\_ptr}+\\{sp\_length}$;\5
$\\{push\_lit\_stk}(\\{make\_string},\39\\{stk\_str})$;\C{and push it onto the
stack}\6
\&{end}\6
\4\&{else} \37\X353:Concatenate them and push when $\\{pop\_lit1},\\{pop%
\_lit2}<\\{cmd\_str\_ptr}$\X;\6
\&{end}\par
\U351.\fi

\M353.
Again, we simply continue the previous module.

\Y\P$\4\X353:Concatenate them and push when $\\{pop\_lit1},\\{pop\_lit2}<\\{cmd%
\_str\_ptr}$\X\S$\6
\&{begin} \37\&{if} $(\\{length}(\\{pop\_lit1})=0)$ \1\&{then}\5
$\\{incr}(\\{lit\_stk\_ptr})$\6
\4\&{else} \&{if} $(\\{length}(\\{pop\_lit2})=0)$ \1\&{then}\5
$\\{push\_lit\_stk}(\\{pop\_lit1},\39\\{stk\_str})$\6
\4\&{else} \C{both strings are nonnull, and both are below \\{cmd\_str\_ptr}}\2%
\2\6
\&{begin} \37$\\{str\_room}(\\{length}(\\{pop\_lit1})+\\{length}(\\{pop%
\_lit2}))$;\5
$\\{sp\_ptr}\K\\{str\_start}[\\{pop\_lit2}]$;\5
$\\{sp\_end}\K\\{str\_start}[\\{pop\_lit2}+1]$;\6
\&{while} $(\\{sp\_ptr}<\\{sp\_end})$ \1\&{do}\C{slide up \\{pop\_lit2}}\6
\&{begin} \37$\\{append\_char}(\\{str\_pool}[\\{sp\_ptr}])$;\5
$\\{incr}(\\{sp\_ptr})$;\6
\&{end};\2\6
$\\{sp\_ptr}\K\\{str\_start}[\\{pop\_lit1}]$;\5
$\\{sp\_end}\K\\{str\_start}[\\{pop\_lit1}+1]$;\6
\&{while} $(\\{sp\_ptr}<\\{sp\_end})$ \1\&{do}\C{slide up \\{pop\_lit1}}\6
\&{begin} \37$\\{append\_char}(\\{str\_pool}[\\{sp\_ptr}])$;\5
$\\{incr}(\\{sp\_ptr})$;\6
\&{end};\2\6
$\\{push\_lit\_stk}(\\{make\_string},\39\\{stk\_str})$;\C{and push it onto the
stack}\6
\&{end};\6
\&{end}\par
\U352.\fi

\M354.
The \\{built\_in} function {\.{:=}} pops the top two literals and assigns
to the first (which must be an \\{int\_entry\_var}, a \\{str\_entry\_var}, an
\\{int\_global\_var}, or a \\{str\_global\_var}) the value of the second;
it complains if the value isn't of the appropriate type.

\Y\P$\4\X354:\\{execute\_fn}({\.{:=}})\X\S$\6
\4\&{procedure}\1\  \37\\{x\_gets};\2\6
\&{begin} \37$\\{pop\_lit\_stk}(\\{pop\_lit1},\39\\{pop\_typ1})$;\5
$\\{pop\_lit\_stk}(\\{pop\_lit2},\39\\{pop\_typ2})$;\6
\&{if} $(\\{pop\_typ1}\I\\{stk\_fn})$ \1\&{then}\5
$\\{print\_wrong\_stk\_lit}(\\{pop\_lit1},\39\\{pop\_typ1},\39\\{stk\_fn})$\6
\4\&{else} \&{if} $((\R\\{mess\_with\_entries})\W((\\{fn\_type}[\\{pop\_lit1}]=%
\\{str\_entry\_var})\V(\\{fn\_type}[\\{pop\_lit1}]=\\{int\_entry\_var})))$ \1%
\&{then}\5
\\{bst\_cant\_mess\_with\_entries\_print}\6
\4\&{else} \&{case} $(\\{fn\_type}[\\{pop\_lit1}])$ \1\&{of}\6
\4\\{int\_entry\_var}: \37\X355:Assign to an \\{int\_entry\_var}\X;\6
\4\\{str\_entry\_var}: \37\X357:Assign to a \\{str\_entry\_var}\X;\6
\4\\{int\_global\_var}: \37\X358:Assign to an \\{int\_global\_var}\X;\6
\4\\{str\_global\_var}: \37\X359:Assign to a \\{str\_global\_var}\X;\6
\4\&{othercases} \37\&{begin} \37$\\{print}(\.{\'You\ can\'}\.{\'t\ assign\ to\
type\ \'})$;\5
$\\{print\_fn\_class}(\\{pop\_lit1})$;\5
$\\{bst\_ex\_warn}(\.{\',\ a\ nonvariable\ function\ class\'})$;\6
\&{end}\2\6
\&{endcases};\2\2\6
\&{end};\par
\U342.\fi

\M355.
This module checks that what we're about to assign is really an
integer, and then assigns.

\Y\P$\4\X355:Assign to an \\{int\_entry\_var}\X\S$\6
\&{if} $(\\{pop\_typ2}\I\\{stk\_int})$ \1\&{then}\5
$\\{print\_wrong\_stk\_lit}(\\{pop\_lit2},\39\\{pop\_typ2},\39\\{stk\_int})$\6
\4\&{else} $\\{entry\_ints}[\\{cite\_ptr}\ast\\{num\_ent\_ints}+\\{fn\_info}[%
\\{pop\_lit1}]]\K\\{pop\_lit2}$\2\par
\U354.\fi

\M356.
It's time for a complaint if either of the two (entry or global)
string lengths is exceeded.

\Y\P\D \37$\\{bst\_string\_size\_exceeded}(\#)\S$\1\6
\&{begin} \37\\{bst\_1print\_string\_size\_exceeded};\5
$\\{print}(\#)$;\5
\\{bst\_2print\_string\_size\_exceeded};\6
\&{end}\2\par
\Y\P$\4\X3:Procedures and functions for all file I/O, error messages, and such%
\X\mathrel{+}\S$\6
\4\&{procedure}\1\  \37\\{bst\_1print\_string\_size\_exceeded};\2\6
\&{begin} \37$\\{print}(\.{\'Warning--you\'}\.{\'ve\ exceeded\ \'})$;\6
\&{end};\7
\4\&{procedure}\1\  \37\\{bst\_2print\_string\_size\_exceeded};\2\6
\&{begin} \37$\\{print}(\.{\'-string-size,\'})$;\5
\\{bst\_mild\_ex\_warn\_print};\5
$\\{print\_ln}(\.{\'*Please\ notify\ the\ bibstyle\ designer*\'})$;\6
\&{end};\par
\fi

\M357.
This module checks that what we're about to assign is really a
string, and then assigns.

\Y\P$\4\X357:Assign to a \\{str\_entry\_var}\X\S$\6
\&{begin} \37\&{if} $(\\{pop\_typ2}\I\\{stk\_str})$ \1\&{then}\5
$\\{print\_wrong\_stk\_lit}(\\{pop\_lit2},\39\\{pop\_typ2},\39\\{stk\_str})$\6
\4\&{else} \&{begin} \37$\\{str\_ent\_ptr}\K\\{cite\_ptr}\ast\\{num\_ent%
\_strs}+\\{fn\_info}[\\{pop\_lit1}]$;\5
$\\{ent\_chr\_ptr}\K0$;\5
$\\{sp\_ptr}\K\\{str\_start}[\\{pop\_lit2}]$;\5
$\\{sp\_xptr1}\K\\{str\_start}[\\{pop\_lit2}+1]$;\6
\&{if} $(\\{sp\_xptr1}-\\{sp\_ptr}>\\{ent\_str\_size})$ \1\&{then}\6
\&{begin} \37$\\{bst\_string\_size\_exceeded}(\\{ent\_str\_size}:0,\39\.{\',\
the\ entry\'})$;\5
$\\{sp\_xptr1}\K\\{sp\_ptr}+\\{ent\_str\_size}$;\6
\&{end};\2\6
\&{while} $(\\{sp\_ptr}<\\{sp\_xptr1})$ \1\&{do}\6
\&{begin} \37\C{copy characters into \\{entry\_strs}}\6
$\\{entry\_strs}[\\{str\_ent\_ptr}][\\{ent\_chr\_ptr}]\K\\{str\_pool}[\\{sp%
\_ptr}]$;\5
$\\{incr}(\\{ent\_chr\_ptr})$;\5
$\\{incr}(\\{sp\_ptr})$;\6
\&{end};\2\6
$\\{entry\_strs}[\\{str\_ent\_ptr}][\\{ent\_chr\_ptr}]\K\\{end\_of\_string}$;\6
\&{end}\2\6
\&{end}\par
\U354.\fi

\M358.
This module checks that what we're about to assign is really an
integer, and then assigns.

\Y\P$\4\X358:Assign to an \\{int\_global\_var}\X\S$\6
\&{if} $(\\{pop\_typ2}\I\\{stk\_int})$ \1\&{then}\5
$\\{print\_wrong\_stk\_lit}(\\{pop\_lit2},\39\\{pop\_typ2},\39\\{stk\_int})$\6
\4\&{else} $\\{fn\_info}[\\{pop\_lit1}]\K\\{pop\_lit2}$\2\par
\U354.\fi

\M359.
This module checks that what we're about to assign is really a
string, and then assigns.

\Y\P$\4\X359:Assign to a \\{str\_global\_var}\X\S$\6
\&{begin} \37\&{if} $(\\{pop\_typ2}\I\\{stk\_str})$ \1\&{then}\5
$\\{print\_wrong\_stk\_lit}(\\{pop\_lit2},\39\\{pop\_typ2},\39\\{stk\_str})$\6
\4\&{else} \&{begin} \37$\\{str\_glb\_ptr}\K\\{fn\_info}[\\{pop\_lit1}]$;\6
\&{if} $(\\{pop\_lit2}<\\{cmd\_str\_ptr})$ \1\&{then}\5
$\\{glb\_str\_ptr}[\\{str\_glb\_ptr}]\K\\{pop\_lit2}$\6
\4\&{else} \&{begin} \37$\\{glb\_str\_ptr}[\\{str\_glb\_ptr}]\K0$;\5
$\\{glob\_chr\_ptr}\K0$;\5
$\\{sp\_ptr}\K\\{str\_start}[\\{pop\_lit2}]$;\5
$\\{sp\_end}\K\\{str\_start}[\\{pop\_lit2}+1]$;\6
\&{if} $(\\{sp\_end}-\\{sp\_ptr}>\\{glob\_str\_size})$ \1\&{then}\6
\&{begin} \37$\\{bst\_string\_size\_exceeded}(\\{glob\_str\_size}:0,\39\.{\',\
the\ global\'})$;\5
$\\{sp\_end}\K\\{sp\_ptr}+\\{glob\_str\_size}$;\6
\&{end};\2\6
\&{while} $(\\{sp\_ptr}<\\{sp\_end})$ \1\&{do}\6
\&{begin} \37\C{copy characters into \\{global\_strs}}\6
$\\{global\_strs}[\\{str\_glb\_ptr}][\\{glob\_chr\_ptr}]\K\\{str\_pool}[\\{sp%
\_ptr}]$;\5
$\\{incr}(\\{glob\_chr\_ptr})$;\5
$\\{incr}(\\{sp\_ptr})$;\6
\&{end};\2\6
$\\{glb\_str\_end}[\\{str\_glb\_ptr}]\K\\{glob\_chr\_ptr}$;\6
\&{end};\2\6
\&{end}\2\6
\&{end}\par
\U354.\fi

\M360.
The \\{built\_in} function {\.{add.period\$}} pops the top (string)
literal, adds a \\{period} to a nonnull string if its last
non\\{right\_brace} character isn't a \\{period}, \\{question\_mark}, or
\\{exclamation\_mark}, and pushes this resulting string back onto the
stack.  If the literal isn't a string, it complains and pushes the
null string.

\Y\P$\4\X360:\\{execute\_fn}({\.{add.period\$}})\X\S$\6
\4\&{procedure}\1\  \37\\{x\_add\_period};\6
\4\&{label} \37\\{loop\_exit};\2\6
\&{begin} \37$\\{pop\_lit\_stk}(\\{pop\_lit1},\39\\{pop\_typ1})$;\6
\&{if} $(\\{pop\_typ1}\I\\{stk\_str})$ \1\&{then}\6
\&{begin} \37$\\{print\_wrong\_stk\_lit}(\\{pop\_lit1},\39\\{pop\_typ1},\39%
\\{stk\_str})$;\5
$\\{push\_lit\_stk}(\\{s\_null},\39\\{stk\_str})$;\6
\&{end}\6
\4\&{else} \&{if} $(\\{length}(\\{pop\_lit1})=0)$ \1\&{then}\C{don't add %
\\{period} to the null string}\6
$\\{push\_lit\_stk}(\\{s\_null},\39\\{stk\_str})$\6
\4\&{else} \X361:Add the \\{period}, if necessary, and push\X;\2\2\6
\&{end};\par
\U342.\fi

\M361.
Here we scan backwards from the end of the string, skipping
non\\{right\_brace} characters, to see if we have to add the \\{period}.

\Y\P$\4\X361:Add the \\{period}, if necessary, and push\X\S$\6
\&{begin} \37$\\{sp\_ptr}\K\\{str\_start}[\\{pop\_lit1}+1]$;\5
$\\{sp\_end}\K\\{str\_start}[\\{pop\_lit1}]$;\6
\&{while} $(\\{sp\_ptr}>\\{sp\_end})$ \1\&{do}\C{find a non\\{right\_brace}}\6
\&{begin} \37$\\{decr}(\\{sp\_ptr})$;\6
\&{if} $(\\{str\_pool}[\\{sp\_ptr}]\I\\{right\_brace})$ \1\&{then}\5
\&{goto} \37\\{loop\_exit};\2\6
\&{end};\2\6
\4\\{loop\_exit}: \37\&{case} $(\\{str\_pool}[\\{sp\_ptr}])$ \1\&{of}\6
\4$\\{period},\39\\{question\_mark},\39\\{exclamation\_mark}$: \37\\{repush%
\_string};\6
\4\&{othercases} \37\X362:Add the \\{period} (it's necessary) and push\X\2\6
\&{endcases};\6
\&{end}\par
\U360.\fi

\M362.
Ok guys, we really have to do it.

\Y\P$\4\X362:Add the \\{period} (it's necessary) and push\X\S$\6
\&{begin} \37\&{if} $(\\{pop\_lit1}<\\{cmd\_str\_ptr})$ \1\&{then}\6
\&{begin} \37$\\{str\_room}(\\{length}(\\{pop\_lit1})+1)$;\5
$\\{sp\_ptr}\K\\{str\_start}[\\{pop\_lit1}]$;\5
$\\{sp\_end}\K\\{str\_start}[\\{pop\_lit1}+1]$;\6
\&{while} $(\\{sp\_ptr}<\\{sp\_end})$ \1\&{do}\C{slide \\{pop\_lit1} atop the
string pool}\6
\&{begin} \37$\\{append\_char}(\\{str\_pool}[\\{sp\_ptr}])$;\5
$\\{incr}(\\{sp\_ptr})$;\6
\&{end};\2\6
\&{end}\6
\4\&{else} \C{the string is already there}\2\6
\&{begin} \37$\\{pool\_ptr}\K\\{str\_start}[\\{pop\_lit1}+1]$;\5
$\\{str\_room}(1)$;\6
\&{end};\5
$\\{append\_char}(\\{period})$;\5
$\\{push\_lit\_stk}(\\{make\_string},\39\\{stk\_str})$;\6
\&{end}\par
\U361.\fi

\M363.
The \\{built\_in} function {\.{call.type\$}} executes the function
specified in \\{type\_list} for this entry unless it's \\{undefined}, in
which case it executes the default function \.{default.type} defined
in the \.{.bst} file, or unless it's \\{empty}, in which case it does
nothing.

\Y\P$\4\X363:\\{execute\_fn}({\.{call.type\$}})\X\S$\6
\&{begin} \37\&{if} $(\R\\{mess\_with\_entries})$ \1\&{then}\5
\\{bst\_cant\_mess\_with\_entries\_print}\6
\4\&{else} \&{if} $(\\{type\_list}[\\{cite\_ptr}]=\\{undefined})$ \1\&{then}\5
$\\{execute\_fn}(\\{b\_default})$\6
\4\&{else} \&{if} $(\\{type\_list}[\\{cite\_ptr}]=\\{empty})$ \1\&{then}\5
\\{do\_nothing}\6
\4\&{else} $\\{execute\_fn}(\\{type\_list}[\\{cite\_ptr}])$;\2\2\2\6
\&{end}\par
\U341.\fi

\M364.
The \\{built\_in} function {\.{change.case\$}} pops the top two (string)
literals; it changes the case of the second according to the
specifications of the first, as follows.  (Note: The word `letters' in
the next sentence refers only to those at brace-level~0, the top-most
brace level; no other characters are changed, except perhaps for
special characters, described shortly.)  If the first literal is the
string~\.{t}, it converts to lower case all letters except the very
first character in the string, which it leaves alone, and except the
first character following any \\{colon} and then nonnull \\{white\_space},
which it also leaves alone; if it's the string~\.{l}, it converts all
letters to lower case; if it's the string~\.{u}, it converts all
letters to upper case; and if it's anything else, it complains and
does no conversion.  It then pushes this resulting string.  If either
type is incorrect, it complains and pushes the null string; however,
if both types are correct but the specification string (i.e., the
first string) isn't one of the legal ones, it merely pushes the second
back onto the stack, after complaining.  (Another note: It ignores
case differences in the specification string; for example, the strings
\.{t} and \.{T} are equivalent for the purposes of this \\{built\_in}
function.)

\Y\P\D \37$\\{ok\_pascal\_i\_give\_up}=21$\par
\Y\P$\4\X364:\\{execute\_fn}({\.{change.case\$}})\X\S$\6
\4\&{procedure}\1\  \37\\{x\_change\_case};\6
\4\&{label} \37\\{ok\_pascal\_i\_give\_up};\2\6
\&{begin} \37$\\{pop\_lit\_stk}(\\{pop\_lit1},\39\\{pop\_typ1})$;\5
$\\{pop\_lit\_stk}(\\{pop\_lit2},\39\\{pop\_typ2})$;\6
\&{if} $(\\{pop\_typ1}\I\\{stk\_str})$ \1\&{then}\6
\&{begin} \37$\\{print\_wrong\_stk\_lit}(\\{pop\_lit1},\39\\{pop\_typ1},\39%
\\{stk\_str})$;\5
$\\{push\_lit\_stk}(\\{s\_null},\39\\{stk\_str})$;\6
\&{end}\6
\4\&{else} \&{if} $(\\{pop\_typ2}\I\\{stk\_str})$ \1\&{then}\6
\&{begin} \37$\\{print\_wrong\_stk\_lit}(\\{pop\_lit2},\39\\{pop\_typ2},\39%
\\{stk\_str})$;\5
$\\{push\_lit\_stk}(\\{s\_null},\39\\{stk\_str})$;\6
\&{end}\6
\4\&{else} \&{begin} \37\X366:Determine the case-conversion type\X;\6
$\\{ex\_buf\_length}\K0$;\5
$\\{add\_buf\_pool}(\\{pop\_lit2})$;\5
\X370:Perform the case conversion\X;\6
\\{add\_pool\_buf\_and\_push};\C{push this string onto the stack}\6
\&{end};\2\2\6
\&{end};\par
\U342.\fi

\M365.
First we define a few variables for case conversion.  The constant
definitions, to be used in   \&{case}  statements, are in order of probable
frequency.

\Y\P\D \37$\\{title\_lowers}=0$\C{representing the string \.{t}}\par
\P\D \37$\\{all\_lowers}=1$\C{representing the string \.{l}}\par
\P\D \37$\\{all\_uppers}=2$\C{representing the string \.{u}}\par
\P\D \37$\\{bad\_conversion}=3$\C{representing any illegal case-conversion
string}\par
\Y\P$\4\X16:Globals in the outer block\X\mathrel{+}\S$\6
\4\\{conversion\_type}: \37$0\to\\{bad\_conversion}$;\C{the possible cases}\6
\4\\{prev\_colon}: \37\\{boolean};\C{\\{true} if just past a \\{colon}}\par
\fi

\M366.
Now we determine which of the three case-conversion types we're
dealing with: \.{t},~\.{l}, or~\.{u}.

\Y\P$\4\X366:Determine the case-conversion type\X\S$\6
\&{begin} \37\&{case} $(\\{str\_pool}[\\{str\_start}[\\{pop\_lit1}]])$ \1\&{of}%
\6
\4$\.{"t"},\39\.{"T"}$: \37$\\{conversion\_type}\K\\{title\_lowers}$;\6
\4$\.{"l"},\39\.{"L"}$: \37$\\{conversion\_type}\K\\{all\_lowers}$;\6
\4$\.{"u"},\39\.{"U"}$: \37$\\{conversion\_type}\K\\{all\_uppers}$;\6
\4\&{othercases} \37$\\{conversion\_type}\K\\{bad\_conversion}$\2\6
\&{endcases};\6
\&{if} $((\\{length}(\\{pop\_lit1})\I1)\V(\\{conversion\_type}=\\{bad%
\_conversion}))$ \1\&{then}\6
\&{begin} \37$\\{conversion\_type}\K\\{bad\_conversion}$;\5
$\\{print\_pool\_str}(\\{pop\_lit1})$;\5
$\\{bst\_ex\_warn}(\.{\'\ is\ an\ illegal\ case-conversion\ string\'})$;\6
\&{end};\2\6
\&{end}\par
\U364.\fi

\M367.
This procedure complains if the just-encountered \\{right\_brace} would
make \\{brace\_level} negative.

\Y\P$\4\X367:Procedures and functions for name-string processing\X\S$\6
\4\&{procedure}\1\  \37$\\{decr\_brace\_level}(\\{pop\_lit\_var}:\\{str%
\_number})$;\2\6
\&{begin} \37\&{if} $(\\{brace\_level}=0)$ \1\&{then}\5
$\\{braces\_unbalanced\_complaint}(\\{pop\_lit\_var})$\6
\4\&{else} $\\{decr}(\\{brace\_level})$;\2\6
\&{end};\par
\As369, 384, 397, 401, 404, 406, 418\ETs420.
\U12.\fi

\M368.
This complaint often arises because the style designer has to type
lots of braces.

\Y\P$\4\X3:Procedures and functions for all file I/O, error messages, and such%
\X\mathrel{+}\S$\6
\4\&{procedure}\1\  \37$\\{braces\_unbalanced\_complaint}(\\{pop\_lit\_var}:%
\\{str\_number})$;\2\6
\&{begin} \37$\\{print}(\.{\'Warning--"\'})$;\5
$\\{print\_pool\_str}(\\{pop\_lit\_var})$;\5
$\\{bst\_mild\_ex\_warn}(\.{\'"\ isn\'}\.{\'t\ a\ brace-balanced\ string\'})$;\6
\&{end};\par
\fi

\M369.
This one makes sure that $\\{brace\_level}=0$ (it's called at a point in a
string where braces must be balanced).

\Y\P$\4\X367:Procedures and functions for name-string processing\X\mathrel{+}%
\S$\6
\4\&{procedure}\1\  \37$\\{check\_brace\_level}(\\{pop\_lit\_var}:\\{str%
\_number})$;\2\6
\&{begin} \37\&{if} $(\\{brace\_level}>0)$ \1\&{then}\5
$\\{braces\_unbalanced\_complaint}(\\{pop\_lit\_var})$;\2\6
\&{end};\par
\fi

\M370.
Here's where we actually go through the string and do the case
conversion.

\Y\P$\4\X370:Perform the case conversion\X\S$\6
\&{begin} \37$\\{brace\_level}\K0$;\C{this is the top level}\6
$\\{ex\_buf\_ptr}\K0$;\C{we start with the string's first character}\6
\&{while} $(\\{ex\_buf\_ptr}<\\{ex\_buf\_length})$ \1\&{do}\6
\&{begin} \37\&{if} $(\\{ex\_buf}[\\{ex\_buf\_ptr}]=\\{left\_brace})$ \1%
\&{then}\6
\&{begin} \37$\\{incr}(\\{brace\_level})$;\6
\&{if} $(\\{brace\_level}\I1)$ \1\&{then}\5
\&{goto} \37\\{ok\_pascal\_i\_give\_up};\2\6
\&{if} $(\\{ex\_buf\_ptr}+4>\\{ex\_buf\_length})$ \1\&{then}\5
\&{goto} \37\\{ok\_pascal\_i\_give\_up}\6
\4\&{else} \&{if} $(\\{ex\_buf}[\\{ex\_buf\_ptr}+1]\I\\{backslash})$ \1\&{then}%
\5
\&{goto} \37\\{ok\_pascal\_i\_give\_up};\2\2\6
\&{if} $(\\{conversion\_type}=\\{title\_lowers})$ \1\&{then}\6
\&{if} $(\\{ex\_buf\_ptr}=0)$ \1\&{then}\5
\&{goto} \37\\{ok\_pascal\_i\_give\_up}\6
\4\&{else} \&{if} $((\\{prev\_colon})\W(\\{lex\_class}[\\{ex\_buf}[\\{ex\_buf%
\_ptr}-1]]=\\{white\_space}))$ \1\&{then}\5
\&{goto} \37\\{ok\_pascal\_i\_give\_up};\2\2\2\6
\X371:Convert a special character\X;\6
\4\\{ok\_pascal\_i\_give\_up}: \37$\\{prev\_colon}\K\\{false}$;\6
\&{end}\6
\4\&{else} \&{if} $(\\{ex\_buf}[\\{ex\_buf\_ptr}]=\\{right\_brace})$ \1\&{then}%
\6
\&{begin} \37$\\{decr\_brace\_level}(\\{pop\_lit2})$;\5
$\\{prev\_colon}\K\\{false}$;\6
\&{end}\6
\4\&{else} \&{if} $(\\{brace\_level}=0)$ \1\&{then}\5
\X376:Convert a $\\{brace\_level}=0$ character\X;\2\2\2\6
$\\{incr}(\\{ex\_buf\_ptr})$;\6
\&{end};\2\6
$\\{check\_brace\_level}(\\{pop\_lit2})$;\6
\&{end}\par
\U364.\fi

\M371.
We're dealing with a special character (usually either an undotted
`\i' or `\j', or an accent like one in Table~3.1 of the \LaTeX\
manual, or a foreign character like one in Table~3.2) if the first
character after the \\{left\_brace} is a \\{backslash}; the special
character ends with the matching \\{right\_brace}.  How we handle what's
in between depends on the special character.  In general, this code
will do reasonably well if there is other stuff, too, between braces,
but it doesn't try to do anything special with \\{colon}s.

\Y\P$\4\X371:Convert a special character\X\S$\6
\&{begin} \37$\\{incr}(\\{ex\_buf\_ptr})$;\C{skip over the \\{left\_brace}}\6
\&{while} $((\\{ex\_buf\_ptr}<\\{ex\_buf\_length})\W(\\{brace\_level}>0))$ \1%
\&{do}\6
\&{begin} \37$\\{incr}(\\{ex\_buf\_ptr})$;\C{skip over the \\{backslash}}\6
$\\{ex\_buf\_xptr}\K\\{ex\_buf\_ptr}$;\6
\&{while} $((\\{ex\_buf\_ptr}<\\{ex\_buf\_length})\W(\\{lex\_class}[\\{ex%
\_buf}[\\{ex\_buf\_ptr}]]=\\{alpha}))$ \1\&{do}\5
$\\{incr}(\\{ex\_buf\_ptr})$;\C{this scans the control sequence}\2\6
$\\{control\_seq\_loc}\K\\{str\_lookup}(\\{ex\_buf},\39\\{ex\_buf\_xptr},\39%
\\{ex\_buf\_ptr}-\\{ex\_buf\_xptr},\39\\{control\_seq\_ilk},\39\\{dont%
\_insert})$;\6
\&{if} $(\\{hash\_found})$ \1\&{then}\5
\X372:Convert the accented or foreign character, if necessary\X;\2\6
$\\{ex\_buf\_xptr}\K\\{ex\_buf\_ptr}$;\6
\&{while} $((\\{ex\_buf\_ptr}<\\{ex\_buf\_length})\W(\\{brace\_level}>0)\W(%
\\{ex\_buf}[\\{ex\_buf\_ptr}]\I\\{backslash}))$ \1\&{do}\6
\&{begin} \37\C{this scans to the next control sequence}\6
\&{if} $(\\{ex\_buf}[\\{ex\_buf\_ptr}]=\\{right\_brace})$ \1\&{then}\5
$\\{decr}(\\{brace\_level})$\6
\4\&{else} \&{if} $(\\{ex\_buf}[\\{ex\_buf\_ptr}]=\\{left\_brace})$ \1\&{then}\5
$\\{incr}(\\{brace\_level})$;\2\2\6
$\\{incr}(\\{ex\_buf\_ptr})$;\6
\&{end};\2\6
\X375:Convert a noncontrol sequence\X;\6
\&{end};\2\6
$\\{decr}(\\{ex\_buf\_ptr})$;\C{unskip the \\{right\_brace}}\6
\&{end}\par
\U370.\fi

\M372.
A control sequence, for the purposes of this program, consists just of
the consecutive alphabetic characters following the \\{backslash}; it
might be empty (although ones in this section aren't).

\Y\P$\4\X372:Convert the accented or foreign character, if necessary\X\S$\6
\&{begin} \37\&{case} $(\\{conversion\_type})$ \1\&{of}\6
\4$\\{title\_lowers},\39\\{all\_lowers}$: \37\&{case} $(\\{ilk\_info}[%
\\{control\_seq\_loc}])$ \1\&{of}\6
\4$\\{n\_l\_upper},\39\\{n\_o\_upper},\39\\{n\_oe\_upper},\39\\{n\_ae\_upper},%
\39\\{n\_aa\_upper}$: \37$\\{lower\_case}(\\{ex\_buf},\39\\{ex\_buf\_xptr},\39%
\\{ex\_buf\_ptr}-\\{ex\_buf\_xptr})$;\6
\4\&{othercases} \37\\{do\_nothing}\2\6
\&{endcases};\6
\4\\{all\_uppers}: \37\&{case} $(\\{ilk\_info}[\\{control\_seq\_loc}])$ \1%
\&{of}\6
\4$\\{n\_l},\39\\{n\_o},\39\\{n\_oe},\39\\{n\_ae},\39\\{n\_aa}$: \37$\\{upper%
\_case}(\\{ex\_buf},\39\\{ex\_buf\_xptr},\39\\{ex\_buf\_ptr}-\\{ex\_buf%
\_xptr})$;\6
\4$\\{n\_i},\39\\{n\_j},\39\\{n\_ss}$: \37\X374:Convert, then remove the
control sequence\X;\6
\4\&{othercases} \37\\{do\_nothing}\2\6
\&{endcases};\6
\4\\{bad\_conversion}: \37\\{do\_nothing};\6
\4\&{othercases} \37\\{case\_conversion\_confusion}\2\6
\&{endcases};\6
\&{end}\par
\U371.\fi

\M373.
Another bug complaint.

\Y\P$\4\X3:Procedures and functions for all file I/O, error messages, and such%
\X\mathrel{+}\S$\6
\4\&{procedure}\1\  \37\\{case\_conversion\_confusion};\2\6
\&{begin} \37$\\{confusion}(\.{\'Unknown\ type\ of\ case\ conversion\'})$;\6
\&{end};\par
\fi

\M374.
After converting the control sequence, we need to remove the preceding
\\{backslash} and any following \\{white\_space}.

\Y\P$\4\X374:Convert, then remove the control sequence\X\S$\6
\&{begin} \37$\\{upper\_case}(\\{ex\_buf},\39\\{ex\_buf\_xptr},\39\\{ex\_buf%
\_ptr}-\\{ex\_buf\_xptr})$;\6
\&{while} $(\\{ex\_buf\_xptr}<\\{ex\_buf\_ptr})$ \1\&{do}\6
\&{begin} \37\C{remove preceding \\{backslash} and shift down}\6
$\\{ex\_buf}[\\{ex\_buf\_xptr}-1]\K\\{ex\_buf}[\\{ex\_buf\_xptr}]$;\5
$\\{incr}(\\{ex\_buf\_xptr})$;\6
\&{end};\2\6
$\\{decr}(\\{ex\_buf\_xptr})$;\6
\&{while} $((\\{ex\_buf\_ptr}<\\{ex\_buf\_length})\W(\\{lex\_class}[\\{ex%
\_buf}[\\{ex\_buf\_ptr}]]=\\{white\_space}))$ \1\&{do}\5
$\\{incr}(\\{ex\_buf\_ptr})$;\C{remove \\{white\_space} trailing the control
seq}\2\6
$\\{tmp\_ptr}\K\\{ex\_buf\_ptr}$;\6
\&{while} $(\\{tmp\_ptr}<\\{ex\_buf\_length})$ \1\&{do}\6
\&{begin} \37\C{more shifting down}\6
$\\{ex\_buf}[\\{tmp\_ptr}-(\\{ex\_buf\_ptr}-\\{ex\_buf\_xptr})]\K\\{ex\_buf}[%
\\{tmp\_ptr}]$;\5
$\\{incr}(\\{tmp\_ptr})$\6
\&{end};\2\6
$\\{ex\_buf\_length}\K\\{tmp\_ptr}-(\\{ex\_buf\_ptr}-\\{ex\_buf\_xptr})$;\5
$\\{ex\_buf\_ptr}\K\\{ex\_buf\_xptr}$;\6
\&{end}\par
\U372.\fi

\M375.
There are no control sequences in what we're about to convert,
so a straight conversion suffices.

\Y\P$\4\X375:Convert a noncontrol sequence\X\S$\6
\&{begin} \37\&{case} $(\\{conversion\_type})$ \1\&{of}\6
\4$\\{title\_lowers},\39\\{all\_lowers}$: \37$\\{lower\_case}(\\{ex\_buf},\39%
\\{ex\_buf\_xptr},\39\\{ex\_buf\_ptr}-\\{ex\_buf\_xptr})$;\6
\4\\{all\_uppers}: \37$\\{upper\_case}(\\{ex\_buf},\39\\{ex\_buf\_xptr},\39%
\\{ex\_buf\_ptr}-\\{ex\_buf\_xptr})$;\6
\4\\{bad\_conversion}: \37\\{do\_nothing};\6
\4\&{othercases} \37\\{case\_conversion\_confusion}\2\6
\&{endcases};\6
\&{end}\par
\U371.\fi

\M376.
This code does any needed conversion for an ordinary character; it
won't touch nonletters.

\Y\P$\4\X376:Convert a $\\{brace\_level}=0$ character\X\S$\6
\&{begin} \37\&{case} $(\\{conversion\_type})$ \1\&{of}\6
\4\\{title\_lowers}: \37\&{begin} \37\&{if} $(\\{ex\_buf\_ptr}=0)$ \1\&{then}\5
\\{do\_nothing}\6
\4\&{else} \&{if} $((\\{prev\_colon})\W(\\{lex\_class}[\\{ex\_buf}[\\{ex\_buf%
\_ptr}-1]]=\\{white\_space}))$ \1\&{then}\5
\\{do\_nothing}\6
\4\&{else} $\\{lower\_case}(\\{ex\_buf},\39\\{ex\_buf\_ptr},\391)$;\2\2\6
\&{if} $(\\{ex\_buf}[\\{ex\_buf\_ptr}]=\\{colon})$ \1\&{then}\5
$\\{prev\_colon}\K\\{true}$\6
\4\&{else} \&{if} $(\\{lex\_class}[\\{ex\_buf}[\\{ex\_buf\_ptr}]]\I\\{white%
\_space})$ \1\&{then}\5
$\\{prev\_colon}\K\\{false}$;\2\2\6
\&{end};\6
\4\\{all\_lowers}: \37$\\{lower\_case}(\\{ex\_buf},\39\\{ex\_buf\_ptr},\391)$;\6
\4\\{all\_uppers}: \37$\\{upper\_case}(\\{ex\_buf},\39\\{ex\_buf\_ptr},\391)$;\6
\4\\{bad\_conversion}: \37\\{do\_nothing};\6
\4\&{othercases} \37\\{case\_conversion\_confusion}\2\6
\&{endcases};\6
\&{end}\par
\U370.\fi

\M377.
The \\{built\_in} function {\.{chr.to.int\$}} pops the top (string)
literal, makes sure it's a single character, converts it to the
corresponding \\{ASCII\_code} integer, and pushes this integer.  If the
literal isn't an appropriate string, it complains and pushes the
integer~0.

\Y\P$\4\X377:\\{execute\_fn}({\.{chr.to.int\$}})\X\S$\6
\4\&{procedure}\1\  \37\\{x\_chr\_to\_int};\2\6
\&{begin} \37$\\{pop\_lit\_stk}(\\{pop\_lit1},\39\\{pop\_typ1})$;\6
\&{if} $(\\{pop\_typ1}\I\\{stk\_str})$ \1\&{then}\6
\&{begin} \37$\\{print\_wrong\_stk\_lit}(\\{pop\_lit1},\39\\{pop\_typ1},\39%
\\{stk\_str})$;\5
$\\{push\_lit\_stk}(0,\39\\{stk\_int})$;\6
\&{end}\6
\4\&{else} \&{if} $(\\{length}(\\{pop\_lit1})\I1)$ \1\&{then}\6
\&{begin} \37$\\{print}(\.{\'"\'})$;\5
$\\{print\_pool\_str}(\\{pop\_lit1})$;\5
$\\{bst\_ex\_warn}(\.{\'"\ isn\'}\.{\'t\ a\ single\ character\'})$;\5
$\\{push\_lit\_stk}(0,\39\\{stk\_int})$;\6
\&{end}\6
\4\&{else} $\\{push\_lit\_stk}(\\{str\_pool}[\\{str\_start}[\\{pop\_lit1}]],\39%
\\{stk\_int})$;\C{push the (\\{ASCII\_code}) integer}\2\2\6
\&{end};\par
\U342.\fi

\M378.
The \\{built\_in} function {\.{cite\$}} pushes the appropriate string
from \\{cite\_list} onto the stack.

\Y\P$\4\X378:\\{execute\_fn}({\.{cite\$}})\X\S$\6
\4\&{procedure}\1\  \37\\{x\_cite};\2\6
\&{begin} \37\&{if} $(\R\\{mess\_with\_entries})$ \1\&{then}\5
\\{bst\_cant\_mess\_with\_entries\_print}\6
\4\&{else} $\\{push\_lit\_stk}(\\{cur\_cite\_str},\39\\{stk\_str})$;\2\6
\&{end};\par
\U342.\fi

\M379.
The \\{built\_in} function {\.{duplicate\$}} pops the top literal from
the stack and pushes two copies of it.

\Y\P$\4\X379:\\{execute\_fn}({\.{duplicate\$}})\X\S$\6
\4\&{procedure}\1\  \37\\{x\_duplicate};\2\6
\&{begin} \37$\\{pop\_lit\_stk}(\\{pop\_lit1},\39\\{pop\_typ1})$;\6
\&{if} $(\\{pop\_typ1}\I\\{stk\_str})$ \1\&{then}\6
\&{begin} \37$\\{push\_lit\_stk}(\\{pop\_lit1},\39\\{pop\_typ1})$;\5
$\\{push\_lit\_stk}(\\{pop\_lit1},\39\\{pop\_typ1})$;\6
\&{end}\6
\4\&{else} \&{begin} \37\\{repush\_string};\6
\&{if} $(\\{pop\_lit1}<\\{cmd\_str\_ptr})$ \1\&{then}\5
$\\{push\_lit\_stk}(\\{pop\_lit1},\39\\{pop\_typ1})$\6
\4\&{else} \&{begin} \37$\\{str\_room}(\\{length}(\\{pop\_lit1}))$;\5
$\\{sp\_ptr}\K\\{str\_start}[\\{pop\_lit1}]$;\5
$\\{sp\_end}\K\\{str\_start}[\\{pop\_lit1}+1]$;\6
\&{while} $(\\{sp\_ptr}<\\{sp\_end})$ \1\&{do}\6
\&{begin} \37$\\{append\_char}(\\{str\_pool}[\\{sp\_ptr}])$;\5
$\\{incr}(\\{sp\_ptr})$;\6
\&{end};\2\6
$\\{push\_lit\_stk}(\\{make\_string},\39\\{stk\_str})$;\C{and push it onto the
stack}\6
\&{end};\2\6
\&{end};\2\6
\&{end};\par
\U342.\fi

\M380.
The \\{built\_in} function {\.{empty\$}} pops the top literal and pushes
the integer 1 if it's a missing field or a string having no
non\\{white\_space} characters, 0 otherwise.  If the literal isn't a
missing field or a string, it complains and pushes 0.

\Y\P$\4\X380:\\{execute\_fn}({\.{empty\$}})\X\S$\6
\4\&{procedure}\1\  \37\\{x\_empty};\6
\4\&{label} \37\\{exit};\2\6
\&{begin} \37$\\{pop\_lit\_stk}(\\{pop\_lit1},\39\\{pop\_typ1})$;\6
\&{case} $(\\{pop\_typ1})$ \1\&{of}\6
\4\\{stk\_str}: \37\X381:Push 0 if the string has a non\\{white\_space} char,
else 1\X;\6
\4\\{stk\_field\_missing}: \37$\\{push\_lit\_stk}(1,\39\\{stk\_int})$;\6
\4\\{stk\_empty}: \37$\\{push\_lit\_stk}(0,\39\\{stk\_int})$;\6
\4\&{othercases} \37\&{begin} \37$\\{print\_stk\_lit}(\\{pop\_lit1},\39\\{pop%
\_typ1})$;\5
$\\{bst\_ex\_warn}(\.{\',\ not\ a\ string\ or\ missing\ field,\'})$;\5
$\\{push\_lit\_stk}(0,\39\\{stk\_int})$;\6
\&{end}\2\6
\&{endcases};\6
\4\\{exit}: \37\&{end};\par
\U342.\fi

\M381.
When we arrive here we're dealing with a legitimate string.  If it has
no characters, or has nothing but \\{white\_space} characters, we push~1,
otherwise we push~0.

\Y\P$\4\X381:Push 0 if the string has a non\\{white\_space} char, else 1\X\S$\6
\&{begin} \37$\\{sp\_ptr}\K\\{str\_start}[\\{pop\_lit1}]$;\5
$\\{sp\_end}\K\\{str\_start}[\\{pop\_lit1}+1]$;\6
\&{while} $(\\{sp\_ptr}<\\{sp\_end})$ \1\&{do}\6
\&{begin} \37\&{if} $(\\{lex\_class}[\\{str\_pool}[\\{sp\_ptr}]]\I\\{white%
\_space})$ \1\&{then}\6
\&{begin} \37$\\{push\_lit\_stk}(0,\39\\{stk\_int})$;\5
\&{return};\6
\&{end};\2\6
$\\{incr}(\\{sp\_ptr})$;\6
\&{end};\2\6
$\\{push\_lit\_stk}(1,\39\\{stk\_int})$;\6
\&{end}\par
\U380.\fi

\M382.
The \\{built\_in} function {\.{format.name\$}} pops the top three
literals (they are a string, an integer, and a string literal, in that
order).  The last string literal represents a name list (each name
corresponding to a person), the integer literal specifies which name
to pick from this list, and the first string literal specifies how to
format this name, as described in the \BibTeX\ documentation.
Finally, this function pushes the formatted name.  If any of the types
is incorrect, it complains and pushes the null string.

\Y\P\D \37$\\{von\_found}=52$\C{for when a von token is found}\par
\Y\P$\4\X382:\\{execute\_fn}({\.{format.name\$}})\X\S$\6
\4\&{procedure}\1\  \37\\{x\_format\_name};\6
\4\&{label} \37$\\{loop1\_exit},\39\\{loop2\_exit},\39\\{von\_found}$;\2\6
\&{begin} \37$\\{pop\_lit\_stk}(\\{pop\_lit1},\39\\{pop\_typ1})$;\5
$\\{pop\_lit\_stk}(\\{pop\_lit2},\39\\{pop\_typ2})$;\5
$\\{pop\_lit\_stk}(\\{pop\_lit3},\39\\{pop\_typ3})$;\6
\&{if} $(\\{pop\_typ1}\I\\{stk\_str})$ \1\&{then}\6
\&{begin} \37$\\{print\_wrong\_stk\_lit}(\\{pop\_lit1},\39\\{pop\_typ1},\39%
\\{stk\_str})$;\5
$\\{push\_lit\_stk}(\\{s\_null},\39\\{stk\_str})$;\6
\&{end}\6
\4\&{else} \&{if} $(\\{pop\_typ2}\I\\{stk\_int})$ \1\&{then}\6
\&{begin} \37$\\{print\_wrong\_stk\_lit}(\\{pop\_lit2},\39\\{pop\_typ2},\39%
\\{stk\_int})$;\5
$\\{push\_lit\_stk}(\\{s\_null},\39\\{stk\_str})$;\6
\&{end}\6
\4\&{else} \&{if} $(\\{pop\_typ3}\I\\{stk\_str})$ \1\&{then}\6
\&{begin} \37$\\{print\_wrong\_stk\_lit}(\\{pop\_lit3},\39\\{pop\_typ3},\39%
\\{stk\_str})$;\5
$\\{push\_lit\_stk}(\\{s\_null},\39\\{stk\_str})$;\6
\&{end}\6
\4\&{else} \&{begin} \37$\\{ex\_buf\_length}\K0$;\5
$\\{add\_buf\_pool}(\\{pop\_lit3})$;\5
\X383:Isolate the desired name\X;\6
\X387:Copy name and count \\{comma}s to determine syntax\X;\6
\X395:Find the parts of the name\X;\6
$\\{ex\_buf\_length}\K0$;\5
$\\{add\_buf\_pool}(\\{pop\_lit1})$;\5
\\{figure\_out\_the\_formatted\_name};\6
\\{add\_pool\_buf\_and\_push};\C{push the formatted string onto the stack}\6
\&{end};\2\2\2\6
\&{end};\par
\U342.\fi

\M383.
This module skips over undesired names in \\{pop\_lit3} and it throws
away the ``and'' from the end of the name if it exists.  When it's
done, \\{ex\_buf\_xptr} points to its first character and \\{ex\_buf\_ptr}
points just past its last.

\Y\P$\4\X383:Isolate the desired name\X\S$\6
\&{begin} \37$\\{ex\_buf\_ptr}\K0$;\5
$\\{num\_names}\K0$;\6
\&{while} $((\\{num\_names}<\\{pop\_lit2})\W(\\{ex\_buf\_ptr}<\\{ex\_buf%
\_length}))$ \1\&{do}\6
\&{begin} \37$\\{incr}(\\{num\_names})$;\5
$\\{ex\_buf\_xptr}\K\\{ex\_buf\_ptr}$;\5
$\\{name\_scan\_for\_and}(\\{pop\_lit3})$;\6
\&{end};\2\6
\&{if} $(\\{ex\_buf\_ptr}<\\{ex\_buf\_length})$ \1\&{then}\C{remove the
``and''}\6
$\\{ex\_buf\_ptr}\K\\{ex\_buf\_ptr}-4$;\2\6
\&{if} $(\\{num\_names}<\\{pop\_lit2})$ \1\&{then}\6
\&{begin} \37\&{if} $(\\{pop\_lit2}=1)$ \1\&{then}\5
$\\{print}(\.{\'There\ is\ no\ name\ in\ "\'})$\6
\4\&{else} $\\{print}(\.{\'There\ aren\'}\.{\'t\ \'},\39\\{pop\_lit2}:0,\39\.{%
\'\ names\ in\ "\'})$;\2\6
$\\{print\_pool\_str}(\\{pop\_lit3})$;\5
$\\{bst\_ex\_warn}(\.{\'"\'})$;\6
\&{end}\2\6
\&{end}\par
\U382.\fi

\M384.
This module, starting at \\{ex\_buf\_ptr}, looks in \\{ex\_buf} for an
``and'' surrounded by nonnull \\{white\_space}.  It stops either at
\\{ex\_buf\_length} or just past the ``and'', whichever comes first,
setting \\{ex\_buf\_ptr} accordingly.  Its parameter \\{pop\_lit\_var} is
either \\{pop\_lit3} or \\{pop\_lit1}, depending on whether
{\.{format.name\$}} or {\.{num.names\$}} calls it.

\Y\P$\4\X367:Procedures and functions for name-string processing\X\mathrel{+}%
\S$\6
\4\&{procedure}\1\  \37$\\{name\_scan\_for\_and}(\\{pop\_lit\_var}:\\{str%
\_number})$;\2\6
\&{begin} \37$\\{brace\_level}\K0$;\5
$\\{preceding\_white}\K\\{false}$;\5
$\\{and\_found}\K\\{false}$;\6
\&{while} $((\R\\{and\_found})\W(\\{ex\_buf\_ptr}<\\{ex\_buf\_length}))$ \1%
\&{do}\6
\&{case} $(\\{ex\_buf}[\\{ex\_buf\_ptr}])$ \1\&{of}\6
\4$\.{"a"},\39\.{"A"}$: \37\&{begin} \37$\\{incr}(\\{ex\_buf\_ptr})$;\6
\&{if} $(\\{preceding\_white})$ \1\&{then}\5
\X386:See if we have an ``and''\X;\C{if so, $\\{and\_found}\K\\{true}$}\2\6
$\\{preceding\_white}\K\\{false}$;\6
\&{end};\6
\4\\{left\_brace}: \37\&{begin} \37$\\{incr}(\\{brace\_level})$;\5
$\\{incr}(\\{ex\_buf\_ptr})$;\5
\X385:Skip over \\{ex\_buf} stuff at $\\{brace\_level}>0$\X;\6
$\\{preceding\_white}\K\\{false}$;\6
\&{end};\6
\4\\{right\_brace}: \37\&{begin} \37$\\{decr\_brace\_level}(\\{pop\_lit%
\_var})$;\C{this checks for an error}\6
$\\{incr}(\\{ex\_buf\_ptr})$;\5
$\\{preceding\_white}\K\\{false}$;\6
\&{end};\6
\4\&{othercases} \37\&{if} $(\\{lex\_class}[\\{ex\_buf}[\\{ex\_buf\_ptr}]]=%
\\{white\_space})$ \1\&{then}\6
\&{begin} \37$\\{incr}(\\{ex\_buf\_ptr})$;\5
$\\{preceding\_white}\K\\{true}$;\6
\&{end}\6
\4\&{else} \&{begin} \37$\\{incr}(\\{ex\_buf\_ptr})$;\5
$\\{preceding\_white}\K\\{false}$;\6
\&{end}\2\2\6
\&{endcases};\2\6
$\\{check\_brace\_level}(\\{pop\_lit\_var})$;\6
\&{end};\par
\fi

\M385.
When we come here \\{ex\_buf\_ptr} is just past the \\{left\_brace}, and when
we leave it's either at \\{ex\_buf\_length} or just past the matching
\\{right\_brace}.

\Y\P$\4\X385:Skip over \\{ex\_buf} stuff at $\\{brace\_level}>0$\X\S$\6
\&{while} $((\\{brace\_level}>0)\W(\\{ex\_buf\_ptr}<\\{ex\_buf\_length}))$ \1%
\&{do}\6
\&{begin} \37\&{if} $(\\{ex\_buf}[\\{ex\_buf\_ptr}]=\\{right\_brace})$ \1%
\&{then}\5
$\\{decr}(\\{brace\_level})$\6
\4\&{else} \&{if} $(\\{ex\_buf}[\\{ex\_buf\_ptr}]=\\{left\_brace})$ \1\&{then}\5
$\\{incr}(\\{brace\_level})$;\2\2\6
$\\{incr}(\\{ex\_buf\_ptr})$;\6
\&{end}\2\par
\U384.\fi

\M386.
When we come here \\{ex\_buf\_ptr} is just past the ``a'' or ``A'', and when
we leave it's either at the same place or, if we found an ``and'', at
the following \\{white\_space} character.

\Y\P$\4\X386:See if we have an ``and''\X\S$\6
\&{begin} \37\&{if} $(\\{ex\_buf\_ptr}\L(\\{ex\_buf\_length}-3))$ \1\&{then}%
\C{enough characters are left}\6
\&{if} $((\\{ex\_buf}[\\{ex\_buf\_ptr}]=\.{"n"})\V(\\{ex\_buf}[\\{ex\_buf%
\_ptr}]=\.{"N"}))$ \1\&{then}\6
\&{if} $((\\{ex\_buf}[\\{ex\_buf\_ptr}+1]=\.{"d"})\V(\\{ex\_buf}[\\{ex\_buf%
\_ptr}+1]=\.{"D"}))$ \1\&{then}\6
\&{if} $(\\{lex\_class}[\\{ex\_buf}[\\{ex\_buf\_ptr}+2]]=\\{white\_space})$ \1%
\&{then}\6
\&{begin} \37$\\{ex\_buf\_ptr}\K\\{ex\_buf\_ptr}+2$;\5
$\\{and\_found}\K\\{true}$;\6
\&{end};\2\2\2\2\6
\&{end}\par
\U384.\fi

\M387.
When we arrive here, the desired name is in $\\{ex\_buf}[\\{ex\_buf\_xptr}]$
through $\\{ex\_buf}[\\{ex\_buf\_ptr}-1]$.  This module does its thing for
characters only at $\\{brace\_level}=0$; the rest get processed verbatim.
It removes leading \\{white\_space} (and \\{sep\_char}s), and trailing
\\{white\_space} (and \\{sep\_char}s) and \\{comma}s, complaining for each
trailing \\{comma}.  It then copies the name into \\{name\_buf}, removing
all \\{white\_space}, \\{sep\_char}s and \\{comma}s, counting \\{comma}s, and
constructing a list of name tokens, which are sequences of characters
separated (at $\\{brace\_level}=0$) by \\{white\_space}, \\{sep\_char}s or
\\{comma}s.  Each name token but the first has an associated
\\{name\_sep\_char}, the character that separates it from the preceding
token.  If there are too many (more than two) \\{comma}s, a complaint is
in order.

\Y\P$\4\X387:Copy name and count \\{comma}s to determine syntax\X\S$\6
\&{begin} \37\X388:Remove leading and trailing junk, complaining if necessary%
\X;\6
$\\{name\_bf\_ptr}\K0$;\5
$\\{num\_commas}\K0$;\5
$\\{num\_tokens}\K0$;\6
$\\{token\_starting}\K\\{true}$;\C{to indicate that a name token is starting}\6
\&{while} $(\\{ex\_buf\_xptr}<\\{ex\_buf\_ptr})$ \1\&{do}\6
\&{case} $(\\{ex\_buf}[\\{ex\_buf\_xptr}])$ \1\&{of}\6
\4\\{comma}: \37\X389:Name-process a \\{comma}\X;\6
\4\\{left\_brace}: \37\X390:Name-process a \\{left\_brace}\X;\6
\4\\{right\_brace}: \37\X391:Name-process a \\{right\_brace}\X;\6
\4\&{othercases} \37\&{case} $(\\{lex\_class}[\\{ex\_buf}[\\{ex\_buf\_xptr}]])$
\1\&{of}\6
\4\\{white\_space}: \37\X392:Name-process a \\{white\_space}\X;\6
\4\\{sep\_char}: \37\X393:Name-process a \\{sep\_char}\X;\6
\4\&{othercases} \37\X394:Name-process some other character\X\2\6
\&{endcases}\2\6
\&{endcases};\2\6
$\\{name\_tok}[\\{num\_tokens}]\K\\{name\_bf\_ptr}$;\C{this is an end-marker}\6
\&{end}\par
\U382.\fi

\M388.
This module removes all leading \\{white\_space} (and \\{sep\_char}s), and
trailing \\{white\_space} (and \\{sep\_char}s) and \\{comma}s.  It complains
for each trailing \\{comma}.

\Y\P$\4\X388:Remove leading and trailing junk, complaining if necessary\X\S$\6
\&{begin} \37\&{while} $((\\{ex\_buf\_xptr}<\\{ex\_buf\_ptr})\W(\\{lex\_class}[%
\\{ex\_buf}[\\{ex\_buf\_ptr}]]=\\{white\_space})\W(\\{lex\_class}[\\{ex\_buf}[%
\\{ex\_buf\_ptr}]]=\\{sep\_char}))$ \1\&{do}\5
$\\{incr}(\\{ex\_buf\_xptr})$;\C{this removes leading stuff}\2\6
\&{while} $(\\{ex\_buf\_ptr}>\\{ex\_buf\_xptr})$ \1\&{do}\C{now remove trailing
stuff}\6
\&{case} $(\\{lex\_class}[\\{ex\_buf}[\\{ex\_buf\_ptr}-1]])$ \1\&{of}\6
\4$\\{white\_space},\39\\{sep\_char}$: \37$\\{decr}(\\{ex\_buf\_ptr})$;\6
\4\&{othercases} \37\&{if} $(\\{ex\_buf}[\\{ex\_buf\_ptr}-1]=\\{comma})$ \1%
\&{then}\6
\&{begin} \37$\\{print}(\.{\'Name\ \'},\39\\{pop\_lit2}:0,\39\.{\'\ in\ "\'})$;%
\5
$\\{print\_pool\_str}(\\{pop\_lit3})$;\5
$\\{print}(\.{\'"\ has\ a\ comma\ at\ the\ end\'})$;\5
\\{bst\_ex\_warn\_print};\5
$\\{decr}(\\{ex\_buf\_ptr})$;\6
\&{end}\6
\4\&{else} \&{goto} \37\\{loop1\_exit}\2\2\6
\&{endcases};\2\6
\4\\{loop1\_exit}: \37\&{end}\par
\U387.\fi

\M389.
Here we mark the token number at which this comma has occurred.

\Y\P$\4\X389:Name-process a \\{comma}\X\S$\6
\&{begin} \37\&{if} $(\\{num\_commas}=2)$ \1\&{then}\6
\&{begin} \37$\\{print}(\.{\'Too\ many\ commas\ in\ name\ \'},\39\\{pop%
\_lit2}:0,\39\.{\'\ of\ "\'})$;\5
$\\{print\_pool\_str}(\\{pop\_lit3})$;\5
$\\{print}(\.{\'"\'})$;\5
\\{bst\_ex\_warn\_print};\6
\&{end}\6
\4\&{else} \&{begin} \37$\\{incr}(\\{num\_commas})$;\6
\&{if} $(\\{num\_commas}=1)$ \1\&{then}\5
$\\{comma1}\K\\{num\_tokens}$\6
\4\&{else} $\\{comma2}\K\\{num\_tokens}$;\C{$\\{num\_commas}=2$}\2\6
$\\{name\_sep\_char}[\\{num\_tokens}]\K\\{comma}$;\6
\&{end};\2\6
$\\{incr}(\\{ex\_buf\_xptr})$;\5
$\\{token\_starting}\K\\{true}$;\6
\&{end}\par
\U387.\fi

\M390.
We copy the stuff up through the matching \\{right\_brace} verbatim.

\Y\P$\4\X390:Name-process a \\{left\_brace}\X\S$\6
\&{begin} \37$\\{incr}(\\{brace\_level})$;\6
\&{if} $(\\{token\_starting})$ \1\&{then}\6
\&{begin} \37$\\{name\_tok}[\\{num\_tokens}]\K\\{name\_bf\_ptr}$;\5
$\\{incr}(\\{num\_tokens})$;\6
\&{end};\2\6
$\\{name\_buf}[\\{name\_bf\_ptr}]\K\\{ex\_buf}[\\{ex\_buf\_xptr}]$;\5
$\\{incr}(\\{name\_bf\_ptr})$;\5
$\\{incr}(\\{ex\_buf\_xptr})$;\6
\&{while} $((\\{brace\_level}>0)\W(\\{ex\_buf\_xptr}<\\{ex\_buf\_ptr}))$ \1%
\&{do}\6
\&{begin} \37\&{if} $(\\{ex\_buf}[\\{ex\_buf\_xptr}]=\\{right\_brace})$ \1%
\&{then}\5
$\\{decr}(\\{brace\_level})$\6
\4\&{else} \&{if} $(\\{ex\_buf}[\\{ex\_buf\_xptr}]=\\{left\_brace})$ \1\&{then}%
\5
$\\{incr}(\\{brace\_level})$;\2\2\6
$\\{name\_buf}[\\{name\_bf\_ptr}]\K\\{ex\_buf}[\\{ex\_buf\_xptr}]$;\5
$\\{incr}(\\{name\_bf\_ptr})$;\5
$\\{incr}(\\{ex\_buf\_xptr})$;\6
\&{end};\2\6
$\\{token\_starting}\K\\{false}$;\6
\&{end}\par
\U387.\fi

\M391.
We don't copy an extra \\{right\_brace}; this code will almost never be
executed.

\Y\P$\4\X391:Name-process a \\{right\_brace}\X\S$\6
\&{begin} \37\&{if} $(\\{token\_starting})$ \1\&{then}\6
\&{begin} \37$\\{name\_tok}[\\{num\_tokens}]\K\\{name\_bf\_ptr}$;\5
$\\{incr}(\\{num\_tokens})$;\6
\&{end};\2\6
$\\{print}(\.{\'Name\ \'},\39\\{pop\_lit2}:0,\39\.{\'\ of\ "\'})$;\5
$\\{print\_pool\_str}(\\{pop\_lit3})$;\5
$\\{bst\_ex\_warn}(\.{\'"\ isn\'}\.{\'t\ brace\ balanced\'})$;\5
$\\{incr}(\\{ex\_buf\_xptr})$;\5
$\\{token\_starting}\K\\{false}$;\6
\&{end}\par
\U387.\fi

\M392.
A token will be starting soon in a buffer near you, one way$\ldots$

\Y\P$\4\X392:Name-process a \\{white\_space}\X\S$\6
\&{begin} \37\&{if} $(\R\\{token\_starting})$ \1\&{then}\5
$\\{name\_sep\_char}[\\{num\_tokens}]\K\\{space}$;\2\6
$\\{incr}(\\{ex\_buf\_xptr})$;\5
$\\{token\_starting}\K\\{true}$;\6
\&{end}\par
\U387.\fi

\M393.
or another.  If one of the valid \\{sep\_char}s appears between tokens,
we usually use it instead of a \\{space}.  If the user has been silly
enough to have multiple \\{sep\_char}s, or to have both \\{white\_space} and
a \\{sep\_char}, we use the first such character.

\Y\P$\4\X393:Name-process a \\{sep\_char}\X\S$\6
\&{begin} \37\&{if} $(\R\\{token\_starting})$ \1\&{then}\5
$\\{name\_sep\_char}[\\{num\_tokens}]\K\\{ex\_buf}[\\{ex\_buf\_xptr}]$;\2\6
$\\{incr}(\\{ex\_buf\_xptr})$;\5
$\\{token\_starting}\K\\{true}$;\6
\&{end}\par
\U387.\fi

\M394.
For ordinary characters, we just copy the character.

\Y\P$\4\X394:Name-process some other character\X\S$\6
\&{begin} \37\&{if} $(\\{token\_starting})$ \1\&{then}\6
\&{begin} \37$\\{name\_tok}[\\{num\_tokens}]\K\\{name\_bf\_ptr}$;\5
$\\{incr}(\\{num\_tokens})$;\6
\&{end};\2\6
$\\{name\_buf}[\\{name\_bf\_ptr}]\K\\{ex\_buf}[\\{ex\_buf\_xptr}]$;\5
$\\{incr}(\\{name\_bf\_ptr})$;\5
$\\{incr}(\\{ex\_buf\_xptr})$;\5
$\\{token\_starting}\K\\{false}$;\6
\&{end}\par
\U387.\fi

\M395.
Here we set all the pointers for the various parts of the name,
depending on which of the three possible syntaxes this name uses.

\Y\P$\4\X395:Find the parts of the name\X\S$\6
\&{begin} \37\&{if} $(\\{num\_commas}=0)$ \1\&{then}\6
\&{begin} \37$\\{first\_start}\K0$;\5
$\\{last\_end}\K\\{num\_tokens}$;\5
$\\{jr\_end}\K\\{last\_end}$;\5
\X396:Determine where the first name ends and von name starts and ends\X;\6
\&{end}\6
\4\&{else} \&{if} $(\\{num\_commas}=1)$ \1\&{then}\6
\&{begin} \37$\\{von\_start}\K0$;\5
$\\{last\_end}\K\\{comma1}$;\5
$\\{jr\_end}\K\\{last\_end}$;\5
$\\{first\_start}\K\\{jr\_end}$;\5
$\\{first\_end}\K\\{num\_tokens}$;\5
\\{von\_name\_ends\_and\_last\_name\_starts\_stuff};\6
\&{end}\6
\4\&{else} \&{if} $(\\{num\_commas}=2)$ \1\&{then}\6
\&{begin} \37$\\{von\_start}\K0$;\5
$\\{last\_end}\K\\{comma1}$;\5
$\\{jr\_end}\K\\{comma2}$;\5
$\\{first\_start}\K\\{jr\_end}$;\5
$\\{first\_end}\K\\{num\_tokens}$;\5
\\{von\_name\_ends\_and\_last\_name\_starts\_stuff};\6
\&{end}\6
\4\&{else} $\\{confusion}(\.{\'Illegal\ number\ of\ comma,s\'})$;\2\2\2\6
\&{end}\par
\U382.\fi

\M396.
When there are no brace-level-0 \\{comma}s in the name, the von name
starts with the first nonlast token whose first brace-level-0 letter
is in lower case (for the purposes of this determination, an accented
or foreign character at brace-level-1 that's in lower case will do, as
well).  A module following this one determines where the von name ends
and the last starts.

\Y\P$\4\X396:Determine where the first name ends and von name starts and ends\X%
\S$\6
\&{begin} \37$\\{von\_start}\K0$;\6
\&{while} $(\\{von\_start}<\\{last\_end}-1)$ \1\&{do}\6
\&{begin} \37$\\{name\_bf\_ptr}\K\\{name\_tok}[\\{von\_start}]$;\5
$\\{name\_bf\_xptr}\K\\{name\_tok}[\\{von\_start}+1]$;\6
\&{if} $(\\{von\_token\_found})$ \1\&{then}\6
\&{begin} \37\\{von\_name\_ends\_and\_last\_name\_starts\_stuff};\5
\&{goto} \37\\{von\_found};\6
\&{end};\2\6
$\\{incr}(\\{von\_start})$;\6
\&{end};\C{there's no von name, so}\2\6
\&{while} $(\\{von\_start}>0)$ \1\&{do}\C{backtrack if there are connected
tokens}\6
\&{begin} \37\&{if} $((\\{lex\_class}[\\{name\_sep\_char}[\\{von\_start}]]\I%
\\{sep\_char})\V(\\{name\_sep\_char}[\\{von\_start}]=\\{tie}))$ \1\&{then}\5
\&{goto} \37\\{loop2\_exit};\2\6
$\\{decr}(\\{von\_start})$;\6
\&{end};\2\6
\4\\{loop2\_exit}: \37$\\{von\_end}\K\\{von\_start}$;\6
\4\\{von\_found}: \37$\\{first\_end}\K\\{von\_start}$;\6
\&{end}\par
\U395.\fi

\M397.
It's a von token if there exists a first brace-level-0 letter (or
brace-level-1 special character), and it's in lower case; in this case
we return \\{true}.  The token is in \\{name\_buf}, starting at
\\{name\_bf\_ptr} and ending just before \\{name\_bf\_xptr}.

\Y\P\D \37$\\{return\_von\_found}\S$\1\6
\&{begin} \37$\\{von\_token\_found}\K\\{true}$;\5
\&{return};\6
\&{end}\2\par
\Y\P$\4\X367:Procedures and functions for name-string processing\X\mathrel{+}%
\S$\6
\4\&{function}\1\  \37\\{von\_token\_found}: \37\\{boolean};\6
\4\&{label} \37\\{exit};\2\6
\&{begin} \37$\\{nm\_brace\_level}\K0$;\5
$\\{von\_token\_found}\K\\{false}$;\C{now it's easy to exit if necessary}\6
\&{while} $(\\{name\_bf\_ptr}<\\{name\_bf\_xptr})$ \1\&{do}\6
\&{if} $((\\{name\_buf}[\\{name\_bf\_ptr}]\G\.{"A"})\W(\\{name\_buf}[\\{name%
\_bf\_ptr}]\L\.{"Z"}))$ \1\&{then}\5
\&{return}\6
\4\&{else} \&{if} $((\\{name\_buf}[\\{name\_bf\_ptr}]\G\.{"a"})\W(\\{name%
\_buf}[\\{name\_bf\_ptr}]\L\.{"z"}))$ \1\&{then}\5
\\{return\_von\_found}\6
\4\&{else} \&{if} $(\\{name\_buf}[\\{name\_bf\_ptr}]=\\{left\_brace})$ \1%
\&{then}\6
\&{begin} \37$\\{incr}(\\{nm\_brace\_level})$;\5
$\\{incr}(\\{name\_bf\_ptr})$;\6
\&{if} $((\\{name\_bf\_ptr}+2<\\{name\_bf\_xptr})\W(\\{name\_buf}[\\{name\_bf%
\_ptr}]=\\{backslash}))$ \1\&{then}\5
\X398:Check the special character (and \&{return})\X\6
\4\&{else} \X400:Skip over \\{name\_buf} stuff at $\\{nm\_brace\_level}>0$\X;\2%
\6
\&{end}\6
\4\&{else} $\\{incr}(\\{name\_bf\_ptr})$;\2\2\2\2\6
\4\\{exit}: \37\&{end};\par
\fi

\M398.
When we come here \\{name\_bf\_ptr} is just past the \\{left\_brace},
but we always leave by \&{return}ing.

\Y\P$\4\X398:Check the special character (and \&{return})\X\S$\6
\&{begin} \37$\\{incr}(\\{name\_bf\_ptr})$;\C{skip over the \\{backslash}}\6
$\\{name\_bf\_yptr}\K\\{name\_bf\_ptr}$;\6
\&{while} $((\\{name\_bf\_ptr}<\\{name\_bf\_xptr})\W(\\{lex\_class}[\\{name%
\_buf}[\\{name\_bf\_ptr}]]=\\{alpha}))$ \1\&{do}\5
$\\{incr}(\\{name\_bf\_ptr})$;\C{this scans the control sequence}\2\6
$\\{control\_seq\_loc}\K\\{str\_lookup}(\\{name\_buf},\39\\{name\_bf\_yptr},\39%
\\{name\_bf\_ptr}-\\{name\_bf\_yptr},\39\\{control\_seq\_ilk},\39\\{dont%
\_insert})$;\6
\&{if} $(\\{hash\_found})$ \1\&{then}\5
\X399:Handle this accented or foreign character (and \&{return})\X;\2\6
\&{while} $((\\{name\_bf\_ptr}<\\{name\_bf\_xptr})\W(\\{nm\_brace\_level}>0))$ %
\1\&{do}\6
\&{begin} \37\&{if} $((\\{name\_buf}[\\{name\_bf\_ptr}]\G\.{"A"})\W(\\{name%
\_buf}[\\{name\_bf\_ptr}]\L\.{"Z"}))$ \1\&{then}\5
\&{return}\6
\4\&{else} \&{if} $((\\{name\_buf}[\\{name\_bf\_ptr}]\G\.{"a"})\W(\\{name%
\_buf}[\\{name\_bf\_ptr}]\L\.{"z"}))$ \1\&{then}\5
\\{return\_von\_found}\6
\4\&{else} \&{if} $(\\{name\_buf}[\\{name\_bf\_ptr}]=\\{right\_brace})$ \1%
\&{then}\5
$\\{decr}(\\{nm\_brace\_level})$\6
\4\&{else} \&{if} $(\\{name\_buf}[\\{name\_bf\_ptr}]=\\{left\_brace})$ \1%
\&{then}\5
$\\{incr}(\\{nm\_brace\_level})$;\2\2\2\2\6
$\\{incr}(\\{name\_bf\_ptr})$;\6
\&{end};\2\6
\&{return};\6
\&{end}\par
\U397.\fi

\M399.
The accented or foreign character is either `\.{\\i}' or `\.{\\j}' or
one of the eleven alphabetic foreign characters in Table~3.2 of the
\LaTeX\ manual.

\Y\P$\4\X399:Handle this accented or foreign character (and \&{return})\X\S$\6
\&{begin} \37\&{case} $(\\{ilk\_info}[\\{control\_seq\_loc}])$ \1\&{of}\6
\4$\\{n\_oe\_upper},\39\\{n\_ae\_upper},\39\\{n\_aa\_upper},\39\\{n\_o\_upper},%
\39\\{n\_l\_upper}$: \37\&{return};\6
\4$\\{n\_i},\39\\{n\_j},\39\\{n\_oe},\39\\{n\_ae},\39\\{n\_aa},\39\\{n\_o},\39%
\\{n\_l},\39\\{n\_ss}$: \37\\{return\_von\_found};\6
\4\&{othercases} \37$\\{confusion}(\.{\'Control-sequence\ hash\ error\'})$\2\6
\&{endcases};\6
\&{end}\par
\U398.\fi

\M400.
When we come here \\{name\_bf\_ptr} is just past the \\{left\_brace}; when we
leave it's either at \\{name\_bf\_xptr} or just past the matching
\\{right\_brace}.

\Y\P$\4\X400:Skip over \\{name\_buf} stuff at $\\{nm\_brace\_level}>0$\X\S$\6
\&{while} $((\\{nm\_brace\_level}>0)\W(\\{name\_bf\_ptr}<\\{name\_bf\_xptr}))$ %
\1\&{do}\6
\&{begin} \37\&{if} $(\\{name\_buf}[\\{name\_bf\_ptr}]=\\{right\_brace})$ \1%
\&{then}\5
$\\{decr}(\\{nm\_brace\_level})$\6
\4\&{else} \&{if} $(\\{name\_buf}[\\{name\_bf\_ptr}]=\\{left\_brace})$ \1%
\&{then}\5
$\\{incr}(\\{nm\_brace\_level})$;\2\2\6
$\\{incr}(\\{name\_bf\_ptr})$;\6
\&{end}\2\par
\U397.\fi

\M401.
The last name starts just past the last token, before the first
\\{comma} (if there is no \\{comma}, there is deemed to be one at the end
of the string), for which there exists a first brace-level-0 letter
(or brace-level-1 special character), and it's in lower case, unless
this last token is also the last token before the \\{comma}, in which
case the last name starts with this token (unless this last token is
connected by a \\{sep\_char} other than a \\{tie} to the previous token, in
which case the last name starts with as many tokens earlier as are
connected by non\\{tie}s to this last one (except on Tuesdays
$\ldots\,$), although this module never sees such a case).  Note that
if there are any tokens in either the von or last names, then the last
name has at least one, even if it starts with a lower-case letter.

\Y\P$\4\X367:Procedures and functions for name-string processing\X\mathrel{+}%
\S$\6
\4\&{procedure}\1\  \37\\{von\_name\_ends\_and\_last\_name\_starts\_stuff};\6
\4\&{label} \37\\{exit};\2\6
\&{begin} \37\C{there may or may not be a von name}\6
$\\{von\_end}\K\\{last\_end}-1$;\6
\&{while} $(\\{von\_end}>\\{von\_start})$ \1\&{do}\6
\&{begin} \37$\\{name\_bf\_ptr}\K\\{name\_tok}[\\{von\_end}-1]$;\5
$\\{name\_bf\_xptr}\K\\{name\_tok}[\\{von\_end}]$;\6
\&{if} $(\\{von\_token\_found})$ \1\&{then}\5
\&{return};\2\6
$\\{decr}(\\{von\_end})$;\6
\&{end};\2\6
\4\\{exit}: \37\&{end};\par
\fi

\M402.
This module uses the information in \\{pop\_lit1} to format the name.
Everything at $\\{sp\_brace\_level}=0$ is copied verbatim to the formatted
string; the rest is described in the succeeding modules.

\Y\P$\4\X402:Figure out the formatted name\X\S$\6
\&{begin} \37$\\{ex\_buf\_ptr}\K0$;\5
$\\{sp\_brace\_level}\K0$;\5
$\\{sp\_ptr}\K\\{str\_start}[\\{pop\_lit1}]$;\5
$\\{sp\_end}\K\\{str\_start}[\\{pop\_lit1}+1]$;\6
\&{while} $(\\{sp\_ptr}<\\{sp\_end})$ \1\&{do}\6
\&{if} $(\\{str\_pool}[\\{sp\_ptr}]=\\{left\_brace})$ \1\&{then}\6
\&{begin} \37$\\{incr}(\\{sp\_brace\_level})$;\5
$\\{incr}(\\{sp\_ptr})$;\5
\X403:Format this part of the name\X;\6
\&{end}\6
\4\&{else} \&{if} $(\\{str\_pool}[\\{sp\_ptr}]=\\{right\_brace})$ \1\&{then}\6
\&{begin} \37$\\{braces\_unbalanced\_complaint}(\\{pop\_lit1})$;\5
$\\{incr}(\\{sp\_ptr})$;\6
\&{end}\6
\4\&{else} \&{begin} \37$\\{append\_ex\_buf\_char\_and\_check}(\\{str\_pool}[%
\\{sp\_ptr}])$;\5
$\\{incr}(\\{sp\_ptr})$;\6
\&{end};\2\2\2\6
\&{if} $(\\{sp\_brace\_level}>0)$ \1\&{then}\5
$\\{braces\_unbalanced\_complaint}(\\{pop\_lit1})$;\2\6
$\\{ex\_buf\_length}\K\\{ex\_buf\_ptr}$;\6
\&{end}\par
\U420.\fi

\M403.
When we arrive here we're at $\\{sp\_brace\_level}=1$, just past the
\\{left\_brace}.  Letters at this \\{sp\_brace\_level} other than those
denoting the parts of the name (i.e., the first letters of `first,'
`last,' `von,' and `jr,' ignoring case) are illegal.  We do two passes
over this group; the first determines whether we're to output
anything, and, if we are, the second actually outputs it.

\Y\P$\4\X403:Format this part of the name\X\S$\6
\&{begin} \37$\\{sp\_xptr1}\K\\{sp\_ptr}$;\5
$\\{alpha\_found}\K\\{false}$;\5
$\\{double\_letter}\K\\{false}$;\5
$\\{end\_of\_group}\K\\{false}$;\5
$\\{to\_be\_written}\K\\{true}$;\6
\&{while} $((\R\\{end\_of\_group})\W(\\{sp\_ptr}<\\{sp\_end}))$ \1\&{do}\6
\&{if} $(\\{lex\_class}[\\{str\_pool}[\\{sp\_ptr}]]=\\{alpha})$ \1\&{then}\6
\&{begin} \37$\\{incr}(\\{sp\_ptr})$;\5
\X405:Figure out what this letter means\X;\6
\&{end}\6
\4\&{else} \&{if} $(\\{str\_pool}[\\{sp\_ptr}]=\\{right\_brace})$ \1\&{then}\6
\&{begin} \37$\\{decr}(\\{sp\_brace\_level})$;\5
$\\{incr}(\\{sp\_ptr})$;\5
$\\{end\_of\_group}\K\\{true}$;\6
\&{end}\6
\4\&{else} \&{if} $(\\{str\_pool}[\\{sp\_ptr}]=\\{left\_brace})$ \1\&{then}\6
\&{begin} \37$\\{incr}(\\{sp\_brace\_level})$;\5
$\\{incr}(\\{sp\_ptr})$;\5
\\{skip\_stuff\_at\_sp\_brace\_level\_greater\_than\_one};\6
\&{end}\6
\4\&{else} $\\{incr}(\\{sp\_ptr})$;\2\2\2\2\6
\&{if} $((\\{end\_of\_group})\W(\\{to\_be\_written}))$ \1\&{then}\C{do the
second pass}\6
\X411:Finally format this part of the name\X;\2\6
\&{end}\par
\U402.\fi

\M404.
When we come here \\{sp\_ptr} is just past the \\{left\_brace}, and when we
leave it's either at \\{sp\_end} or just past the matching \\{right\_brace}.

\Y\P$\4\X367:Procedures and functions for name-string processing\X\mathrel{+}%
\S$\6
\4\&{procedure}\1\  \37\\{skip\_stuff\_at\_sp\_brace\_level\_greater\_than%
\_one};\2\6
\&{begin} \37\&{while} $((\\{sp\_brace\_level}>1)\W(\\{sp\_ptr}<\\{sp\_end}))$ %
\1\&{do}\6
\&{begin} \37\&{if} $(\\{str\_pool}[\\{sp\_ptr}]=\\{right\_brace})$ \1\&{then}\5
$\\{decr}(\\{sp\_brace\_level})$\6
\4\&{else} \&{if} $(\\{str\_pool}[\\{sp\_ptr}]=\\{left\_brace})$ \1\&{then}\5
$\\{incr}(\\{sp\_brace\_level})$;\2\2\6
$\\{incr}(\\{sp\_ptr})$;\6
\&{end};\2\6
\&{end};\par
\fi

\M405.
We won't output anything for this part of the name if this is a second
occurrence of an $\\{sp\_brace\_level}=1$ letter, if it's an illegal
letter, or if there are no tokens corresponding to this part.  We also
determine if we're we to output complete tokens (indicated by a double
letter).

\Y\P$\4\X405:Figure out what this letter means\X\S$\6
\&{begin} \37\&{if} $(\\{alpha\_found})$ \1\&{then}\6
\&{begin} \37\\{brace\_lvl\_one\_letters\_complaint};\5
$\\{to\_be\_written}\K\\{false}$;\6
\&{end}\6
\4\&{else} \&{begin} \37\&{case} $(\\{str\_pool}[\\{sp\_ptr}-1])$ \1\&{of}\6
\4$\.{"f"},\39\.{"F"}$: \37\X407:Figure out what tokens we'll output for the
`first' name\X;\6
\4$\.{"v"},\39\.{"V"}$: \37\X408:Figure out what tokens we'll output for the
`von' name\X;\6
\4$\.{"l"},\39\.{"L"}$: \37\X409:Figure out what tokens we'll output for the
`last' name\X;\6
\4$\.{"j"},\39\.{"J"}$: \37\X410:Figure out what tokens we'll output for the
`jr' name\X;\6
\4\&{othercases} \37\&{begin} \37\\{brace\_lvl\_one\_letters\_complaint};\5
$\\{to\_be\_written}\K\\{false}$;\6
\&{end}\2\6
\&{endcases};\6
\&{if} $(\\{double\_letter})$ \1\&{then}\5
$\\{incr}(\\{sp\_ptr})$;\2\6
\&{end};\2\6
$\\{alpha\_found}\K\\{true}$;\6
\&{end}\par
\U403.\fi

\M406.
At most one of the important letters, perhaps doubled, may appear at
$\\{sp\_brace\_level}=1$.

\Y\P$\4\X367:Procedures and functions for name-string processing\X\mathrel{+}%
\S$\6
\4\&{procedure}\1\  \37\\{brace\_lvl\_one\_letters\_complaint};\2\6
\&{begin} \37$\\{print}(\.{\'The\ format\ string\ "\'})$;\5
$\\{print\_pool\_str}(\\{pop\_lit1})$;\5
$\\{bst\_ex\_warn}(\.{\'"\ has\ an\ illegal\ brace-level-1\ letter\'})$;\6
\&{end};\par
\fi

\M407.
Here we set pointers into \\{name\_tok} and note whether we'll be dealing
with a full first-name tokens ($\\{double\_letter}=\\{true}$) or
abbreviations ($\\{double\_letter}=\\{false}$).

\Y\P$\4\X407:Figure out what tokens we'll output for the `first' name\X\S$\6
\&{begin} \37$\\{cur\_token}\K\\{first\_start}$;\5
$\\{last\_token}\K\\{first\_end}$;\6
\&{if} $(\\{cur\_token}=\\{last\_token})$ \1\&{then}\5
$\\{to\_be\_written}\K\\{false}$;\2\6
\&{if} $((\\{str\_pool}[\\{sp\_ptr}]=\.{"f"})\V(\\{str\_pool}[\\{sp\_ptr}]=%
\.{"F"}))$ \1\&{then}\5
$\\{double\_letter}\K\\{true}$;\2\6
\&{end}\par
\U405.\fi

\M408.
The same as above but for von-name tokens.

\Y\P$\4\X408:Figure out what tokens we'll output for the `von' name\X\S$\6
\&{begin} \37$\\{cur\_token}\K\\{von\_start}$;\5
$\\{last\_token}\K\\{von\_end}$;\6
\&{if} $(\\{cur\_token}=\\{last\_token})$ \1\&{then}\5
$\\{to\_be\_written}\K\\{false}$;\2\6
\&{if} $((\\{str\_pool}[\\{sp\_ptr}]=\.{"v"})\V(\\{str\_pool}[\\{sp\_ptr}]=%
\.{"V"}))$ \1\&{then}\5
$\\{double\_letter}\K\\{true}$;\2\6
\&{end}\par
\U405.\fi

\M409.
The same as above but for last-name tokens.

\Y\P$\4\X409:Figure out what tokens we'll output for the `last' name\X\S$\6
\&{begin} \37$\\{cur\_token}\K\\{von\_end}$;\5
$\\{last\_token}\K\\{last\_end}$;\6
\&{if} $(\\{cur\_token}=\\{last\_token})$ \1\&{then}\5
$\\{to\_be\_written}\K\\{false}$;\2\6
\&{if} $((\\{str\_pool}[\\{sp\_ptr}]=\.{"l"})\V(\\{str\_pool}[\\{sp\_ptr}]=%
\.{"L"}))$ \1\&{then}\5
$\\{double\_letter}\K\\{true}$;\2\6
\&{end}\par
\U405.\fi

\M410.
The same as above but for jr-name tokens.

\Y\P$\4\X410:Figure out what tokens we'll output for the `jr' name\X\S$\6
\&{begin} \37$\\{cur\_token}\K\\{last\_end}$;\5
$\\{last\_token}\K\\{jr\_end}$;\6
\&{if} $(\\{cur\_token}=\\{last\_token})$ \1\&{then}\5
$\\{to\_be\_written}\K\\{false}$;\2\6
\&{if} $((\\{str\_pool}[\\{sp\_ptr}]=\.{"j"})\V(\\{str\_pool}[\\{sp\_ptr}]=%
\.{"J"}))$ \1\&{then}\5
$\\{double\_letter}\K\\{true}$;\2\6
\&{end}\par
\U405.\fi

\M411.
This is the second pass over this part of the name; here we actually
write stuff out to \\{ex\_buf}.

\Y\P$\4\X411:Finally format this part of the name\X\S$\6
\&{begin} \37$\\{ex\_buf\_xptr}\K\\{ex\_buf\_ptr}$;\5
$\\{sp\_ptr}\K\\{sp\_xptr1}$;\5
$\\{sp\_brace\_level}\K1$;\6
\&{while} $(\\{sp\_brace\_level}>0)$ \1\&{do}\6
\&{if} $((\\{lex\_class}[\\{str\_pool}[\\{sp\_ptr}]]=\\{alpha})\W(\\{sp\_brace%
\_level}=1))$ \1\&{then}\6
\&{begin} \37$\\{incr}(\\{sp\_ptr})$;\5
\X412:Figure out how to output the name tokens, and do it\X;\6
\&{end}\6
\4\&{else} \&{if} $(\\{str\_pool}[\\{sp\_ptr}]=\\{right\_brace})$ \1\&{then}\6
\&{begin} \37$\\{decr}(\\{sp\_brace\_level})$;\5
$\\{incr}(\\{sp\_ptr})$;\6
\&{if} $(\\{sp\_brace\_level}>0)$ \1\&{then}\5
$\\{append\_ex\_buf\_char\_and\_check}(\\{right\_brace})$;\2\6
\&{end}\6
\4\&{else} \&{if} $(\\{str\_pool}[\\{sp\_ptr}]=\\{left\_brace})$ \1\&{then}\6
\&{begin} \37$\\{incr}(\\{sp\_brace\_level})$;\5
$\\{incr}(\\{sp\_ptr})$;\5
$\\{append\_ex\_buf\_char\_and\_check}(\\{left\_brace})$;\6
\&{end}\6
\4\&{else} \&{begin} \37$\\{append\_ex\_buf\_char\_and\_check}(\\{str\_pool}[%
\\{sp\_ptr}])$;\5
$\\{incr}(\\{sp\_ptr})$;\6
\&{end};\2\2\2\2\6
\&{if} $(\\{ex\_buf\_ptr}>0)$ \1\&{then}\6
\&{if} $(\\{ex\_buf}[\\{ex\_buf\_ptr}-1]=\\{tie})$ \1\&{then}\5
\X419:Handle a discretionary \\{tie}\X;\2\2\6
\&{end}\par
\U403.\fi

\M412.
When we come here, \\{sp\_ptr} is just past the letter indicating the
part of the name for which we're about to output tokens.  When we
leave, it's at the first character of the rest of the group.

\Y\P$\4\X412:Figure out how to output the name tokens, and do it\X\S$\6
\&{begin} \37\&{if} $(\\{double\_letter})$ \1\&{then}\5
$\\{incr}(\\{sp\_ptr})$;\2\6
$\\{use\_default}\K\\{true}$;\5
$\\{sp\_xptr2}\K\\{sp\_ptr}$;\6
\&{if} $(\\{str\_pool}[\\{sp\_ptr}]=\\{left\_brace})$ \1\&{then}\C{find the
inter-token string}\6
\&{begin} \37$\\{use\_default}\K\\{false}$;\5
$\\{incr}(\\{sp\_brace\_level})$;\5
$\\{incr}(\\{sp\_ptr})$;\5
$\\{sp\_xptr1}\K\\{sp\_ptr}$;\5
\\{skip\_stuff\_at\_sp\_brace\_level\_greater\_than\_one};\5
$\\{sp\_xptr2}\K\\{sp\_ptr}-1$;\6
\&{end};\2\6
\X413:Finally output the name tokens\X;\6
\&{if} $(\R\\{use\_default})$ \1\&{then}\5
$\\{sp\_ptr}\K\\{sp\_xptr2}+1$;\2\6
\&{end}\par
\U411.\fi

\M413.
Here, for each token in this part, we output either a full or an
abbreviated token and the inter-token string for all but the last
token of this part.

\Y\P$\4\X413:Finally output the name tokens\X\S$\6
\&{while} $(\\{cur\_token}<\\{last\_token})$ \1\&{do}\6
\&{begin} \37\&{if} $(\\{double\_letter})$ \1\&{then}\5
\X414:Finally output a full token\X\6
\4\&{else} \X415:Finally output an abbreviated token\X;\2\6
$\\{incr}(\\{cur\_token})$;\6
\&{if} $(\\{cur\_token}<\\{last\_token})$ \1\&{then}\5
\X417:Finally output the inter-token string\X;\2\6
\&{end}\2\par
\U412.\fi

\M414.
Here we output all the characters in the token, verbatim.

\Y\P$\4\X414:Finally output a full token\X\S$\6
\&{begin} \37$\\{name\_bf\_ptr}\K\\{name\_tok}[\\{cur\_token}]$;\5
$\\{name\_bf\_xptr}\K\\{name\_tok}[\\{cur\_token}+1]$;\6
\&{if} $(\\{ex\_buf\_length}+(\\{name\_bf\_xptr}-\\{name\_bf\_ptr})>\\{buf%
\_size})$ \1\&{then}\5
\\{buffer\_overflow};\2\6
\&{while} $(\\{name\_bf\_ptr}<\\{name\_bf\_xptr})$ \1\&{do}\6
\&{begin} \37$\\{append\_ex\_buf\_char}(\\{name\_buf}[\\{name\_bf\_ptr}])$;\5
$\\{incr}(\\{name\_bf\_ptr})$;\6
\&{end};\2\6
\&{end}\par
\U413.\fi

\M415.
Here we output the first alphabetic or special character of the token;
brace level is irrelevant for an alphabetic (but not a special)
character.

\Y\P$\4\X415:Finally output an abbreviated token\X\S$\6
\&{begin} \37$\\{name\_bf\_ptr}\K\\{name\_tok}[\\{cur\_token}]$;\5
$\\{name\_bf\_xptr}\K\\{name\_tok}[\\{cur\_token}+1]$;\6
\&{while} $(\\{name\_bf\_ptr}<\\{name\_bf\_xptr})$ \1\&{do}\6
\&{begin} \37\&{if} $(\\{lex\_class}[\\{name\_buf}[\\{name\_bf\_ptr}]]=%
\\{alpha})$ \1\&{then}\6
\&{begin} \37$\\{append\_ex\_buf\_char\_and\_check}(\\{name\_buf}[\\{name\_bf%
\_ptr}])$;\5
\&{goto} \37\\{loop\_exit};\6
\&{end}\6
\4\&{else} \&{if} $((\\{name\_buf}[\\{name\_bf\_ptr}]=\\{left\_brace})\W(%
\\{name\_bf\_ptr}+1<\\{name\_bf\_xptr}))$ \1\&{then}\6
\&{if} $(\\{name\_buf}[\\{name\_bf\_ptr}+1]=\\{backslash})$ \1\&{then}\5
\X416:Finally output a special character and exit loop\X;\2\2\2\6
$\\{incr}(\\{name\_bf\_ptr})$;\6
\&{end};\2\6
\4\\{loop\_exit}: \37\&{end}\par
\U413.\fi

\M416.
We output a special character here even if the user has been silly
enough to make it nonalphabetic (and even if the user has been sillier
still by not having a matching \\{right\_brace}).

\Y\P$\4\X416:Finally output a special character and exit loop\X\S$\6
\&{begin} \37\&{if} $(\\{ex\_buf\_ptr}+2>\\{buf\_size})$ \1\&{then}\5
\\{buffer\_overflow};\2\6
$\\{append\_ex\_buf\_char}(\\{left\_brace})$;\5
$\\{append\_ex\_buf\_char}(\\{backslash})$;\5
$\\{name\_bf\_ptr}\K\\{name\_bf\_ptr}+2$;\5
$\\{nm\_brace\_level}\K1$;\6
\&{while} $((\\{name\_bf\_ptr}<\\{name\_bf\_xptr})\W(\\{nm\_brace\_level}>0))$ %
\1\&{do}\6
\&{begin} \37\&{if} $(\\{name\_buf}[\\{name\_bf\_ptr}]=\\{right\_brace})$ \1%
\&{then}\5
$\\{decr}(\\{nm\_brace\_level})$\6
\4\&{else} \&{if} $(\\{name\_buf}[\\{name\_bf\_ptr}]=\\{left\_brace})$ \1%
\&{then}\5
$\\{incr}(\\{nm\_brace\_level})$;\2\2\6
$\\{append\_ex\_buf\_char\_and\_check}(\\{name\_buf}[\\{name\_bf\_ptr}])$;\5
$\\{incr}(\\{name\_bf\_ptr})$;\6
\&{end};\2\6
\&{goto} \37\\{loop\_exit};\6
\&{end}\par
\U415.\fi

\M417.
Here we output either the \.{.bst} given string if it exists, or else
the \.{.bib} \\{sep\_char} if it exists, or else the default string.  A
\\{tie} is the default space character between the last two tokens of
the name part, and between the first two tokens if the first token is
short enough; otherwise, a \\{space} is the default.

\Y\P\D \37$\\{long\_token}=3$\C{a token this length or longer is ``long''}\par
\Y\P$\4\X417:Finally output the inter-token string\X\S$\6
\&{begin} \37\&{if} $(\\{use\_default})$ \1\&{then}\6
\&{begin} \37\&{if} $(\R\\{double\_letter})$ \1\&{then}\5
$\\{append\_ex\_buf\_char\_and\_check}(\\{period})$;\2\6
\&{if} $(\\{lex\_class}[\\{name\_sep\_char}[\\{cur\_token}]]=\\{sep\_char})$ \1%
\&{then}\5
$\\{append\_ex\_buf\_char\_and\_check}(\\{name\_sep\_char}[\\{cur\_token}])$\6
\4\&{else} \&{if} $((\\{cur\_token}=\\{last\_token}-1)\V(\R\\{enough\_text%
\_chars}(\\{long\_token})))$ \1\&{then}\5
$\\{append\_ex\_buf\_char\_and\_check}(\\{tie})$\6
\4\&{else} $\\{append\_ex\_buf\_char\_and\_check}(\\{space})$;\2\2\6
\&{end}\6
\4\&{else} \&{begin} \37\&{if} $(\\{ex\_buf\_length}+(\\{sp\_xptr2}-\\{sp%
\_xptr1})>\\{buf\_size})$ \1\&{then}\5
\\{buffer\_overflow};\2\6
$\\{sp\_ptr}\K\\{sp\_xptr1}$;\6
\&{while} $(\\{sp\_ptr}<\\{sp\_xptr2})$ \1\&{do}\6
\&{begin} \37$\\{append\_ex\_buf\_char}(\\{str\_pool}[\\{sp\_ptr}])$;\5
$\\{incr}(\\{sp\_ptr})$;\6
\&{end}\2\6
\&{end};\2\6
\&{end}\par
\U413.\fi

\M418.
This function looks at the string in \\{ex\_buf}, starting at
\\{ex\_buf\_xptr} and ending just before \\{ex\_buf\_ptr}, and it returns
\\{true} if there are \\{enough\_chars}, where a special character (even if
it's missing its matching \\{right\_brace}) counts as a single charcter.
This procedure is called only for strings that don't have too many
\\{right\_brace}s.

\Y\P$\4\X367:Procedures and functions for name-string processing\X\mathrel{+}%
\S$\6
\4\&{function}\1\  \37$\\{enough\_text\_chars}(\\{enough\_chars}:\\{buf%
\_pointer})$: \37\\{boolean};\2\6
\&{begin} \37$\\{num\_text\_chars}\K0$;\5
$\\{ex\_buf\_yptr}\K\\{ex\_buf\_xptr}$;\6
\&{while} $((\\{ex\_buf\_yptr}<\\{ex\_buf\_ptr})\W(\\{num\_text\_chars}<%
\\{enough\_chars}))$ \1\&{do}\6
\&{begin} \37$\\{incr}(\\{ex\_buf\_yptr})$;\6
\&{if} $(\\{ex\_buf}[\\{ex\_buf\_yptr}-1]=\\{left\_brace})$ \1\&{then}\6
\&{begin} \37$\\{incr}(\\{brace\_level})$;\6
\&{if} $((\\{brace\_level}=1)\W(\\{ex\_buf\_yptr}<\\{ex\_buf\_ptr}))$ \1%
\&{then}\6
\&{if} $(\\{ex\_buf}[\\{ex\_buf\_yptr}]=\\{backslash})$ \1\&{then}\6
\&{begin} \37$\\{incr}(\\{ex\_buf\_yptr})$;\C{skip over the \\{backslash}}\6
\&{while} $((\\{ex\_buf\_yptr}<\\{ex\_buf\_ptr})\W(\\{brace\_level}>0))$ \1%
\&{do}\6
\&{begin} \37\&{if} $(\\{ex\_buf}[\\{ex\_buf\_yptr}]=\\{right\_brace})$ \1%
\&{then}\5
$\\{decr}(\\{brace\_level})$\6
\4\&{else} \&{if} $(\\{ex\_buf}[\\{ex\_buf\_yptr}]=\\{left\_brace})$ \1\&{then}%
\5
$\\{incr}(\\{brace\_level})$;\2\2\6
$\\{incr}(\\{ex\_buf\_yptr})$;\6
\&{end};\2\6
\&{end};\2\2\6
\&{end}\6
\4\&{else} \&{if} $(\\{ex\_buf}[\\{ex\_buf\_yptr}-1]=\\{right\_brace})$ \1%
\&{then}\5
$\\{decr}(\\{brace\_level})$;\2\2\6
$\\{incr}(\\{num\_text\_chars})$;\6
\&{end};\2\6
\&{if} $(\\{num\_text\_chars}<\\{enough\_chars})$ \1\&{then}\5
$\\{enough\_text\_chars}\K\\{false}$\6
\4\&{else} $\\{enough\_text\_chars}\K\\{true}$;\2\6
\&{end};\par
\fi

\M419.
If the last character output for this name part is a \\{tie} but the
previous character it isn't, we're dealing with a discretionary \\{tie};
thus we replace it by a \\{space} if there are enough characters in the
rest of the name part.

\Y\P\D \37$\\{long\_name}=3$\C{a name this length or longer is ``long''}\par
\Y\P$\4\X419:Handle a discretionary \\{tie}\X\S$\6
\&{begin} \37$\\{decr}(\\{ex\_buf\_ptr})$;\C{remove the previous \\{tie}}\6
\&{if} $(\\{ex\_buf}[\\{ex\_buf\_ptr}-1]=\\{tie})$ \1\&{then}\C{it's not a
discretionary \\{tie}}\6
\\{do\_nothing}\6
\4\&{else} \&{if} $(\R\\{enough\_text\_chars}(\\{long\_name}))$ \1\&{then}%
\C{this is a short name part}\6
$\\{incr}(\\{ex\_buf\_ptr})$\C{so restore the \\{tie}}\6
\4\&{else} \C{replace it by a \\{space}}\2\2\6
$\\{append\_ex\_buf\_char}(\\{space})$;\6
\&{end}\par
\U411.\fi

\M420.
This is a procedure so that \\{x\_format\_name} is smaller.

\Y\P$\4\X367:Procedures and functions for name-string processing\X\mathrel{+}%
\S$\6
\4\&{procedure}\1\  \37\\{figure\_out\_the\_formatted\_name};\6
\4\&{label} \37\\{loop\_exit};\2\6
\&{begin} \37\X402:Figure out the formatted name\X;\6
\&{end};\par
\fi

\M421.
The \\{built\_in} function {\.{if\$}} pops the top three literals (they
are two function literals and an integer literal, in that order); if
the integer is greater than 0, it executes the second literal, else it
executes the first.  If any of the types is incorrect, it complains
but does nothing else.

\Y\P$\4\X421:\\{execute\_fn}({\.{if\$}})\X\S$\6
\&{begin} \37$\\{pop\_lit\_stk}(\\{pop\_lit1},\39\\{pop\_typ1})$;\5
$\\{pop\_lit\_stk}(\\{pop\_lit2},\39\\{pop\_typ2})$;\5
$\\{pop\_lit\_stk}(\\{pop\_lit3},\39\\{pop\_typ3})$;\6
\&{if} $(\\{pop\_typ1}\I\\{stk\_fn})$ \1\&{then}\5
$\\{print\_wrong\_stk\_lit}(\\{pop\_lit1},\39\\{pop\_typ1},\39\\{stk\_fn})$\6
\4\&{else} \&{if} $(\\{pop\_typ2}\I\\{stk\_fn})$ \1\&{then}\5
$\\{print\_wrong\_stk\_lit}(\\{pop\_lit2},\39\\{pop\_typ2},\39\\{stk\_fn})$\6
\4\&{else} \&{if} $(\\{pop\_typ3}\I\\{stk\_int})$ \1\&{then}\5
$\\{print\_wrong\_stk\_lit}(\\{pop\_lit3},\39\\{pop\_typ3},\39\\{stk\_int})$\6
\4\&{else} \&{if} $(\\{pop\_lit3}>0)$ \1\&{then}\5
$\\{execute\_fn}(\\{pop\_lit2})$\6
\4\&{else} $\\{execute\_fn}(\\{pop\_lit1})$;\2\2\2\2\6
\&{end}\par
\U341.\fi

\M422.
The \\{built\_in} function {\.{int.to.chr\$}} pops the top (integer)
literal, interpreted as the \\{ASCII\_code} of a single character,
converts it to the corresponding single-character string, and pushes
this string.  If the literal isn't an appropriate integer, it
complains and pushes the null string.

\Y\P$\4\X422:\\{execute\_fn}({\.{int.to.chr\$}})\X\S$\6
\4\&{procedure}\1\  \37\\{x\_int\_to\_chr};\2\6
\&{begin} \37$\\{pop\_lit\_stk}(\\{pop\_lit1},\39\\{pop\_typ1})$;\6
\&{if} $(\\{pop\_typ1}\I\\{stk\_int})$ \1\&{then}\6
\&{begin} \37$\\{print\_wrong\_stk\_lit}(\\{pop\_lit1},\39\\{pop\_typ1},\39%
\\{stk\_int})$;\5
$\\{push\_lit\_stk}(\\{s\_null},\39\\{stk\_str})$;\6
\&{end}\6
\4\&{else} \&{if} $((\\{pop\_lit1}<0)\V(\\{pop\_lit1}>127))$ \1\&{then}\6
\&{begin} \37$\\{bst\_ex\_warn}(\\{pop\_lit1}:0,\39\.{\'\ isn\'}\.{\'t\ valid\
ASCII\'})$;\5
$\\{push\_lit\_stk}(\\{s\_null},\39\\{stk\_str})$;\6
\&{end}\6
\4\&{else} \&{begin} \37$\\{str\_room}(1)$;\5
$\\{append\_char}(\\{pop\_lit1})$;\5
$\\{push\_lit\_stk}(\\{make\_string},\39\\{stk\_str})$;\6
\&{end};\2\2\6
\&{end};\par
\U342.\fi

\M423.
The \\{built\_in} function {\.{int.to.str\$}} pops the top (integer)
literal, converts it to its (unique) string equivalent, and pushes
this string.  If the literal isn't an integer, it complains and pushes
the null string.

\Y\P$\4\X423:\\{execute\_fn}({\.{int.to.str\$}})\X\S$\6
\4\&{procedure}\1\  \37\\{x\_int\_to\_str};\2\6
\&{begin} \37$\\{pop\_lit\_stk}(\\{pop\_lit1},\39\\{pop\_typ1})$;\6
\&{if} $(\\{pop\_typ1}\I\\{stk\_int})$ \1\&{then}\6
\&{begin} \37$\\{print\_wrong\_stk\_lit}(\\{pop\_lit1},\39\\{pop\_typ1},\39%
\\{stk\_int})$;\5
$\\{push\_lit\_stk}(\\{s\_null},\39\\{stk\_str})$;\6
\&{end}\6
\4\&{else} \&{begin} \37$\\{int\_to\_ASCII}(\\{pop\_lit1},\39\\{ex\_buf},\390,%
\39\\{ex\_buf\_length})$;\6
\\{add\_pool\_buf\_and\_push};\C{push this string onto the stack}\6
\&{end};\2\6
\&{end};\par
\U342.\fi

\M424.
The \\{built\_in} function {\.{missing\$}} pops the top literal and
pushes the integer 1 if it's a missing field, 0 otherwise.  If the
literal isn't a missing field or a string, it complains and pushes 0.
Unlike \.{empty\$}, this function should be called only when
\\{mess\_with\_entries} is true.

\Y\P$\4\X424:\\{execute\_fn}({\.{missing\$}})\X\S$\6
\4\&{procedure}\1\  \37\\{x\_missing};\2\6
\&{begin} \37$\\{pop\_lit\_stk}(\\{pop\_lit1},\39\\{pop\_typ1})$;\6
\&{if} $(\R\\{mess\_with\_entries})$ \1\&{then}\5
\\{bst\_cant\_mess\_with\_entries\_print}\6
\4\&{else} \&{if} $((\\{pop\_typ1}\I\\{stk\_str})\W(\\{pop\_typ1}\I\\{stk%
\_field\_missing}))$ \1\&{then}\6
\&{begin} \37\&{if} $(\\{pop\_typ1}\I\\{stk\_empty})$ \1\&{then}\6
\&{begin} \37$\\{print\_stk\_lit}(\\{pop\_lit1},\39\\{pop\_typ1})$;\5
$\\{bst\_ex\_warn}(\.{\',\ not\ a\ string\ or\ missing\ field,\'})$;\6
\&{end};\2\6
$\\{push\_lit\_stk}(0,\39\\{stk\_int})$;\6
\&{end}\6
\4\&{else} \&{if} $(\\{pop\_typ1}=\\{stk\_field\_missing})$ \1\&{then}\5
$\\{push\_lit\_stk}(1,\39\\{stk\_int})$\6
\4\&{else} $\\{push\_lit\_stk}(0,\39\\{stk\_int})$;\2\2\2\6
\&{end};\par
\U342.\fi

\M425.
The \\{built\_in} function {\.{newline\$}} writes whatever has
accumulated in the output buffer \\{out\_buf} onto the \.{.bbl} file.

\Y\P$\4\X425:\\{execute\_fn}({\.{newline\$}})\X\S$\6
\&{begin} \37\\{output\_bbl\_line};\6
\&{end}\par
\U341.\fi

\M426.
The \\{built\_in} function {\.{num.names\$}} pops the top (string)
literal; it pushes the number of names the string represents---one
plus the number of occurrences of the substring ``and'' (ignoring case
differences) surrounded by nonnull \\{white\_space} at the top brace
level.  If the literal isn't a string, it complains and pushes the
value 0.

\Y\P$\4\X426:\\{execute\_fn}({\.{num.names\$}})\X\S$\6
\4\&{procedure}\1\  \37\\{x\_num\_names};\2\6
\&{begin} \37$\\{pop\_lit\_stk}(\\{pop\_lit1},\39\\{pop\_typ1})$;\6
\&{if} $(\\{pop\_typ1}\I\\{stk\_str})$ \1\&{then}\6
\&{begin} \37$\\{print\_wrong\_stk\_lit}(\\{pop\_lit1},\39\\{pop\_typ1},\39%
\\{stk\_str})$;\5
$\\{push\_lit\_stk}(0,\39\\{stk\_int})$;\6
\&{end}\6
\4\&{else} \&{begin} \37$\\{ex\_buf\_length}\K0$;\5
$\\{add\_buf\_pool}(\\{pop\_lit1})$;\5
\X427:Determine the number of names\X;\6
$\\{push\_lit\_stk}(\\{num\_names},\39\\{stk\_int})$;\6
\&{end};\2\6
\&{end};\par
\U342.\fi

\M427.
This module, while scanning the list of names, counts the occurrences
of ``and'' (ignoring case differences) surrounded by nonnull
\\{white\_space}, and adds 1.

\Y\P$\4\X427:Determine the number of names\X\S$\6
\&{begin} \37$\\{ex\_buf\_ptr}\K0$;\5
$\\{num\_names}\K0$;\6
\&{while} $(\\{ex\_buf\_ptr}<\\{ex\_buf\_length})$ \1\&{do}\6
\&{begin} \37$\\{name\_scan\_for\_and}(\\{pop\_lit1})$;\5
$\\{incr}(\\{num\_names})$;\6
\&{end};\2\6
\&{end}\par
\U426.\fi

\M428.
The \\{built\_in} function {\.{pop\$}} pops the top of the stack but
doesn't print it.

\Y\P$\4\X428:\\{execute\_fn}({\.{pop\$}})\X\S$\6
\&{begin} \37$\\{pop\_lit\_stk}(\\{pop\_lit1},\39\\{pop\_typ1})$;\6
\&{end}\par
\U341.\fi

\M429.
The \\{built\_in} function {\.{preamble\$}} pushes onto the stack the
concatenation of all the \.{preamble} strings read from the database
files.

\Y\P$\4\X429:\\{execute\_fn}({\.{preamble\$}})\X\S$\6
\4\&{procedure}\1\  \37\\{x\_preamble};\2\6
\&{begin} \37$\\{ex\_buf\_length}\K0$;\5
$\\{preamble\_ptr}\K0$;\6
\&{while} $(\\{preamble\_ptr}<\\{num\_preamble\_strings})$ \1\&{do}\6
\&{begin} \37$\\{add\_buf\_pool}(\\{s\_preamble}[\\{preamble\_ptr}])$;\5
$\\{incr}(\\{preamble\_ptr})$;\6
\&{end};\2\6
\\{add\_pool\_buf\_and\_push};\C{push the concatenation string onto the stack}\6
\&{end};\par
\U342.\fi

\M430.
The \\{built\_in} function {\.{purify\$}} pops the top (string) literal,
removes nonalphanumeric characters except for \\{white\_space} and
\\{sep\_char} characters (these get converted to a \\{space}) and removes
certain alphabetic characters contained in the control sequences
associated with a special character, and pushes the resulting string.
If the literal isn't a string, it complains and pushes the null
string.

\Y\P$\4\X430:\\{execute\_fn}({\.{purify\$}})\X\S$\6
\4\&{procedure}\1\  \37\\{x\_purify};\2\6
\&{begin} \37$\\{pop\_lit\_stk}(\\{pop\_lit1},\39\\{pop\_typ1})$;\6
\&{if} $(\\{pop\_typ1}\I\\{stk\_str})$ \1\&{then}\6
\&{begin} \37$\\{print\_wrong\_stk\_lit}(\\{pop\_lit1},\39\\{pop\_typ1},\39%
\\{stk\_str})$;\5
$\\{push\_lit\_stk}(\\{s\_null},\39\\{stk\_str})$;\6
\&{end}\6
\4\&{else} \&{begin} \37$\\{ex\_buf\_length}\K0$;\5
$\\{add\_buf\_pool}(\\{pop\_lit1})$;\5
\X431:Perform the purification\X;\6
\\{add\_pool\_buf\_and\_push};\C{push this string onto the stack}\6
\&{end};\2\6
\&{end};\par
\U342.\fi

\M431.
The resulting string has nonalphanumeric characters removed, and each
\\{white\_space} or \\{sep\_char} character converted to a \\{space}.  The next
module handles special characters.  This code doesn't complain if the
string isn't brace balanced.

\Y\P$\4\X431:Perform the purification\X\S$\6
\&{begin} \37$\\{brace\_level}\K0$;\C{this is the top level}\6
$\\{ex\_buf\_xptr}\K0$;\C{this pointer is for the purified string}\6
$\\{ex\_buf\_ptr}\K0$;\C{and this one is for the original string}\6
\&{while} $(\\{ex\_buf\_ptr}<\\{ex\_buf\_length})$ \1\&{do}\6
\&{begin} \37\&{case} $(\\{lex\_class}[\\{ex\_buf}[\\{ex\_buf\_ptr}]])$ \1%
\&{of}\6
\4$\\{white\_space},\39\\{sep\_char}$: \37\&{begin} \37$\\{ex\_buf}[\\{ex\_buf%
\_xptr}]\K\\{space}$;\5
$\\{incr}(\\{ex\_buf\_xptr})$;\6
\&{end};\6
\4$\\{alpha},\39\\{numeric}$: \37\&{begin} \37$\\{ex\_buf}[\\{ex\_buf\_xptr}]\K%
\\{ex\_buf}[\\{ex\_buf\_ptr}]$;\5
$\\{incr}(\\{ex\_buf\_xptr})$;\6
\&{end};\6
\4\&{othercases} \37\&{if} $(\\{ex\_buf}[\\{ex\_buf\_ptr}]=\\{left\_brace})$ \1%
\&{then}\6
\&{begin} \37$\\{incr}(\\{brace\_level})$;\6
\&{if} $((\\{brace\_level}=1)\W(\\{ex\_buf\_ptr}+1<\\{ex\_buf\_length}))$ \1%
\&{then}\6
\&{if} $(\\{ex\_buf}[\\{ex\_buf\_ptr}+1]=\\{backslash})$ \1\&{then}\5
\X432:Purify a special character\X;\2\2\6
\&{end}\6
\4\&{else} \&{if} $(\\{ex\_buf}[\\{ex\_buf\_ptr}]=\\{right\_brace})$ \1\&{then}%
\6
\&{if} $(\\{brace\_level}>0)$ \1\&{then}\5
$\\{decr}(\\{brace\_level})$\2\2\2\2\6
\&{endcases};\5
$\\{incr}(\\{ex\_buf\_ptr})$;\6
\&{end};\2\6
$\\{ex\_buf\_length}\K\\{ex\_buf\_xptr}$;\6
\&{end}\par
\U430.\fi

\M432.
Special characters (even without a matching \\{right\_brace}) are
purified by removing the control sequences (but restoring the correct
thing for `\.{\\i}' and `\.{\\j}' as well as the eleven alphabetic
foreign characters in Table~3.2 of the \LaTeX\ manual) and removing
all nonalphanumeric characters (including \\{white\_space} and
\\{sep\_char}s).

\Y\P$\4\X432:Purify a special character\X\S$\6
\&{begin} \37$\\{incr}(\\{ex\_buf\_ptr})$;\C{skip over the \\{left\_brace}}\6
\&{while} $((\\{ex\_buf\_ptr}<\\{ex\_buf\_length})\W(\\{brace\_level}>0))$ \1%
\&{do}\6
\&{begin} \37$\\{incr}(\\{ex\_buf\_ptr})$;\C{skip over the \\{backslash}}\6
$\\{ex\_buf\_yptr}\K\\{ex\_buf\_ptr}$;\C{mark the beginning of the control
sequence}\6
\&{while} $((\\{ex\_buf\_ptr}<\\{ex\_buf\_length})\W(\\{lex\_class}[\\{ex%
\_buf}[\\{ex\_buf\_ptr}]]=\\{alpha}))$ \1\&{do}\6
$\\{incr}(\\{ex\_buf\_ptr})$;\C{this scans the control sequence}\2\6
$\\{control\_seq\_loc}\K\\{str\_lookup}(\\{ex\_buf},\39\\{ex\_buf\_yptr},\39%
\\{ex\_buf\_ptr}-\\{ex\_buf\_yptr},\39\\{control\_seq\_ilk},\39\\{dont%
\_insert})$;\6
\&{if} $(\\{hash\_found})$ \1\&{then}\5
\X433:Purify this accented or foreign character\X;\2\6
\&{while} $((\\{ex\_buf\_ptr}<\\{ex\_buf\_length})\W(\\{brace\_level}>0)\W(%
\\{ex\_buf}[\\{ex\_buf\_ptr}]\I\\{backslash}))$ \1\&{do}\6
\&{begin} \37\C{this scans to the next control sequence}\6
\&{case} $(\\{lex\_class}[\\{ex\_buf}[\\{ex\_buf\_ptr}]])$ \1\&{of}\6
\4$\\{alpha},\39\\{numeric}$: \37\&{begin} \37$\\{ex\_buf}[\\{ex\_buf\_xptr}]\K%
\\{ex\_buf}[\\{ex\_buf\_ptr}]$;\5
$\\{incr}(\\{ex\_buf\_xptr})$;\6
\&{end};\6
\4\&{othercases} \37\&{if} $(\\{ex\_buf}[\\{ex\_buf\_ptr}]=\\{right\_brace})$ %
\1\&{then}\5
$\\{decr}(\\{brace\_level})$\6
\4\&{else} \&{if} $(\\{ex\_buf}[\\{ex\_buf\_ptr}]=\\{left\_brace})$ \1\&{then}\5
$\\{incr}(\\{brace\_level})$\2\2\2\6
\&{endcases};\5
$\\{incr}(\\{ex\_buf\_ptr})$;\6
\&{end};\2\6
\&{end};\2\6
$\\{decr}(\\{ex\_buf\_ptr})$;\C{unskip the \\{right\_brace} (or last
character)}\6
\&{end}\par
\U431.\fi

\M433.
We consider the purified character to be either the first alphabetic
character of its control sequence, or perhaps both alphabetic
characters.

\Y\P$\4\X433:Purify this accented or foreign character\X\S$\6
\&{begin} \37$\\{ex\_buf}[\\{ex\_buf\_xptr}]\K\\{ex\_buf}[\\{ex\_buf\_yptr}]$;%
\C{the first alphabetic character}\6
$\\{incr}(\\{ex\_buf\_xptr})$;\6
\&{case} $(\\{ilk\_info}[\\{control\_seq\_loc}])$ \1\&{of}\6
\4$\\{n\_oe},\39\\{n\_oe\_upper},\39\\{n\_ae},\39\\{n\_ae\_upper},\39\\{n%
\_ss}$: \37\&{begin} \37\C{and the second}\6
$\\{ex\_buf}[\\{ex\_buf\_xptr}]\K\\{ex\_buf}[\\{ex\_buf\_yptr}+1]$;\5
$\\{incr}(\\{ex\_buf\_xptr})$;\6
\&{end};\6
\4\&{othercases} \37\\{do\_nothing}\2\6
\&{endcases};\6
\&{end}\par
\U432.\fi

\M434.
The \\{built\_in} function {\.{quote\$}} pushes the string consisting of
the \\{double\_quote} character.

\Y\P$\4\X434:\\{execute\_fn}({\.{quote\$}})\X\S$\6
\4\&{procedure}\1\  \37\\{x\_quote};\2\6
\&{begin} \37$\\{str\_room}(1)$;\5
$\\{append\_char}(\\{double\_quote})$;\5
$\\{push\_lit\_stk}(\\{make\_string},\39\\{stk\_str})$;\6
\&{end};\par
\U342.\fi

\M435.
The \\{built\_in} function {\.{skip\$}} is a no-op.

\Y\P$\4\X435:\\{execute\_fn}({\.{skip\$}})\X\S$\6
\&{begin} \37\\{do\_nothing};\6
\&{end}\par
\U341.\fi

\M436.
The \\{built\_in} function {\.{stack\$}} pops and prints the whole stack;
it's meant to be used for style designers while debugging.

\Y\P$\4\X436:\\{execute\_fn}({\.{stack\$}})\X\S$\6
\&{begin} \37\\{pop\_whole\_stack};\6
\&{end}\par
\U341.\fi

\M437.
The \\{built\_in} function {\.{substring\$}} pops the top three literals
(they are the two integers literals \\{pop\_lit1} and \\{pop\_lit2} and a
string literal, in that order).  It pushes the substring of the (at
most) \\{pop\_lit1} consecutive characters starting at the \\{pop\_lit2}th
character (assuming 1-based indexing) if \\{pop\_lit2} is positive, and
ending at the $-\\{pop\_lit2}$th character from the end if \\{pop\_lit2} is
negative (where the first character from the end is the last
character).  If any of the types is incorrect, it complain and pushes
the null string.

\Y\P$\4\X437:\\{execute\_fn}({\.{substring\$}})\X\S$\6
\4\&{procedure}\1\  \37\\{x\_substring};\6
\4\&{label} \37\\{exit};\2\6
\&{begin} \37$\\{pop\_lit\_stk}(\\{pop\_lit1},\39\\{pop\_typ1})$;\5
$\\{pop\_lit\_stk}(\\{pop\_lit2},\39\\{pop\_typ2})$;\5
$\\{pop\_lit\_stk}(\\{pop\_lit3},\39\\{pop\_typ3})$;\6
\&{if} $(\\{pop\_typ1}\I\\{stk\_int})$ \1\&{then}\6
\&{begin} \37$\\{print\_wrong\_stk\_lit}(\\{pop\_lit1},\39\\{pop\_typ1},\39%
\\{stk\_int})$;\5
$\\{push\_lit\_stk}(\\{s\_null},\39\\{stk\_str})$;\6
\&{end}\6
\4\&{else} \&{if} $(\\{pop\_typ2}\I\\{stk\_int})$ \1\&{then}\6
\&{begin} \37$\\{print\_wrong\_stk\_lit}(\\{pop\_lit2},\39\\{pop\_typ2},\39%
\\{stk\_int})$;\5
$\\{push\_lit\_stk}(\\{s\_null},\39\\{stk\_str})$;\6
\&{end}\6
\4\&{else} \&{if} $(\\{pop\_typ3}\I\\{stk\_str})$ \1\&{then}\6
\&{begin} \37$\\{print\_wrong\_stk\_lit}(\\{pop\_lit3},\39\\{pop\_typ3},\39%
\\{stk\_str})$;\5
$\\{push\_lit\_stk}(\\{s\_null},\39\\{stk\_str})$;\6
\&{end}\6
\4\&{else} \&{begin} \37$\\{sp\_length}\K\\{length}(\\{pop\_lit3})$;\6
\&{if} $(\\{pop\_lit1}\G\\{sp\_length})$ \1\&{then}\6
\&{if} $((\\{pop\_lit2}=1)\V(\\{pop\_lit2}=-1))$ \1\&{then}\6
\&{begin} \37\\{repush\_string};\5
\&{return};\6
\&{end};\2\2\6
\&{if} $((\\{pop\_lit1}\L0)\V(\\{pop\_lit2}=0)\V(\\{pop\_lit2}>\\{sp\_length})%
\V(\\{pop\_lit2}<-\\{sp\_length}))$ \1\&{then}\6
\&{begin} \37$\\{push\_lit\_stk}(\\{s\_null},\39\\{stk\_str})$;\5
\&{return};\6
\&{end}\6
\4\&{else} \X438:Form the appropriate substring\X;\2\6
\&{end};\2\2\2\6
\4\\{exit}: \37\&{end};\par
\U342.\fi

\M438.
This module finds the substring as described in the last section,
and slides it into place in the string pool, if necessary.

\Y\P$\4\X438:Form the appropriate substring\X\S$\6
\&{begin} \37\&{if} $(\\{pop\_lit2}>0)$ \1\&{then}\6
\&{begin} \37\&{if} $(\\{pop\_lit1}>\\{sp\_length}-(\\{pop\_lit2}-1))$ \1%
\&{then}\5
$\\{pop\_lit1}\K\\{sp\_length}-(\\{pop\_lit2}-1)$;\2\6
$\\{sp\_ptr}\K\\{str\_start}[\\{pop\_lit3}]+(\\{pop\_lit2}-1)$;\5
$\\{sp\_end}\K\\{sp\_ptr}+\\{pop\_lit1}$;\6
\&{if} $(\\{pop\_lit2}=1)$ \1\&{then}\6
\&{if} $(\\{pop\_lit3}\G\\{cmd\_str\_ptr})$ \1\&{then}\C{no shifting---merely
change pointers}\6
\&{begin} \37$\\{str\_start}[\\{pop\_lit3}+1]\K\\{sp\_end}$;\5
\\{unflush\_string};\5
$\\{incr}(\\{lit\_stk\_ptr})$;\5
\&{return};\6
\&{end};\2\2\6
\&{end}\6
\4\&{else} \C{$-\\{ex\_buf\_length}\L\\{pop\_lit2}<0$}\2\6
\&{begin} \37$\\{pop\_lit2}\K-\\{pop\_lit2}$;\6
\&{if} $(\\{pop\_lit1}>\\{sp\_length}-(\\{pop\_lit2}-1))$ \1\&{then}\5
$\\{pop\_lit1}\K\\{sp\_length}-(\\{pop\_lit2}-1)$;\2\6
$\\{sp\_end}\K\\{str\_start}[\\{pop\_lit3}+1]-(\\{pop\_lit2}-1)$;\5
$\\{sp\_ptr}\K\\{sp\_end}-\\{pop\_lit1}$;\6
\&{end};\6
\&{while} $(\\{sp\_ptr}<\\{sp\_end})$ \1\&{do}\C{shift the substring}\6
\&{begin} \37$\\{append\_char}(\\{str\_pool}[\\{sp\_ptr}])$;\5
$\\{incr}(\\{sp\_ptr})$;\6
\&{end};\2\6
$\\{push\_lit\_stk}(\\{make\_string},\39\\{stk\_str})$;\C{and push it onto the
stack}\6
\&{end}\par
\U437.\fi

\M439.
The \\{built\_in} function {\.{swap\$}} pops the top two literals from
the stack and pushes them back swapped.

\Y\P$\4\X439:\\{execute\_fn}({\.{swap\$}})\X\S$\6
\4\&{procedure}\1\  \37\\{x\_swap};\2\6
\&{begin} \37$\\{pop\_lit\_stk}(\\{pop\_lit1},\39\\{pop\_typ1})$;\5
$\\{pop\_lit\_stk}(\\{pop\_lit2},\39\\{pop\_typ2})$;\6
\&{if} $((\\{pop\_typ1}\I\\{stk\_str})\V(\\{pop\_lit1}<\\{cmd\_str\_ptr}))$ \1%
\&{then}\6
\&{begin} \37$\\{push\_lit\_stk}(\\{pop\_lit1},\39\\{pop\_typ1})$;\6
\&{if} $((\\{pop\_typ2}=\\{stk\_str})\W(\\{pop\_lit2}\G\\{cmd\_str\_ptr}))$ \1%
\&{then}\5
\\{unflush\_string};\2\6
$\\{push\_lit\_stk}(\\{pop\_lit2},\39\\{pop\_typ2})$;\6
\&{end}\6
\4\&{else} \&{if} $((\\{pop\_typ2}\I\\{stk\_str})\V(\\{pop\_lit2}<\\{cmd\_str%
\_ptr}))$ \1\&{then}\6
\&{begin} \37\\{unflush\_string};\C{this is \\{pop\_lit1}}\6
$\\{push\_lit\_stk}(\\{pop\_lit1},\39\\{stk\_str})$;\5
$\\{push\_lit\_stk}(\\{pop\_lit2},\39\\{pop\_typ2})$;\6
\&{end}\6
\4\&{else} \C{bummer, both are recent strings}\2\2\6
\X440:Swap the two strings (they're at the end of \\{str\_pool})\X;\6
\&{end};\par
\U342.\fi

\M440.
We have to swap both (a)~the strings at the end of the string pool,
and (b)~their pointers on the literal stack.

\Y\P$\4\X440:Swap the two strings (they're at the end of \\{str\_pool})\X\S$\6
\&{begin} \37$\\{ex\_buf\_length}\K0$;\5
$\\{add\_buf\_pool}(\\{pop\_lit2})$;\C{save the second string}\6
$\\{sp\_ptr}\K\\{str\_start}[\\{pop\_lit1}]$;\5
$\\{sp\_end}\K\\{str\_start}[\\{pop\_lit1}+1]$;\6
\&{while} $(\\{sp\_ptr}<\\{sp\_end})$ \1\&{do}\C{slide the first string down}\6
\&{begin} \37$\\{append\_char}(\\{str\_pool}[\\{sp\_ptr}])$;\5
$\\{incr}(\\{sp\_ptr})$;\6
\&{end};\2\6
$\\{push\_lit\_stk}(\\{make\_string},\39\\{stk\_str})$;\C{and push it onto the
stack}\6
\\{add\_pool\_buf\_and\_push};\C{push second string onto the stack}\6
\&{end}\par
\U439.\fi

\M441.
The \\{built\_in} function {\.{text.length\$}} pops the top (string)
literal, and pushes the number of text characters it contains, where
an accented character (more precisely, a ``special character''$\!$,
defined earlier) counts as a single text character, even if it's
missing its matching \\{right\_brace}, and where braces don't count as
text characters.  If the literal isn't a string, it complains and
pushes the null string.

\Y\P$\4\X441:\\{execute\_fn}({\.{text.length\$}})\X\S$\6
\4\&{procedure}\1\  \37\\{x\_text\_length};\2\6
\&{begin} \37$\\{pop\_lit\_stk}(\\{pop\_lit1},\39\\{pop\_typ1})$;\6
\&{if} $(\\{pop\_typ1}\I\\{stk\_str})$ \1\&{then}\6
\&{begin} \37$\\{print\_wrong\_stk\_lit}(\\{pop\_lit1},\39\\{pop\_typ1},\39%
\\{stk\_str})$;\5
$\\{push\_lit\_stk}(\\{s\_null},\39\\{stk\_str})$;\6
\&{end}\6
\4\&{else} \&{begin} \37$\\{num\_text\_chars}\K0$;\5
\X442:Count the text characters\X;\6
$\\{push\_lit\_stk}(\\{num\_text\_chars},\39\\{stk\_int})$;\C{and push it onto
the stack}\6
\&{end};\2\6
\&{end};\par
\U342.\fi

\M442.
Here we determine the number of text characters in the string, where
an entire special character counts as a single text character (even if
it's missing its matching \\{right\_brace}), and where braces don't count
as text characters.

\Y\P$\4\X442:Count the text characters\X\S$\6
\&{begin} \37$\\{sp\_ptr}\K\\{str\_start}[\\{pop\_lit1}]$;\5
$\\{sp\_end}\K\\{str\_start}[\\{pop\_lit1}+1]$;\5
$\\{sp\_brace\_level}\K0$;\6
\&{while} $(\\{sp\_ptr}<\\{sp\_end})$ \1\&{do}\6
\&{begin} \37$\\{incr}(\\{sp\_ptr})$;\6
\&{if} $(\\{str\_pool}[\\{sp\_ptr}-1]=\\{left\_brace})$ \1\&{then}\6
\&{begin} \37$\\{incr}(\\{sp\_brace\_level})$;\6
\&{if} $((\\{sp\_brace\_level}=1)\W(\\{sp\_ptr}<\\{sp\_end}))$ \1\&{then}\6
\&{if} $(\\{str\_pool}[\\{sp\_ptr}]=\\{backslash})$ \1\&{then}\6
\&{begin} \37$\\{incr}(\\{sp\_ptr})$;\C{skip over the \\{backslash}}\6
\&{while} $((\\{sp\_ptr}<\\{sp\_end})\W(\\{sp\_brace\_level}>0))$ \1\&{do}\6
\&{begin} \37\&{if} $(\\{str\_pool}[\\{sp\_ptr}]=\\{right\_brace})$ \1\&{then}\5
$\\{decr}(\\{sp\_brace\_level})$\6
\4\&{else} \&{if} $(\\{str\_pool}[\\{sp\_ptr}]=\\{left\_brace})$ \1\&{then}\5
$\\{incr}(\\{sp\_brace\_level})$;\2\2\6
$\\{incr}(\\{sp\_ptr})$;\6
\&{end};\2\6
$\\{incr}(\\{num\_text\_chars})$;\6
\&{end};\2\2\6
\&{end}\6
\4\&{else} \&{if} $(\\{str\_pool}[\\{sp\_ptr}-1]=\\{right\_brace})$ \1\&{then}\6
\&{begin} \37\&{if} $(\\{sp\_brace\_level}>0)$ \1\&{then}\5
$\\{decr}(\\{sp\_brace\_level})$;\2\6
\&{end}\6
\4\&{else} $\\{incr}(\\{num\_text\_chars})$;\2\2\6
\&{end};\2\6
\&{end}\par
\U441.\fi

\M443.
The \\{built\_in} function {\.{text.prefix\$}} pops the top two literals
(the integer literal \\{pop\_lit1} and a string literal, in that order).
It pushes the substring of the (at most) \\{pop\_lit1} consecutive text
characters starting from the beginning of the string.  This function
is similar to {\.{substring\$}}, but this one considers an accented
character (or more precisely, a ``special character''$\!$, even if
it's missing its matching \\{right\_brace}) to be a single text character
(rather than however many \\{ASCII\_code} characters it actually
comprises), and this function doesn't consider braces to be text
characters; furthermore, this function appends any needed matching
\\{right\_brace}s.  If any of the types is incorrect, it complains and
pushes the null string.

\Y\P$\4\X443:\\{execute\_fn}({\.{text.prefix\$}})\X\S$\6
\4\&{procedure}\1\  \37\\{x\_text\_prefix};\6
\4\&{label} \37\\{exit};\2\6
\&{begin} \37$\\{pop\_lit\_stk}(\\{pop\_lit1},\39\\{pop\_typ1})$;\5
$\\{pop\_lit\_stk}(\\{pop\_lit2},\39\\{pop\_typ2})$;\6
\&{if} $(\\{pop\_typ1}\I\\{stk\_int})$ \1\&{then}\6
\&{begin} \37$\\{print\_wrong\_stk\_lit}(\\{pop\_lit1},\39\\{pop\_typ1},\39%
\\{stk\_int})$;\5
$\\{push\_lit\_stk}(\\{s\_null},\39\\{stk\_str})$;\6
\&{end}\6
\4\&{else} \&{if} $(\\{pop\_typ2}\I\\{stk\_str})$ \1\&{then}\6
\&{begin} \37$\\{print\_wrong\_stk\_lit}(\\{pop\_lit2},\39\\{pop\_typ2},\39%
\\{stk\_str})$;\5
$\\{push\_lit\_stk}(\\{s\_null},\39\\{stk\_str})$;\6
\&{end}\6
\4\&{else} \&{if} $(\\{pop\_lit1}\L0)$ \1\&{then}\6
\&{begin} \37$\\{push\_lit\_stk}(\\{s\_null},\39\\{stk\_str})$;\5
\&{return};\6
\&{end}\6
\4\&{else} \X444:Form the appropriate prefix\X;\2\2\2\6
\4\\{exit}: \37\&{end};\par
\U342.\fi

\M444.
This module finds the prefix as described in the last section, and
appends any needed matching \\{right\_brace}s.

\Y\P$\4\X444:Form the appropriate prefix\X\S$\6
\&{begin} \37$\\{sp\_ptr}\K\\{str\_start}[\\{pop\_lit2}]$;\5
$\\{sp\_end}\K\\{str\_start}[\\{pop\_lit2}+1]$;\C{this may change}\6
\X445:Scan the appropriate number of characters\X;\6
\&{if} $(\\{pop\_lit2}\G\\{cmd\_str\_ptr})$ \1\&{then}\C{no shifting---merely
change pointers}\6
$\\{pool\_ptr}\K\\{sp\_end}$\6
\4\&{else} \&{while} $(\\{sp\_ptr}<\\{sp\_end})$ \1\&{do}\C{shift the
substring}\6
\&{begin} \37$\\{append\_char}(\\{str\_pool}[\\{sp\_ptr}])$;\5
$\\{incr}(\\{sp\_ptr})$;\6
\&{end};\2\2\6
\&{while} $(\\{sp\_brace\_level}>0)$ \1\&{do}\C{add matching \\{right\_brace}s}%
\6
\&{begin} \37$\\{append\_char}(\\{right\_brace})$;\5
$\\{decr}(\\{sp\_brace\_level})$;\6
\&{end};\2\6
$\\{push\_lit\_stk}(\\{make\_string},\39\\{stk\_str})$;\C{and push it onto the
stack}\6
\&{end}\par
\U443.\fi

\M445.
This section scans \\{pop\_lit1} text characters, where an entire special
character counts as a single text character (even if it's missing its
matching \\{right\_brace}), and where braces don't count as text
characters.

\Y\P$\4\X445:Scan the appropriate number of characters\X\S$\6
\&{begin} \37$\\{num\_text\_chars}\K0$;\5
$\\{sp\_brace\_level}\K0$;\5
$\\{sp\_xptr1}\K\\{sp\_ptr}$;\6
\&{while} $((\\{sp\_xptr1}<\\{sp\_end})\W(\\{num\_text\_chars}<\\{pop\_lit1}))$
\1\&{do}\6
\&{begin} \37$\\{incr}(\\{sp\_xptr1})$;\6
\&{if} $(\\{str\_pool}[\\{sp\_xptr1}-1]=\\{left\_brace})$ \1\&{then}\6
\&{begin} \37$\\{incr}(\\{sp\_brace\_level})$;\6
\&{if} $((\\{sp\_brace\_level}=1)\W(\\{sp\_xptr1}<\\{sp\_end}))$ \1\&{then}\6
\&{if} $(\\{str\_pool}[\\{sp\_xptr1}]=\\{backslash})$ \1\&{then}\6
\&{begin} \37$\\{incr}(\\{sp\_xptr1})$;\C{skip over the \\{backslash}}\6
\&{while} $((\\{sp\_xptr1}<\\{sp\_end})\W(\\{sp\_brace\_level}>0))$ \1\&{do}\6
\&{begin} \37\&{if} $(\\{str\_pool}[\\{sp\_xptr1}]=\\{right\_brace})$ \1%
\&{then}\5
$\\{decr}(\\{sp\_brace\_level})$\6
\4\&{else} \&{if} $(\\{str\_pool}[\\{sp\_xptr1}]=\\{left\_brace})$ \1\&{then}\5
$\\{incr}(\\{sp\_brace\_level})$;\2\2\6
$\\{incr}(\\{sp\_xptr1})$;\6
\&{end};\2\6
$\\{incr}(\\{num\_text\_chars})$;\6
\&{end};\2\2\6
\&{end}\6
\4\&{else} \&{if} $(\\{str\_pool}[\\{sp\_xptr1}-1]=\\{right\_brace})$ \1%
\&{then}\6
\&{begin} \37\&{if} $(\\{sp\_brace\_level}>0)$ \1\&{then}\5
$\\{decr}(\\{sp\_brace\_level})$;\2\6
\&{end}\6
\4\&{else} $\\{incr}(\\{num\_text\_chars})$;\2\2\6
\&{end};\2\6
$\\{sp\_end}\K\\{sp\_xptr1}$;\6
\&{end}\par
\U444.\fi

\M446.
The \\{built\_in} function {\.{top\$}} pops and prints the top of the
stack.

\Y\P$\4\X446:\\{execute\_fn}({\.{top\$}})\X\S$\6
\&{begin} \37\\{pop\_top\_and\_print};\6
\&{end}\par
\U341.\fi

\M447.
The \\{built\_in} function {\.{type\$}} pushes the appropriate string
from \\{type\_list} onto the stack (unless either it's \\{undefined} or
\\{empty}, in which case it pushes the null string).

\Y\P$\4\X447:\\{execute\_fn}({\.{type\$}})\X\S$\6
\4\&{procedure}\1\  \37\\{x\_type};\2\6
\&{begin} \37\&{if} $(\R\\{mess\_with\_entries})$ \1\&{then}\5
\\{bst\_cant\_mess\_with\_entries\_print}\6
\4\&{else} \&{if} $((\\{type\_list}[\\{cite\_ptr}]=\\{undefined})\V(\\{type%
\_list}[\\{cite\_ptr}]=\\{empty}))$ \1\&{then}\5
$\\{push\_lit\_stk}(\\{s\_null},\39\\{stk\_str})$\6
\4\&{else} $\\{push\_lit\_stk}(\\{hash\_text}[\\{type\_list}[\\{cite\_ptr}]],%
\39\\{stk\_str})$;\2\2\6
\&{end};\par
\U342.\fi

\M448.
The \\{built\_in} function {\.{warning\$}} pops the top (string) literal
and prints it following a warning message.  This is implemented as a
special \\{built\_in} function rather than using the {\.{top\$}} function
so that it can \\{mark\_warning}.

\Y\P$\4\X448:\\{execute\_fn}({\.{warning\$}})\X\S$\6
\4\&{procedure}\1\  \37\\{x\_warning};\2\6
\&{begin} \37$\\{pop\_lit\_stk}(\\{pop\_lit1},\39\\{pop\_typ1})$;\6
\&{if} $(\\{pop\_typ1}\I\\{stk\_str})$ \1\&{then}\5
$\\{print\_wrong\_stk\_lit}(\\{pop\_lit1},\39\\{pop\_typ1},\39\\{stk\_str})$\6
\4\&{else} \&{begin} \37$\\{print}(\.{\'Warning--\'})$;\5
$\\{print\_lit}(\\{pop\_lit1},\39\\{pop\_typ1})$;\5
\\{mark\_warning};\6
\&{end};\2\6
\&{end};\par
\U342.\fi

\M449.
The \\{built\_in} function {\.{while\$}} pops the top two (function)
literals, and keeps executing the second as long as the (integer)
value left on the stack by executing the first is greater than 0.  If
either type is incorrect, it complains but does nothing else.

\Y\P$\4\X449:\\{execute\_fn}({\.{while\$}})\X\S$\6
\&{begin} \37$\\{pop\_lit\_stk}(\\{r\_pop\_lt1},\39\\{r\_pop\_tp1})$;\5
$\\{pop\_lit\_stk}(\\{r\_pop\_lt2},\39\\{r\_pop\_tp2})$;\6
\&{if} $(\\{r\_pop\_tp1}\I\\{stk\_fn})$ \1\&{then}\5
$\\{print\_wrong\_stk\_lit}(\\{r\_pop\_lt1},\39\\{r\_pop\_tp1},\39\\{stk\_fn})$%
\6
\4\&{else} \&{if} $(\\{r\_pop\_tp2}\I\\{stk\_fn})$ \1\&{then}\5
$\\{print\_wrong\_stk\_lit}(\\{r\_pop\_lt2},\39\\{r\_pop\_tp2},\39\\{stk\_fn})$%
\6
\4\&{else} \~ \1\&{loop}\6
\&{begin} \37$\\{execute\_fn}(\\{r\_pop\_lt2})$;\C{this is the \.{while\$}
test}\6
$\\{pop\_lit\_stk}(\\{pop\_lit1},\39\\{pop\_typ1})$;\6
\&{if} $(\\{pop\_typ1}\I\\{stk\_int})$ \1\&{then}\6
\&{begin} \37$\\{print\_wrong\_stk\_lit}(\\{pop\_lit1},\39\\{pop\_typ1},\39%
\\{stk\_int})$;\5
\&{goto} \37\\{end\_while};\6
\&{end}\6
\4\&{else} \&{if} $(\\{pop\_lit1}>0)$ \1\&{then}\5
$\\{execute\_fn}(\\{r\_pop\_lt1})$\C{this is the \.{while\$} body}\6
\4\&{else} \&{goto} \37\\{end\_while};\2\2\6
\&{end};\2\2\2\6
\4\\{end\_while}: \37\C{justifies this \\{mean\_while}}\6
\&{end}\par
\U341.\fi

\M450.
The \\{built\_in} function {\.{width\$}} pops the top (string) literal
and pushes the integer that represents its width in units specified by
the \\{char\_width} array.  This function takes the literal literally;
that is, it assumes each character in the string is to be printed as
is, regardless of whether the character has a special meaning to \TeX,
except that special characters (even without their \\{right\_brace}s) are
handled specially.  If the literal isn't a string, it complains and
pushes~0.

\Y\P$\4\X450:\\{execute\_fn}({\.{width\$}})\X\S$\6
\4\&{procedure}\1\  \37\\{x\_width};\2\6
\&{begin} \37$\\{pop\_lit\_stk}(\\{pop\_lit1},\39\\{pop\_typ1})$;\6
\&{if} $(\\{pop\_typ1}\I\\{stk\_str})$ \1\&{then}\6
\&{begin} \37$\\{print\_wrong\_stk\_lit}(\\{pop\_lit1},\39\\{pop\_typ1},\39%
\\{stk\_str})$;\5
$\\{push\_lit\_stk}(0,\39\\{stk\_int})$;\6
\&{end}\6
\4\&{else} \&{begin} \37$\\{ex\_buf\_length}\K0$;\5
$\\{add\_buf\_pool}(\\{pop\_lit1})$;\5
$\\{string\_width}\K0$;\5
\X451:Add up the \\{char\_width}s in this string\X;\6
$\\{push\_lit\_stk}(\\{string\_width},\39\\{stk\_int})$;\6
\&{end}\2\6
\&{end};\par
\U342.\fi

\M451.
We use the natural width for all but special characters, and we
complain if the string isn't brace-balanced.

\Y\P$\4\X451:Add up the \\{char\_width}s in this string\X\S$\6
\&{begin} \37$\\{brace\_level}\K0$;\C{we're at the top level}\6
$\\{ex\_buf\_ptr}\K0$;\C{and the beginning of string}\6
\&{while} $(\\{ex\_buf\_ptr}<\\{ex\_buf\_length})$ \1\&{do}\6
\&{begin} \37\&{if} $(\\{ex\_buf}[\\{ex\_buf\_ptr}]=\\{left\_brace})$ \1%
\&{then}\6
\&{begin} \37$\\{incr}(\\{brace\_level})$;\6
\&{if} $((\\{brace\_level}=1)\W(\\{ex\_buf\_ptr}+1<\\{ex\_buf\_length}))$ \1%
\&{then}\6
\&{if} $(\\{ex\_buf}[\\{ex\_buf\_ptr}+1]=\\{backslash})$ \1\&{then}\5
\X452:Determine the width of this special character\X\6
\4\&{else} $\\{string\_width}\K\\{string\_width}+\\{char\_width}[\\{left%
\_brace}]$\2\6
\4\&{else} $\\{string\_width}\K\\{string\_width}+\\{char\_width}[\\{left%
\_brace}]$;\2\6
\&{end}\6
\4\&{else} \&{if} $(\\{ex\_buf}[\\{ex\_buf\_ptr}]=\\{right\_brace})$ \1\&{then}%
\6
\&{begin} \37$\\{decr\_brace\_level}(\\{pop\_lit1})$;\5
$\\{string\_width}\K\\{string\_width}+\\{char\_width}[\\{right\_brace}]$;\6
\&{end}\6
\4\&{else} $\\{string\_width}\K\\{string\_width}+\\{char\_width}[\\{ex\_buf}[%
\\{ex\_buf\_ptr}]]$;\2\2\6
$\\{incr}(\\{ex\_buf\_ptr})$;\6
\&{end};\2\6
$\\{check\_brace\_level}(\\{pop\_lit1})$;\6
\&{end}\par
\U450.\fi

\M452.
We use the natural widths of all characters except that some
characters have no width: braces, control sequences (except for the
usual 13 accented and foreign characters, whose widths are given in
the next module), and \\{white\_space} following control sequences (even
a null control sequence).

\Y\P$\4\X452:Determine the width of this special character\X\S$\6
\&{begin} \37$\\{incr}(\\{ex\_buf\_ptr})$;\C{skip over the \\{left\_brace}}\6
\&{while} $((\\{ex\_buf\_ptr}<\\{ex\_buf\_length})\W(\\{brace\_level}>0))$ \1%
\&{do}\6
\&{begin} \37$\\{incr}(\\{ex\_buf\_ptr})$;\C{skip over the \\{backslash}}\6
$\\{ex\_buf\_xptr}\K\\{ex\_buf\_ptr}$;\6
\&{while} $((\\{ex\_buf\_ptr}<\\{ex\_buf\_length})\W(\\{lex\_class}[\\{ex%
\_buf}[\\{ex\_buf\_ptr}]]=\\{alpha}))$ \1\&{do}\6
$\\{incr}(\\{ex\_buf\_ptr})$;\C{this scans the control sequence}\2\6
\&{if} $((\\{ex\_buf\_ptr}<\\{ex\_buf\_length})\W(\\{ex\_buf\_ptr}=\\{ex\_buf%
\_xptr}))$ \1\&{then}\5
$\\{incr}(\\{ex\_buf\_ptr})$\C{this skips a nonalpha control seq}\6
\4\&{else} \&{begin} \37$\\{control\_seq\_loc}\K\\{str\_lookup}(\\{ex\_buf},\39%
\\{ex\_buf\_xptr},\39\\{ex\_buf\_ptr}-\\{ex\_buf\_xptr},\39\\{control\_seq%
\_ilk},\39\\{dont\_insert})$;\6
\&{if} $(\\{hash\_found})$ \1\&{then}\5
\X453:Determine the width of this accented or foreign character\X;\2\6
\&{end};\2\6
\&{while} $((\\{ex\_buf\_ptr}<\\{ex\_buf\_length})\W(\\{lex\_class}[\\{ex%
\_buf}[\\{ex\_buf\_ptr}]]=\\{white\_space}))$ \1\&{do}\5
$\\{incr}(\\{ex\_buf\_ptr})$;\C{this skips following \\{white\_space}}\2\6
\&{while} $((\\{ex\_buf\_ptr}<\\{ex\_buf\_length})\W(\\{brace\_level}>0)\W(%
\\{ex\_buf}[\\{ex\_buf\_ptr}]\I\\{backslash}))$ \1\&{do}\6
\&{begin} \37\C{this scans to the next control sequence}\6
\&{if} $(\\{ex\_buf}[\\{ex\_buf\_ptr}]=\\{right\_brace})$ \1\&{then}\5
$\\{decr}(\\{brace\_level})$\6
\4\&{else} \&{if} $(\\{ex\_buf}[\\{ex\_buf\_ptr}]=\\{left\_brace})$ \1\&{then}\5
$\\{incr}(\\{brace\_level})$\6
\4\&{else} $\\{string\_width}\K\\{string\_width}+\\{char\_width}[\\{ex\_buf}[%
\\{ex\_buf\_ptr}]]$;\2\2\6
$\\{incr}(\\{ex\_buf\_ptr})$;\6
\&{end};\2\6
\&{end};\2\6
$\\{decr}(\\{ex\_buf\_ptr})$;\C{unskip the \\{right\_brace}}\6
\&{end}\par
\U451.\fi

\M453.
Five of the 13 possibilities resort to special information not present
in the \\{char\_width} array; the other eight simply use \\{char\_width}'s
information for the first letter of the control sequence.

\Y\P$\4\X453:Determine the width of this accented or foreign character\X\S$\6
\&{begin} \37\&{case} $(\\{ilk\_info}[\\{control\_seq\_loc}])$ \1\&{of}\6
\4\\{n\_ss}: \37$\\{string\_width}\K\\{string\_width}+\\{ss\_width}$;\6
\4\\{n\_ae}: \37$\\{string\_width}\K\\{string\_width}+\\{ae\_width}$;\6
\4\\{n\_oe}: \37$\\{string\_width}\K\\{string\_width}+\\{oe\_width}$;\6
\4\\{n\_ae\_upper}: \37$\\{string\_width}\K\\{string\_width}+\\{upper\_ae%
\_width}$;\6
\4\\{n\_oe\_upper}: \37$\\{string\_width}\K\\{string\_width}+\\{upper\_oe%
\_width}$;\6
\4\&{othercases} \37$\\{string\_width}\K\\{string\_width}+\\{char\_width}[\\{ex%
\_buf}[\\{ex\_buf\_xptr}]]$\2\6
\&{endcases};\6
\&{end}\par
\U452.\fi

\M454.
The \\{built\_in} function {\.{write\$}} pops the top (string) literal
and writes it onto the output buffer \\{out\_buf} (which will result in
stuff being written onto the \.{.bbl} file if the buffer fills up).  If
the literal isn't a string, it complains but does nothing else.

\Y\P$\4\X454:\\{execute\_fn}({\.{write\$}})\X\S$\6
\4\&{procedure}\1\  \37\\{x\_write};\2\6
\&{begin} \37$\\{pop\_lit\_stk}(\\{pop\_lit1},\39\\{pop\_typ1})$;\6
\&{if} $(\\{pop\_typ1}\I\\{stk\_str})$ \1\&{then}\5
$\\{print\_wrong\_stk\_lit}(\\{pop\_lit1},\39\\{pop\_typ1},\39\\{stk\_str})$\6
\4\&{else} $\\{add\_out\_pool}(\\{pop\_lit1})$;\2\6
\&{end};\par
\U342.\fi

\N455.  Cleaning up.
This section does any last-minute printing and ends the program.

\Y\P$\4\X455:Clean up and leave\X\S$\6
\&{begin} \37\&{if} $((\\{read\_performed})\W(\R\\{reading\_completed}))$ \1%
\&{then}\6
\&{begin} \37$\\{print}(\.{\'Aborted\ at\ line\ \'},\39\\{bib\_line\_num}:0,\39%
\.{\'\ of\ file\ \'})$;\5
\\{print\_bib\_name};\6
\&{end};\2\6
\\{trace\_and\_stat\_printing};\5
\X466:Print the job \\{history}\X;\6
$\\{a\_close}(\\{log\_file})$;\C{turn out the lights, the fat lady has sung;
it's over, Yogi}\6
\&{end}\par
\U10.\fi

\M456.
Here we print  \&{trace}  and/or  \&{stat}  information, if desired.

\Y\P$\4\X3:Procedures and functions for all file I/O, error messages, and such%
\X\mathrel{+}\S$\6
\4\&{procedure}\1\  \37\\{trace\_and\_stat\_printing};\2\6
\&{begin} \37\&{trace} \37\X457:Print all \.{.bib}- and \.{.bst}-file
information\X;\6
\X458:Print all \\{cite\_list} and entry information\X;\6
\X463:Print the \\{wiz\_defined} functions\X;\6
\X464:Print the string pool\X;\6
\&{ecart}\7
\&{stat} \37\X465:Print usage statistics\X;\6
\&{tats}\7
\&{end};\par
\fi

\M457.
This prints information obtained from the \.{.aux} file about the
other files.

\Y\P$\4\X457:Print all \.{.bib}- and \.{.bst}-file information\X\S$\6
\&{begin} \37\&{if} $(\\{num\_bib\_files}=1)$ \1\&{then}\5
$\\{trace\_pr\_ln}(\.{\'The\ 1\ database\ file\ is\'})$\6
\4\&{else} $\\{trace\_pr\_ln}(\.{\'The\ \'},\39\\{num\_bib\_files}:0,\39\.{\'\
database\ files\ are\'})$;\2\6
\&{if} $(\\{num\_bib\_files}=0)$ \1\&{then}\5
$\\{trace\_pr\_ln}(\.{\'\ \ \ undefined\'})$\6
\4\&{else} \&{begin} \37$\\{bib\_ptr}\K0$;\6
\&{while} $(\\{bib\_ptr}<\\{num\_bib\_files})$ \1\&{do}\6
\&{begin} \37$\\{trace\_pr}(\.{\'\ \ \ \'})$;\5
$\\{trace\_pr\_pool\_str}(\\{cur\_bib\_str})$;\5
$\\{trace\_pr\_pool\_str}(\\{s\_bib\_extension})$;\5
\\{trace\_pr\_newline};\5
$\\{incr}(\\{bib\_ptr})$;\6
\&{end};\2\6
\&{end};\2\6
$\\{trace\_pr}(\.{\'The\ style\ file\ is\ \'})$;\6
\&{if} $(\\{bst\_str}=0)$ \1\&{then}\5
$\\{trace\_pr\_ln}(\.{\'undefined\'})$\6
\4\&{else} \&{begin} \37$\\{trace\_pr\_pool\_str}(\\{bst\_str})$;\5
$\\{trace\_pr\_pool\_str}(\\{s\_bst\_extension})$;\5
\\{trace\_pr\_newline};\6
\&{end};\2\6
\&{end}\par
\U456.\fi

\M458.
In entry-sorted order, this prints an entry's \\{cite\_list} string and,
indirectly, its entry type and entry variables.

\Y\P$\4\X458:Print all \\{cite\_list} and entry information\X\S$\6
\&{begin} \37\&{if} $(\\{all\_entries})$ \1\&{then}\5
$\\{trace\_pr}(\.{\'all\_marker=\'},\39\\{all\_marker}:0,\39\.{\',\ \'})$;\2\6
\&{if} $(\\{read\_performed})$ \1\&{then}\5
$\\{trace\_pr\_ln}(\.{\'old\_num\_cites=\'},\39\\{old\_num\_cites}:0)$\6
\4\&{else} \\{trace\_pr\_newline};\2\6
$\\{trace\_pr}(\.{\'The\ \'},\39\\{num\_cites}:0)$;\6
\&{if} $(\\{num\_cites}=1)$ \1\&{then}\5
$\\{trace\_pr\_ln}(\.{\'\ entry:\'})$\6
\4\&{else} $\\{trace\_pr\_ln}(\.{\'\ entries:\'})$;\2\6
\&{if} $(\\{num\_cites}=0)$ \1\&{then}\5
$\\{trace\_pr\_ln}(\.{\'\ \ \ undefined\'})$\6
\4\&{else} \&{begin} \37$\\{sort\_cite\_ptr}\K0$;\6
\&{while} $(\\{sort\_cite\_ptr}<\\{num\_cites})$ \1\&{do}\6
\&{begin} \37\&{if} $(\R\\{read\_completed})$ \1\&{then}\C{we didn't finish the
\.{read} command}\6
$\\{cite\_ptr}\K\\{sort\_cite\_ptr}$\6
\4\&{else} $\\{cite\_ptr}\K\\{sorted\_cites}[\\{sort\_cite\_ptr}]$;\2\6
$\\{trace\_pr\_pool\_str}(\\{cur\_cite\_str})$;\6
\&{if} $(\\{read\_performed})$ \1\&{then}\5
\X459:Print entry information\X\6
\4\&{else} \\{trace\_pr\_newline};\2\6
$\\{incr}(\\{sort\_cite\_ptr})$;\6
\&{end};\2\6
\&{end};\2\6
\&{end}\par
\U456.\fi

\M459.
This prints information gathered while reading the \.{.bst} and
\.{.bib} files.

\Y\P$\4\X459:Print entry information\X\S$\6
\&{begin} \37$\\{trace\_pr}(\.{\',\ entry-type\ \'})$;\6
\&{if} $(\\{type\_list}[\\{cite\_ptr}]=\\{undefined})$ \1\&{then}\6
\4\\{undefined}: \37$\\{trace\_pr}(\.{\'unknown\'})$\6
\4\&{else} \&{if} $(\\{type\_list}[\\{cite\_ptr}]=\\{empty})$ \1\&{then}\5
$\\{trace\_pr}(\.{\'---\ no\ type\ found\'})$\6
\4\&{else} $\\{trace\_pr\_pool\_str}(\\{hash\_text}[\\{type\_list}[\\{cite%
\_ptr}]])$;\2\2\6
$\\{trace\_pr\_ln}(\.{\',\ has\ entry\ strings\'})$;\5
\X460:Print entry strings\X;\6
$\\{trace\_pr}(\.{\'\ \ has\ entry\ integers\'})$;\5
\X461:Print entry integers\X;\6
$\\{trace\_pr\_ln}(\.{\'\ \ and\ has\ fields\'})$;\5
\X462:Print fields\X;\6
\&{end}\par
\U458.\fi

\M460.
This prints, for the current entry, the strings declared by the
\.{entry} command.

\Y\P$\4\X460:Print entry strings\X\S$\6
\&{begin} \37\&{if} $(\\{num\_ent\_strs}=0)$ \1\&{then}\5
$\\{trace\_pr\_ln}(\.{\'\ \ \ \ undefined\'})$\6
\4\&{else} \&{if} $(\R\\{read\_completed})$ \1\&{then}\5
$\\{trace\_pr\_ln}(\.{\'\ \ \ \ uninitialized\'})$\6
\4\&{else} \&{begin} \37$\\{str\_ent\_ptr}\K\\{cite\_ptr}\ast\\{num\_ent%
\_strs}$;\6
\&{while} $(\\{str\_ent\_ptr}<(\\{cite\_ptr}+1)\ast\\{num\_ent\_strs})$ \1%
\&{do}\6
\&{begin} \37$\\{ent\_chr\_ptr}\K0$;\5
$\\{trace\_pr}(\.{\'\ \ \ \ "\'})$;\6
\&{while} $(\\{entry\_strs}[\\{str\_ent\_ptr}][\\{ent\_chr\_ptr}]\I\\{end\_of%
\_string})$ \1\&{do}\6
\&{begin} \37$\\{trace\_pr}(\\{xchr}[\\{entry\_strs}[\\{str\_ent\_ptr}][\\{ent%
\_chr\_ptr}]])$;\5
$\\{incr}(\\{ent\_chr\_ptr})$;\6
\&{end};\2\6
$\\{trace\_pr\_ln}(\.{\'"\'})$;\5
$\\{incr}(\\{str\_ent\_ptr})$;\6
\&{end};\2\6
\&{end};\2\2\6
\&{end}\par
\U459.\fi

\M461.
This prints, for the current entry, the integers declared by the
\.{entry} command.

\Y\P$\4\X461:Print entry integers\X\S$\6
\&{begin} \37\&{if} $(\\{num\_ent\_ints}=0)$ \1\&{then}\5
$\\{trace\_pr}(\.{\'\ undefined\'})$\6
\4\&{else} \&{if} $(\R\\{read\_completed})$ \1\&{then}\5
$\\{trace\_pr}(\.{\'\ uninitialized\'})$\6
\4\&{else} \&{begin} \37$\\{int\_ent\_ptr}\K\\{cite\_ptr}\ast\\{num\_ent%
\_ints}$;\6
\&{while} $(\\{int\_ent\_ptr}<(\\{cite\_ptr}+1)\ast\\{num\_ent\_ints})$ \1%
\&{do}\6
\&{begin} \37$\\{trace\_pr}(\.{\'\ \'},\39\\{entry\_ints}[\\{int\_ent%
\_ptr}]:0)$;\5
$\\{incr}(\\{int\_ent\_ptr})$;\6
\&{end};\2\6
\&{end};\2\2\6
\\{trace\_pr\_newline};\6
\&{end}\par
\U459.\fi

\M462.
This prints the fields stored for the current entry.

\Y\P$\4\X462:Print fields\X\S$\6
\&{begin} \37\&{if} $(\R\\{read\_performed})$ \1\&{then}\5
$\\{trace\_pr\_ln}(\.{\'\ \ \ \ uninitialized\'})$\6
\4\&{else} \&{begin} \37$\\{field\_ptr}\K\\{cite\_ptr}\ast\\{num\_fields}$;\5
$\\{field\_end\_ptr}\K\\{field\_ptr}+\\{num\_fields}$;\5
$\\{no\_fields}\K\\{true}$;\6
\&{while} $(\\{field\_ptr}<\\{field\_end\_ptr})$ \1\&{do}\6
\&{begin} \37\&{if} $(\\{field\_info}[\\{field\_ptr}]\I\\{missing})$ \1\&{then}%
\6
\&{begin} \37$\\{trace\_pr}(\.{\'\ \ \ \ "\'})$;\5
$\\{trace\_pr\_pool\_str}(\\{field\_info}[\\{field\_ptr}])$;\5
$\\{trace\_pr\_ln}(\.{\'"\'})$;\5
$\\{no\_fields}\K\\{false}$;\6
\&{end};\2\6
$\\{incr}(\\{field\_ptr})$;\6
\&{end};\2\6
\&{if} $(\\{no\_fields})$ \1\&{then}\5
$\\{trace\_pr\_ln}(\.{\'\ \ \ \ missing\'})$;\2\6
\&{end};\2\6
\&{end}\par
\U459.\fi

\M463.
This gives all the \\{wiz\_defined} functions that appeared in the
\.{.bst} file.

\Y\P$\4\X463:Print the \\{wiz\_defined} functions\X\S$\6
\&{begin} \37$\\{trace\_pr\_ln}(\.{\'The\ wiz-defined\ functions\ are\'})$;\6
\&{if} $(\\{wiz\_def\_ptr}=0)$ \1\&{then}\5
$\\{trace\_pr\_ln}(\.{\'\ \ \ nonexistent\'})$\6
\4\&{else} \&{begin} \37$\\{wiz\_fn\_ptr}\K0$;\6
\&{while} $(\\{wiz\_fn\_ptr}<\\{wiz\_def\_ptr})$ \1\&{do}\6
\&{begin} \37\&{if} $(\\{wiz\_functions}[\\{wiz\_fn\_ptr}]=\\{end\_of\_def})$ %
\1\&{then}\5
$\\{trace\_pr\_ln}(\\{wiz\_fn\_ptr}:0,\39\.{\'--end-of-def--\'})$\6
\4\&{else} \&{if} $(\\{wiz\_functions}[\\{wiz\_fn\_ptr}]=\\{quote\_next\_fn})$ %
\1\&{then}\5
$\\{trace\_pr}(\\{wiz\_fn\_ptr}:0,\39\.{\'\ \ quote\_next\_function\ \ \ \ %
\'})$\6
\4\&{else} \&{begin} \37$\\{trace\_pr}(\\{wiz\_fn\_ptr}:0,\39\.{\'\ \ \`\'})$;\5
$\\{trace\_pr\_pool\_str}(\\{hash\_text}[\\{wiz\_functions}[\\{wiz\_fn%
\_ptr}]])$;\5
$\\{trace\_pr\_ln}(\.{\'\'}\.{\'\'})$;\6
\&{end};\2\2\6
$\\{incr}(\\{wiz\_fn\_ptr})$;\6
\&{end};\2\6
\&{end};\2\6
\&{end}\par
\U456.\fi

\M464.
This includes all the `static' strings (that is, those that are also
in the hash table), but none of the dynamic strings (that is, those
put on the stack while executing \.{.bst} commands).

\Y\P$\4\X464:Print the string pool\X\S$\6
\&{begin} \37$\\{trace\_pr\_ln}(\.{\'The\ string\ pool\ is\'})$;\5
$\\{str\_num}\K1$;\6
\&{while} $(\\{str\_num}<\\{str\_ptr})$ \1\&{do}\6
\&{begin} \37$\\{trace\_pr}(\\{str\_num}:4,\39\\{str\_start}[\\{str\_num}]:6,%
\39\.{\'\ "\'})$;\5
$\\{trace\_pr\_pool\_str}(\\{str\_num})$;\5
$\\{trace\_pr\_ln}(\.{\'"\'})$;\5
$\\{incr}(\\{str\_num})$;\6
\&{end};\2\6
\&{end}\par
\U456.\fi

\M465.
These statistics can help determine how large some of the constants
should be and can tell how useful certain \\{built\_in} functions are.
They are written to the same files as tracing information.

\Y\P\D \37$\\{stat\_pr}\S\\{trace\_pr}$\par
\P\D \37$\\{stat\_pr\_ln}\S\\{trace\_pr\_ln}$\par
\P\D \37$\\{stat\_pr\_pool\_str}\S\\{trace\_pr\_pool\_str}$\par
\Y\P$\4\X465:Print usage statistics\X\S$\6
\&{begin} \37$\\{stat\_pr}(\.{\'You\'}\.{\'ve\ used\ \'},\39\\{num\_cites}:0)$;%
\6
\&{if} $(\\{num\_cites}=1)$ \1\&{then}\5
$\\{stat\_pr\_ln}(\.{\'\ entry,\'})$\6
\4\&{else} $\\{stat\_pr\_ln}(\.{\'\ entries,\'})$;\2\6
$\\{stat\_pr\_ln}(\.{\'\ \ \ \ \ \ \ \ \ \ \ \ \'},\39\\{wiz\_def\_ptr}:0,\39%
\.{\'\ wiz\_defined-function\ locations,\'})$;\5
$\\{stat\_pr\_ln}(\.{\'\ \ \ \ \ \ \ \ \ \ \ \ \'},\39\\{str\_ptr}:0,\39\.{\'\
strings\ with\ \'},\39\\{str\_start}[\\{str\_ptr}]:0,\39\.{\'\ characters,%
\'})$;\5
$\\{blt\_in\_ptr}\K0$;\5
$\\{total\_ex\_count}\K0$;\6
\&{while} $(\\{blt\_in\_ptr}<\\{num\_blt\_in\_fns})$ \1\&{do}\6
\&{begin} \37$\\{total\_ex\_count}\K\\{total\_ex\_count}+\\{execution\_count}[%
\\{blt\_in\_ptr}]$;\5
$\\{incr}(\\{blt\_in\_ptr})$;\6
\&{end};\2\6
$\\{stat\_pr\_ln}(\.{\'and\ the\ built\_in\ function-call\ counts,\ \'},\39%
\\{total\_ex\_count}:0,\39\.{\'\ in\ all,\ are:\'})$;\5
$\\{blt\_in\_ptr}\K0$;\6
\&{while} $(\\{blt\_in\_ptr}<\\{num\_blt\_in\_fns})$ \1\&{do}\6
\&{begin} \37$\\{stat\_pr\_pool\_str}(\\{hash\_text}[\\{blt\_in\_loc}[\\{blt%
\_in\_ptr}]])$;\5
$\\{stat\_pr\_ln}(\.{\'\ --\ \'},\39\\{execution\_count}[\\{blt\_in\_ptr}]:0)$;%
\5
$\\{incr}(\\{blt\_in\_ptr})$;\6
\&{end};\2\6
\&{end}\par
\U456.\fi

\M466.
Some implementations may wish to pass the \\{history} value to the
operating system so that it can be used to govern whether or not other
programs are started. Here we simply report the history to the user.

\Y\P$\4\X466:Print the job \\{history}\X\S$\6
\&{case} $(\\{history})$ \1\&{of}\6
\4\\{spotless}: \37\\{do\_nothing};\6
\4\\{warning\_message}: \37\&{begin} \37\&{if} $(\\{err\_count}=1)$ \1\&{then}\5
$\\{print\_ln}(\.{\'(There\ was\ 1\ warning)\'})$\6
\4\&{else} $\\{print\_ln}(\.{\'(There\ were\ \'},\39\\{err\_count}:0,\39\.{\'\
warnings)\'})$;\2\6
\&{end};\6
\4\\{error\_message}: \37\&{begin} \37\&{if} $(\\{err\_count}=1)$ \1\&{then}\5
$\\{print\_ln}(\.{\'(There\ was\ 1\ error\ message)\'})$\6
\4\&{else} $\\{print\_ln}(\.{\'(There\ were\ \'},\39\\{err\_count}:0,\39\.{\'\
error\ messages)\'})$;\2\6
\&{end};\6
\4\\{fatal\_message}: \37$\\{print\_ln}(\.{\'(That\ was\ a\ fatal\ error)\'})$;%
\6
\4\&{othercases} \37\&{begin} \37$\\{print}(\.{\'History\ is\ bunk\'})$;\5
\\{print\_confusion};\6
\&{end}\2\6
\&{endcases}\par
\U455.\fi

\N467.  System-dependent changes.
This section should be replaced, if necessary, by changes to the program
that are necessary to make \BibTeX\ work at a particular installation.
It is usually best to design your change file so that all changes to
previous sections preserve the section numbering; then everybody's version
will be consistent with the printed program. More extensive changes,
which introduce new sections, can be inserted here; then only the index
itself will get a new section number.



\fi

\N468.  Index.
Here is where you can find all uses of each identifier in the program,
with underlined entries pointing to where the identifier was defined.
If the identifier is only one letter long, however, you get to see only
the underlined entries. All references are to section numbers instead of
page numbers.

This index also lists a few error messages and other aspects of the
program that you might want to look up some day. For example, the
entry for ``system dependencies'' lists all sections that should
receive special attention from people who are installing \TeX\ in a
new operating environment. A list of various things that can't happen
appears under ``this can't happen''$\!$.
\fi


\inx
\:\\{a\_close}, \[39], 142, 151, 223, 455.
\:\\{a\_minus}, 331.
\:\\{a\_open\_in}, \[38], 106, 123, 127, 141.
\:\\{a\_open\_out}, \[38], 106.
\:{add a built-in function}, 331, 333, 334, 341, 342.
\:\\{add\_area}, \[61], 123, 127.
\:\\{add\_buf\_pool}, \[320], 364, 382, 426, 429, 430, 440, 450.
\:\\{add\_database\_cite}, 264, \[265], 272.
\:\\{add\_extension}, \[60], 106, 107, 123, 127.
\:\\{add\_out\_pool}, \[322], 454.
\:\\{add\_pool\_buf\_and\_push}, \[318], 329, 364, 382, 423, 429, 430, 440.
\:\\{ae\_width}, \[35], 453.
\:\\{all\_entries}, \[129], 131, 134, 145, 219, 227, 263, 264, 265, 267, 268,
269, 270, 272, 279, 283, 458.
\:\\{all\_lowers}, 337, \[365], 366, 372, 375, 376.
\:\\{all\_marker}, \[129], 134, 227, 268, 270, 272, 286, 458.
\:\\{all\_uppers}, 337, \[365], 366, 372, 375, 376.
\:\\{alpha}, \[31], 32, 88, 371, 398, 403, 411, 415, 431, 432, 452.
\:\\{alpha\_file}, \[36], 38, 39, 47, 51, 82, 104, 117, 124.
\:\\{alpha\_found}, \[344], 403, 405.
\:\\{already\_seen\_function\_print}, \[169].
\:\\{and\_found}, \[344], 384, 386.
\:\\{any\_value}, \[9], 227.
\:\\{append\_char}, \[53], 71, 318, 330, 351, 352, 353, 362, 379, 422, 434,
438, 440, 444.
\:\\{append\_ex\_buf\_char}, \[319], 320, 329, 414, 416, 417, 419.
\:\\{append\_ex\_buf\_char\_and\_check}, \[319], 402, 411, 415, 416, 417.
\:\\{append\_int\_char}, \[197], 198.
\:\\{area}, \[61].
\:\\{arg1}, \[301].
\:\\{arg2}, \[301].
\:{ASCII code}, 21.
\:\\{ASCII\_code}, \[22], 23, 24, 30, 31, 34, 40, 41, 42, 47, 48, 53, 83, 84,
85, 86, 87, 90, 161, 198, 216, 219, 230, 301, 344, 377, 422, 443.
\:\\{at\_bib\_command}, \[219], 221, 236, 239, 259, 261.
\:\\{at\_sign}, \[29], 218, 237, 238.
\:\\{aux\_bib\_data\_command}, 116, \[120].
\:\\{aux\_bib\_style\_command}, 116, \[126].
\:\\{aux\_citation\_command}, 116, \[132].
\:\\{aux\_command\_ilk}, \[64], 79, 116.
\:\\{aux\_done}, \[109], 110, 142.
\:\\{aux\_end\_err}, \[144], 145.
\:\\{aux\_end1\_err\_print}, \[144].
\:\\{aux\_end2\_err\_print}, \[144].
\:\\{aux\_err}, \[111], 122.
\:\\{aux\_err\_illegal\_another}, \[112], 120, 126.
\:\\{aux\_err\_illegal\_another\_print}, \[112].
\:\\{aux\_err\_no\_right\_brace}, \[113], 120, 126, 132, 139.
\:\\{aux\_err\_no\_right\_brace\_print}, \[113].
\:\\{aux\_err\_print}, \[111].
\:\\{aux\_err\_return}, \[111], 112, 113, 114, 115, 122, 127, 134, 135, 140,
141.
\:\\{aux\_err\_stuff\_after\_right\_brace}, \[114], 120, 126, 132, 139.
\:\\{aux\_err\_stuff\_after\_right\_brace\_print}, \[114].
\:\\{aux\_err\_white\_space\_in\_argument}, \[115], 120, 126, 132, 139.
\:\\{aux\_err\_white\_space\_in\_argument\_print}, \[115].
\:\\{aux\_extension\_ok}, \[139], 140.
\:\\{aux\_file}, \[104].
\:\\{aux\_file\_ilk}, \[64], 107, 140.
\:\\{aux\_found}, \[97], 100, 103.
\:\\{aux\_input\_command}, 116, \[139].
\:\\{aux\_list}, \[104], 105, 107.
\:\\{aux\_ln\_stack}, \[104].
\:\\{aux\_name\_length}, \[97], 98, 100, 103, 106, 107.
\:\\{aux\_not\_found}, \[97], 98, 99, \[100].
\:\\{aux\_number}, 104, \[105].
\:\\{aux\_ptr}, \[104], 106, 140, 141, 142.
\:\\{aux\_stack\_size}, \[14], 104, 105, 109, 140.
\:{auxiliary-file commands}, 109, 116.
\:\9{auxiliary-file commands}{\quad \.{\\\AT!input}}, 139.
\:\9{auxiliary-file commands}{\quad \.{\\bibdata}}, 120.
\:\9{auxiliary-file commands}{\quad \.{\\bibstyle}}, 126.
\:\9{auxiliary-file commands}{\quad \.{\\citation}}, 132.
\:\\{b\_}, 331.
\:\\{b\_add\_period}, \[331], 334.
\:\\{b\_call\_type}, \[331], 334.
\:\\{b\_change\_case}, \[331], 334.
\:\\{b\_chr\_to\_int}, \[331], 334.
\:\\{b\_cite}, \[331], 334.
\:\\{b\_concatenate}, \[331], 334.
\:\\{b\_default}, 182, \[331], 339, 363.
\:\\{b\_duplicate}, \[331], 334.
\:\\{b\_empty}, \[331], 334.
\:\\{b\_equals}, \[331], 334.
\:\\{b\_format\_name}, \[331], 334.
\:\\{b\_gat}, 331.
\:\\{b\_gets}, \[331], 334.
\:\\{b\_greater\_than}, \[331], 334.
\:\\{b\_if}, \[331], 334.
\:\\{b\_int\_to\_chr}, \[331], 334.
\:\\{b\_int\_to\_str}, \[331], 334.
\:\\{b\_less\_than}, \[331], 334.
\:\\{b\_minus}, \[331], 334.
\:\\{b\_missing}, \[331], 334.
\:\\{b\_newline}, \[331], 334.
\:\\{b\_num\_names}, \[331], 334.
\:\\{b\_plus}, \[331], 334.
\:\\{b\_pop}, \[331], 334.
\:\\{b\_preamble}, \[331], 334.
\:\\{b\_purify}, \[331], 334.
\:\\{b\_quote}, \[331], 334.
\:\\{b\_skip}, \[331], 334, 339.
\:\\{b\_stack}, \[331], 334.
\:\\{b\_substring}, \[331], 334.
\:\\{b\_swap}, \[331], 334.
\:\\{b\_text\_length}, \[331], 334.
\:\\{b\_text\_prefix}, \[331], 334.
\:\\{b\_top\_stack}, \[331], 334.
\:\\{b\_type}, \[331], 334.
\:\\{b\_warning}, \[331], 334.
\:\\{b\_while}, \[331], 334.
\:\\{b\_width}, \[331], 334.
\:\\{b\_write}, \[331], 334.
\:\\{backslash}, \[29], 370, 371, 372, 374, 397, 398, 415, 416, 418, 431, 432,
442, 445, 451, 452.
\:\\{bad}, 13, \[16], 17, 302.
\:\\{bad\_argument\_token}, \[177], 179, 204, 213.
\:\\{bad\_conversion}, \[365], 366, 372, 375, 376.
\:\\{bad\_cross\_reference\_print}, \[280], 281, 282.
\:\\{banner}, \[1], 10.
\:\\{bbl\_file}, \[104], 106, 151, 321.
\:\\{bbl\_line\_num}, \[147], 151, 321.
\:\&{begin}, 4.
\:\\{bf\_ptr}, \[56], \[62], \[63], \[95].
\:\\{bib\_brace\_level}, \[247], 253, 254, 255, 256, 257.
\:\\{bib\_cmd\_confusion}, 239, \[240], 262.
\:\\{bib\_command\_ilk}, \[64], 79, 238.
\:\\{bib\_equals\_sign\_expected\_err}, \[231], 246, 275.
\:\\{bib\_equals\_sign\_print}, \[231].
\:\\{bib\_err}, \[221], 229, 230, 231, 232, 233, 235, 242, 246, 268.
\:\\{bib\_err\_print}, \[221].
\:\\{bib\_field\_too\_long\_err}, \[233], 251.
\:\\{bib\_field\_too\_long\_print}, \[233].
\:\\{bib\_file}, \[117].
\:\\{bib\_file\_ilk}, \[64], 123.
\:\\{bib\_id\_print}, \[235].
\:\\{bib\_identifier\_scan\_check}, \[235], 238, 244, 259, 275.
\:\\{bib\_line\_num}, \[219], 220, 223, 228, 237, 252, 455.
\:\\{bib\_list}, \[117], 118, 119, 123.
\:\\{bib\_ln\_num\_print}, \[220], 221, 222.
\:\\{bib\_number}, 117, \[118], 219, 337.
\:\\{bib\_one\_of\_two\_expected\_err}, \[230], 242, 244, 266, 274.
\:\\{bib\_one\_of\_two\_print}, \[230].
\:\\{bib\_ptr}, \[117], 119, 123, 145, 223, 457.
\:\\{bib\_seen}, \[117], 119, 120, 145.
\:\\{bib\_unbalanced\_braces\_err}, \[232], 254, 256.
\:\\{bib\_unbalanced\_braces\_print}, \[232].
\:\\{bib\_warn}, \[222].
\:\\{bib\_warn\_newline}, \[222], 234, 263, 273.
\:\\{bib\_warn\_print}, \[222].
\:{biblical procreation}, 331.
\:\\{BibTEX}, \[10].
\:\.{BibTeX capacity exceeded}, 44.
\:\9{BibTeX capacity exceeded}{\quad buffer size}, 46, 47, 197, 319, 320, 414,
416, 417.
\:\9{BibTeX capacity exceeded}{\quad file name size}, 58, 59, 60, 61.
\:\9{BibTeX capacity exceeded}{\quad hash size}, 71.
\:\9{BibTeX capacity exceeded}{\quad literal-stack size}, 307.
\:\9{BibTeX capacity exceeded}{\quad number of \.{.aux} files}, 140.
\:\9{BibTeX capacity exceeded}{\quad number of \.{.bib} files}, 123.
\:\9{BibTeX capacity exceeded}{\quad number of cite keys}, 138.
\:\9{BibTeX capacity exceeded}{\quad number of string global-variables}, 216.
\:\9{BibTeX capacity exceeded}{\quad number of strings}, 54.
\:\9{BibTeX capacity exceeded}{\quad output buffer size}, 322.
\:\9{BibTeX capacity exceeded}{\quad pool size}, 53.
\:\9{BibTeX capacity exceeded}{\quad single function space}, 188.
\:\9{BibTeX capacity exceeded}{\quad total number of fields}, 226.
\:\9{BibTeX capacity exceeded}{\quad total number of integer entry-variables},
287.
\:\9{BibTeX capacity exceeded}{\quad total number of string entry-variables},
288.
\:\9{BibTeX capacity exceeded}{\quad wizard-defined function space}, 200.
\:\9{BibTeX documentation}{\BibTeX\ documentation}, \[1].
\:\9{BibTeX}{\BibTeX}, \[1].
\:\\{blt\_in\_loc}, \[331], 335, 465.
\:\\{blt\_in\_num}, \[335].
\:\\{blt\_in\_ptr}, \[331], 465.
\:\\{blt\_in\_range}, 331, \[332], 335.
\:\\{boolean}, 38, 47, 56, 57, 65, 68, 83, 84, 85, 86, 87, 88, 92, 93, 94, 101,
117, 124, 129, 139, 152, 161, 163, 177, 219, 228, 249, 250, 252, 253, 278, 290,
301, 322, 344, 365, 397, 418.
\:{bottom up}, 12.
\:\\{brace\_level}, \[290], 367, 369, 370, 371, 384, 385, 387, 390, 418, 431,
432, 451, 452.
\:\\{brace\_lvl\_one\_letters\_complaint}, 405, \[406].
\:\\{braces\_unbalanced\_complaint}, 367, \[368], 369, 402.
\:\\{break\_pt\_found}, 322, 323, 324.
\:\\{break\_ptr}, \[322], 323.
\:\\{bst\_cant\_mess\_with\_entries\_print}, \[295], 327, 328, 329, 354, 363,
378, 424, 447.
\:\\{bst\_command\_ilk}, \[64], 79, 154.
\:\\{bst\_done}, \[146], 149, 151.
\:\\{bst\_entry\_command}, 155, \[170].
\:\\{bst\_err}, \[149], 153, 154, 166, 167, 168, 169, 170, 177, 178, 203, 205,
207, 208, 209, 211, 212, 214.
\:\\{bst\_err\_print\_and\_look\_for\_blank\_line}, \[149].
\:\\{bst\_err\_print\_and\_look\_for\_blank\_line\_return}, \[149], 169, 177.
\:\\{bst\_ex\_warn}, \[293], 295, 309, 317, 345, 354, 366, 377, 380, 383, 391,
406, 422, 424.
\:\\{bst\_ex\_warn\_print}, \[293], 312, 388, 389.
\:\\{bst\_execute\_command}, 155, \[178].
\:\\{bst\_file}, \[124], 127, 149, 151, 152.
\:\\{bst\_file\_ilk}, \[64], 127.
\:\\{bst\_fn\_ilk}, \[64], 156, 172, 174, 176, 177, 182, 192, 194, 199, 202,
216, 238, 275, 335, 340.
\:\\{bst\_function\_command}, 155, \[180].
\:\\{bst\_get\_and\_check\_left\_brace}, \[167], 171, 173, 175, 178, 180, 181,
201, 203, 206, 208, 212, 215.
\:\\{bst\_get\_and\_check\_right\_brace}, \[168], 178, 181, 203, 206, 208, 212.
\:\\{bst\_id\_print}, \[166].
\:\\{bst\_identifier\_scan}, \[166], 171, 173, 175, 178, 181, 201, 203, 206,
212, 215.
\:\\{bst\_integers\_command}, 155, \[201].
\:\\{bst\_iterate\_command}, 155, \[203].
\:\\{bst\_left\_brace\_print}, \[167].
\:\\{bst\_line\_num}, \[147], 148, 149, 151, 152.
\:\\{bst\_ln\_num\_print}, \[148], 149, 150, 183, 293.
\:\\{bst\_macro\_command}, 155, \[205].
\:\\{bst\_mild\_ex\_warn}, \[294], 368.
\:\\{bst\_mild\_ex\_warn\_print}, \[294], 356.
\:\\{bst\_read\_command}, 155, \[211].
\:\\{bst\_reverse\_command}, 155, \[212].
\:\\{bst\_right\_brace\_print}, \[168].
\:\\{bst\_seen}, \[124], 125, 126, 145.
\:\\{bst\_sort\_command}, 155, \[214].
\:\\{bst\_str}, \[124], 125, 127, 128, 145, 151, 457.
\:\\{bst\_string\_size\_exceeded}, \[356], 357, 359.
\:\\{bst\_strings\_command}, 155, \[215].
\:\\{bst\_warn}, \[150], 170, 294.
\:\\{bst\_warn\_print}, \[150].
\:\\{bst\_1print\_string\_size\_exceeded}, \[356].
\:\\{bst\_2print\_string\_size\_exceeded}, \[356].
\:\\{buf}, \[56], \[62], \[63], \[68], 69, 70, 71.
\:\\{buf\_pointer}, 41, \[42], 43, 56, 62, 63, 68, 80, 82, 95, 187, 198, 290,
322, 344, 418.
\:\\{buf\_ptr1}, \[80], 81, 82, 83, 84, 85, 86, 87, 88, 90, 92, 93, 116, 123,
127, 133, 134, 135, 136, 140, 154, 172, 174, 176, 177, 182, 190, 191, 192, 199,
202, 207, 209, 216, 238, 245, 258, 259, 267, 269, 272, 273, 275.
\:\\{buf\_ptr2}, \[80], 81, 82, 83, 84, 85, 86, 87, 88, 90, 92, 93, 94, 95,
116, 120, 126, 132, 133, 139, 140, 149, 151, 152, 167, 168, 171, 173, 175, 187,
190, 191, 192, 194, 201, 209, 211, 215, 223, 228, 237, 238, 242, 244, 246, 249,
252, 253, 254, 255, 256, 257, 258, 266, 267, 274, 275.
\:\\{buf\_size}, \[14], 17, 42, 46, 47, 197, 233, 251, 319, 320, 322, 414, 416,
417.
\:\\{buf\_type}, 41, \[42], 43, 56, 62, 63, 68, 198, 290.
\:\\{buffer}, \[41], 42, 47, 68, 77, 80, 81, 82, 83, 95, 107, 116, 123, 127,
133, 134, 135, 136, 140, 154, 172, 174, 176, 177, 182, 190, 191, 192, 199, 202,
207, 209, 211, 216, 238, 245, 258, 259, 267, 269, 272, 273, 275.
\:\\{buffer\_overflow}, \[46], 47, 197, 319, 320, 414, 416, 417.
\:\\{build\_in}, 334, \[335].
\:\\{built\_in}, 43, 50, \[156], 158, 159, 177, 178, 179, 182, 203, 204, 212,
213, 325, 331, 332, 333, 334, 335, 337, 341, 342, 343, 345, 346, 347, 348, 349,
350, 354, 360, 363, 364, 377, 378, 379, 380, 382, 421, 422, 423, 424, 425, 426,
428, 429, 430, 434, 435, 436, 437, 439, 441, 443, 446, 447, 448, 449, 450, 454,
465.
\:{bunk, history}, 466.
\:{case mismatch}, 132.
\:{case mismatch errors}, 135, 273.
\:\\{case\_conversion\_confusion}, 372, \[373], 375, 376.
\:\\{case\_difference}, \[62], 63.
\:{Casey Stengel would be proud}, 401.
\:\\{char}, 23, 37, 73, 97.
\:\\{char\_ptr}, \[301].
\:\\{char\_value}, \[91], 92, 93.
\:\\{char\_width}, \[34], 35, 450, 451, 452, 453.
\:{character set dependencies}, 23, 25, 26, 27, 32, 33, 35.
\:\\{char1}, \[83], \[84], \[85], \[86], \[87], \[90], \[230], \[301].
\:\\{char2}, \[85], \[86], \[87], \[90], \[230], \[301].
\:\\{char3}, \[87], \[90].
\:\\{check\_brace\_level}, \[369], 370, 384, 451.
\:\\{check\_cite\_overflow}, 136, \[138], 265.
\:\\{check\_cmnd\_line}, 100, \[101].
\:\\{check\_command\_execution}, 296, 297, 298, \[317].
\:\\{check\_field\_overflow}, 225, \[226], 265.
\:\\{check\_for\_already\_seen\_function}, \[169], 172, 174, 176, 182, 202, 216.
\:\\{check\_for\_and\_compress\_bib\_white\_space}, \[252], 253, 256, 257.
\:{child entry}, 277.
\:\\{chr}, 23, 24, 27, 28, 58, 60, 61.
\:\\{citation\_seen}, \[129], 131, 132, 145.
\:\\{cite\_already\_set}, \[236], 272.
\:\\{cite\_found}, \[129].
\:\\{cite\_hash\_found}, \[219], 278, 279, 285.
\:\\{cite\_ilk}, \[64], 135, 136, 264, 269, 272, 273, 278.
\:\\{cite\_info}, \[219], 227, 264, 270, 279, 283, 286, 289, 290.
\:\\{cite\_key\_disappeared\_confusion}, 270, \[271], 285.
\:\\{cite\_list}, 14, 64, \[129], 130, 131, 133, 135, 136, 138, 219, 224, 227,
263, 264, 265, 267, 268, 269, 272, 273, 278, 279, 281, 282, 283, 284, 285, 286,
297, 298, 302, 306, 378, 458.
\:\\{cite\_loc}, \[129], 136, 138, 264, 265, 269, 272, 277, 278, 279, 285.
\:\\{cite\_number}, 129, \[130], 138, 161, 219, 265, 290, 300, 301, 303.
\:\\{cite\_parent\_ptr}, \[161], 277, 279, 282.
\:\\{cite\_ptr}, \[129], 131, 134, 136, 145, 227, 264, 272, 276, 277, 279, 283,
285, 286, 289, 297, 298, 327, 328, 329, 355, 357, 363, 447, 458, 459, 460, 461,
462.
\:\\{cite\_str}, \[278].
\:\\{cite\_xptr}, \[161], 283, 285.
\:{clich\'e-\`a-trois}, 455.
\:\\{close}, 39.
\:\\{close\_up\_shop}, \[10], 44, 45.
\:\\{cmd\_num}, \[112].
\:\\{cmd\_str\_ptr}, \[290], 308, 309, 316, 317, 351, 352, 353, 359, 362, 379,
438, 439, 444.
\:\\{colon}, \[29], 364, 365, 371, 376.
\:\\{comma}, \[29], 33, 120, 132, 218, 259, 266, 274, 387, 388, 389, 396, 401.
\:\\{command\_ilk}, 64.
\:\\{command\_num}, \[78], 116, 154, 155, 238, 239, 259, 262.
\:\\{comma1}, \[344], 389, 395.
\:\\{comma2}, \[344], 389, 395.
\:\\{comment}, \[29], 33, 152, 166, 183, 190, 191, 192, 199.
\:{commented-out code}, 184, 245, 273.
\:\\{compare\_return}, \[301].
\:\\{compress\_bib\_white}, \[252].
\:\\{concat\_char}, \[29], 218, 242, 243, 249, 259.
\:\\{confusion}, \[45], 51, 107, 112, 116, 127, 137, 155, 157, 165, 194, 238,
240, 258, 268, 271, 301, 309, 310, 317, 341, 373, 395, 399.
\:{control sequence}, 372.
\:\\{control\_seq\_ilk}, \[64], 339, 371, 398, 432, 452.
\:\\{control\_seq\_loc}, \[344], 371, 372, 398, 399, 432, 433, 452, 453.
\:\\{conversion\_type}, \[365], 366, 370, 372, 375, 376.
\:\\{copy\_char}, \[251], 252, 256, 257, 258, 260.
\:\\{copy\_ptr}, \[187], 200.
\:{cross references}, 277.
\:\.{crossref}, 340.
\:\\{crossref\_num}, \[161], 263, 277, 279, 340.
\:\\{cur\_aux\_file}, \[104], 106, 110, 141, 142.
\:\\{cur\_aux\_line}, \[104], 107, 110, 111, 141.
\:\\{cur\_aux\_str}, \[104], 107, 108, 140, 141.
\:\\{cur\_bib\_file}, \[117], 123, 223, 228, 237, 252.
\:\\{cur\_bib\_str}, \[117], 121, 123, 457.
\:\\{cur\_cite\_str}, \[129], 136, 280, 283, 293, 294, 297, 298, 378, 458.
\:\\{cur\_macro\_loc}, \[219], 245, 259, 262.
\:\\{cur\_token}, \[344], 407, 408, 409, 410, 413, 414, 415, 417.
\:{database-file commands}, 239.
\:\9{database-file commands}{\quad \.{comment}}, 241.
\:\9{database-file commands}{\quad \.{preamble}}, 242.
\:\9{database-file commands}{\quad \.{string}}, 243.
\:\&{debug}, \[4], \[11].
\:{debugging}, 4.
\:\\{decr}, \[9], 47, 55, 61, 71, 140, 141, 142, 198, 253, 255, 257, 261, 298,
306, 309, 321, 323, 352, 361, 367, 371, 374, 385, 388, 390, 396, 398, 400, 401,
403, 404, 411, 416, 418, 419, 431, 432, 442, 444, 445, 452.
\:\\{decr\_brace\_level}, \[367], 370, 384, 451.
\:\.{default.type}, 339.
\:\\{do\_insert}, \[68], 77, 107, 123, 127, 133, 136, 140, 172, 174, 176, 182,
190, 191, 194, 202, 207, 209, 216, 245, 261, 264, 267, 269, 272.
\:\\{do\_nothing}, \[9], 68, 102, 166, 183, 192, 199, 235, 266, 363, 372, 375,
376, 419, 433, 435, 466.
\:{documentation}, 1.
\:\\{dont\_insert}, \[68], 116, 135, 154, 177, 192, 199, 238, 259, 267, 270,
273, 275, 278, 371, 398, 432, 452.
\:\\{double\_letter}, \[344], 403, 405, 407, 408, 409, 410, 412, 413, 417.
\:\\{double\_quote}, \[29], 33, 189, 191, 205, 208, 209, 218, 219, 250, 434.
\:\\{dum\_ptr}, \[307].
\:\\{dummy\_loc}, \[65], 135, 273.
\:\\{eat\_bib\_print}, \[229], 252.
\:\\{eat\_bib\_white\_and\_eof\_check}, \[229], 236, 238, 242, 243, 244, 246,
249, 250, 254, 255, 266, 274, 275.
\:\\{eat\_bib\_white\_space}, \[228], 229, 252.
\:\\{eat\_bst\_print}, \[153].
\:\\{eat\_bst\_white\_and\_eof\_check}, \[153], 170, 171, 173, 175, 178, 180,
181, 187, 201, 203, 205, 206, 208, 212, 215.
\:\\{eat\_bst\_white\_space}, 151, \[152], 153.
\:\&{ecart}, \[4].
\:\&{else}, 5.
\:\\{empty}, \[9], 64, 67, 68, 161, 219, 227, 268, 279, 283, 363, 447, 459.
\:\&{end}, 4, 5.
\:\\{end\_of\_def}, \[160], 188, 200, 326, 463.
\:\\{end\_of\_group}, \[344], 403.
\:\\{end\_of\_num}, \[187], 194.
\:\\{end\_of\_string}, \[216], 288, 301, 329, 357, 460.
\:\\{end\_offset}, \[302], 305.
\:\\{end\_ptr}, \[322], 323, 324.
\:\\{end\_while}, \[343], 449.
\:\&{endcases}, \[5].
\:\\{enough\_chars}, \[418].
\:\\{enough\_text\_chars}, 417, \[418], 419.
\:\\{ent\_chr\_ptr}, \[290], 329, 357, 460.
\:\\{ent\_str\_size}, \[14], 17, 161, 290, 301, 340, 357.
\:{entire database inclusion}, 132.
\:\.{entry string size exceeded}, 357.
\:\.{entry.max\$}, 340.
\:\\{entry\_cite\_ptr}, \[129], 263, 267, 268, 269, 270, 272, 273.
\:\\{entry\_exists}, \[219], 227, 268, 270, 272, 286.
\:\\{entry\_ints}, \[161], 287, 328, 355, 461.
\:\\{entry\_seen}, \[163], 164, 170, 211.
\:\\{entry\_strs}, \[161], 176, 288, 301, 329, 357, 460.
\:\\{entry\_type\_loc}, \[219], 238, 273.
\:\\{eof}, 37, 47, 223.
\:\\{eoln}, 47, 100.
\:\\{equals\_sign}, \[29], 33, 218, 231, 243, 244, 246, 275.
\:\\{err\_count}, 18, \[19], 20, 466.
\:\\{error\_message}, \[18], 19, 20, 293, 294, 466.
\:\\{erstat}, \[38].
\:\\{ex\_buf}, 133, 194, 247, 267, 270, 278, \[290], 318, 319, 320, 344, 370,
371, 372, 374, 375, 376, 384, 385, 386, 387, 388, 390, 393, 394, 411, 418, 419,
423, 431, 432, 433, 451, 452, 453.
\:\\{ex\_buf\_length}, \[290], 318, 320, 329, 364, 370, 371, 374, 382, 383,
384, 385, 386, 402, 414, 417, 423, 426, 427, 429, 430, 431, 432, 438, 440, 450,
451, 452.
\:\\{ex\_buf\_ptr}, 247, 270, 278, \[290], 318, 319, 320, 329, 370, 371, 372,
374, 375, 376, 383, 384, 385, 386, 387, 388, 390, 402, 411, 416, 418, 419, 427,
431, 432, 451, 452.
\:\\{ex\_buf\_xptr}, 247, \[344], 371, 372, 374, 375, 383, 387, 388, 389, 390,
391, 392, 393, 394, 411, 418, 431, 432, 433, 452, 453.
\:\\{ex\_buf\_yptr}, \[344], 418, 432, 433.
\:\\{ex\_buf1}, \[133].
\:\\{ex\_buf2}, \[194].
\:\\{ex\_buf3}, \[267].
\:\\{ex\_buf4}, \[270].
\:\\{ex\_buf4\_ptr}, \[270].
\:\\{ex\_buf5}, \[278].
\:\\{ex\_buf5\_ptr}, \[278].
\:\\{ex\_fn\_loc}, \[325], 326, 327, 328, 329, 330, 341.
\:\\{exclamation\_mark}, \[29], 360, 361.
\:\\{execute\_fn}, 296, 297, 298, \[325], 326, 342, 344, 363, 421, 449.
\:\\{execution\_count}, \[331], 335, 341, 465.
\:\\{exit}, \[6], 9, 56, 57, 111, 116, 120, 126, \[132], 139, 149, 152, 154,
169, 170, 177, 178, 180, \[187], 201, 203, 205, 211, 212, 214, 215, 228, 229,
230, 231, 232, 233, \[236], 249, 250, 252, \[253], 301, \[321], 380, 397, 401,
437, 443.
\:\\{exit\_program}, \[10], 13.
\:\\{ext}, \[60].
\:\\{extern}, 38.
\:\\{extra\_buf}, \[264].
\:\|{f}, \[38], \[39], \[47], \[51], \[82].
\:\\{false}, 38, 47, 56, 57, 68, 83, 84, 85, 86, 87, 88, 92, 93, 94, 100, 119,
125, 131, 140, 152, 164, 177, 227, 228, 236, 238, 249, 250, 252, 253, 259, 264,
267, 272, 275, 278, 296, 301, 322, 323, 324, 370, 376, 384, 390, 391, 394, 397,
403, 405, 407, 408, 409, 410, 412, 418, 462.
\:{fat lady}, 455.
\:\\{fatal\_message}, \[18], 19, 466.
\:{fetish}, 138, 226.
\:\\{field}, \[156], 158, 159, 162, 170, 171, 172, 275, 325, 331, 340.
\:\\{field\_end}, \[247], 249, 251, 253, 260, 261, 264.
\:\\{field\_end\_ptr}, \[161], 277, 285, 462.
\:\\{field\_info}, \[161], 172, 224, 225, 263, 277, 279, 281, 285, 327, 462.
\:\\{field\_loc}, \[160], 161.
\:\\{field\_name\_loc}, \[219], 263, 275.
\:\\{field\_parent\_ptr}, \[161], 277, 279.
\:\\{field\_ptr}, \[161], 225, 263, 277, 279, 281, 285, 327, 462.
\:\\{field\_start}, \[247], 261, 264.
\:\\{field\_val\_loc}, \[219], 261, 262, 263.
\:\\{field\_vl\_str}, \[247], 249, 251, 252, 253, 258, 259, 260, 261, 264.
\:\\{figure\_out\_the\_formatted\_name}, 382, \[420].
\:\\{file\_area\_ilk}, \[64], 75.
\:\\{file\_ext\_ilk}, \[64], 75.
\:\\{file\_name}, \[58].
\:\\{file\_name\_size}, \[15], 37, 58, 59, 60, 61, 97, 100, 103, 141.
\:\\{file\_nm\_size\_overflow}, 58, \[59], 60, 61.
\:\\{find\_cite\_locs\_for\_this\_cite\_key}, 270, 277, \[278], 279, 285.
\:\\{first\_end}, \[344], 395, 396, 407.
\:\\{first\_start}, \[344], 395, 407.
\:\\{first\_text\_char}, \[23], 28.
\:\\{first\_time\_entry}, \[236], 268.
\:\\{flush\_string}, \[55], 309.
\:\\{fn\_class}, \[160], 161, 190, 191, 209, 261.
\:\\{fn\_def\_loc}, \[187].
\:\\{fn\_hash\_loc}, \[187], 200, \[335].
\:\\{fn\_info}, \[161], 172, 174, 176, 190, 191, 200, 202, 216, 263, 325, 326,
327, 328, 329, 330, 335, 340, 341, 355, 357, 358, 359.
\:\\{fn\_loc}, \[158], \[159], \[161], 172, 174, 176, 177, 192, 193, 199, 202,
216, 296, 297, 298.
\:\\{fn\_type}, 158, 159, \[161], 172, 174, 176, 177, 182, 190, 191, 194, 202,
209, 216, 238, 261, 275, 325, 335, 339, 340, 354.
\:{for a good time, try comment-out code}, 184.
\:{for loops}, 7, 69, 71.
\:\\{get}, 37, 47, 100.
\:\\{get\_aux\_command\_and\_process}, 110, \[116].
\:\\{get\_bib\_command\_or\_entry\_and\_process}, 223, \[236].
\:\\{get\_bst\_command\_and\_process}, 151, \[154].
\:\\{get\_the\_top\_level\_aux\_file\_name}, 13, \[100].
\:\\{glb\_str\_end}, \[161], 162, 330, 359.
\:\\{glb\_str\_ptr}, \[161], 162, 330, 359.
\:\\{glob\_chr\_ptr}, \[290], 330, 359.
\:\\{glob\_str\_size}, \[14], 17, 161, 290, 340, 359.
\:\.{global string size exceeded}, 359.
\:\.{global.max\$}, 340.
\:\\{global\_strs}, \[161], 216, 330, 359.
\:{grade inflation}, 331.
\:\&{gubed}, \[4].
\:{gymnastics}, 12, 143, 210, 217, 248, 342.
\:\|{h}, \[68].
\:{ham and eggs}, 261.
\:\\{hash\_}, 68.
\:\\{hash\_base}, \[64], 65, 67, 68, 160, 219.
\:\\{hash\_cite\_confusion}, 136, \[137], 264, 272, 279, 285.
\:\\{hash\_found}, \[65], 68, 70, 107, 116, 123, 127, 133, 135, 136, 140, 154,
169, 177, 190, 192, 194, 199, 207, 219, 238, 245, 259, 264, 267, 268, 269, 270,
272, 273, 275, 278, 371, 398, 432, 452.
\:\\{hash\_ilk}, 64, \[65], 67, 70, 71.
\:\\{hash\_is\_full}, \[64], 71.
\:\\{hash\_loc}, \[64], 65, 66, 68, 76, 129, 158, 159, 160, 161, 169, 187, 219,
325, 331, 335, 344.
\:\\{hash\_max}, \[64], 65, 67, 160, 219.
\:\\{hash\_next}, 64, \[65], 67, 68, 71.
\:\\{hash\_pointer}, \[64], 65.
\:\\{hash\_prime}, \[15], 17, 68, 69.
\:\\{hash\_ptr2}, \[160], 161, 187, 219.
\:\\{hash\_size}, 14, \[15], 17, 64, 69, 71.
\:\\{hash\_text}, 64, \[65], 67, 70, 71, 75, 107, 123, 127, 136, 138, 140, 169,
182, 194, 207, 209, 245, 261, 262, 263, 265, 269, 277, 297, 298, 307, 311, 313,
325, 327, 339, 447, 459, 463, 465.
\:\\{hash\_used}, 64, \[65], 67, 71.
\:\\{history}, 18, \[19], 20, 466.
\:\\{hyphen}, \[29], 32.
\:\|{i}, \[51], \[56], \[62], \[63], \[77], \[82].
\:\\{id\_class}, \[30], 33, 90.
\:\\{id\_null}, \[89], 90, 166, 235.
\:\\{id\_scanning\_confusion}, \[165], 166, 235.
\:\\{id\_type}, 30, \[31].
\:\\{ilk}, 64, 65, \[68], 70, 71, \[77].
\:\\{ilk\_info}, 64, \[65], 67, 78, 79, 116, 135, 136, 154, 161, 207, 209, 238,
245, 260, 262, 264, 265, 267, 269, 272, 277, 279, 285, 339, 372, 399, 433, 453.
\:\\{illegal}, \[31], 32.
\:\\{illegal\_id\_char}, \[31], 33, 90.
\:\\{illegl\_literal\_confusion}, \[310], 311, 312, 313.
\:\\{impl\_fn\_loc}, \[187], 194.
\:\\{impl\_fn\_num}, 194, \[195], 196.
\:{important note}, 75, 79, 334, 339, 340.
\:\\{incr}, \[9], 18, 47, 53, 54, 55, 56, 57, 58, 60, 61, 69, 71, 82, 83, 84,
85, 86, 87, 88, 90, 92, 93, 94, 95, 98, 99, 100, 107, 110, 120, 123, 126, 132,
133, 136, 139, 140, 141, 149, 152, 162, 167, 168, 171, 172, 173, 174, 175, 176,
187, 188, 190, 191, 192, 194, 197, 198, 200, 201, 209, 211, 215, 216, 223, 225,
227, 228, 237, 238, 242, 244, 246, 249, 251, 252, 253, 254, 255, 256, 257, 258,
260, 262, 264, 265, 266, 267, 270, 274, 275, 277, 278, 279, 283, 285, 286, 287,
288, 289, 297, 301, 306, 307, 308, 318, 319, 320, 321, 322, 323, 324, 326, 330,
340, 341, 351, 352, 353, 357, 359, 362, 370, 371, 374, 379, 381, 383, 384, 385,
388, 389, 390, 391, 392, 393, 394, 396, 397, 398, 400, 402, 403, 404, 405, 411,
412, 413, 414, 415, 416, 417, 418, 419, 427, 429, 431, 432, 433, 438, 440, 442,
444, 445, 451, 452, 457, 458, 460, 461, 462, 463, 464, 465.
\:\\{init\_command\_execution}, 296, 297, 298, \[316].
\:\\{initialize}, 10, 12, \[13], 336.
\:\\{innocent\_bystander}, \[300].
\:\\{input\_ln}, 41, \[47], 80, 110, 149, 152, 228, 237, 252.
\:\\{insert\_fn\_loc}, \[188], 190, 191, 193, 194, 199, 200.
\:\\{insert\_it}, \[68].
\:\\{insert\_ptr}, \[303], 304.
\:\\{int}, \[198].
\:\\{int\_begin}, \[198].
\:\\{int\_buf}, 197, \[198].
\:\\{int\_end}, \[198].
\:\\{int\_ent\_loc}, \[160], 161.
\:\\{int\_ent\_ptr}, \[161], 287, 461.
\:\\{int\_entry\_var}, 14, \[156], 158, 159, 160, 161, 162, 170, 173, 174, 287,
325, 328, 354.
\:\\{int\_global\_var}, \[156], 158, 159, 201, 202, 325, 331, 340, 354.
\:\\{int\_literal}, 29, \[156], 158, 159, 189, 190, 325.
\:\\{int\_ptr}, 197, \[198].
\:\\{int\_tmp\_val}, \[198].
\:\\{int\_to\_ASCII}, 194, 197, \[198], 423.
\:\\{int\_xptr}, \[198].
\:\\{integer}, 16, 19, 34, 38, 43, 65, 78, 91, 104, 112, 147, 161, 195, 198,
219, 226, 247, 290, 307, 309, 311, 312, 313, 314, 331, 343, 344.
\:\\{integer\_ilk}, \[64], 156, 190.
\:\\{invalid\_code}, \[26], 28, 32, 216.
\:\|{j}, \[56], \[68].
\:\\{jr\_end}, \[344], 395, 410.
\:\|{k}, \[66], \[68].
\:{kludge}, 43, 51, 133, 194, 247, 264, 267, 270, 278.
\:\|{l}, \[68].
\:\\{last}, \[41], 47, 80, 83, 84, 85, 86, 87, 88, 90, 92, 93, 94, 95, 120,
126, 132, 139, 149, 151, 190, 191, 211, 223, 252.
\:\\{last\_check\_for\_aux\_errors}, 110, \[145].
\:\\{last\_cite}, \[138].
\:\\{last\_end}, \[344], 395, 396, 401, 409, 410.
\:\\{last\_fn\_class}, \[156], 160.
\:\\{last\_ilk}, \[64].
\:\\{last\_lex}, \[31].
\:\\{last\_lit\_type}, \[291].
\:\\{last\_text\_char}, \[23], 28.
\:\\{last\_token}, \[344], 407, 408, 409, 410, 413, 417.
\:\9{LaTeX}{\LaTeX}, 1, 10, 132.
\:\\{lc\_cite\_ilk}, \[64], 133, 264, 267, 270, 278.
\:\\{lc\_cite\_loc}, \[129], 133, 135, 136, 264, 265, 267, 268, 269, 272, 277,
278, 279, 285.
\:\\{lc\_xcite\_loc}, \[129], 268, 270.
\:\\{left}, \[303], 305, 306.
\:\\{left\_brace}, \[29], 33, 116, 126, 139, 167, 171, 173, 175, 178, 181, 189,
194, 201, 203, 206, 208, 212, 215, 238, 242, 244, 250, 254, 255, 256, 257, 266,
370, 371, 384, 385, 387, 390, 397, 398, 400, 402, 403, 404, 411, 412, 415, 416,
418, 431, 432, 442, 445, 451, 452.
\:\\{left\_end}, 302, \[303], 304, 305, 306.
\:\\{left\_paren}, \[29], 33, 238, 242, 244, 266.
\:\\{legal\_id\_char}, \[31], 33, 90.
\:\\{len}, \[56], \[62], \[63], \[77], \[335].
\:\\{length}, \[52], 56, 57, 58, 60, 61, 103, 140, 270, 278, 351, 352, 353,
360, 362, 366, 377, 379, 437.
\:\\{less\_than}, \[301], 304, 305, 306.
\:\\{lex\_class}, \[30], 32, 47, 84, 86, 88, 90, 92, 93, 94, 95, 120, 126, 132,
139, 190, 191, 252, 260, 321, 323, 324, 370, 371, 374, 376, 381, 384, 386, 387,
388, 396, 398, 403, 411, 415, 417, 431, 432, 452.
\:\\{lex\_type}, 30, \[31].
\:\\{lit\_stack}, \[290], 291, 307, 308, 309, 352.
\:\\{lit\_stk\_loc}, 290, \[291], 307.
\:\\{lit\_stk\_ptr}, \[290], 307, 308, 309, 315, 316, 317, 351, 352, 353, 438.
\:\\{lit\_stk\_size}, \[14], 291, 307.
\:\\{lit\_stk\_type}, \[290], 291, 307, 309.
\:{literal literal}, 450.
\:\\{literal\_loc}, \[161], 190, 191.
\:\\{log\_file}, 3, 10, 50, 51, 75, 79, 81, 82, \[104], 106, 334, 339, 340, 455.
\:\\{long\_name}, \[419].
\:\\{long\_token}, \[417].
\:\\{longest\_pds}, \[73], 75, 77, 79, 334, 335, 339, 340.
\:\&{loop}, 6, \[9].
\:\\{loop\_exit}, \[6], 47, \[236], 253, 257, 274, 321, 360, 361, 415, 416, 420.
\:\\{loop1\_exit}, \[6], 322, 324, 382, 388.
\:\\{loop2\_exit}, \[6], 322, 324, \[382], 396.
\:\\{lower\_case}, \[62], 133, 154, 172, 174, 176, 177, 182, 192, 199, 202,
207, 216, 238, 245, 259, 264, 267, 270, 275, 278, 372, 375, 376.
\:\\{macro\_def\_loc}, \[161], 209.
\:\\{macro\_ilk}, \[64], 207, 245, 259.
\:\\{macro\_loc}, 219.
\:\\{macro\_name\_loc}, \[161], 207, 209, 259, 260.
\:\\{macro\_name\_warning}, \[234], 245, 259.
\:\\{macro\_warn\_print}, \[234].
\:\\{make\_string}, \[54], 71, 318, 330, 351, 352, 353, 362, 379, 422, 434,
438, 440, 444.
\:\\{mark\_error}, \[18], 95, 111, 122, 144, 149, 183, 221, 281, 293.
\:\\{mark\_fatal}, \[18], 44, 45.
\:\\{mark\_warning}, \[18], 150, 222, 282, 284, 294, 448.
\:\\{max\_bib\_files}, \[14], 117, 118, 123, 242.
\:\\{max\_cites}, \[14], 17, 129, 130, 138, 219, 227.
\:\\{max\_ent\_ints}, \[14], 160, 287.
\:\\{max\_ent\_strs}, \[14], 160, 288.
\:\\{max\_fields}, \[14], 160, 225, 226.
\:\\{max\_glb\_str\_minus\_1}, \[15], 160.
\:\\{max\_glob\_strs}, \[15], 161, 162, 216.
\:\\{max\_hash\_value}, \[68].
\:\\{max\_pop}, \[50], 51, 331.
\:\\{max\_print\_line}, \[14], 17, 322, 323, 324.
\:\\{max\_strings}, \[14], 15, 17, 49, 51, 54, 219.
\:\\{mean\_while}, 449.
\:\\{mess\_with\_entries}, \[290], 293, 294, 296, 297, 298, 327, 328, 329, 354,
363, 378, 424, 447.
\:\\{middle}, \[303], 305.
\:\\{min\_crossrefs}, \[14], 227, 279, 283.
\:\\{min\_print\_line}, \[14], 17, 323.
\:\\{minus\_sign}, \[29], 64, 93, 190, 198.
\:\\{missing}, \[161], 225, 263, 277, 279, 282, 291, 327, 462.
\:{mooning}, 12.
\:\\{n\_}, 78, 333, 338.
\:\\{n\_aa}, \[338], 339, 372, 399.
\:\\{n\_aa\_upper}, \[338], 339, 372, 399.
\:\\{n\_add\_period}, \[333], 334, 341.
\:\\{n\_ae}, \[338], 339, 372, 399, 433, 453.
\:\\{n\_ae\_upper}, \[338], 339, 372, 399, 433, 453.
\:\\{n\_aux\_bibdata}, \[78], 79, 112, 116, 120.
\:\\{n\_aux\_bibstyle}, \[78], 79, 112, 116, 126.
\:\\{n\_aux\_citation}, \[78], 79, 116.
\:\\{n\_aux\_input}, \[78], 79, 116.
\:\\{n\_bib\_comment}, \[78], 79, 239.
\:\\{n\_bib\_preamble}, \[78], 79, 239, 262.
\:\\{n\_bib\_string}, \[78], 79, 239, 259, 262.
\:\\{n\_bst\_entry}, \[78], 79, 155.
\:\\{n\_bst\_execute}, \[78], 79, 155.
\:\\{n\_bst\_function}, \[78], 79, 155.
\:\\{n\_bst\_integers}, \[78], 79, 155.
\:\\{n\_bst\_iterate}, \[78], 79, 155.
\:\\{n\_bst\_macro}, \[78], 79, 155.
\:\\{n\_bst\_read}, \[78], 79, 155.
\:\\{n\_bst\_reverse}, \[78], 79, 155.
\:\\{n\_bst\_sort}, \[78], 79, 155.
\:\\{n\_bst\_strings}, \[78], 79, 155.
\:\\{n\_call\_type}, \[333], 334, 341.
\:\\{n\_change\_case}, \[333], 334, 341.
\:\\{n\_chr\_to\_int}, \[333], 334, 341.
\:\\{n\_cite}, \[333], 334, 341.
\:\\{n\_concatenate}, \[333], 334, 341.
\:\\{n\_duplicate}, \[333], 334, 341.
\:\\{n\_empty}, \[333], 334, 341.
\:\\{n\_equals}, \[333], 334, 341.
\:\\{n\_format\_name}, \[333], 334, 341.
\:\\{n\_gets}, \[333], 334, 341.
\:\\{n\_greater\_than}, \[333], 334, 341.
\:\\{n\_i}, \[338], 339, 372, 399.
\:\\{n\_if}, \[333], 334, 341.
\:\\{n\_int\_to\_chr}, \[333], 334, 341.
\:\\{n\_int\_to\_str}, \[333], 334, 341.
\:\\{n\_j}, \[338], 339, 372, 399.
\:\\{n\_l}, \[338], 339, 372, 399.
\:\\{n\_l\_upper}, \[338], 339, 372, 399.
\:\\{n\_less\_than}, \[333], 334, 341.
\:\\{n\_minus}, \[333], 334, 341.
\:\\{n\_missing}, \[333], 334, 341.
\:\\{n\_newline}, \[333], 334, 341.
\:\\{n\_num\_names}, \[333], 334, 341.
\:\\{n\_o}, \[338], 339, 372, 399.
\:\\{n\_o\_upper}, \[338], 339, 372, 399.
\:\\{n\_oe}, \[338], 339, 372, 399, 433, 453.
\:\\{n\_oe\_upper}, \[338], 339, 372, 399, 433, 453.
\:\\{n\_plus}, \[333], 334, 341.
\:\\{n\_pop}, \[333], 334, 341.
\:\\{n\_preamble}, \[333], 334, 341.
\:\\{n\_purify}, \[333], 334, 341.
\:\\{n\_quote}, \[333], 334, 341.
\:\\{n\_skip}, \[333], 334, 341.
\:\\{n\_ss}, \[338], 339, 372, 399, 433, 453.
\:\\{n\_stack}, \[333], 334, 341.
\:\\{n\_substring}, \[333], 334, 341.
\:\\{n\_swap}, \[333], 334, 341.
\:\\{n\_text\_length}, \[333], 334, 341.
\:\\{n\_text\_prefix}, \[333], 334, 341.
\:\\{n\_top\_stack}, \[333], 334, 341.
\:\\{n\_type}, \[333], 334, 341.
\:\\{n\_warning}, \[333], 334, 341.
\:\\{n\_while}, \[333], 334, 341.
\:\\{n\_width}, \[333], 334, 341.
\:\\{n\_write}, \[333], 334, 341.
\:\\{name\_bf\_ptr}, \[344], 387, 390, 391, 394, 396, 397, 398, 400, 401, 414,
415, 416.
\:\\{name\_bf\_xptr}, \[344], 396, 397, 398, 400, 401, 414, 415, 416.
\:\\{name\_bf\_yptr}, \[344], 398.
\:\\{name\_buf}, 43, \[344], 387, 390, 394, 397, 398, 400, 414, 415, 416.
\:\\{name\_length}, \[37], 58, 60, 61, 99, 106, 107, 141.
\:\\{name\_of\_file}, \[37], 38, 58, 60, 61, 97, 98, 99, 100, 107, 141.
\:\\{name\_ptr}, \[37], 58, 60, 61, 98, 99, 107, 141.
\:\\{name\_scan\_for\_and}, 383, \[384], 427.
\:\\{name\_sep\_char}, \[344], 387, 389, 392, 393, 396, 417.
\:\\{name\_tok}, \[344], 387, 390, 391, 394, 396, 401, 407, 414, 415.
\:\\{negative}, \[93].
\:{nested cross references}, 277.
\:\\{new\_cite}, \[265].
\:\\{newline}, 108, 121, 128.
\:\\{next\_cite}, \[132], 134.
\:\\{next\_insert}, \[303], 304.
\:\\{next\_token}, \[183], 184, 185, 186, 187.
\:\&{nil}, 9.
\:\\{nm\_brace\_level}, \[344], 397, 398, 400, 416.
\:\\{no\_bst\_file}, \[146], 151.
\:\\{no\_fields}, \[161], 462.
\:\\{nonexistent\_cross\_reference\_error}, 279, \[281].
\:\\{null\_code}, \[26].
\:\\{num\_bib\_files}, \[117], 145, 223, 457.
\:\\{num\_blt\_in\_fns}, 332, \[333], 335, 465.
\:\\{num\_cites}, \[129], 145, 225, 227, 276, 277, 279, 283, 287, 288, 289,
297, 298, 299, 458, 465.
\:\\{num\_commas}, \[344], 387, 389, 395.
\:\\{num\_ent\_ints}, \[161], 162, 174, 287, 328, 355, 461.
\:\\{num\_ent\_strs}, \[161], 162, 176, 288, 301, 329, 340, 357, 460.
\:\\{num\_fields}, \[161], 162, 170, 172, 225, 263, 265, 277, 279, 285, 327,
340, 462.
\:\\{num\_glb\_strs}, \[161], 162, 216.
\:\\{num\_names}, \[344], 383, 426, 427.
\:\\{num\_pre\_defined\_fields}, \[161], 170, 277, 340.
\:\\{num\_preamble\_strings}, \[219], 276, 429.
\:\\{num\_text\_chars}, \[344], 418, 441, 442, 445.
\:\\{num\_tokens}, \[344], 387, 389, 390, 391, 392, 393, 394, 395.
\:\\{number\_sign}, \[29], 33, 189, 190.
\:\\{numeric}, \[31], 32, 90, 92, 93, 190, 250, 431, 432.
\:\\{oe\_width}, \[35], 453.
\:\\{ok\_pascal\_i\_give\_up}, \[364], 370.
\:\\{old\_num\_cites}, \[129], 227, 264, 268, 269, 279, 283, 286, 458.
\:\\{old\_string}, \[68], 70, 71.
\:\\{open\_bibdata\_aux\_err}, \[122], 123.
\:\\{ord}, 24.
\:\\{other\_char\_adjacent}, \[89], 90, 166, 235.
\:\\{other\_lex}, \[31], 32.
\:\&{othercases}, \[5].
\:\\{others}, 5.
\:\\{out\_buf}, 264, \[290], 321, 322, 323, 324, 425, 454.
\:\\{out\_buf\_length}, \[290], 292, 321, 322, 323.
\:\\{out\_buf\_ptr}, \[290], 321, 322, 323, 324.
\:\\{out\_pool\_str}, 50, \[51].
\:\\{out\_token}, 81, \[82].
\:\\{output\_bbl\_line}, \[321], 323, 425.
\:\\{overflow}, \[44], 46, 53, 54, 59, 71, 123, 138, 140, 188, 200, 216, 226,
287, 288, 307, 322.
\:{overflow in arithmetic}, 11.
\:\|{p}, \[68].
\:\\{p\_ptr}, \[58], \[60], \[61].
\:\\{p\_ptr1}, \[48], 57, 320, 322.
\:\\{p\_ptr2}, \[48], 57, 320, 322.
\:\\{p\_str}, \[320], \[322].
\:{parent entry}, 277.
\:\\{partition}, \[303], 306.
\:\9{PASCAL H}{\ph}, 38.
\:\\{pds}, \[77], \[335].
\:\\{pds\_len}, \[73], 77, 335.
\:\\{pds\_loc}, \[73].
\:\\{pds\_type}, \[73], 77, 335.
\:\\{period}, \[29], 360, 361, 362, 417.
\:\\{pool\_file}, 48, 72.
\:\\{pool\_overflow}, \[53].
\:\\{pool\_pointer}, 48, \[49], 51, 56, 58, 60, 61, 344.
\:\\{pool\_ptr}, \[48], 53, 54, 55, 72, 351, 352, 362, 444.
\:\\{pool\_size}, \[14], 49, 53.
\:\\{pop\_lit}, \[309].
\:\\{pop\_lit\_stack}, 312.
\:\\{pop\_lit\_stk}, \[309], 314, 345, 346, 347, 348, 349, 350, 354, 360, 364,
377, 379, 380, 382, 421, 422, 423, 424, 426, 428, 430, 437, 439, 441, 443, 448,
449, 450, 454.
\:\\{pop\_lit\_var}, \[367], \[368], \[369], \[384].
\:\\{pop\_lit1}, \[344], 345, 346, 347, 348, 349, 350, 351, 352, 353, 354, 355,
357, 358, 359, 360, 361, 362, 364, 366, 377, 379, 380, 381, 382, 384, 402, 406,
421, 422, 423, 424, 426, 427, 428, 430, 437, 438, 439, 440, 441, 442, 443, 445,
448, 449, 450, 451, 454.
\:\\{pop\_lit2}, \[344], 345, 346, 347, 348, 349, 350, 351, 352, 353, 354, 355,
357, 358, 359, 364, 370, 382, 383, 388, 389, 391, 421, 437, 438, 439, 440, 443,
444.
\:\\{pop\_lit3}, \[344], 382, 383, 384, 388, 389, 391, 421, 437, 438.
\:\\{pop\_the\_aux\_stack}, 110, \[142].
\:\\{pop\_top\_and\_print}, \[314], 315, 446.
\:\\{pop\_type}, \[309].
\:\\{pop\_typ1}, \[344], 345, 346, 347, 348, 349, 350, 354, 360, 364, 377, 379,
380, 382, 421, 422, 423, 424, 426, 428, 430, 437, 439, 441, 443, 448, 449, 450,
454.
\:\\{pop\_typ2}, \[344], 345, 346, 347, 348, 349, 350, 354, 355, 357, 358, 359,
364, 382, 421, 437, 439, 443.
\:\\{pop\_typ3}, \[344], 382, 421, 437.
\:\\{pop\_whole\_stack}, \[315], 317, 436.
\:\\{pre\_def\_certain\_strings}, 13, \[336].
\:\\{pre\_def\_loc}, 75, \[76], 77, 79, 335, 339, 340.
\:\\{pre\_define}, 75, \[77], 79, 335, 339, 340.
\:\\{preamble\_ptr}, \[219], 242, 262, 276, 339, 429.
\:\\{preceding\_white}, \[344], 384.
\:\\{prev\_colon}, \[365], 370, 376.
\:\\{print}, \[3], 44, 45, 58, 60, 61, 95, 96, 110, 111, 112, 113, 114, 115,
122, 127, 135, 140, 141, 144, 148, 149, 150, 153, 158, 166, 167, 168, 169, 177,
183, 184, 185, 186, 200, 220, 221, 222, 223, 234, 235, 263, 273, 280, 281, 282,
284, 287, 288, 293, 294, 311, 312, 345, 354, 356, 368, 377, 383, 388, 389, 391,
406, 448, 455, 466.
\:\\{print\_}, 3.
\:\\{print\_a\_newline}, \[3].
\:\\{print\_a\_pool\_str}, 50, \[51].
\:\\{print\_a\_token}, 81, \[82].
\:\\{print\_aux\_name}, 107, \[108], 110, 111, 140, 141, 144.
\:\\{print\_bad\_input\_line}, \[95], 111, 149, 221.
\:\\{print\_bib\_name}, \[121], 122, 220, 223, 455.
\:\\{print\_bst\_name}, 127, \[128], 148.
\:\\{print\_confusion}, \[45], 466.
\:\\{print\_fn\_class}, \[158], 169, 177, 354.
\:\\{print\_lit}, \[313], 314, 448.
\:\\{print\_ln}, \[3], 10, 44, 45, 58, 60, 95, 111, 134, 138, 169, 184, 221,
222, 226, 280, 281, 282, 284, 313, 314, 317, 356, 466.
\:\\{print\_missing\_entry}, 283, \[284], 286.
\:\\{print\_newline}, \[3], 95, 108, 121, 128, 135, 293, 294, 313, 345.
\:\\{print\_overflow}, \[44].
\:\\{print\_pool\_str}, \[50], 58, 60, 61, 108, 121, 128, 135, 138, 169, 263,
273, 280, 284, 293, 294, 311, 313, 366, 368, 377, 383, 388, 389, 391, 406.
\:\\{print\_recursion\_illegal}, \[184].
\:\\{print\_skipping\_whatever\_remains}, \[96], 111, 221.
\:\\{print\_stk\_lit}, \[311], 312, 313, 345, 380, 424.
\:\\{print\_token}, \[81], 135, 140, 154, 177, 184, 185, 207, 234, 273.
\:\\{print\_wrong\_stk\_lit}, \[312], 346, 347, 348, 349, 350, 354, 355, 357,
358, 359, 360, 364, 377, 382, 421, 422, 423, 426, 430, 437, 441, 443, 448, 449,
450, 454.
\:{program conventions}, 8.
\:\\{ptr1}, \[301].
\:\\{ptr2}, \[301].
\:{push the literal stack}, 308, 351, 352, 353, 361, 379, 437, 438, 444.
\:\\{push\_lit\_stack}, 308.
\:\\{push\_lit\_stk}, \[307], 318, 325, 326, 327, 328, 330, 345, 346, 347, 348,
349, 350, 351, 352, 353, 360, 362, 364, 377, 378, 379, 380, 381, 382, 422, 423,
424, 426, 430, 434, 437, 438, 439, 440, 441, 443, 444, 447, 450.
\:\\{push\_lt}, \[307].
\:\\{push\_type}, \[307].
\:\\{put}, 37, 40.
\:\\{question\_mark}, \[29], 360, 361.
\:\\{quick\_sort}, 299, 300, 302, \[303], 306.
\:\\{quote\_next\_fn}, \[160], 188, 193, 194, 326, 463.
\:\\{r\_pop\_lt1}, \[343], 449.
\:\\{r\_pop\_lt2}, \[343], 449.
\:\\{r\_pop\_tp1}, \[343], 449.
\:\\{r\_pop\_tp2}, \[343], 449.
\:{raisin}, 278.
\:\\{read}, 100.
\:\\{read\_completed}, \[163], 164, 223, 458, 460, 461.
\:\\{read\_ln}, 100.
\:\\{read\_performed}, \[163], 164, 223, 455, 458, 462.
\:\\{read\_seen}, \[163], 164, 178, 203, 205, 211, 212, 214.
\:\\{reading\_completed}, \[163], 164, 223, 455.
\:\\{repush\_string}, \[308], 361, 379, 437.
\:\\{reset}, 37, 38.
\:\\{reset\_OK}, \[38].
\:\&{return}, 6, \[9].
\:\\{return\_von\_found}, \[397], 398, 399.
\:\\{rewrite}, 37, 38.
\:\\{rewrite\_OK}, \[38].
\:\\{right}, \[303], 304, 305, 306.
\:\\{right\_brace}, \[29], 33, 113, 114, 116, 120, 126, 132, 139, 166, 168,
171, 173, 175, 178, 181, 183, 187, 190, 191, 192, 199, 201, 203, 206, 208, 212,
215, 219, 242, 244, 250, 254, 255, 256, 257, 266, 360, 361, 367, 370, 371, 384,
385, 387, 390, 391, 398, 400, 402, 403, 404, 411, 416, 418, 431, 432, 441, 442,
443, 444, 445, 450, 451, 452.
\:\\{right\_end}, 302, \[303], 304, 305, 306.
\:\\{right\_outer\_delim}, \[219], 242, 244, 246, 259, 266, 274.
\:\\{right\_paren}, \[29], 33, 219, 242, 244, 266.
\:\\{right\_str\_delim}, \[219], 250, 253, 254, 255, 256.
\:\|{s}, \[51], \[56], \[280], \[284].
\:\\{s\_}, 74, 337.
\:\\{s\_aux\_extension}, \[74], 75, 103, 106, 107, 139, 140.
\:\\{s\_bbl\_extension}, \[74], 75, 103, 106.
\:\\{s\_bib\_area}, \[74], 75, 123.
\:\\{s\_bib\_extension}, \[74], 75, 121, 123, 457.
\:\\{s\_bst\_area}, \[74], 75, 127.
\:\\{s\_bst\_extension}, \[74], 75, 127, 128, 457.
\:\\{s\_default}, 182, \[337], 339.
\:\\{s\_l}, \[337].
\:\\{s\_log\_extension}, \[74], 75, 103, 106.
\:\\{s\_null}, \[337], 339, 350, 360, 364, 382, 422, 423, 430, 437, 441, 443,
447.
\:\\{s\_preamble}, 219, 262, \[337], 339, 429.
\:\\{s\_t}, \[337].
\:\\{s\_u}, \[337].
\:\\{sam\_too\_long\_file\_name\_print}, \[98].
\:\\{sam\_wrong\_file\_name\_print}, \[99].
\:\\{sam\_you\_made\_the\_file\_name\_too\_long}, \[98], 100, 103.
\:\\{sam\_you\_made\_the\_file\_name\_wrong}, \[99], 106.
\:{save space}, 42, 161.
\:\\{scan\_a\_field\_token\_and\_eat\_white}, 249, \[250].
\:\\{scan\_alpha}, \[88], 154.
\:\\{scan\_and\_store\_the\_field\_value\_and\_eat\_white}, 242, 246, 247, 248,
\[249], 274.
\:\\{scan\_balanced\_braces}, 250, \[253].
\:\\{scan\_char}, \[80], 83, 84, 85, 86, 87, 88, 90, 91, 92, 93, 94, 120, 126,
132, 139, 152, 154, 166, 167, 168, 171, 173, 175, 186, 187, 189, 190, 191, 201,
208, 215, 235, 238, 242, 244, 246, 249, 250, 252, 254, 255, 256, 257, 266, 274,
275.
\:\\{scan\_fn\_def}, 180, \[187], 189, 194.
\:\\{scan\_identifier}, 89, \[90], 166, 238, 244, 259, 275.
\:\\{scan\_integer}, \[93], 190.
\:\\{scan\_nonneg\_integer}, \[92], 258.
\:\\{scan\_result}, \[89], 90, 166, 235.
\:\\{scan\_white\_space}, \[94], 152, 228, 252.
\:\\{scan1}, \[83], 85, 116, 191, 209, 237.
\:\\{scan1\_white}, \[84], 126, 139, 266.
\:\\{scan2}, \[85], 87, 255.
\:\\{scan2\_white}, \[86], 120, 132, 183, 192, 199, 266.
\:\\{scan3}, \[87], 254.
\:{secret agent man}, 172.
\:\\{seen\_fn\_loc}, \[169].
\:\\{sep\_char}, \[31], 32, 387, 388, 393, 396, 401, 417, 430, 431, 432.
\:\\{short\_list}, \[302], 303, 304.
\:\\{sign\_length}, \[93].
\:\\{singl\_fn\_overflow}, \[188].
\:\\{singl\_function}, \[187], 188, 200.
\:\\{single\_fn\_space}, \[14], 187, 188.
\:\\{single\_ptr}, \[187], 188, 200.
\:\\{single\_quote}, \[29], 33, 189, 192, 194.
\:\\{skip\_illegal\_stuff\_after\_token\_print}, \[186].
\:\\{skip\_recursive\_token}, \[184], 193, 199.
\:\\{skip\_stuff\_at\_sp\_brace\_level\_greater\_than\_one}, 403, \[404], 412.
\:\\{skip\_token}, \[183], 190, 191.
\:\\{skip\_token\_illegal\_stuff\_after\_literal}, \[186], 190, 191.
\:\\{skip\_token\_print}, \[183], 184, 185, 186.
\:\\{skip\_token\_unknown\_function}, \[185], 192, 199.
\:\\{skp\_token\_unknown\_function\_print}, \[185].
\:\.{sort.key\$}, 340.
\:\\{sort\_cite\_ptr}, \[290], 297, 298, 458.
\:\\{sort\_key\_num}, \[290], 301, 340.
\:\\{sorted\_cites}, 219, \[289], 290, 297, 298, 300, 302, 303, 304, 305, 306,
458.
\:\\{sp\_brace\_level}, \[344], 402, 403, 404, 405, 406, 411, 412, 442, 444,
445.
\:\\{sp\_end}, \[344], 351, 352, 353, 359, 361, 362, 379, 381, 402, 403, 404,
438, 440, 442, 444, 445.
\:\\{sp\_length}, \[344], 352, 437, 438.
\:\\{sp\_ptr}, \[344], 351, 352, 353, 357, 359, 361, 362, 379, 381, 402, 403,
404, 405, 407, 408, 409, 410, 411, 412, 417, 438, 440, 442, 444, 445.
\:\\{sp\_xptr1}, \[344], 352, 357, 403, 411, 412, 417, 445.
\:\\{sp\_xptr2}, \[344], 412, 417.
\:\\{space}, \[26], 31, 32, 33, 35, 95, 249, 252, 253, 256, 260, 261, 322, 323,
392, 393, 417, 419, 430, 431.
\:{space savings}, 1, 14, 15, 42, 161.
\:{special character}, 371, 397, 398, 401, 415, 416, 418, 430, 431, 432, 441,
442, 443, 445, 450, 452.
\:\\{specified\_char\_adjacent}, \[89], 90, 166, 235.
\:\\{spotless}, \[18], 19, 20, 466.
\:\\{sp2\_length}, 344, 352.
\:\\{ss\_width}, \[35], 453.
\:\\{star}, \[29], 134.
\:\\{start\_name}, \[58], 123, 127, 141.
\:\&{stat}, \[4].
\:\\{stat\_pr}, \[465].
\:\\{stat\_pr\_ln}, \[465].
\:\\{stat\_pr\_pool\_str}, \[465].
\:{statistics}, 4, 465.
\:\\{stk\_empty}, \[291], 307, 309, 311, 312, 313, 314, 345, 380, 424.
\:\\{stk\_field\_missing}, \[291], 307, 311, 312, 313, 327, 380, 424.
\:\\{stk\_fn}, \[291], 307, 311, 312, 313, 326, 354, 421, 449.
\:\\{stk\_int}, \[291], 307, 311, 312, 313, 325, 328, 345, 346, 347, 348, 349,
355, 358, 377, 380, 381, 382, 421, 422, 423, 424, 426, 437, 441, 443, 449, 450.
\:\\{stk\_lt}, \[311], \[312], \[313], \[314].
\:\\{stk\_str}, \[291], 307, \[309], 311, 312, 313, 318, 325, 327, 330, 345,
350, 351, 352, 353, 357, 359, 360, 362, 364, 377, 378, 379, 380, 382, 422, 423,
424, 426, 430, 434, 437, 438, 439, 440, 441, 443, 444, 447, 448, 450, 454.
\:\\{stk\_tp}, \[311], \[313], \[314].
\:\\{stk\_tp1}, \[312].
\:\\{stk\_tp2}, \[312].
\:\\{stk\_type}, 290, \[291], 307, 309, 311, 312, 313, 314, 343, 344.
\:\\{store\_entry}, \[219], 267, 275.
\:\\{store\_field}, \[219], 242, 246, 249, 253, 258, 259, 275.
\:\\{store\_token}, \[219], 259.
\:\\{str\_delim}, 247.
\:\\{str\_ent\_loc}, \[160], 161, 290, 301.
\:\\{str\_ent\_ptr}, \[161], 288, 329, 357, 460.
\:\\{str\_entry\_var}, 14, \[156], 158, 159, 160, 161, 162, 170, 175, 176, 288,
290, 302, 325, 329, 331, 340, 354.
\:\\{str\_eq\_buf}, \[56], 70, 140.
\:\\{str\_eq\_str}, \[57], 345.
\:\\{str\_found}, \[68], 70.
\:\\{str\_glb\_ptr}, \[161], 162, 330, 359.
\:\\{str\_glob\_loc}, \[160], 161.
\:\\{str\_global\_var}, 14, 15, \[156], 158, 159, 160, 161, 162, 215, 216, 290,
325, 330, 354.
\:\\{str\_ilk}, \[64], 65, 68, 70, 77.
\:\\{str\_literal}, \[156], 158, 159, 180, 189, 191, 205, 209, 261, 325, 339.
\:\\{str\_lookup}, 65, \[68], 76, 77, 107, 116, 123, 127, 133, 135, 136, 140,
154, 172, 174, 176, 177, 182, 190, 191, 192, 194, 199, 202, 207, 209, 216, 238,
245, 259, 261, 264, 267, 269, 270, 272, 273, 275, 278, 371, 398, 432, 452.
\:\\{str\_not\_found}, \[68].
\:\\{str\_num}, \[48], \[68], 70, 71, 464.
\:\\{str\_number}, 48, \[49], 51, 54, 56, 57, 58, 60, 61, 65, 68, 74, 104, 117,
124, 129, 161, 219, 278, 280, 284, 290, 320, 322, 337, 367, 368, 369, 384.
\:\\{str\_pool}, \[48], 49, 50, 51, 53, 54, 56, 57, 58, 60, 61, 64, 68, 71, 72,
73, 74, 75, 104, 117, 129, 260, 270, 278, 291, 309, 316, 317, 318, 320, 322,
329, 330, 334, 337, 344, 351, 352, 353, 357, 359, 361, 362, 366, 377, 379, 381,
402, 403, 404, 405, 407, 408, 409, 410, 411, 412, 417, 438, 440, 442, 444, 445.
\:\\{str\_ptr}, \[48], 51, 54, 55, 72, 290, 309, 316, 317, 464, 465.
\:\\{str\_room}, \[53], 71, 318, 330, 351, 352, 353, 362, 379, 422, 434.
\:\\{str\_start}, \[48], 49, 51, 52, 54, 55, 56, 57, 58, 60, 61, 64, 67, 72,
260, 270, 278, 320, 322, 351, 352, 353, 357, 359, 361, 362, 366, 377, 379, 381,
402, 438, 440, 442, 444, 464, 465.
\:{string pool}, 72.
\:\.{String size exceeded}, 356.
\:\9{String size exceeded}{\quad entry string size}, 357.
\:\9{String size exceeded}{\quad global string size}, 359.
\:\\{string\_width}, \[34], 450, 451, 452, 453.
\:{style-file commands}, 155, 163.
\:\9{style-file commands}{\quad \.{entry}}, 170.
\:\9{style-file commands}{\quad \.{execute}}, 178.
\:\9{style-file commands}{\quad \.{function}}, 180.
\:\9{style-file commands}{\quad \.{integers}}, 201.
\:\9{style-file commands}{\quad \.{iterate}}, 203.
\:\9{style-file commands}{\quad \.{macro}}, 205.
\:\9{style-file commands}{\quad \.{read}}, 211.
\:\9{style-file commands}{\quad \.{reverse}}, 212.
\:\9{style-file commands}{\quad \.{sort}}, 214.
\:\9{style-file commands}{\quad \.{strings}}, 215.
\:\\{sv\_buffer}, \[43], 211, 344.
\:\\{sv\_ptr1}, \[43], 211.
\:\\{sv\_ptr2}, \[43], 211.
\:\\{swap}, \[300], 304, 305, 306.
\:\\{swap1}, \[300].
\:\\{swap2}, \[300].
\:{system dependencies}, 1, 2, 3, 5, 10, 11, 14, 15, 23, 25, 26, 27, 32, 33,
35, 37, 38, 39, 42, 51, 75, 82, 97, 98, 99, 100, 101, 102, 106, 161, 466, 467.
\:\\{s1}, \[57].
\:\\{s2}, \[57].
\:\\{tab}, \[26], 27, 32, 33.
\:\&{tats}, \[4].
\:\\{term\_in}, \[2], 100.
\:\\{term\_out}, \[2], 3, 13, 51, 82, 98, 99, 100.
\:\9{TeXbook}{\sl The \TeX book}, 27.
\:\\{text\_char}, \[23], 24, 36, 38.
\:\\{text\_ilk}, \[64], 75, 107, 156, 191, 209, 261, 339.
\:\.{this can't happen}, 45, 468.
\:\9{this can't happen}{\quad A cite key disappeared}, 270, 271, 285.
\:\9{this can't happen}{\quad A digit disappeared}, 258.
\:\9{this can't happen}{\quad Already encountered auxiliary file}, 107.
\:\9{this can't happen}{\quad Already encountered implicit function}, 194.
\:\9{this can't happen}{\quad Already encountered style file}, 127.
\:\9{this can't happen}{\quad An at-sign disappeared}, 238.
\:\9{this can't happen}{\quad Cite hash error}, 136, 137, 264, 272, 279, 285.
\:\9{this can't happen}{\quad Control-sequence hash error}, 399.
\:\9{this can't happen}{\quad Duplicate sort key}, 301.
\:\9{this can't happen}{\quad History is bunk}, 466.
\:\9{this can't happen}{\quad Identifier scanning error}, 165, 166, 235.
\:\9{this can't happen}{\quad Illegal auxiliary-file command}, 112.
\:\9{this can't happen}{\quad Illegal literal type}, 310, 311, 312, 313.
\:\9{this can't happen}{\quad Illegal number of comma,s}, 395.
\:\9{this can't happen}{\quad Illegal string number}, 51.
\:\9{this can't happen}{\quad Nonempty empty string stack}, 317.
\:\9{this can't happen}{\quad Nontop top of string stack}, 309.
\:\9{this can't happen}{\quad The cite list is messed up}, 268.
\:\9{this can't happen}{\quad Unknown auxiliary-file command}, 116.
\:\9{this can't happen}{\quad Unknown built-in function}, 341.
\:\9{this can't happen}{\quad Unknown database-file command}, 239, 240, 262.
\:\9{this can't happen}{\quad Unknown function class}, 157, 158, 159, 325.
\:\9{this can't happen}{\quad Unknown literal type}, 307, 310, 311, 312, 313.
\:\9{this can't happen}{\quad Unknown style-file command}, 155.
\:\9{this can't happen}{\quad Unknown type of case conversion}, 372, 373, 375,
376.
\:\\{tie}, \[29], 32, 396, 401, 411, 417, 419.
\:\\{title\_lowers}, 337, \[365], 366, 370, 372, 375, 376.
\:\\{tmp\_end\_ptr}, \[43], 260, 270, 278.
\:\\{tmp\_ptr}, \[43], 133, 211, 258, 260, 264, 267, 270, 278, 285, 323, 374.
\:\\{to\_be\_written}, \[344], 403, 405, 407, 408, 409, 410.
\:\\{token\_len}, \[80], 88, 90, 92, 93, 116, 123, 127, 133, 134, 135, 136,
140, 154, 172, 174, 176, 177, 182, 190, 191, 192, 199, 202, 207, 209, 216, 238,
245, 259, 267, 269, 272, 273, 275.
\:\\{token\_starting}, \[344], 387, 389, 390, 391, 392, 393, 394.
\:\\{token\_value}, \[91], 92, 93, 190.
\:\\{top\_lev\_str}, \[104], 107.
\:\\{total\_ex\_count}, \[331], 465.
\:\\{total\_fields}, \[226].
\:\\{tr\_print}, 161.
\:\&{trace}, 3, \[4].
\:\\{trace\_and\_stat\_printing}, 455, \[456].
\:\\{trace\_pr}, \[3], 133, 159, 190, 191, 192, 193, 199, 209, 261, 297, 298,
307, 325, 457, 458, 459, 460, 461, 462, 463, 464, 465.
\:\\{trace\_pr\_}, 3.
\:\\{trace\_pr\_fn\_class}, \[159], 193, 199.
\:\\{trace\_pr\_ln}, \[3], 110, 123, 134, 135, 172, 174, 176, 179, 182, 190,
191, 194, 202, 204, 207, 209, 213, 216, 223, 238, 245, 261, 267, 275, 299, 303,
307, 325, 457, 458, 459, 460, 462, 463, 464, 465.
\:\\{trace\_pr\_newline}, \[3], 136, 184, 193, 199, 297, 298, 457, 458, 461.
\:\\{trace\_pr\_pool\_str}, \[50], 123, 194, 261, 297, 298, 307, 325, 457, 458,
459, 462, 463, 464, 465.
\:\\{trace\_pr\_token}, \[81], 133, 172, 174, 176, 179, 182, 190, 191, 192,
199, 202, 204, 207, 209, 213, 216, 238, 245, 267, 275.
\:\\{true}, 9, 47, 56, 57, 65, 68, 70, 83, 84, 85, 86, 87, 88, 92, 93, 94, 101,
117, 120, 124, 126, 129, 132, 134, 140, 152, 163, 170, 177, 211, 219, 223, 228,
238, 239, 242, 246, 249, 250, 252, 253, 259, 265, 267, 268, 269, 272, 275, 278,
290, 297, 298, 301, 323, 324, 365, 376, 384, 386, 387, 389, 392, 393, 397, 403,
405, 407, 408, 409, 410, 412, 418, 462.
\:\\{tty}, 2.
\:{Tuesdays}, 325, 401.
\:{turn out lights}, 455.
\:\\{type\_exists}, \[219], 238, 273.
\:\\{type\_list}, \[219], 227, 268, 273, 279, 283, 285, 363, 447, 459.
\:\\{unbreakable\_tail}, 322, 324.
\:\\{undefined}, \[219], 273, 363, 447, 459.
\:\\{unflush\_string}, \[55], 308, 351, 352, 438, 439.
\:\\{unknwn\_function\_class\_confusion}, \[157], 158, 159, 325.
\:\\{unknwn\_literal\_confusion}, 307, \[310], 311, 312, 313.
\:\\{upper\_ae\_width}, \[35], 453.
\:\\{upper\_case}, \[63], 372, 374, 375, 376.
\:\\{upper\_oe\_width}, \[35], 453.
\:\\{use\_default}, \[344], 412, 417.
\:{user abuse}, 98, 99, 393, 416.
\:\\{von\_end}, \[344], 396, 401, 408, 409.
\:\\{von\_found}, \[382], 396.
\:\\{von\_name\_ends\_and\_last\_name\_starts\_stuff}, 395, 396, \[401].
\:\\{von\_start}, \[344], 395, 396, 401, 408.
\:\\{von\_token\_found}, 396, \[397], 401.
\:\\{warning\_message}, \[18], 19, 20, 150, 293, 294, 466.
\:\.{WEB}, 52, 69.
\:\\{white\_adjacent}, \[89], 90, 166, 235.
\:\\{white\_space}, 26, 29, \[31], 32, 35, 47, 84, 86, 90, 94, 95, 115, 120,
126, 132, 139, 152, 170, 180, 183, 187, 190, 191, 192, 199, 201, 205, 215, 218,
228, 243, 246, 249, 252, 253, 254, 256, 257, 260, 321, 322, 323, 324, 364, 370,
374, 376, 380, 381, 384, 386, 387, 388, 393, 426, 427, 430, 431, 432, 452.
\:{whole database inclusion}, 132.
\:{windows}, 325.
\:\\{wiz\_def\_ptr}, \[161], 162, 200, 463, 465.
\:\\{wiz\_defined}, 14, \[156], 158, 159, 160, 161, 162, 177, 178, 179, 180,
181, 182, 184, 187, 194, 203, 204, 212, 213, 238, 325, 326, 463.
\:\\{wiz\_fn\_loc}, \[160], 161, 325.
\:\\{wiz\_fn\_ptr}, \[161], 463.
\:\\{wiz\_fn\_space}, \[14], 160, 200.
\:\\{wiz\_functions}, 160, \[161], 188, 190, 191, 193, 194, 199, 200, 325, 326,
463.
\:\\{wiz\_loc}, \[161], 180, 182, 189, 193, 199.
\:\\{wiz\_ptr}, \[325], 326.
\:{wizard}, 1.
\:\\{write}, 3, 51, 82, 98, 99, 100, 321.
\:\\{write\_ln}, 3, 13, 98, 99, 321.
\:\\{x\_add\_period}, 341, \[360].
\:\\{x\_change\_case}, 341, \[364].
\:\\{x\_chr\_to\_int}, 341, \[377].
\:\\{x\_cite}, 341, \[378].
\:\\{x\_concatenate}, 341, \[350].
\:\\{x\_duplicate}, 341, \[379].
\:\\{x\_empty}, 341, \[380].
\:\\{x\_equals}, 341, \[345].
\:\\{x\_format\_name}, 341, \[382], 420.
\:\\{x\_gets}, 341, \[354].
\:\\{x\_greater\_than}, 341, \[346].
\:\\{x\_int\_to\_chr}, 341, \[422].
\:\\{x\_int\_to\_str}, 341, \[423].
\:\\{x\_less\_than}, 341, \[347].
\:\\{x\_minus}, 341, \[349].
\:\\{x\_missing}, 341, \[424].
\:\\{x\_num\_names}, 341, \[426].
\:\\{x\_plus}, 341, \[348].
\:\\{x\_preamble}, 341, \[429].
\:\\{x\_purify}, 341, \[430].
\:\\{x\_quote}, 341, \[434].
\:\\{x\_substring}, 341, \[437].
\:\\{x\_swap}, 341, \[439].
\:\\{x\_text\_length}, 341, \[441].
\:\\{x\_text\_prefix}, 341, \[443].
\:\\{x\_type}, 341, \[447].
\:\\{x\_warning}, 341, \[448].
\:\\{x\_width}, 341, \[450].
\:\\{x\_write}, 341, \[454].
\:\\{xchr}, \[24], 25, 27, 28, 48, 51, 82, 95, 113, 114, 154, 166, 167, 168,
186, 191, 208, 209, 230, 231, 235, 238, 242, 246, 321, 460.
\:\&{xclause}, 9.
\:\\{xord}, \[24], 28, 47, 77, 107.
\:{Yogi}, 455.
\fin
\:\X277:Add cross-reference information\X
\U276.
\:\X106:Add extensions and open files\X
\U103.
\:\X264:Add or update a cross reference on \\{cite\_list} if necessary\X
\U263.
\:\X362:Add the \\{period} (it's necessary) and push\X
\U361.
\:\X361:Add the \\{period}, if necessary, and push\X
\U360.
\:\X451:Add up the \\{char\_width}s in this string\X
\U450.
\:\X357:Assign to a \\{str\_entry\_var}\X
\U354.
\:\X359:Assign to a \\{str\_global\_var}\X
\U354.
\:\X355:Assign to an \\{int\_entry\_var}\X
\U354.
\:\X358:Assign to an \\{int\_global\_var}\X
\U354.
\:\X323:Break that line\X
\U322.
\:\X324:Break that unbreakably long line\X
\U323.
\:\X193:Check and insert the quoted function\X
\U192.
\:\X267:Check for a database key of interest\X
\U266.
\:\X268:Check for a duplicate or \.{crossref}-matching database key\X
\U267.
\:\X134:Check for entire database inclusion (and thus skip this cite key)\X
\U133.
\:\X179:Check the \.{execute}-command argument token\X
\U178.
\:\X204:Check the \.{iterate}-command argument token\X
\U203.
\:\X213:Check the \.{reverse}-command argument token\X
\U212.
\:\X17, 302:Check the ``constant'' values for consistency\X
\U13.
\:\X133:Check the cite key\X
\U132.
\:\X207:Check the macro name\X
\U206.
\:\X398:Check the special character (and \&{return})\X
\U397.
\:\X182:Check the \\{wiz\_defined} function name\X
\U181.
\:\X135:Cite seen, don't add a cite key\X
\U133.
\:\X136:Cite unseen, add a cite key\X
\U133.
\:\X455:Clean up and leave\X
\U10.
\:\X11:Compiler directives\X
\U10.
\:\X282:Complain about a nested cross reference\X
\U279.
\:\X286:Complain about missing entries whose cite keys got overwritten\X
\U283.
\:\X200:Complete this function's definition\X
\U187.
\:\X69:Compute the hash code \|h\X
\U68.
\:\X351:Concatenate the two strings and push\X
\U350.
\:\X353:Concatenate them and push when $\\{pop\_lit1},\\{pop\_lit2}<\\{cmd\_str%
\_ptr}$\X
\U352.
\:\X352:Concatenate them and push when $\\{pop\_lit2}<\\{cmd\_str\_ptr}$\X
\U351.
\:\X14, 333:Constants in the outer block\X
\U10.
\:\X375:Convert a noncontrol sequence\X
\U371.
\:\X371:Convert a special character\X
\U370.
\:\X376:Convert a $\\{brace\_level}=0$ character\X
\U370.
\:\X372:Convert the accented or foreign character, if necessary\X
\U371.
\:\X374:Convert, then remove the control sequence\X
\U372.
\:\X387:Copy name and count \\{comma}s to determine syntax\X
\U382.
\:\X260:Copy the macro string to \\{field\_vl\_str}\X
\U259.
\:\X442:Count the text characters\X
\U441.
\:\X343:Declarations for executing \\{built\_in} functions\X
\U325.
\:\X366:Determine the case-conversion type\X
\U364.
\:\X427:Determine the number of names\X
\U426.
\:\X453:Determine the width of this accented or foreign character\X
\U452.
\:\X452:Determine the width of this special character\X
\U451.
\:\X396:Determine where the first name ends and von name starts and ends\X
\U395.
\:\X256:Do a full brace-balanced scan\X
\U253.
\:\X257:Do a full scan with $\\{bib\_brace\_level}>0$\X
\U256.
\:\X254:Do a quick brace-balanced scan\X
\U253.
\:\X255:Do a quick scan with $\\{bib\_brace\_level}>0$\X
\U254.
\:\X304:Do a straight insertion sort\X
\U303.
\:\X306:Do the partitioning and the recursive calls\X
\U303.
\:\X305:Draw out the median-of-three partition element\X
\U303.
\:\X327:Execute a field\X
\U325.
\:\X341:Execute a \\{built\_in} function\X
\U325.
\:\X329:Execute a \\{str\_entry\_var}\X
\U325.
\:\X330:Execute a \\{str\_global\_var}\X
\U325.
\:\X326:Execute a \\{wiz\_defined} function\X
\U325.
\:\X328:Execute an \\{int\_entry\_var}\X
\U325.
\:\X412:Figure out how to output the name tokens, and do it\X
\U411.
\:\X402:Figure out the formatted name\X
\U420.
\:\X405:Figure out what this letter means\X
\U403.
\:\X407:Figure out what tokens we'll output for the `first' name\X
\U405.
\:\X410:Figure out what tokens we'll output for the `jr' name\X
\U405.
\:\X409:Figure out what tokens we'll output for the `last' name\X
\U405.
\:\X408:Figure out what tokens we'll output for the `von' name\X
\U405.
\:\X224:Final initialization for \.{.bib} processing\X
\U223.
\:\X276:Final initialization for processing the entries\X
\U223.
\:\X411:Finally format this part of the name\X
\U403.
\:\X414:Finally output a full token\X
\U413.
\:\X416:Finally output a special character and exit loop\X
\U415.
\:\X415:Finally output an abbreviated token\X
\U413.
\:\X417:Finally output the inter-token string\X
\U413.
\:\X413:Finally output the name tokens\X
\U412.
\:\X270:Find the lower-case equivalent of the \\{cite\_info} key\X
\U268.
\:\X395:Find the parts of the name\X
\U382.
\:\X444:Form the appropriate prefix\X
\U443.
\:\X438:Form the appropriate substring\X
\U437.
\:\X403:Format this part of the name\X
\U402.
\:\X275:Get the next field name\X
\U274.
\:\X189:Get the next function of the definition\X
\U187.
\:\X16, 19, 24, 30, 34, 37, 41, 43, 48, 65, 74, 76, 78, 80, 89, 91, 97, 104,
117, 124, 129, 147, 161, 163, 195, 219, 247, 290, 331, 337, 344, 365:Globals in
the outer block\X
\U10.
\:\X419:Handle a discretionary \\{tie}\X
\U411.
\:\X103:Handle this \.{.aux} name\X
\U100.
\:\X399:Handle this accented or foreign character (and \&{return})\X
\U398.
\:\X225:Initialize the \\{field\_info}\X
\U224.
\:\X287:Initialize the \\{int\_entry\_var}s\X
\U276.
\:\X289:Initialize the \\{sorted\_cites}\X
\U276.
\:\X288:Initialize the \\{str\_entry\_var}s\X
\U276.
\:\X227:Initialize things for the \\{cite\_list}\X
\U224.
\:\X172:Insert a \\{field} into the hash table\X
\U171.
\:\X176:Insert a \\{str\_entry\_var} into the hash table\X
\U175.
\:\X216:Insert a \\{str\_global\_var} into the hash table\X
\U215.
\:\X174:Insert an \\{int\_entry\_var} into the hash table\X
\U173.
\:\X202:Insert an \\{int\_global\_var} into the hash table\X
\U201.
\:\X71:Insert pair into hash table and make \|p point to it\X
\U68.
\:\X383:Isolate the desired name\X
\U382.
\:\X109, 146:Labels in the outer block\X
\U10.
\:\X23, 66:Local variables for initialization\X
\U13.
\:\X273:Make sure this entry is ok before proceeding\X
\U267.
\:\X269:Make sure this entry's database key is on \\{cite\_list}\X
\U268.
\:\X389:Name-process a \\{comma}\X
\U387.
\:\X390:Name-process a \\{left\_brace}\X
\U387.
\:\X391:Name-process a \\{right\_brace}\X
\U387.
\:\X393:Name-process a \\{sep\_char}\X
\U387.
\:\X392:Name-process a \\{white\_space}\X
\U387.
\:\X394:Name-process some other character\X
\U387.
\:\X123:Open a \.{.bib} file\X
\U120.
\:\X127:Open the \.{.bst} file\X
\U126.
\:\X141:Open this \.{.aux} file\X
\U140.
\:\X298:Perform a \.{reverse} command\X
\U212.
\:\X299:Perform a \.{sort} command\X
\U214.
\:\X296:Perform an \.{execute} command\X
\U178.
\:\X297:Perform an \.{iterate} command\X
\U203.
\:\X370:Perform the case conversion\X
\U364.
\:\X431:Perform the purification\X
\U430.
\:\X75, 79, 334, 339, 340:Pre-define certain strings\X
\U336.
\:\X457:Print all \.{.bib}- and \.{.bst}-file information\X
\U456.
\:\X458:Print all \\{cite\_list} and entry information\X
\U456.
\:\X459:Print entry information\X
\U458.
\:\X461:Print entry integers\X
\U459.
\:\X460:Print entry strings\X
\U459.
\:\X462:Print fields\X
\U459.
\:\X466:Print the job \\{history}\X
\U455.
\:\X464:Print the string pool\X
\U456.
\:\X463:Print the \\{wiz\_defined} functions\X
\U456.
\:\X465:Print usage statistics\X
\U456.
\:\X12:Procedures and functions for about everything\X
\U10.
\:\X3, 18, 44, 45, 46, 47, 51, 53, 59, 82, 95, 96, 98, 99, 108, 111, 112, 113,
114, 115, 121, 128, 137, 138, 144, 148, 149, 150, 153, 157, 158, 159, 165, 166,
167, 168, 169, 188, 220, 221, 222, 226, 229, 230, 231, 232, 233, 234, 235, 240,
271, 280, 281, 284, 293, 294, 295, 310, 311, 313, 321, 356, 368, 373,
456:Procedures and functions for all file I/O, error messages, and such\X
\U12.
\:\X38, 39, 58, 60, 61:Procedures and functions for file-system interacting\X
\U12.
\:\X54, 56, 57, 62, 63, 68, 77, 198, 265, 278, 300, 301, 303, 335,
336:Procedures and functions for handling numbers, characters, and strings\X
\U12.
\:\X83, 84, 85, 86, 87, 88, 90, 92, 93, 94, 152, 183, 184, 185, 186, 187, 228,
248, 249:Procedures and functions for input scanning\X
\U12.
\:\X367, 369, 384, 397, 401, 404, 406, 418, 420:Procedures and functions for
name-string processing\X
\U12.
\:\X307, 309, 312, 314, 315, 316, 317, 318, 320, 322, 342:Procedures and
functions for style-file function execution\X
\U12.
\:\X100, 120, 126, 132, 139, 142, 143, 145, 170, 177, 178, 180, 201, 203, 205,
210, 211, 212, 214, 215, 217:Procedures and functions for the reading and
processing of input files\X
\U12.
\:\X239:Process a \.{.bib} command\X
\U238.
\:\X241:Process a \.{comment} command\X
\U239.
\:\X242:Process a \.{preamble} command\X
\U239.
\:\X243:Process a \.{string} command\X
\U239.
\:\X102:Process a possible command line\X
\U100.
\:\X155:Process the appropriate \.{.bst} command\X
\U154.
\:\X70:Process the string if we've already encountered it\X
\U68.
\:\X432:Purify a special character\X
\U431.
\:\X433:Purify this accented or foreign character\X
\U432.
\:\X381:Push 0 if the string has a non\\{white\_space} char, else 1\X
\U380.
\:\X140:Push the \.{.aux} stack\X
\U139.
\:\X272:Put this cite key in its place\X
\U267.
\:\X107:Put this name into the hash table\X
\U103.
\:\X151:Read and execute the \.{.bst} file\X
\U10.
\:\X110:Read the \.{.aux} file\X
\U10.
\:\X223:Read the \.{.bib} file(s)\X
\U211.
\:\X388:Remove leading and trailing junk, complaining if necessary\X
\U387.
\:\X283:Remove missing entries or those cross referenced too few times\X
\U276.
\:\X259:Scan a macro name\X
\U250.
\:\X258:Scan a number\X
\U250.
\:\X192:Scan a quoted function\X
\U189.
\:\X191:Scan a \\{str\_literal}\X
\U189.
\:\X199:Scan an already-defined function\X
\U189.
\:\X190:Scan an \\{int\_literal}\X
\U189.
\:\X236:Scan for and process a \.{.bib} command or database entry\X
\U210.
\:\X154:Scan for and process a \.{.bst} command\X
\U217.
\:\X116:Scan for and process an \.{.aux} command\X
\U143.
\:\X445:Scan the appropriate number of characters\X
\U444.
\:\X238:Scan the entry type or scan and process the \.{.bib} command\X
\U236.
\:\X266:Scan the entry's database key\X
\U236.
\:\X274:Scan the entry's list of fields\X
\U236.
\:\X171:Scan the list of \\{field}s\X
\U170.
\:\X173:Scan the list of \\{int\_entry\_var}s\X
\U170.
\:\X175:Scan the list of \\{str\_entry\_var}s\X
\U170.
\:\X209:Scan the macro definition-string\X
\U208.
\:\X206:Scan the macro name\X
\U205.
\:\X208:Scan the macro's definition\X
\U205.
\:\X246:Scan the string's definition field\X
\U243.
\:\X244:Scan the string's name\X
\U243.
\:\X181:Scan the \\{wiz\_defined} function name\X
\U180.
\:\X386:See if we have an ``and''\X
\U384.
\:\X20, 25, 27, 28, 32, 33, 35, 67, 72, 119, 125, 131, 162, 164, 196, 292:Set
initial values of key variables\X
\U13.
\:\X385:Skip over \\{ex\_buf} stuff at $\\{brace\_level}>0$\X
\U384.
\:\X400:Skip over \\{name\_buf} stuff at $\\{nm\_brace\_level}>0$\X
\U397.
\:\X237:Skip to the next database entry or \.{.bib} command\X
\U236.
\:\X285:Slide this cite key down to its permanent spot\X
\U283.
\:\X194:Start a new function definition\X
\U189.
\:\X262:Store the field value for a command\X
\U261.
\:\X263:Store the field value for a database entry\X
\U261.
\:\X261:Store the field value string\X
\U249.
\:\X245:Store the string's name\X
\U244.
\:\X279:Subtract cross-reference information\X
\U276.
\:\X440:Swap the two strings (they're at the end of \\{str\_pool})\X
\U439.
\:\X13:The procedure \\{initialize}\X
\U10.
\:\X252:The scanning function \\{compress\_bib\_white}\X
\U248.
\:\X250:The scanning function \\{scan\_a\_field\_token\_and\_eat\_white}\X
\U248.
\:\X253:The scanning function \\{scan\_balanced\_braces}\X
\U248.
\:\X22, 31, 36, 42, 49, 64, 73, 105, 118, 130, 160, 291, 332:Types in the outer
block\X
\U10.
\:\X101:Variables for possible command-line processing\X
\U100.
\:\X325:\\{execute\_fn} itself\X
\U342.
\:\X350:\\{execute\_fn}({\.{*}})\X
\U342.
\:\X348:\\{execute\_fn}({\.{+}})\X
\U342.
\:\X349:\\{execute\_fn}({\.{-}})\X
\U342.
\:\X354:\\{execute\_fn}({\.{:=}})\X
\U342.
\:\X347:\\{execute\_fn}({\.{<}})\X
\U342.
\:\X345:\\{execute\_fn}({\.{=}})\X
\U342.
\:\X346:\\{execute\_fn}({\.{>}})\X
\U342.
\:\X360:\\{execute\_fn}({\.{add.period\$}})\X
\U342.
\:\X363:\\{execute\_fn}({\.{call.type\$}})\X
\U341.
\:\X364:\\{execute\_fn}({\.{change.case\$}})\X
\U342.
\:\X377:\\{execute\_fn}({\.{chr.to.int\$}})\X
\U342.
\:\X378:\\{execute\_fn}({\.{cite\$}})\X
\U342.
\:\X379:\\{execute\_fn}({\.{duplicate\$}})\X
\U342.
\:\X380:\\{execute\_fn}({\.{empty\$}})\X
\U342.
\:\X382:\\{execute\_fn}({\.{format.name\$}})\X
\U342.
\:\X421:\\{execute\_fn}({\.{if\$}})\X
\U341.
\:\X422:\\{execute\_fn}({\.{int.to.chr\$}})\X
\U342.
\:\X423:\\{execute\_fn}({\.{int.to.str\$}})\X
\U342.
\:\X424:\\{execute\_fn}({\.{missing\$}})\X
\U342.
\:\X425:\\{execute\_fn}({\.{newline\$}})\X
\U341.
\:\X426:\\{execute\_fn}({\.{num.names\$}})\X
\U342.
\:\X428:\\{execute\_fn}({\.{pop\$}})\X
\U341.
\:\X429:\\{execute\_fn}({\.{preamble\$}})\X
\U342.
\:\X430:\\{execute\_fn}({\.{purify\$}})\X
\U342.
\:\X434:\\{execute\_fn}({\.{quote\$}})\X
\U342.
\:\X435:\\{execute\_fn}({\.{skip\$}})\X
\U341.
\:\X436:\\{execute\_fn}({\.{stack\$}})\X
\U341.
\:\X437:\\{execute\_fn}({\.{substring\$}})\X
\U342.
\:\X439:\\{execute\_fn}({\.{swap\$}})\X
\U342.
\:\X441:\\{execute\_fn}({\.{text.length\$}})\X
\U342.
\:\X443:\\{execute\_fn}({\.{text.prefix\$}})\X
\U342.
\:\X446:\\{execute\_fn}({\.{top\$}})\X
\U341.
\:\X447:\\{execute\_fn}({\.{type\$}})\X
\U342.
\:\X448:\\{execute\_fn}({\.{warning\$}})\X
\U342.
\:\X449:\\{execute\_fn}({\.{while\$}})\X
\U341.
\:\X450:\\{execute\_fn}({\.{width\$}})\X
\U342.
\:\X454:\\{execute\_fn}({\.{write\$}})\X
\U342.
\con
